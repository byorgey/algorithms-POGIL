% -*- compile-command: "pdflatex AA-intro-metadata.tex" -*-

\documentclass{tufte-handout}

\title{Introduction to Asymptotic Analysis: abstract and metadata}

\begin{document}

\maketitle

\section{Abstract}

This is one of the first activities in my Algorithms course, a
required course for (mostly junior) computer science majors, with
prerequisites of CS1, Data Structures, and Discrete Math.  The
activity introduces big-O, big-Omega, and big-Theta notation by
example.  The students have all seen big-O notation in Data Structures
before, but the activity attempts to help them start building up to a
more nuanced understanding from scratch.  The next activity after this
one actually introduces the formal definition of the notations; this
activity is just trying to help them understand the important ideas in
preparation for seeing the real definition.  After doing this
activity, students should be able to correctly classify functions
which are similar to the given examples as being $O(n^2)$,
$\Omega(n^2)$, or $\Theta(n^2)$.

\section{Metadata}

\begin{tabular}{rl}
  \textbf{Level} & Undergraduate \\
  \textbf{Setting} & Classroom \\
  \textbf{Activity Type} & Learning Cycle \\
  \textbf{Discipline} & Computer Science \\
  \textbf{Course} & Algorithms \\
  \textbf{Keywords} & Big-O, asymptotic analysis
\end{tabular}



\end{document}