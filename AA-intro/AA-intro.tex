% -*- compile-command: "rubber -d --unsafe AA-intro.tex" -*-
\documentclass{tufte-handout}

\usepackage[answers]{../algo-activity}

\usepackage{pgfplots}
\pgfplotsset{width=10cm,compat=1.9}
\usepgfplotslibrary{external}
\tikzexternalize

\title{\thecourse: Introduction to Asymptotic Analysis}
\date{}

\begin{document}

\maketitle

\definecolor{cb_orange}{RGB}{230,159,0}
\definecolor{cb_skyblue}{RGB}{86,180,233}
\definecolor{cb_bluegreen}{RGB}{0,158,115}
\definecolor{cb_yellow}{RGB}{240,228,66}
\definecolor{cb_blue}{RGB}{0,114,178}
\definecolor{cb_vermillion}{RGB}{213,94,0}
\definecolor{cb_redpurple}{RGB}{204,121,167}

\begin{model}{Big-O and Big-$\Omega$}{bigO}
  \begin{center}
    \begin{pgfpicture}
\pgfpathrectangle{\pgfpointorigin}{\pgfqpoint{300.0000bp}{199.0000bp}}
\pgfusepath{use as bounding box}
\begin{pgfscope}
\definecolor{fc}{rgb}{0.0000,0.0000,0.0000}
\pgfsetfillcolor{fc}
\pgftransformshift{\pgfqpoint{220.0000bp}{166.6667bp}}
\pgftransformscale{1.6667}
\pgftext[]{$\Omega(n^2)$}
\end{pgfscope}
\begin{pgfscope}
\definecolor{fc}{rgb}{0.0000,0.0000,0.0000}
\pgfsetfillcolor{fc}
\pgftransformshift{\pgfqpoint{80.0000bp}{166.6667bp}}
\pgftransformscale{1.6667}
\pgftext[]{$O(n^2)$}
\end{pgfscope}
\begin{pgfscope}
\definecolor{fc}{rgb}{0.0000,0.0000,0.0000}
\pgfsetfillcolor{fc}
\pgfsetfillopacity{0.0000}
\pgfsetlinewidth{0.9798bp}
\definecolor{sc}{rgb}{0.0000,0.0000,0.0000}
\pgfsetstrokecolor{sc}
\pgfsetmiterjoin
\pgfsetbuttcap
\pgfpathqmoveto{300.0000bp}{100.0000bp}
\pgfpathqcurveto{300.0000bp}{155.2285bp}{255.2285bp}{200.0000bp}{200.0000bp}{200.0000bp}
\pgfpathqcurveto{144.7715bp}{200.0000bp}{100.0000bp}{155.2285bp}{100.0000bp}{100.0000bp}
\pgfpathqcurveto{100.0000bp}{44.7715bp}{144.7715bp}{0.0000bp}{200.0000bp}{0.0000bp}
\pgfpathqcurveto{255.2285bp}{-0.0000bp}{300.0000bp}{44.7715bp}{300.0000bp}{100.0000bp}
\pgfpathclose
\pgfusepathqfillstroke
\end{pgfscope}
\begin{pgfscope}
\definecolor{fc}{rgb}{0.0000,0.0000,0.0000}
\pgfsetfillcolor{fc}
\pgfsetfillopacity{0.0000}
\pgfsetlinewidth{0.9798bp}
\definecolor{sc}{rgb}{0.0000,0.0000,0.0000}
\pgfsetstrokecolor{sc}
\pgfsetmiterjoin
\pgfsetbuttcap
\pgfpathqmoveto{200.0000bp}{100.0000bp}
\pgfpathqcurveto{200.0000bp}{155.2285bp}{155.2285bp}{200.0000bp}{100.0000bp}{200.0000bp}
\pgfpathqcurveto{44.7715bp}{200.0000bp}{0.0000bp}{155.2285bp}{0.0000bp}{100.0000bp}
\pgfpathqcurveto{0.0000bp}{44.7715bp}{44.7715bp}{0.0000bp}{100.0000bp}{0.0000bp}
\pgfpathqcurveto{155.2285bp}{-0.0000bp}{200.0000bp}{44.7715bp}{200.0000bp}{100.0000bp}
\pgfpathclose
\pgfusepathqfillstroke
\end{pgfscope}
\begin{pgfscope}
\definecolor{fc}{rgb}{0.0000,0.0000,0.0000}
\pgfsetfillcolor{fc}
\pgftransformshift{\pgfqpoint{250.0000bp}{65.0000bp}}
\pgftransformscale{1.2500}
\pgftext[]{$2^n$}
\end{pgfscope}
\begin{pgfscope}
\definecolor{fc}{rgb}{0.0000,0.0000,0.0000}
\pgfsetfillcolor{fc}
\pgftransformshift{\pgfqpoint{250.0000bp}{88.3333bp}}
\pgftransformscale{1.2500}
\pgftext[]{$\frac{n^3}{1000}$}
\end{pgfscope}
\begin{pgfscope}
\definecolor{fc}{rgb}{0.0000,0.0000,0.0000}
\pgfsetfillcolor{fc}
\pgftransformshift{\pgfqpoint{250.0000bp}{111.6667bp}}
\pgftransformscale{1.2500}
\pgftext[]{$n^4 - 3n^2$}
\end{pgfscope}
\begin{pgfscope}
\definecolor{fc}{rgb}{0.0000,0.0000,0.0000}
\pgfsetfillcolor{fc}
\pgftransformshift{\pgfqpoint{250.0000bp}{135.0000bp}}
\pgftransformscale{1.2500}
\pgftext[]{$n^3$}
\end{pgfscope}
\begin{pgfscope}
\definecolor{fc}{rgb}{0.0000,0.0000,0.0000}
\pgfsetfillcolor{fc}
\pgftransformshift{\pgfqpoint{150.0000bp}{65.0000bp}}
\pgftransformscale{1.2500}
\pgftext[]{$\frac{n^2}{2} - n$}
\end{pgfscope}
\begin{pgfscope}
\definecolor{fc}{rgb}{0.0000,0.0000,0.0000}
\pgfsetfillcolor{fc}
\pgftransformshift{\pgfqpoint{150.0000bp}{88.3333bp}}
\pgftransformscale{1.2500}
\pgftext[]{$2n^2 + n + 1$}
\end{pgfscope}
\begin{pgfscope}
\definecolor{fc}{rgb}{0.0000,0.0000,0.0000}
\pgfsetfillcolor{fc}
\pgftransformshift{\pgfqpoint{150.0000bp}{111.6667bp}}
\pgftransformscale{1.2500}
\pgftext[]{$\frac{n^3 + 3}{n}$}
\end{pgfscope}
\begin{pgfscope}
\definecolor{fc}{rgb}{0.0000,0.0000,0.0000}
\pgfsetfillcolor{fc}
\pgftransformshift{\pgfqpoint{150.0000bp}{135.0000bp}}
\pgftransformscale{1.2500}
\pgftext[]{$n^2$}
\end{pgfscope}
\begin{pgfscope}
\definecolor{fc}{rgb}{0.0000,0.0000,0.0000}
\pgfsetfillcolor{fc}
\pgftransformshift{\pgfqpoint{50.0000bp}{65.0000bp}}
\pgftransformscale{1.2500}
\pgftext[]{$\frac{n^2 + 2}{n}$}
\end{pgfscope}
\begin{pgfscope}
\definecolor{fc}{rgb}{0.0000,0.0000,0.0000}
\pgfsetfillcolor{fc}
\pgftransformshift{\pgfqpoint{50.0000bp}{88.3333bp}}
\pgftransformscale{1.2500}
\pgftext[]{$2\sqrt n$}
\end{pgfscope}
\begin{pgfscope}
\definecolor{fc}{rgb}{0.0000,0.0000,0.0000}
\pgfsetfillcolor{fc}
\pgftransformshift{\pgfqpoint{50.0000bp}{111.6667bp}}
\pgftransformscale{1.2500}
\pgftext[]{$n$}
\end{pgfscope}
\begin{pgfscope}
\definecolor{fc}{rgb}{0.0000,0.0000,0.0000}
\pgfsetfillcolor{fc}
\pgftransformshift{\pgfqpoint{50.0000bp}{135.0000bp}}
\pgftransformscale{1.2500}
\pgftext[]{$6$}
\end{pgfscope}
\end{pgfpicture}


    \vspace{0.5in}

\begin{minipage}{\textwidth}
\begin{tikzpicture}
\begin{axis}[
    axis lines = left,
    xlabel = $n$,
%    ylabel = {$f(n)$},
    width = 5.5cm, height = 5.5cm,
    every axis plot/.append style={thick},
    legend style={at={(0.5, -0.7)},anchor=south},
]
\addplot [
    domain=1:5,
    samples=100,
    color=cb_blue,
    ]
    {(x^2+2)/x};
\addlegendentry{$f(n) = (n^2+2)/n$}
\addplot [
    domain=1:5,
    samples=100,
    color=cb_redpurple,
    style=densely dotted,
]
{(x^2)/2 - x};
\addlegendentry{$g(n) = n^2/2 - n$}
\addplot [
    domain=1:5,
    samples=100,
    color=cb_orange,
    style=densely dashed,
]
{x^3 / 1000};
\addlegendentry{$h(n) = n^3/1000$}
\end{axis}
\end{tikzpicture}
\hfill
\begin{tikzpicture}
\begin{axis}[
    axis lines = left,
    xlabel = $n$,
%    ylabel = {$f(n)$},
    width = 5.5cm, height = 5.5cm,
    every axis plot/.append style={thick},
    legend style={at={(0.5, -0.7)},anchor=south},
]
\addplot [
    domain=1:40,
    samples=100,
    color=cb_blue,
    ]
    {(x^2+2)/x};
\addlegendentry{$f$}
\addplot [
    domain=1:20,
    samples=100,
    color=cb_redpurple,
    style=densely dotted,
]
{(x^2)/2 - x};
\addlegendentry{$g$}
\addplot [
    domain=1:40,
    samples=100,
    color=cb_orange,
    style=densely dashed,
]
{x^3 / 1000};
\addlegendentry{$h$}
\end{axis}
\end{tikzpicture}
\hfill
\begin{tikzpicture}
\begin{axis}[
    axis lines = left,
    xlabel = $n$,
%    ylabel = {$f(n)$},
    width = 5.5cm, height = 5.5cm,
    every axis plot/.append style={thick},
    legend style={at={(0.5, -0.7)},anchor=south},
]
\addplot [
    domain=1:550,
    samples=100,
    color=cb_blue,
    ]
    {(x^2+2)/x};
\addlegendentry{$f$}
\addplot [
    domain=1:550,
    samples=100,
    color=cb_redpurple,
    style=densely dotted,
]
{(x^2)/2 - x};
\addlegendentry{$g$}
\addplot [
    domain=1:550,
    samples=100,
    color=cb_orange,
    style=densely dashed,
]
{x^3 / 1000};
\addlegendentry{$h$}
\end{axis}
\end{tikzpicture}
\end{minipage}

\end{center}
\end{model}

\newpage
\section{Critical Thinking Questions I}
\begin{objective}
  Students will describe asymptotic behavior of functions
  using big-$O$, big-$\Theta$, and big-$\Omega$ notation.
\end{objective}

\textbf{Important note}: although any previous experience you have
with big-$O$ notation may be helpful, I am not assuming that you
remember anything in particular.  When answering the following
questions, as much as possible, try to rely on the information
provided in Model 1 rather than on your memory.

\begin{questions}
\item Based on the Venn diagram in the model, say whether each
  function is $O(n^2)$, $\Omega(n^2)$, or both. \marginnote{$\Omega$
    is pronounced ``big omega'' (amusingly, ``o-mega'' is itself Greek
    for ``big O'', although they meant ``big'' in the sense of a long
    vowel, not uppercase).}
  \begin{subquestions}
  \item $2\sqrt n$
    \begin{answer}According to the Venn diagram, $2\sqrt n$ is $O(n^2)$.\end{answer}
  \item $n^3$
    \begin{answer}$\Omega(n^2)$\end{answer}
  \item $2n^2 + n + 1$
    \begin{answer}Both $O(n^2)$ and $\Omega(n^2)$.\end{answer}
  \item $2^n$
    \begin{answer}$\Omega(n^2)$.\end{answer}
  \end{subquestions}
\end{questions}

Consider the functions
\begin{align*}
  f(n) &= (n^2 + 2)/n, \\ g(n) &= n^2/2 - n, \text{and} \\ h(n) &= n^3/1000
\end{align*}
for which graphs are shown in the model.
\begin{questions}
\item On each of the following intervals, use the provided graphs to
  list the functions $f$, $g$, and $h$ in order from largest to
  smallest.
  \begin{subquestions}
  \item $2 \leq n \leq 4$
    \begin{answer}On this interval, $h(n) \leq g(n) \leq f(n)$.\end{answer}
  \item $5 \leq n \leq 30$
    \begin{answer}On the interval $5 \leq n \leq 30$, $h(n) \leq f(n) \leq g(n)$.\end{answer}
  \item $35 \leq n \leq 450$
    \begin{answer}$f(n) \leq h(n) \leq g(n)$.\end{answer}
  \end{subquestions}
\item Which function do you think is largest, and which the smallest,
  at $n = 600$?
  \begin{answer}At $n = 600$, $h$ is largest and $f$ is smallest.\end{answer}

\item Does this relative order continue for all $n \geq 600$, or do
  the functions ever change places again?  Justify your answer.
  \begin{answer}The functions never change places again.  Since $h$
    is proportional to $n^3$ it will continue to grow faster than
    $g$, which will in turn continue to grow faster than
    $f$.\end{answer}
\item How do you think your answers to the previous questions relate
  to whether each of $f$, $g$, and $h$ is $O(n^2)$, $\Omega(n^2)$, or
  both?
  \begin{answer}
    From the Venn diagram we can see that $f$ is $O(n^2)$, $g$ is
    both, and $h$ is $\Omega(n^2)$.  Eventually, $O(n^2)$ functions
    will be smallest, $\Omega(n^2)$ functions will be biggest, and
    functions that are both will be in the middle.
  \end{answer}
\end{questions}

\noindent Say whether you think each of the following statements is
true or false.  Give a short justification for each answer.

\begin{questions}
\item If $f(n)$ is $O(n^2)$, then it has $n^2$ in its definition.
  \begin{answer} False; \eg $6$ is $O(n^2)$ but does not have $n^2$
    in its definition.\end{answer}
\item If $f(n)$ has $n^2$ in its definition, then $f(n)$ is $O(n^2)$.
  \begin{answer} False; \eg $n^4 - 3n^2$ has $n^2$ in its definition
    but it is not $O(n^2)$.\end{answer}
\item If $f(n)$ is both $O(n^2)$ and $\Omega(n^2)$, then it has $n^2$
  in its definition.
  \begin{answer} False; \eg $\frac{n^3 + 3}{n}$. \end{answer}
\item If $f(n) \leq n^2$ for all $n \geq 0$, then $f(n)$ is $O(n^2)$.
  \begin{answer} This is true. All the functions in the model with
    this property are $O(n^2)$. \end{answer}
\item If $f(n)$ is $O(n^2)$, then $f(n) \leq n^2$ for all $n \geq 0$.
  \begin{answer}
    False.  For example, $2n^2 + n + 1$ is $O(n^2)$ but it is never
    $\leq n^2$.
  \end{answer}
\item If $f(n) \leq n^2$ for all $n$ that are sufficiently large, then
  $f(n)$ is $O(n^2)$.
  \begin{answer}
    True.  For example, $\frac{n^2 + 2}{n}$ is only $\leq n^2$ for $n
    > 1$.
  \end{answer}
\item If $f(n)$ is $O(n^2)$ and $g(n)$ is $\Omega(n^2)$, then $f(n)
  \leq g(n)$ for all $n \geq 0$.
  \begin{answer}
    False.  For example, consider $(n^2 + 2)/n$ (which is $O(n^2)$)
    and $n^3/1000$ (which is $\Omega(n^2)$).  From the graphs we know
    that the first function is actually greater than the second until
    around $n = 30$.
  \end{answer}
\item Every function $f(n)$ is either $O(n^2)$ or $\Omega(n^2)$ (or
  both).
  \begin{answer}
    This is true, though there is no particular way to answer this
    from the model alone; student answers may vary.
  \end{answer}
\item Using one or more complete English sentences and appropriate
  mathematical formalism, propose a definition of $O(n^2)$ by
  completing the following statement.

  \emph{A function $f(n)$ is $O(n^2)$ if and only if\dots}

  \begin{answer}
    Answers may vary.  Students may talk about how it only matters
    what happens to $f(n)$ when $n$ is big enough; how $f(n)$ should
    ``grow at a similar rate'' to $n^2$, or always be $\leq kn^2$ for
    some constant $k$; they may note that only the ``biggest term'' of
    $f(n)$ matters.
  \end{answer}
\end{questions}

\pause

%%%%%%%%%%%%%%%%%%%%%%%%%%%%%%%%%%%%%%%%%%%%%%%%%%%%%%%%%%%%

% Fall 2017: skipped these questions

\section{Critical Thinking Questions II}

\begin{questions}
\item In what way(s) do you think the definition of $\Omega(n^2)$ is similar to
  that of $O(n^2)$?
  \begin{answer}
    Answers may vary.  For example, the definitions both have
    something to do with $n$ being ``sufficiently large'', and they
    both involve comparing something to $n^2$.
  \end{answer}
\item In what way(s) do you think it is different?
  \begin{answer}
    One involves something being $\leq$ something else, and the other
    involves $\geq$.
  \end{answer}
% \item Using complete English sentences, propose a definition for
%   $\Omega(n^2)$.
\item If a function is both $O(n^2)$ and $\Omega(n^2)$, we say it is
  $\Theta(n^2)$.
  \marginnote{$\Theta$ is pronounced ``big theta''.}
  For each of the below functions, say whether you
  think it is $\Theta(n^2)$.  Justify your answers.
  \begin{subquestions}
  \item $3n^2 + 2n - 10$
    \begin{answer}
      This is $\Theta(n)$.  It is very similar to $2n^2 + n + 1$ which
      we know is $\Theta(n)$.
    \end{answer}
  \item $\displaystyle \frac{n^3 - 5}{n}$
    \begin{answer}
      This is also $\Theta(n)$.  It is similar to $(n^3 + 3)/n$.
    \end{answer}
  \item $\displaystyle \frac{n^3 - 5}{\sqrt n}$
    \begin{answer}
      This is $\Omega(n^2)$ but not $O(n^2)$.  Divding $n^3$ by $\sqrt
      n$ produces something that still grows faster than $n^2$.
    \end{answer}
  \item $(n+1)(n-2)$
    \begin{answer}
      This is $\Theta(n)$. It is equal to $n^2 - n - 2$.
    \end{answer}
  \item $n + n \sqrt n$
    \begin{answer}
      This is $O(n^2)$ but not $\Omega(n^2)$.  $n \sqrt n = n \cdot
      n^{1/2} = n^{1.5}$ which grows more slowly than $n^2$.
    \end{answer}
  \end{subquestions}
\item Do you think $n^2 \cdot \log_2 n$ is $O(n^2)$, $\Omega(n^2)$, or
  both?  Why?
  \begin{answer}
    Answers may vary.  In fact, it is $\Omega(n^2)$ but not $O(n^2)$;
    multiplying by $\log_2 n$ means it grows strictly faster than
    $n^2$.
  \end{answer}
\end{questions}

\end{document}
