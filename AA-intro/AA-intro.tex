% -*- compile-command: "stack exec -- rubber -d --unsafe AA-intro.tex" -*-
\documentclass{tufte-handout}

\usepackage{algo-activity}

\usepackage{pgfplots}
\pgfplotsset{width=10cm,compat=1.9}
\usepgfplotslibrary{external}
\tikzexternalize

\title{\thecourse: Introduction to Asymptotic Analysis}
\date{}

\begin{document}

\maketitle

\definecolor{cb_orange}{RGB}{230,159,0}
\definecolor{cb_skyblue}{RGB}{86,180,233}
\definecolor{cb_bluegreen}{RGB}{0,158,115}
\definecolor{cb_yellow}{RGB}{240,228,66}
\definecolor{cb_blue}{RGB}{0,114,178}
\definecolor{cb_vermillion}{RGB}{213,94,0}
\definecolor{cb_redpurple}{RGB}{204,121,167}

\begin{model}{Big-O and Big-$\Omega$}{bigO}
\begin{center}
\begin{diagram}[width=300]
  dia :: Diagram B
  dia = mconcat
    [ bigO # centerY # translateX (-15)
    , bigTheta # centerY # translateX 15  -- # translateY (-5)
    , bigOmega # centerY # translateX 45
    , circle 30
    , circle 30 # translateX 30
    , text "$O(n^2)$" # fontSizeL 4 # translate ((-6) ^& 20)
    -- , text "$\\Theta(n^2)$" # fontSizeL 4 # translate (15 ^& 15)
    , text "$\\Omega(n^2)$" # fontSizeL 4 # translate (36 ^& 20)
    ]
    where
      mkColumn = vsep 7 . map (fontSizeL 3 . text . ("$"++) . (++"$"))
      bigO = mkColumn
        [ "6"
        , "n"
        , "2\\sqrt n"
        , "\\frac{n^2 + 2}{n}"
        ]
      bigTheta = mkColumn
        [ "n^2"
        , "\\frac{n^3 + 3}{n}"
        , "2n^2 + n + 1"
        , "\\frac{n^2}{2} - n"
        ]
      bigOmega = mkColumn
        [ "n^3"
        , "n^4 - 3n^2"
        , "\\frac{n^3}{1000}"
        , "2^n"
        ]
\end{diagram}

\vspace{0.5in}

\begin{minipage}{\textwidth}
\begin{tikzpicture}
\begin{axis}[
    axis lines = left,
    xlabel = $n$,
%    ylabel = {$f(n)$},
    width = 5.5cm, height = 5.5cm,
    every axis plot/.append style={thick},
    legend style={at={(0.5, -0.7)},anchor=south},
]
\addplot [
    domain=1:5,
    samples=100,
    color=cb_blue,
    ]
    {(x^2+2)/x};
\addlegendentry{$f(n) = (n^2+2)/n$}
\addplot [
    domain=1:5,
    samples=100,
    color=cb_redpurple,
    style=densely dotted,
]
{(x^2)/2 - x};
\addlegendentry{$g(n) = n^2/2 - n$}
\addplot [
    domain=1:5,
    samples=100,
    color=cb_orange,
    style=densely dashed,
]
{x^3 / 1000};
\addlegendentry{$h(n) = n^3/1000$}
\end{axis}
\end{tikzpicture}
\hfill
\begin{tikzpicture}
\begin{axis}[
    axis lines = left,
    xlabel = $n$,
%    ylabel = {$f(n)$},
    width = 5.5cm, height = 5.5cm,
    every axis plot/.append style={thick},
    legend style={at={(0.5, -0.7)},anchor=south},
]
\addplot [
    domain=1:40,
    samples=100,
    color=cb_blue,
    ]
    {(x^2+2)/x};
\addlegendentry{$f$}
\addplot [
    domain=1:20,
    samples=100,
    color=cb_redpurple,
    style=densely dotted,
]
{(x^2)/2 - x};
\addlegendentry{$g$}
\addplot [
    domain=1:40,
    samples=100,
    color=cb_orange,
    style=densely dashed,
]
{x^3 / 1000};
\addlegendentry{$h$}
\end{axis}
\end{tikzpicture}
\hfill
\begin{tikzpicture}
\begin{axis}[
    axis lines = left,
    xlabel = $n$,
%    ylabel = {$f(n)$},
    width = 5.5cm, height = 5.5cm,
    every axis plot/.append style={thick},
    legend style={at={(0.5, -0.7)},anchor=south},
]
\addplot [
    domain=1:550,
    samples=100,
    color=cb_blue,
    ]
    {(x^2+2)/x};
\addlegendentry{$f$}
\addplot [
    domain=1:550,
    samples=100,
    color=cb_redpurple,
    style=densely dotted,
]
{(x^2)/2 - x};
\addlegendentry{$g$}
\addplot [
    domain=1:550,
    samples=100,
    color=cb_orange,
    style=densely dashed,
]
{x^3 / 1000};
\addlegendentry{$h$}
\end{axis}
\end{tikzpicture}
\end{minipage}

\end{center}
\end{model}

\newpage
\section{Critical Thinking Questions I}
\begin{objective}
  Students will describe asymptotic behavior of functions
  using big-$O$, big-$\Theta$, and big-$\Omega$ notation.
\end{objective}

\textbf{Important note}: although your previous experience with
big-$O$ notation will certainly be helpful, I am not assuming that you
remember anything in particular.  When answering the following
questions, as much as possible, try to rely on the information
provided in Model 1 rather than on your memory.

\begin{questions}
\item Based on the Venn diagram in the model, say whether each
  function is $O(n^2)$, $\Omega(n^2)$, or both. \marginnote{$\Omega$
    is pronounced ``big omega'' (amusingly, ``o-mega'' is itself Greek
    for ``big O'', although they meant ``big'' in the sense of a long
    vowel, not uppercase).}
  \begin{enumerate}[label=(\alph*)]
    \item $2\sqrt n$
    \item $n^3$
    \item $2n^2 + n + 1$
    \item $2^n$
    \end{enumerate}
\end{questions}
Consider the functions
\begin{align*}
  f(n) &= (n^2 + 2)/n, \\ g(n) &= n^2/2 - n, \text{and} \\ h(n) &= n^3/1000
\end{align*}
for which graphs are shown in the model.
\begin{questions}
  \item On each of the following intervals, list the functions $f$,
    $g$, and $h$ from largest to smallest.
  \begin{subquestions}
  \item $n \in [2,4]$
  \item $n \in [5,30]$
  \item $n \in [35,450]$
  \end{subquestions}
\item Which function is largest, and which the smallest, at $n = 600$?
\item Does this relative order continue for all $n \geq 600$, or do
  the functions ever change places again?  Justify your answer.
  \newpage
\item How do you think your answers to the previous questions relate
  to whether each of $f$, $g$, and $h$ is $O(n^2)$, $\Omega(n^2)$, or both?
\end{questions}

\noindent Say whether you think each of the following statements is
true or false.  Give a short justification for each answer.

\begin{questions}
\item If $f(n)$ is $O(n^2)$, then it has $n^2$ in its definition.
\item If $f(n)$ has $n^2$ in its definition, then $f(n)$ is $O(n^2)$.
\item If $f(n)$ is both $O(n^2)$ and $\Omega(n^2)$, then it has $n^2$
  in its definition.
\item If $f(n) \leq n^2$ for all $n \geq 0$, then $f(n)$ is $O(n^2)$.
\item If $f(n)$ is $O(n^2)$, then $f(n) \leq n^2$ for all $n \geq 0$.
\item If $f(n) \leq n^2$ for all $n$ that are sufficiently large, then
  $f(n)$ is $O(n^2)$.
\item If $f(n)$ is $O(n^2)$ and $g(n)$ is $\Omega(n^2)$, then $f(n)
  \leq g(n)$ for all $n \geq 0$.
\item Every function $f(n)$ is either $O(n^2)$ or $\Omega(n^2)$ (or
  both).
  \newpage
\item Using one or more complete English sentences and appropriate
  mathematical formalism, propose a definition of $O(n^2)$ by
  completing the following statement.

  \emph{A function $f(n)$ is $O(n^2)$ if and only if\dots}
\end{questions}

\pause

%%%%%%%%%%%%%%%%%%%%%%%%%%%%%%%%%%%%%%%%%%%%%%%%%%%%%%%%%%%%

% Fall 2017: skipped these questions

\section{Critical Thinking Questions II}

\begin{questions}
\item In what way(s) do you think the definition of $\Omega(n^2)$ is similar to
  that of $O(n^2)$?
\item In what way(s) do you think it is different?
% \item Using complete English sentences, propose a definition for
%   $\Omega(n^2)$.
\item If a function is both $O(n^2)$ and $\Omega(n^2)$, we say it is
  $\Theta(n^2)$.
  \marginnote{$\Theta$ is pronounced ``big theta''.}
  For each of the below functions, say whether you
  think it is $\Theta(n^2)$.  Justify your answers.
  \begin{subquestions}
  \item $3n^2 + 2n - 10$
  \item $\displaystyle \frac{n^3 - 5}{n}$
  \item $\displaystyle \frac{n^3 - 5}{\sqrt n}$
  \item $(n+1)(n-2)$
  \item $n + n \sqrt n$
  \end{subquestions}
\item Do you think $n^2 \cdot \log_2 n$ is $O(n^2)$, $\Omega(n^2)$, or
  both?  Why?
\end{questions}

\end{document}
