% -*- compile-command: "rubber -d --unsafe AA-intro.tex" -*-
\documentclass{tufte-handout}

\usepackage{algo-activity}

\usepackage{pgfplots}
\pgfplotsset{width=10cm,compat=1.9}
\usepgfplotslibrary{external}
\tikzexternalize

\title{\thecourse: Introduction to Asymptotic Analysis}
\date{}

\begin{document}

\maketitle

\definecolor{cb_orange}{RGB}{230,159,0}
\definecolor{cb_skyblue}{RGB}{86,180,233}
\definecolor{cb_bluegreen}{RGB}{0,158,115}
\definecolor{cb_yellow}{RGB}{240,228,66}
\definecolor{cb_blue}{RGB}{0,114,178}
\definecolor{cb_vermillion}{RGB}{213,94,0}
\definecolor{cb_redpurple}{RGB}{204,121,167}

\begin{model}{Big-O and Big-$\Omega$}{bigO}
  \begin{center}
    \begin{pgfpicture}
\pgfpathrectangle{\pgfpointorigin}{\pgfqpoint{300.0000bp}{199.0000bp}}
\pgfusepath{use as bounding box}
\begin{pgfscope}
\definecolor{fc}{rgb}{0.0000,0.0000,0.0000}
\pgfsetfillcolor{fc}
\pgftransformshift{\pgfqpoint{220.0000bp}{166.6667bp}}
\pgftransformscale{1.6667}
\pgftext[]{$\Omega(n^2)$}
\end{pgfscope}
\begin{pgfscope}
\definecolor{fc}{rgb}{0.0000,0.0000,0.0000}
\pgfsetfillcolor{fc}
\pgftransformshift{\pgfqpoint{80.0000bp}{166.6667bp}}
\pgftransformscale{1.6667}
\pgftext[]{$O(n^2)$}
\end{pgfscope}
\begin{pgfscope}
\definecolor{fc}{rgb}{0.0000,0.0000,0.0000}
\pgfsetfillcolor{fc}
\pgfsetfillopacity{0.0000}
\pgfsetlinewidth{0.9798bp}
\definecolor{sc}{rgb}{0.0000,0.0000,0.0000}
\pgfsetstrokecolor{sc}
\pgfsetmiterjoin
\pgfsetbuttcap
\pgfpathqmoveto{300.0000bp}{100.0000bp}
\pgfpathqcurveto{300.0000bp}{155.2285bp}{255.2285bp}{200.0000bp}{200.0000bp}{200.0000bp}
\pgfpathqcurveto{144.7715bp}{200.0000bp}{100.0000bp}{155.2285bp}{100.0000bp}{100.0000bp}
\pgfpathqcurveto{100.0000bp}{44.7715bp}{144.7715bp}{0.0000bp}{200.0000bp}{0.0000bp}
\pgfpathqcurveto{255.2285bp}{-0.0000bp}{300.0000bp}{44.7715bp}{300.0000bp}{100.0000bp}
\pgfpathclose
\pgfusepathqfillstroke
\end{pgfscope}
\begin{pgfscope}
\definecolor{fc}{rgb}{0.0000,0.0000,0.0000}
\pgfsetfillcolor{fc}
\pgfsetfillopacity{0.0000}
\pgfsetlinewidth{0.9798bp}
\definecolor{sc}{rgb}{0.0000,0.0000,0.0000}
\pgfsetstrokecolor{sc}
\pgfsetmiterjoin
\pgfsetbuttcap
\pgfpathqmoveto{200.0000bp}{100.0000bp}
\pgfpathqcurveto{200.0000bp}{155.2285bp}{155.2285bp}{200.0000bp}{100.0000bp}{200.0000bp}
\pgfpathqcurveto{44.7715bp}{200.0000bp}{0.0000bp}{155.2285bp}{0.0000bp}{100.0000bp}
\pgfpathqcurveto{0.0000bp}{44.7715bp}{44.7715bp}{0.0000bp}{100.0000bp}{0.0000bp}
\pgfpathqcurveto{155.2285bp}{-0.0000bp}{200.0000bp}{44.7715bp}{200.0000bp}{100.0000bp}
\pgfpathclose
\pgfusepathqfillstroke
\end{pgfscope}
\begin{pgfscope}
\definecolor{fc}{rgb}{0.0000,0.0000,0.0000}
\pgfsetfillcolor{fc}
\pgftransformshift{\pgfqpoint{250.0000bp}{65.0000bp}}
\pgftransformscale{1.2500}
\pgftext[]{$2^n$}
\end{pgfscope}
\begin{pgfscope}
\definecolor{fc}{rgb}{0.0000,0.0000,0.0000}
\pgfsetfillcolor{fc}
\pgftransformshift{\pgfqpoint{250.0000bp}{88.3333bp}}
\pgftransformscale{1.2500}
\pgftext[]{$\frac{n^3}{1000}$}
\end{pgfscope}
\begin{pgfscope}
\definecolor{fc}{rgb}{0.0000,0.0000,0.0000}
\pgfsetfillcolor{fc}
\pgftransformshift{\pgfqpoint{250.0000bp}{111.6667bp}}
\pgftransformscale{1.2500}
\pgftext[]{$n^4 - 3n^2$}
\end{pgfscope}
\begin{pgfscope}
\definecolor{fc}{rgb}{0.0000,0.0000,0.0000}
\pgfsetfillcolor{fc}
\pgftransformshift{\pgfqpoint{250.0000bp}{135.0000bp}}
\pgftransformscale{1.2500}
\pgftext[]{$n^3$}
\end{pgfscope}
\begin{pgfscope}
\definecolor{fc}{rgb}{0.0000,0.0000,0.0000}
\pgfsetfillcolor{fc}
\pgftransformshift{\pgfqpoint{150.0000bp}{65.0000bp}}
\pgftransformscale{1.2500}
\pgftext[]{$\frac{n^2}{2} - n$}
\end{pgfscope}
\begin{pgfscope}
\definecolor{fc}{rgb}{0.0000,0.0000,0.0000}
\pgfsetfillcolor{fc}
\pgftransformshift{\pgfqpoint{150.0000bp}{88.3333bp}}
\pgftransformscale{1.2500}
\pgftext[]{$2n^2 + n + 1$}
\end{pgfscope}
\begin{pgfscope}
\definecolor{fc}{rgb}{0.0000,0.0000,0.0000}
\pgfsetfillcolor{fc}
\pgftransformshift{\pgfqpoint{150.0000bp}{111.6667bp}}
\pgftransformscale{1.2500}
\pgftext[]{$\frac{n^3 + 3}{n}$}
\end{pgfscope}
\begin{pgfscope}
\definecolor{fc}{rgb}{0.0000,0.0000,0.0000}
\pgfsetfillcolor{fc}
\pgftransformshift{\pgfqpoint{150.0000bp}{135.0000bp}}
\pgftransformscale{1.2500}
\pgftext[]{$n^2$}
\end{pgfscope}
\begin{pgfscope}
\definecolor{fc}{rgb}{0.0000,0.0000,0.0000}
\pgfsetfillcolor{fc}
\pgftransformshift{\pgfqpoint{50.0000bp}{65.0000bp}}
\pgftransformscale{1.2500}
\pgftext[]{$\frac{n^2 + 2}{n}$}
\end{pgfscope}
\begin{pgfscope}
\definecolor{fc}{rgb}{0.0000,0.0000,0.0000}
\pgfsetfillcolor{fc}
\pgftransformshift{\pgfqpoint{50.0000bp}{88.3333bp}}
\pgftransformscale{1.2500}
\pgftext[]{$2\sqrt n$}
\end{pgfscope}
\begin{pgfscope}
\definecolor{fc}{rgb}{0.0000,0.0000,0.0000}
\pgfsetfillcolor{fc}
\pgftransformshift{\pgfqpoint{50.0000bp}{111.6667bp}}
\pgftransformscale{1.2500}
\pgftext[]{$n$}
\end{pgfscope}
\begin{pgfscope}
\definecolor{fc}{rgb}{0.0000,0.0000,0.0000}
\pgfsetfillcolor{fc}
\pgftransformshift{\pgfqpoint{50.0000bp}{135.0000bp}}
\pgftransformscale{1.2500}
\pgftext[]{$6$}
\end{pgfscope}
\end{pgfpicture}


    \vspace{0.5in}

\begin{minipage}{\textwidth}
\begin{tikzpicture}
\begin{axis}[
    axis lines = left,
    xlabel = $n$,
    width = 5.5cm, height = 5.5cm,
    every axis plot/.append style={thick},
    legend style={at={(0.5, -0.7)},anchor=south},
]
\addplot [
    domain=1:5,
    samples=100,
    color=cb_blue,
    ]
    {(x^2+2)/x};
\addlegendentry{$f(n) = (n^2+2)/n$}
\addplot [
    domain=1:5,
    samples=100,
    color=cb_redpurple,
    style=densely dotted,
]
{(x^2)/2 - x};
\addlegendentry{$g(n) = n^2/2 - n$}
\addplot [
    domain=1:5,
    samples=100,
    color=cb_orange,
    style=densely dashed,
]
{x^3 / 1000};
\addlegendentry{$h(n) = n^3/1000$}
\end{axis}
\end{tikzpicture}
\hfill
\begin{tikzpicture}
\begin{axis}[
    axis lines = left,
    xlabel = $n$,
%    ylabel = {$f(n)$},
    width = 5.5cm, height = 5.5cm,
    every axis plot/.append style={thick},
    legend style={at={(0.5, -0.7)},anchor=south},
]
\addplot [
    domain=1:40,
    samples=100,
    color=cb_blue,
    ]
    {(x^2+2)/x};
\addlegendentry{$f$}
\addplot [
    domain=1:20,
    samples=100,
    color=cb_redpurple,
    style=densely dotted,
]
{(x^2)/2 - x};
\addlegendentry{$g$}
\addplot [
    domain=1:40,
    samples=100,
    color=cb_orange,
    style=densely dashed,
]
{x^3 / 1000};
\addlegendentry{$h$}
\end{axis}
\end{tikzpicture}
\hfill
\begin{tikzpicture}
\begin{axis}[
    axis lines = left,
    xlabel = $n$,
%    ylabel = {$f(n)$},
    width = 5.5cm, height = 5.5cm,
    every axis plot/.append style={thick},
    legend style={at={(0.5, -0.7)},anchor=south},
]
\addplot [
    domain=1:550,
    samples=100,
    color=cb_blue,
    ]
    {(x^2+2)/x};
\addlegendentry{$f$}
\addplot [
    domain=1:550,
    samples=100,
    color=cb_redpurple,
    style=densely dotted,
]
{(x^2)/2 - x};
\addlegendentry{$g$}
\addplot [
    domain=1:550,
    samples=100,
    color=cb_orange,
    style=densely dashed,
]
{x^3 / 1000};
\addlegendentry{$h$}
\end{axis}
\end{tikzpicture}
\end{minipage}

\end{center}
\end{model}

\newpage
\section{Critical Thinking Questions I (20 minutes)}

\newcommand{\lobjectiveA}{Extrapolating from examples, students
  will develop and apply informal definitions to classify functions as
  $O(n^2)$, $\Omega(n^2)$, or $\Theta(n^2)$.}

\newcommand{\pobjectiveA}{Students will process information from a
  model to explore the meaning of big-O and big-Omega notation.}

\newcommand{\pobjectiveB}{Students will think critically to discover
  counterexamples and assemble evidence.}

\begin{objective}
  \lobjectiveA
\end{objective}

\begin{pobjective}
  \pobjectiveA
\end{pobjective}

\begin{pobjective}
  \pobjectiveB
\end{pobjective}

\textbf{Important note}: although any previous experience you have
with big-$O$ notation may be helpful, it is \textbf{not} assumed that
you remember anything in particular!  When answering the following
questions, as much as possible, try to rely on the information
provided in Model 1 rather than on your memory.

\begin{questions}
\item \textbf{Working together}, based on the \textbf{Venn diagram} in
  the model, say whether each function is $O(n^2)$, $\Omega(n^2)$, or
  both. \marginnote{$\Omega$ is pronounced ``big omega'' (amusingly,
    ``o-mega'' is itself Greek for ``big O'', although they meant
    ``big'' in the sense of a long vowel, not uppercase).}
  \begin{subquestions}
  \item $\frac{n^2 + 2}{n}$
    \begin{answer}According to the Venn diagram, $\frac{n^2 + 2}{n}$ is $O(n^2)$.\end{answer}
  \item $\frac{n^3}{1000}$
    \begin{answer}$\Omega(n^2)$\end{answer}
  \item $\frac{n^2}{2} - n$
    \begin{answer}Both $O(n^2)$ and $\Omega(n^2)$.\end{answer}
  \item $2^n$
    \begin{answer}$\Omega(n^2)$.\end{answer}
  \end{subquestions}
\end{questions}

\hrule \bigskip

For Questions \ref{q:biggest}--\ref{q:synthesize}, consider the functions
\begin{align*}
  f(n) &= (n^2 + 2)/n, \\ g(n) &= n^2/2 - n, \text{and} \\ h(n) &= n^3/1000.
\end{align*}
Graphs of these functions are shown in the model (or rather,
\emph{one} graph is shown three times at different zoom levels).
\begin{questions}
\item \label{q:biggest} Look at the graphs to determine which function is
  biggest when $2 \leq n \leq 4$.
  \begin{answer}
    The left-hand graph shows that $f(n)$ (the blue line) is biggest
    on this interval.
  \end{answer}
\item \label{q:table} The following table has four columns
  representing different intervals for $n$.  For each interval, the
  table is supposed to show which function is smallest, which is
  biggest, and which is in between. A couple entries have already been
  filled in for you.  Using the graphs in the model, fill in the rest
  of the table. Note that the graphs do not quite show what happens at
  $n = 600$; when filling in the last column of the table, simply use
  your best judgment to predict what will happen.

  \textbf{Make sure your group agrees} on the best way to fill in the table.

  \setlength{\tabcolsep}{20pt}
  \renewcommand{\arraystretch}{2}
  \begin{fullwidth}
  \begin{tabular}{c|cccc}
    biggest & $f$ & & & \\
    mediumest & & $f$ & & \\
    smallest & & & & \\
    \hline
        & $2 \leq n \leq 4$ & $5 \leq n \leq 30$ & $35 \leq n \leq 450$ & $n = 600$
  \end{tabular}
  \end{fullwidth}

  \begin{answer}
    \setlength{\tabcolsep}{20pt}
    \renewcommand{\arraystretch}{1}
    \begin{tabular}{cccc}
    $f$ & $g$ & $g$ & $h$ \\
    $g$ & $f$ & $h$ & $g$ \\
    $h$ & $h$ & $f$ & $f$
    \end{tabular}
  \end{answer}

\item Does the same relative order continue for all $n \geq 600$, or do
  the functions ever change places again?  Justify your answer.
  \begin{answer}The functions never change places again.  Since $h$
    is proportional to $n^3$ it will continue to grow faster than
    $g$, which will in turn continue to grow faster than
    $f$.\end{answer}

\item Look at all the functions in the Venn diagram which are
  \emph{both} $O(n^2)$ \emph{and} $\Omega(n^2)$.  What do they have in
  common?

\item Now look at the functions which are $O(n^2)$ but \emph{not}
  $\Omega(n^2)$.  What do they have in common?

% \item Label each of the following statements as True or False, and
%   write a sentence or phrase explaining your reasoning.
%   \begin{subquestions}
%   \item XXX
%   \end{subquestions}

\item Suppose we have three algorithms to solve a particular problem:
  \begin{itemize}
  \item Algorithm F takes $f(n) = (n^2 + 2)/n$ seconds to solve
    the problem on a particular computer when given an input of size $n$.
  \item Algorithm G takes $g(n) = n^2/2 - n$ seconds to solve
    the problem on a particular computer when given an input of size $n$.
  \item Algorithm H takes $h(n) = n^3/1000$ seconds to solve
    the problem on a particular computer when given an input of size $n$.
  \end{itemize}

  Label each of the following statements as True or False, and write
  a sentence or phrase explaining your reasoning.
  \begin{subquestions}
    \item Algorithm H would be the best choice if we only ever need to
      solve the problem for small ($n \leq 10$) inputs.
      \begin{answer}
        True.  For $n \leq 10$, $h(n)$ is much smaller than $f(n)$ and
        $g(n)$, so Algorithm H would be fastest.
      \end{answer}
    \item Algorithm G would be the best choice if we need to solve the
      problem for very large ($n \geq 10^6$) inputs.
      \begin{answer}
        False.  $h(n) > g(n) > f(n)$ for $n \geq 10^6$, so Algorithm F
        would actually be the best choice.
      \end{answer}
    \item Algorithm G will take exactly $g(n) = n^2/2 - n$ seconds to
      solve a problem of size $n$ on \emph{any} computer.
      \begin{answer}
        False.  Different computers execute instructions at different
        speeds, so Algorithm G would not take exactly this long on
        every possible computer.
      \end{answer}
    \item If we run Algorithm G on a different computer, there is some
      positive real number $k$ such that Algorithm G will take
      $k (n^2/2 - n)$ seconds to solve a problem of size $n$ on that
      computer.
      \begin{answer}
        True\dots sort of.  Under a simple model where different
        computers simply execute the same machine instructions in
        sequence and each instruction takes the same amount of time,
        the running times of an algorithm on different computers would
        indeed be related to each other by a constant factor.  This is
        a perfectly acceptable mental model for students to have in
        this course.

        In actual practice, there are any number of reasons for this
        to be false.  For example, the instruction set on the two
        computers might be different, leading to very different
        compiled executables with different performance
        characteristics.  Things like cache size can also impact
        running times in ways that are hard to predict.  Students are
        not expected to come up with examples like this, but if they
        do it could lead to an interesting discussion.
      \end{answer}
    \item If we need to solve inputs of any size $n \geq 10$ in at most
      $n^2/4 - n/2$ seconds, we could run Algorithm G but on a computer
      that is twice as fast.
      \begin{answer}
        True.
      \end{answer}
    \item If we need to solve inputs of any size $n \geq 10$ in at most
      $n^2/4 - n/2$ seconds, we could use Algorithm F on a sufficiently
      fast computer.
      \begin{answer}
        This is true also, since $n^2$ grows faster than $n$, and
        there is some constant $k$ for which
        $k(n + 2/n) \leq n^2/4 - n/2$.  Graphically, since $n^2$ grows
        faster than $n$, we can shrink $(n + 2/n)$ enough so it fits
        under $n^2/4 - n/2$.
      \end{answer}
    \item If we need to solve inputs of any size $n \geq 10$ in at most
      $n^2/4 - n/2$ seconds, we could run Algorithm H on a sufficiently
      fast computer.
      \begin{answer}
        False.  Intuitively, no matter how fast the computer is, once
        $n$ gets big enough Algorithm H will be too slow.  Formally,
        no matter what constant $k$ we pick, $kn^3/1000 > n^2/4 - n/2$
        for some sufficiently large $n$.
      \end{answer}
    \item Suppose Algorithm J solves the problem for inputs of size
      $n$ in some amount of time that is $O(n^2)$.  In general,
      assuming $n$ may be large, we would prefer Algorithm J over
      Algorithm H.
      \begin{answer}
        True.  A function which is $O(n^2)$ will grow more slowly than
        $h(n) = n^3/1000$.  For large enough $n$, Algorithm J will be
        faster.
      \end{answer}
  \end{subquestions}

  %% XXX
  %% - Eventually, for big enough n
  %% - grows at the same rate
  %% - look at fastest-growing term??
\newpage
\item \label{q:synthesize} Based on the model and your answers to the
  previous questions, match each statement on the left with an
  appropriate informal\marginnote{You will see more formal definitions
    on the next activity!} definition on the right. $q(n)$ represents an
  arbitrary function.  \bigskip

  \begin{fullwidth}
  \begin{center}
  \begin{minipage}{0.4\linewidth}
    A function $q(n)$ is $O(n^2)$ \bigskip

    A function $q(n)$ is $\Omega(n^2)$
  \end{minipage}
  \begin{minipage}{0.4\linewidth}
     $q(n)$ is greater than $n^2$ for all $n \geq 0$ \bigskip

     Eventually, for big enough values of $n$, $q(n)$ grows at
      a similar or lower rate than $n^2$ \bigskip

     $q(n)$ grows more slowly than $n^2$ \bigskip

     $q(n) \geq n^2$ for big enough values of $n$ \bigskip

     The definition of $q(n)$ has $n^2$ in it \bigskip

    $q(n)$ eventually grows at a similar or higher rate
      than $n^2$
  \end{minipage}
  \end{center}
  \end{fullwidth}

  \begin{answer}
    \begin{itemize}
    \item ``$q(n)$ is $O(n^2)$'' goes with ``Eventually, for big enough
      values of $n$, $q(n)$ grows at the same rate or more slowly than
      $n^2$''.  The function $g(n)$ grows at the same rate as $n^2$,
      and $f(n)$ grows more slowly than $n^2$; both are $O(n^2)$.  We
      also know that it only matters what happens for big enough $n$,
      because both $f(n)$ and $g(n)$ start out larger than $h(n)$, but
      eventually $h(n)$ becomes larger, and $h(n)$ is not $O(n^2)$.
    \item ``$q(n)$ is $\Omega(n^2)$'' goes with ``$q(n)$ eventually
      grows at a similar rate or more quickly than $n^2$''.  The
      reasoning is similar to that for $O(n^2)$.
    \end{itemize}
  \end{answer}

\item \label{q:incorrect} Choose two \emph{incorrect} definitions from
  the previous question.  For each one, write one or two sentences
  explaining why it is incorrect.  Be sure to mention evidence from
  your previous answers.
  \begin{answer}
    \begin{itemize}
    \item ``$q(n)$ is greater than $n^2$ for all $n \geq 0$'': this is
      an incorrect definition of $\Omega(n^2)$ since it requires
      something to be true for all $n \geq 0$, whereas the real
      definition only cares what happens once $n$ gets big enough.
      For example, we know from the model that $h(n) = n^3/1000$ is
      $\Omega(n^2)$, but it is only greater than $n^2$ for $n > 1000$.
    \item ``$q(n)$ grows more slowly than $n^2$'': a function which is
      $O(n^2)$ may grow more slowly \emph{or at the same rate} as
      $n^2$.  For example, $2n^2 + n + 1$ is $O(n^2)$ but does not
      grow more slowly than $n^2$.
    \item ``$q(n) \geq n^2$ for big enough values of $n$'': for $q(n)$
      to be $\Omega(n^2)$, we only require that it grows at a similar
      or greater rate as $n^2$, not that it is literally greater than
      $n^2$.  For example, from the model, $g(n) = n^2/2 - n$ is
      $\Omega(n^2)$, but it is \emph{never} greater than $n^2$.
    \item ``The definition of $q(n)$ has $n^2$ in it'': there are
      clearly functions in the model which are $O(n^2)$ or
      $\Omega(n^2)$ that do not mention $n^2$; for example, $2^n$ is
      $\Omega(n^2)$ and $n$ is $O(n^2)$.
    \end{itemize}
  \end{answer}
\end{questions}

\pause

\section{Critical Thinking Questions II (10 minutes)}

\begin{questions}
\item \label{q:classify} If a function is both $O(n^2)$ and
  $\Omega(n^2)$, we say it is $\Theta(n^2)$.  \marginnote{$\Theta$ is
    pronounced ``big theta''.}  For each of the below functions, say
  whether you think it is $\Theta(n^2)$, only $O(n^2)$, or
  only $\Omega(n^2)$.  Justify your answers.
  \begin{subquestions}
  \item $3n^2 + 2n - 10$
    \begin{answer}
      This is $\Theta(n^2)$.  It is very similar to $2n^2 + n + 1$ which
      we know is $\Theta(n^2)$.  It grows at a similar rate to $n^2$.
    \end{answer}
  \item $\displaystyle \frac{n^3 - 5}{n}$
    \begin{answer}
      This is also $\Theta(n^2)$.  It is similar to $(n^3 + 3)/n$.  If
      we rewrite it as $n^2 + 3/n$ we can see that it will also grow
      at a similar rate to $n^2$.
    \end{answer}
  \item $\displaystyle \frac{n^3 - 5}{\sqrt n}$
    \begin{answer}
      This is $\Omega(n^2)$ but not $O(n^2)$.  Divding $n^3$ by $\sqrt
      n$ produces something that still grows faster than $n^2$.
    \end{answer}
  \item $(n+1)(n-2)$
    \begin{answer}
      This is $\Theta(n^2)$. It is equal to $n^2 - n - 2$.
    \end{answer}
  \item $n + n \sqrt n$
    \begin{answer}
      This is $O(n^2)$ but not $\Omega(n^2)$.  $n \sqrt n = n \cdot
      n^{1/2} = n^{1.5}$ which grows more slowly than $n^2$.
    \end{answer}
  \item $n^2 \cdot \log_2 n$
    \begin{answer}
      This is $\Omega(n^2)$ but not $O(n^2)$; multiplying by
      $\log_2 n$ means it grows strictly faster than $n^2$.
    \end{answer}
  \end{subquestions}
\item In your answers to \pref{q:classify}, in which cases did you
  make use of evidence from the model (the Venn diagram or graphs) to
  justify your answers?  In which cases did you make use of team
  members' previous knowledge?
  \begin{answer}
    Student answers will vary.
  \end{answer}
\end{questions}

\newpage

\section{Facilitation plan}
\label{sec:facilitation}

\section{Learning Objectives}

\subsection{Content objectives}

\begin{itemize}
\item \lobjectiveA
\end{itemize}

\subsection{Process objectives}

\begin{itemize}
\item \pobjectiveA
\item \pobjectiveB
\end{itemize}

\section{Announcements (2 minutes)}

\begin{itemize}
\item Remember HW 1 due Friday.  Start early, come ask for help if you
  need it.
\item Today, take a role you haven't had.  Review duties.
\item Remind managers to look at the time limits on the activities,
  make sure you stay on track!
\end{itemize}

\section{CTQs I (Big-O) (30 mins: 20 activity + 10 discussion)}

(Up to 3 minutes to get started, look at role cards, etc.)

\begin{itemize}
\item Share and discuss answers to \ref{q:synthesize},
  \ref{q:compare}, and \ref{q:incorrect}.  Note that the next activity
  will present the real definitions, so it is not critical that
  students converge on an exactly correct definitions; the goal is to
  get them to think about the important issues.
\end{itemize}

\section{CTQs II (Big-Theta, classification) (15 mins: 10 activity + wrap-up)}

\begin{itemize}
\item Discuss answers as necessary.
\item Wrap-up: today was about building intuition and
  examples. Promise we will see the real definitions next time!
\end{itemize}

\newpage

\section{Author notes}
\label{sec:author}

In the past when I have used a previous version of the activity in a
50-minute class, I only made it through CTQ I and never made it to CTQ
II.  I hope that
\begin{itemize}
\item this version is more streamlined, and
\item encouraging managers to keep track of time will help so that we
  can get to the application questions.
\end{itemize}

\end{document}
