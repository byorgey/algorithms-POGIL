% -*- compile-command: "rubber -d --unsafe amortized-intro.tex" -*-
\documentclass{tufte-handout}

\usepackage{algo-activity}
\usepackage{fourier}

\title{\thecourse: Amortized Analysis}
\date{}

\graphicspath{{images/}{../images/}}

\begin{document}

\maketitle

% \begin{objective}
%   Students will XXX.
% \end{objective}

\begin{model*}{Incrementing a binary counter}{bin-inc}
  \begin{center}
    \begin{pgfpicture}
  \pgfpathrectangle{\pgfpointorigin}{\pgfqpoint{100.0000bp}{333.0000bp}}
  \pgfusepath{use as bounding box}
  \begin{pgfscope}
    \definecolor{fc}{rgb}{0.0000,0.0000,0.0000}
    \pgfsetfillcolor{fc}
    \pgfsetfillopacity{0.0000}
    \pgfsetlinewidth{0.7303bp}
    \definecolor{sc}{rgb}{0.0000,0.0000,0.0000}
    \pgfsetstrokecolor{sc}
    \pgfsetmiterjoin
    \pgfsetbuttcap
    \pgfpathqmoveto{100.0000bp}{0.0000bp}
    \pgfpathqlineto{100.0000bp}{13.3333bp}
    \pgfpathqlineto{86.6667bp}{13.3333bp}
    \pgfpathqlineto{86.6667bp}{0.0000bp}
    \pgfpathqlineto{100.0000bp}{0.0000bp}
    \pgfpathclose
    \pgfusepathqfillstroke
  \end{pgfscope}
  \begin{pgfscope}
    \definecolor{fc}{rgb}{0.0000,0.0000,0.0000}
    \pgfsetfillcolor{fc}
    \pgftransformshift{\pgfqpoint{93.3333bp}{6.6667bp}}
    \pgftransformscale{1.5000}
    \pgftext[]{$1$}
  \end{pgfscope}
  \begin{pgfscope}
    \definecolor{fc}{rgb}{0.0000,0.0000,0.0000}
    \pgfsetfillcolor{fc}
    \pgfsetfillopacity{0.0000}
    \pgfsetlinewidth{0.7303bp}
    \definecolor{sc}{rgb}{0.0000,0.0000,0.0000}
    \pgfsetstrokecolor{sc}
    \pgfsetmiterjoin
    \pgfsetbuttcap
    \pgfpathqmoveto{86.6667bp}{0.0000bp}
    \pgfpathqlineto{86.6667bp}{13.3333bp}
    \pgfpathqlineto{73.3333bp}{13.3333bp}
    \pgfpathqlineto{73.3333bp}{0.0000bp}
    \pgfpathqlineto{86.6667bp}{0.0000bp}
    \pgfpathclose
    \pgfusepathqfillstroke
  \end{pgfscope}
  \begin{pgfscope}
    \definecolor{fc}{rgb}{0.0000,0.0000,0.0000}
    \pgfsetfillcolor{fc}
    \pgftransformshift{\pgfqpoint{80.0000bp}{6.6667bp}}
    \pgftransformscale{1.5000}
    \pgftext[]{$1$}
  \end{pgfscope}
  \begin{pgfscope}
    \definecolor{fc}{rgb}{0.0000,0.0000,0.0000}
    \pgfsetfillcolor{fc}
    \pgfsetfillopacity{0.0000}
    \pgfsetlinewidth{0.7303bp}
    \definecolor{sc}{rgb}{0.0000,0.0000,0.0000}
    \pgfsetstrokecolor{sc}
    \pgfsetmiterjoin
    \pgfsetbuttcap
    \pgfpathqmoveto{73.3333bp}{0.0000bp}
    \pgfpathqlineto{73.3333bp}{13.3333bp}
    \pgfpathqlineto{60.0000bp}{13.3333bp}
    \pgfpathqlineto{60.0000bp}{0.0000bp}
    \pgfpathqlineto{73.3333bp}{0.0000bp}
    \pgfpathclose
    \pgfusepathqfillstroke
  \end{pgfscope}
  \begin{pgfscope}
    \definecolor{fc}{rgb}{0.0000,0.0000,0.0000}
    \pgfsetfillcolor{fc}
    \pgftransformshift{\pgfqpoint{66.6667bp}{6.6667bp}}
    \pgftransformscale{1.5000}
    \pgftext[]{$1$}
  \end{pgfscope}
  \begin{pgfscope}
    \definecolor{fc}{rgb}{0.0000,0.0000,0.0000}
    \pgfsetfillcolor{fc}
    \pgfsetfillopacity{0.0000}
    \pgfsetlinewidth{0.7303bp}
    \definecolor{sc}{rgb}{0.0000,0.0000,0.0000}
    \pgfsetstrokecolor{sc}
    \pgfsetmiterjoin
    \pgfsetbuttcap
    \pgfpathqmoveto{60.0000bp}{0.0000bp}
    \pgfpathqlineto{60.0000bp}{13.3333bp}
    \pgfpathqlineto{46.6667bp}{13.3333bp}
    \pgfpathqlineto{46.6667bp}{0.0000bp}
    \pgfpathqlineto{60.0000bp}{0.0000bp}
    \pgfpathclose
    \pgfusepathqfillstroke
  \end{pgfscope}
  \begin{pgfscope}
    \definecolor{fc}{rgb}{0.0000,0.0000,0.0000}
    \pgfsetfillcolor{fc}
    \pgftransformshift{\pgfqpoint{53.3333bp}{6.6667bp}}
    \pgftransformscale{1.5000}
    \pgftext[]{$1$}
  \end{pgfscope}
  \begin{pgfscope}
    \definecolor{fc}{rgb}{0.0000,0.0000,0.0000}
    \pgfsetfillcolor{fc}
    \pgfsetfillopacity{0.0000}
    \pgfsetlinewidth{0.7303bp}
    \definecolor{sc}{rgb}{0.0000,0.0000,0.0000}
    \pgfsetstrokecolor{sc}
    \pgfsetmiterjoin
    \pgfsetbuttcap
    \pgfpathqmoveto{46.6667bp}{0.0000bp}
    \pgfpathqlineto{46.6667bp}{13.3333bp}
    \pgfpathqlineto{33.3333bp}{13.3333bp}
    \pgfpathqlineto{33.3333bp}{0.0000bp}
    \pgfpathqlineto{46.6667bp}{0.0000bp}
    \pgfpathclose
    \pgfusepathqfillstroke
  \end{pgfscope}
  \begin{pgfscope}
    \definecolor{fc}{rgb}{0.0000,0.0000,0.0000}
    \pgfsetfillcolor{fc}
    \pgftransformshift{\pgfqpoint{40.0000bp}{6.6667bp}}
    \pgftransformscale{1.5000}
    \pgftext[]{$0$}
  \end{pgfscope}
  \begin{pgfscope}
    \definecolor{fc}{rgb}{0.0000,0.0000,0.0000}
    \pgfsetfillcolor{fc}
    \pgfsetfillopacity{0.0000}
    \pgfsetlinewidth{0.7303bp}
    \definecolor{sc}{rgb}{0.0000,0.0000,0.0000}
    \pgfsetstrokecolor{sc}
    \pgfsetmiterjoin
    \pgfsetbuttcap
    \pgfpathqmoveto{33.3333bp}{0.0000bp}
    \pgfpathqlineto{33.3333bp}{13.3333bp}
    \pgfpathqlineto{20.0000bp}{13.3333bp}
    \pgfpathqlineto{20.0000bp}{0.0000bp}
    \pgfpathqlineto{33.3333bp}{0.0000bp}
    \pgfpathclose
    \pgfusepathqfillstroke
  \end{pgfscope}
  \begin{pgfscope}
    \definecolor{fc}{rgb}{0.0000,0.0000,0.0000}
    \pgfsetfillcolor{fc}
    \pgftransformshift{\pgfqpoint{26.6667bp}{6.6667bp}}
    \pgftransformscale{1.5000}
    \pgftext[]{$0$}
  \end{pgfscope}
  \begin{pgfscope}
    \definecolor{fc}{rgb}{0.0000,0.0000,0.0000}
    \pgfsetfillcolor{fc}
    \pgfsetfillopacity{0.0000}
    \pgfpathqmoveto{13.3333bp}{0.0000bp}
    \pgfpathqlineto{13.3333bp}{13.3333bp}
    \pgfpathqlineto{-0.0000bp}{13.3333bp}
    \pgfpathqlineto{-0.0000bp}{0.0000bp}
    \pgfpathqlineto{13.3333bp}{0.0000bp}
    \pgfpathclose
    \pgfusepathqfill
  \end{pgfscope}
  \begin{pgfscope}
    \definecolor{fc}{rgb}{0.0000,0.0000,0.0000}
    \pgfsetfillcolor{fc}
    \pgftransformshift{\pgfqpoint{6.6667bp}{6.6667bp}}
    \pgftransformscale{0.8333}
    \pgftext[]{$15$}
  \end{pgfscope}
  \begin{pgfscope}
    \definecolor{fc}{rgb}{0.0000,0.0000,0.0000}
    \pgfsetfillcolor{fc}
    \pgfsetfillopacity{0.0000}
    \pgfsetlinewidth{0.7303bp}
    \definecolor{sc}{rgb}{0.0000,0.0000,0.0000}
    \pgfsetstrokecolor{sc}
    \pgfsetmiterjoin
    \pgfsetbuttcap
    \pgfpathqmoveto{100.0000bp}{20.0000bp}
    \pgfpathqlineto{100.0000bp}{33.3333bp}
    \pgfpathqlineto{86.6667bp}{33.3333bp}
    \pgfpathqlineto{86.6667bp}{20.0000bp}
    \pgfpathqlineto{100.0000bp}{20.0000bp}
    \pgfpathclose
    \pgfusepathqfillstroke
  \end{pgfscope}
  \begin{pgfscope}
    \definecolor{fc}{rgb}{0.0000,0.0000,0.0000}
    \pgfsetfillcolor{fc}
    \pgftransformshift{\pgfqpoint{93.3333bp}{26.6667bp}}
    \pgftransformscale{1.5000}
    \pgftext[]{$0$}
  \end{pgfscope}
  \begin{pgfscope}
    \definecolor{fc}{rgb}{0.0000,0.0000,0.0000}
    \pgfsetfillcolor{fc}
    \pgfsetfillopacity{0.0000}
    \pgfsetlinewidth{0.7303bp}
    \definecolor{sc}{rgb}{0.0000,0.0000,0.0000}
    \pgfsetstrokecolor{sc}
    \pgfsetmiterjoin
    \pgfsetbuttcap
    \pgfpathqmoveto{86.6667bp}{20.0000bp}
    \pgfpathqlineto{86.6667bp}{33.3333bp}
    \pgfpathqlineto{73.3333bp}{33.3333bp}
    \pgfpathqlineto{73.3333bp}{20.0000bp}
    \pgfpathqlineto{86.6667bp}{20.0000bp}
    \pgfpathclose
    \pgfusepathqfillstroke
  \end{pgfscope}
  \begin{pgfscope}
    \definecolor{fc}{rgb}{0.0000,0.0000,0.0000}
    \pgfsetfillcolor{fc}
    \pgftransformshift{\pgfqpoint{80.0000bp}{26.6667bp}}
    \pgftransformscale{1.5000}
    \pgftext[]{$1$}
  \end{pgfscope}
  \begin{pgfscope}
    \definecolor{fc}{rgb}{0.0000,0.0000,0.0000}
    \pgfsetfillcolor{fc}
    \pgfsetfillopacity{0.0000}
    \pgfsetlinewidth{0.7303bp}
    \definecolor{sc}{rgb}{0.0000,0.0000,0.0000}
    \pgfsetstrokecolor{sc}
    \pgfsetmiterjoin
    \pgfsetbuttcap
    \pgfpathqmoveto{73.3333bp}{20.0000bp}
    \pgfpathqlineto{73.3333bp}{33.3333bp}
    \pgfpathqlineto{60.0000bp}{33.3333bp}
    \pgfpathqlineto{60.0000bp}{20.0000bp}
    \pgfpathqlineto{73.3333bp}{20.0000bp}
    \pgfpathclose
    \pgfusepathqfillstroke
  \end{pgfscope}
  \begin{pgfscope}
    \definecolor{fc}{rgb}{0.0000,0.0000,0.0000}
    \pgfsetfillcolor{fc}
    \pgftransformshift{\pgfqpoint{66.6667bp}{26.6667bp}}
    \pgftransformscale{1.5000}
    \pgftext[]{$1$}
  \end{pgfscope}
  \begin{pgfscope}
    \definecolor{fc}{rgb}{0.0000,0.0000,0.0000}
    \pgfsetfillcolor{fc}
    \pgfsetfillopacity{0.0000}
    \pgfsetlinewidth{0.7303bp}
    \definecolor{sc}{rgb}{0.0000,0.0000,0.0000}
    \pgfsetstrokecolor{sc}
    \pgfsetmiterjoin
    \pgfsetbuttcap
    \pgfpathqmoveto{60.0000bp}{20.0000bp}
    \pgfpathqlineto{60.0000bp}{33.3333bp}
    \pgfpathqlineto{46.6667bp}{33.3333bp}
    \pgfpathqlineto{46.6667bp}{20.0000bp}
    \pgfpathqlineto{60.0000bp}{20.0000bp}
    \pgfpathclose
    \pgfusepathqfillstroke
  \end{pgfscope}
  \begin{pgfscope}
    \definecolor{fc}{rgb}{0.0000,0.0000,0.0000}
    \pgfsetfillcolor{fc}
    \pgftransformshift{\pgfqpoint{53.3333bp}{26.6667bp}}
    \pgftransformscale{1.5000}
    \pgftext[]{$1$}
  \end{pgfscope}
  \begin{pgfscope}
    \definecolor{fc}{rgb}{0.0000,0.0000,0.0000}
    \pgfsetfillcolor{fc}
    \pgfsetfillopacity{0.0000}
    \pgfsetlinewidth{0.7303bp}
    \definecolor{sc}{rgb}{0.0000,0.0000,0.0000}
    \pgfsetstrokecolor{sc}
    \pgfsetmiterjoin
    \pgfsetbuttcap
    \pgfpathqmoveto{46.6667bp}{20.0000bp}
    \pgfpathqlineto{46.6667bp}{33.3333bp}
    \pgfpathqlineto{33.3333bp}{33.3333bp}
    \pgfpathqlineto{33.3333bp}{20.0000bp}
    \pgfpathqlineto{46.6667bp}{20.0000bp}
    \pgfpathclose
    \pgfusepathqfillstroke
  \end{pgfscope}
  \begin{pgfscope}
    \definecolor{fc}{rgb}{0.0000,0.0000,0.0000}
    \pgfsetfillcolor{fc}
    \pgftransformshift{\pgfqpoint{40.0000bp}{26.6667bp}}
    \pgftransformscale{1.5000}
    \pgftext[]{$0$}
  \end{pgfscope}
  \begin{pgfscope}
    \definecolor{fc}{rgb}{0.0000,0.0000,0.0000}
    \pgfsetfillcolor{fc}
    \pgfsetfillopacity{0.0000}
    \pgfsetlinewidth{0.7303bp}
    \definecolor{sc}{rgb}{0.0000,0.0000,0.0000}
    \pgfsetstrokecolor{sc}
    \pgfsetmiterjoin
    \pgfsetbuttcap
    \pgfpathqmoveto{33.3333bp}{20.0000bp}
    \pgfpathqlineto{33.3333bp}{33.3333bp}
    \pgfpathqlineto{20.0000bp}{33.3333bp}
    \pgfpathqlineto{20.0000bp}{20.0000bp}
    \pgfpathqlineto{33.3333bp}{20.0000bp}
    \pgfpathclose
    \pgfusepathqfillstroke
  \end{pgfscope}
  \begin{pgfscope}
    \definecolor{fc}{rgb}{0.0000,0.0000,0.0000}
    \pgfsetfillcolor{fc}
    \pgftransformshift{\pgfqpoint{26.6667bp}{26.6667bp}}
    \pgftransformscale{1.5000}
    \pgftext[]{$0$}
  \end{pgfscope}
  \begin{pgfscope}
    \definecolor{fc}{rgb}{0.0000,0.0000,0.0000}
    \pgfsetfillcolor{fc}
    \pgfsetfillopacity{0.0000}
    \pgfpathqmoveto{13.3333bp}{20.0000bp}
    \pgfpathqlineto{13.3333bp}{33.3333bp}
    \pgfpathqlineto{-0.0000bp}{33.3333bp}
    \pgfpathqlineto{-0.0000bp}{20.0000bp}
    \pgfpathqlineto{13.3333bp}{20.0000bp}
    \pgfpathclose
    \pgfusepathqfill
  \end{pgfscope}
  \begin{pgfscope}
    \definecolor{fc}{rgb}{0.0000,0.0000,0.0000}
    \pgfsetfillcolor{fc}
    \pgftransformshift{\pgfqpoint{6.6667bp}{26.6667bp}}
    \pgftransformscale{0.8333}
    \pgftext[]{$14$}
  \end{pgfscope}
  \begin{pgfscope}
    \definecolor{fc}{rgb}{0.0000,0.0000,0.0000}
    \pgfsetfillcolor{fc}
    \pgfsetfillopacity{0.0000}
    \pgfsetlinewidth{0.7303bp}
    \definecolor{sc}{rgb}{0.0000,0.0000,0.0000}
    \pgfsetstrokecolor{sc}
    \pgfsetmiterjoin
    \pgfsetbuttcap
    \pgfpathqmoveto{100.0000bp}{40.0000bp}
    \pgfpathqlineto{100.0000bp}{53.3333bp}
    \pgfpathqlineto{86.6667bp}{53.3333bp}
    \pgfpathqlineto{86.6667bp}{40.0000bp}
    \pgfpathqlineto{100.0000bp}{40.0000bp}
    \pgfpathclose
    \pgfusepathqfillstroke
  \end{pgfscope}
  \begin{pgfscope}
    \definecolor{fc}{rgb}{0.0000,0.0000,0.0000}
    \pgfsetfillcolor{fc}
    \pgftransformshift{\pgfqpoint{93.3333bp}{46.6667bp}}
    \pgftransformscale{1.5000}
    \pgftext[]{$1$}
  \end{pgfscope}
  \begin{pgfscope}
    \definecolor{fc}{rgb}{0.0000,0.0000,0.0000}
    \pgfsetfillcolor{fc}
    \pgfsetfillopacity{0.0000}
    \pgfsetlinewidth{0.7303bp}
    \definecolor{sc}{rgb}{0.0000,0.0000,0.0000}
    \pgfsetstrokecolor{sc}
    \pgfsetmiterjoin
    \pgfsetbuttcap
    \pgfpathqmoveto{86.6667bp}{40.0000bp}
    \pgfpathqlineto{86.6667bp}{53.3333bp}
    \pgfpathqlineto{73.3333bp}{53.3333bp}
    \pgfpathqlineto{73.3333bp}{40.0000bp}
    \pgfpathqlineto{86.6667bp}{40.0000bp}
    \pgfpathclose
    \pgfusepathqfillstroke
  \end{pgfscope}
  \begin{pgfscope}
    \definecolor{fc}{rgb}{0.0000,0.0000,0.0000}
    \pgfsetfillcolor{fc}
    \pgftransformshift{\pgfqpoint{80.0000bp}{46.6667bp}}
    \pgftransformscale{1.5000}
    \pgftext[]{$0$}
  \end{pgfscope}
  \begin{pgfscope}
    \definecolor{fc}{rgb}{0.0000,0.0000,0.0000}
    \pgfsetfillcolor{fc}
    \pgfsetfillopacity{0.0000}
    \pgfsetlinewidth{0.7303bp}
    \definecolor{sc}{rgb}{0.0000,0.0000,0.0000}
    \pgfsetstrokecolor{sc}
    \pgfsetmiterjoin
    \pgfsetbuttcap
    \pgfpathqmoveto{73.3333bp}{40.0000bp}
    \pgfpathqlineto{73.3333bp}{53.3333bp}
    \pgfpathqlineto{60.0000bp}{53.3333bp}
    \pgfpathqlineto{60.0000bp}{40.0000bp}
    \pgfpathqlineto{73.3333bp}{40.0000bp}
    \pgfpathclose
    \pgfusepathqfillstroke
  \end{pgfscope}
  \begin{pgfscope}
    \definecolor{fc}{rgb}{0.0000,0.0000,0.0000}
    \pgfsetfillcolor{fc}
    \pgftransformshift{\pgfqpoint{66.6667bp}{46.6667bp}}
    \pgftransformscale{1.5000}
    \pgftext[]{$1$}
  \end{pgfscope}
  \begin{pgfscope}
    \definecolor{fc}{rgb}{0.0000,0.0000,0.0000}
    \pgfsetfillcolor{fc}
    \pgfsetfillopacity{0.0000}
    \pgfsetlinewidth{0.7303bp}
    \definecolor{sc}{rgb}{0.0000,0.0000,0.0000}
    \pgfsetstrokecolor{sc}
    \pgfsetmiterjoin
    \pgfsetbuttcap
    \pgfpathqmoveto{60.0000bp}{40.0000bp}
    \pgfpathqlineto{60.0000bp}{53.3333bp}
    \pgfpathqlineto{46.6667bp}{53.3333bp}
    \pgfpathqlineto{46.6667bp}{40.0000bp}
    \pgfpathqlineto{60.0000bp}{40.0000bp}
    \pgfpathclose
    \pgfusepathqfillstroke
  \end{pgfscope}
  \begin{pgfscope}
    \definecolor{fc}{rgb}{0.0000,0.0000,0.0000}
    \pgfsetfillcolor{fc}
    \pgftransformshift{\pgfqpoint{53.3333bp}{46.6667bp}}
    \pgftransformscale{1.5000}
    \pgftext[]{$1$}
  \end{pgfscope}
  \begin{pgfscope}
    \definecolor{fc}{rgb}{0.0000,0.0000,0.0000}
    \pgfsetfillcolor{fc}
    \pgfsetfillopacity{0.0000}
    \pgfsetlinewidth{0.7303bp}
    \definecolor{sc}{rgb}{0.0000,0.0000,0.0000}
    \pgfsetstrokecolor{sc}
    \pgfsetmiterjoin
    \pgfsetbuttcap
    \pgfpathqmoveto{46.6667bp}{40.0000bp}
    \pgfpathqlineto{46.6667bp}{53.3333bp}
    \pgfpathqlineto{33.3333bp}{53.3333bp}
    \pgfpathqlineto{33.3333bp}{40.0000bp}
    \pgfpathqlineto{46.6667bp}{40.0000bp}
    \pgfpathclose
    \pgfusepathqfillstroke
  \end{pgfscope}
  \begin{pgfscope}
    \definecolor{fc}{rgb}{0.0000,0.0000,0.0000}
    \pgfsetfillcolor{fc}
    \pgftransformshift{\pgfqpoint{40.0000bp}{46.6667bp}}
    \pgftransformscale{1.5000}
    \pgftext[]{$0$}
  \end{pgfscope}
  \begin{pgfscope}
    \definecolor{fc}{rgb}{0.0000,0.0000,0.0000}
    \pgfsetfillcolor{fc}
    \pgfsetfillopacity{0.0000}
    \pgfsetlinewidth{0.7303bp}
    \definecolor{sc}{rgb}{0.0000,0.0000,0.0000}
    \pgfsetstrokecolor{sc}
    \pgfsetmiterjoin
    \pgfsetbuttcap
    \pgfpathqmoveto{33.3333bp}{40.0000bp}
    \pgfpathqlineto{33.3333bp}{53.3333bp}
    \pgfpathqlineto{20.0000bp}{53.3333bp}
    \pgfpathqlineto{20.0000bp}{40.0000bp}
    \pgfpathqlineto{33.3333bp}{40.0000bp}
    \pgfpathclose
    \pgfusepathqfillstroke
  \end{pgfscope}
  \begin{pgfscope}
    \definecolor{fc}{rgb}{0.0000,0.0000,0.0000}
    \pgfsetfillcolor{fc}
    \pgftransformshift{\pgfqpoint{26.6667bp}{46.6667bp}}
    \pgftransformscale{1.5000}
    \pgftext[]{$0$}
  \end{pgfscope}
  \begin{pgfscope}
    \definecolor{fc}{rgb}{0.0000,0.0000,0.0000}
    \pgfsetfillcolor{fc}
    \pgfsetfillopacity{0.0000}
    \pgfpathqmoveto{13.3333bp}{40.0000bp}
    \pgfpathqlineto{13.3333bp}{53.3333bp}
    \pgfpathqlineto{-0.0000bp}{53.3333bp}
    \pgfpathqlineto{-0.0000bp}{40.0000bp}
    \pgfpathqlineto{13.3333bp}{40.0000bp}
    \pgfpathclose
    \pgfusepathqfill
  \end{pgfscope}
  \begin{pgfscope}
    \definecolor{fc}{rgb}{0.0000,0.0000,0.0000}
    \pgfsetfillcolor{fc}
    \pgftransformshift{\pgfqpoint{6.6667bp}{46.6667bp}}
    \pgftransformscale{0.8333}
    \pgftext[]{$13$}
  \end{pgfscope}
  \begin{pgfscope}
    \definecolor{fc}{rgb}{0.0000,0.0000,0.0000}
    \pgfsetfillcolor{fc}
    \pgfsetfillopacity{0.0000}
    \pgfsetlinewidth{0.7303bp}
    \definecolor{sc}{rgb}{0.0000,0.0000,0.0000}
    \pgfsetstrokecolor{sc}
    \pgfsetmiterjoin
    \pgfsetbuttcap
    \pgfpathqmoveto{100.0000bp}{60.0000bp}
    \pgfpathqlineto{100.0000bp}{73.3333bp}
    \pgfpathqlineto{86.6667bp}{73.3333bp}
    \pgfpathqlineto{86.6667bp}{60.0000bp}
    \pgfpathqlineto{100.0000bp}{60.0000bp}
    \pgfpathclose
    \pgfusepathqfillstroke
  \end{pgfscope}
  \begin{pgfscope}
    \definecolor{fc}{rgb}{0.0000,0.0000,0.0000}
    \pgfsetfillcolor{fc}
    \pgftransformshift{\pgfqpoint{93.3333bp}{66.6667bp}}
    \pgftransformscale{1.5000}
    \pgftext[]{$0$}
  \end{pgfscope}
  \begin{pgfscope}
    \definecolor{fc}{rgb}{0.0000,0.0000,0.0000}
    \pgfsetfillcolor{fc}
    \pgfsetfillopacity{0.0000}
    \pgfsetlinewidth{0.7303bp}
    \definecolor{sc}{rgb}{0.0000,0.0000,0.0000}
    \pgfsetstrokecolor{sc}
    \pgfsetmiterjoin
    \pgfsetbuttcap
    \pgfpathqmoveto{86.6667bp}{60.0000bp}
    \pgfpathqlineto{86.6667bp}{73.3333bp}
    \pgfpathqlineto{73.3333bp}{73.3333bp}
    \pgfpathqlineto{73.3333bp}{60.0000bp}
    \pgfpathqlineto{86.6667bp}{60.0000bp}
    \pgfpathclose
    \pgfusepathqfillstroke
  \end{pgfscope}
  \begin{pgfscope}
    \definecolor{fc}{rgb}{0.0000,0.0000,0.0000}
    \pgfsetfillcolor{fc}
    \pgftransformshift{\pgfqpoint{80.0000bp}{66.6667bp}}
    \pgftransformscale{1.5000}
    \pgftext[]{$0$}
  \end{pgfscope}
  \begin{pgfscope}
    \definecolor{fc}{rgb}{0.0000,0.0000,0.0000}
    \pgfsetfillcolor{fc}
    \pgfsetfillopacity{0.0000}
    \pgfsetlinewidth{0.7303bp}
    \definecolor{sc}{rgb}{0.0000,0.0000,0.0000}
    \pgfsetstrokecolor{sc}
    \pgfsetmiterjoin
    \pgfsetbuttcap
    \pgfpathqmoveto{73.3333bp}{60.0000bp}
    \pgfpathqlineto{73.3333bp}{73.3333bp}
    \pgfpathqlineto{60.0000bp}{73.3333bp}
    \pgfpathqlineto{60.0000bp}{60.0000bp}
    \pgfpathqlineto{73.3333bp}{60.0000bp}
    \pgfpathclose
    \pgfusepathqfillstroke
  \end{pgfscope}
  \begin{pgfscope}
    \definecolor{fc}{rgb}{0.0000,0.0000,0.0000}
    \pgfsetfillcolor{fc}
    \pgftransformshift{\pgfqpoint{66.6667bp}{66.6667bp}}
    \pgftransformscale{1.5000}
    \pgftext[]{$1$}
  \end{pgfscope}
  \begin{pgfscope}
    \definecolor{fc}{rgb}{0.0000,0.0000,0.0000}
    \pgfsetfillcolor{fc}
    \pgfsetfillopacity{0.0000}
    \pgfsetlinewidth{0.7303bp}
    \definecolor{sc}{rgb}{0.0000,0.0000,0.0000}
    \pgfsetstrokecolor{sc}
    \pgfsetmiterjoin
    \pgfsetbuttcap
    \pgfpathqmoveto{60.0000bp}{60.0000bp}
    \pgfpathqlineto{60.0000bp}{73.3333bp}
    \pgfpathqlineto{46.6667bp}{73.3333bp}
    \pgfpathqlineto{46.6667bp}{60.0000bp}
    \pgfpathqlineto{60.0000bp}{60.0000bp}
    \pgfpathclose
    \pgfusepathqfillstroke
  \end{pgfscope}
  \begin{pgfscope}
    \definecolor{fc}{rgb}{0.0000,0.0000,0.0000}
    \pgfsetfillcolor{fc}
    \pgftransformshift{\pgfqpoint{53.3333bp}{66.6667bp}}
    \pgftransformscale{1.5000}
    \pgftext[]{$1$}
  \end{pgfscope}
  \begin{pgfscope}
    \definecolor{fc}{rgb}{0.0000,0.0000,0.0000}
    \pgfsetfillcolor{fc}
    \pgfsetfillopacity{0.0000}
    \pgfsetlinewidth{0.7303bp}
    \definecolor{sc}{rgb}{0.0000,0.0000,0.0000}
    \pgfsetstrokecolor{sc}
    \pgfsetmiterjoin
    \pgfsetbuttcap
    \pgfpathqmoveto{46.6667bp}{60.0000bp}
    \pgfpathqlineto{46.6667bp}{73.3333bp}
    \pgfpathqlineto{33.3333bp}{73.3333bp}
    \pgfpathqlineto{33.3333bp}{60.0000bp}
    \pgfpathqlineto{46.6667bp}{60.0000bp}
    \pgfpathclose
    \pgfusepathqfillstroke
  \end{pgfscope}
  \begin{pgfscope}
    \definecolor{fc}{rgb}{0.0000,0.0000,0.0000}
    \pgfsetfillcolor{fc}
    \pgftransformshift{\pgfqpoint{40.0000bp}{66.6667bp}}
    \pgftransformscale{1.5000}
    \pgftext[]{$0$}
  \end{pgfscope}
  \begin{pgfscope}
    \definecolor{fc}{rgb}{0.0000,0.0000,0.0000}
    \pgfsetfillcolor{fc}
    \pgfsetfillopacity{0.0000}
    \pgfsetlinewidth{0.7303bp}
    \definecolor{sc}{rgb}{0.0000,0.0000,0.0000}
    \pgfsetstrokecolor{sc}
    \pgfsetmiterjoin
    \pgfsetbuttcap
    \pgfpathqmoveto{33.3333bp}{60.0000bp}
    \pgfpathqlineto{33.3333bp}{73.3333bp}
    \pgfpathqlineto{20.0000bp}{73.3333bp}
    \pgfpathqlineto{20.0000bp}{60.0000bp}
    \pgfpathqlineto{33.3333bp}{60.0000bp}
    \pgfpathclose
    \pgfusepathqfillstroke
  \end{pgfscope}
  \begin{pgfscope}
    \definecolor{fc}{rgb}{0.0000,0.0000,0.0000}
    \pgfsetfillcolor{fc}
    \pgftransformshift{\pgfqpoint{26.6667bp}{66.6667bp}}
    \pgftransformscale{1.5000}
    \pgftext[]{$0$}
  \end{pgfscope}
  \begin{pgfscope}
    \definecolor{fc}{rgb}{0.0000,0.0000,0.0000}
    \pgfsetfillcolor{fc}
    \pgfsetfillopacity{0.0000}
    \pgfpathqmoveto{13.3333bp}{60.0000bp}
    \pgfpathqlineto{13.3333bp}{73.3333bp}
    \pgfpathqlineto{-0.0000bp}{73.3333bp}
    \pgfpathqlineto{-0.0000bp}{60.0000bp}
    \pgfpathqlineto{13.3333bp}{60.0000bp}
    \pgfpathclose
    \pgfusepathqfill
  \end{pgfscope}
  \begin{pgfscope}
    \definecolor{fc}{rgb}{0.0000,0.0000,0.0000}
    \pgfsetfillcolor{fc}
    \pgftransformshift{\pgfqpoint{6.6667bp}{66.6667bp}}
    \pgftransformscale{0.8333}
    \pgftext[]{$12$}
  \end{pgfscope}
  \begin{pgfscope}
    \definecolor{fc}{rgb}{0.0000,0.0000,0.0000}
    \pgfsetfillcolor{fc}
    \pgfsetfillopacity{0.0000}
    \pgfsetlinewidth{0.7303bp}
    \definecolor{sc}{rgb}{0.0000,0.0000,0.0000}
    \pgfsetstrokecolor{sc}
    \pgfsetmiterjoin
    \pgfsetbuttcap
    \pgfpathqmoveto{100.0000bp}{80.0000bp}
    \pgfpathqlineto{100.0000bp}{93.3333bp}
    \pgfpathqlineto{86.6667bp}{93.3333bp}
    \pgfpathqlineto{86.6667bp}{80.0000bp}
    \pgfpathqlineto{100.0000bp}{80.0000bp}
    \pgfpathclose
    \pgfusepathqfillstroke
  \end{pgfscope}
  \begin{pgfscope}
    \definecolor{fc}{rgb}{0.0000,0.0000,0.0000}
    \pgfsetfillcolor{fc}
    \pgftransformshift{\pgfqpoint{93.3333bp}{86.6667bp}}
    \pgftransformscale{1.5000}
    \pgftext[]{$1$}
  \end{pgfscope}
  \begin{pgfscope}
    \definecolor{fc}{rgb}{0.0000,0.0000,0.0000}
    \pgfsetfillcolor{fc}
    \pgfsetfillopacity{0.0000}
    \pgfsetlinewidth{0.7303bp}
    \definecolor{sc}{rgb}{0.0000,0.0000,0.0000}
    \pgfsetstrokecolor{sc}
    \pgfsetmiterjoin
    \pgfsetbuttcap
    \pgfpathqmoveto{86.6667bp}{80.0000bp}
    \pgfpathqlineto{86.6667bp}{93.3333bp}
    \pgfpathqlineto{73.3333bp}{93.3333bp}
    \pgfpathqlineto{73.3333bp}{80.0000bp}
    \pgfpathqlineto{86.6667bp}{80.0000bp}
    \pgfpathclose
    \pgfusepathqfillstroke
  \end{pgfscope}
  \begin{pgfscope}
    \definecolor{fc}{rgb}{0.0000,0.0000,0.0000}
    \pgfsetfillcolor{fc}
    \pgftransformshift{\pgfqpoint{80.0000bp}{86.6667bp}}
    \pgftransformscale{1.5000}
    \pgftext[]{$1$}
  \end{pgfscope}
  \begin{pgfscope}
    \definecolor{fc}{rgb}{0.0000,0.0000,0.0000}
    \pgfsetfillcolor{fc}
    \pgfsetfillopacity{0.0000}
    \pgfsetlinewidth{0.7303bp}
    \definecolor{sc}{rgb}{0.0000,0.0000,0.0000}
    \pgfsetstrokecolor{sc}
    \pgfsetmiterjoin
    \pgfsetbuttcap
    \pgfpathqmoveto{73.3333bp}{80.0000bp}
    \pgfpathqlineto{73.3333bp}{93.3333bp}
    \pgfpathqlineto{60.0000bp}{93.3333bp}
    \pgfpathqlineto{60.0000bp}{80.0000bp}
    \pgfpathqlineto{73.3333bp}{80.0000bp}
    \pgfpathclose
    \pgfusepathqfillstroke
  \end{pgfscope}
  \begin{pgfscope}
    \definecolor{fc}{rgb}{0.0000,0.0000,0.0000}
    \pgfsetfillcolor{fc}
    \pgftransformshift{\pgfqpoint{66.6667bp}{86.6667bp}}
    \pgftransformscale{1.5000}
    \pgftext[]{$0$}
  \end{pgfscope}
  \begin{pgfscope}
    \definecolor{fc}{rgb}{0.0000,0.0000,0.0000}
    \pgfsetfillcolor{fc}
    \pgfsetfillopacity{0.0000}
    \pgfsetlinewidth{0.7303bp}
    \definecolor{sc}{rgb}{0.0000,0.0000,0.0000}
    \pgfsetstrokecolor{sc}
    \pgfsetmiterjoin
    \pgfsetbuttcap
    \pgfpathqmoveto{60.0000bp}{80.0000bp}
    \pgfpathqlineto{60.0000bp}{93.3333bp}
    \pgfpathqlineto{46.6667bp}{93.3333bp}
    \pgfpathqlineto{46.6667bp}{80.0000bp}
    \pgfpathqlineto{60.0000bp}{80.0000bp}
    \pgfpathclose
    \pgfusepathqfillstroke
  \end{pgfscope}
  \begin{pgfscope}
    \definecolor{fc}{rgb}{0.0000,0.0000,0.0000}
    \pgfsetfillcolor{fc}
    \pgftransformshift{\pgfqpoint{53.3333bp}{86.6667bp}}
    \pgftransformscale{1.5000}
    \pgftext[]{$1$}
  \end{pgfscope}
  \begin{pgfscope}
    \definecolor{fc}{rgb}{0.0000,0.0000,0.0000}
    \pgfsetfillcolor{fc}
    \pgfsetfillopacity{0.0000}
    \pgfsetlinewidth{0.7303bp}
    \definecolor{sc}{rgb}{0.0000,0.0000,0.0000}
    \pgfsetstrokecolor{sc}
    \pgfsetmiterjoin
    \pgfsetbuttcap
    \pgfpathqmoveto{46.6667bp}{80.0000bp}
    \pgfpathqlineto{46.6667bp}{93.3333bp}
    \pgfpathqlineto{33.3333bp}{93.3333bp}
    \pgfpathqlineto{33.3333bp}{80.0000bp}
    \pgfpathqlineto{46.6667bp}{80.0000bp}
    \pgfpathclose
    \pgfusepathqfillstroke
  \end{pgfscope}
  \begin{pgfscope}
    \definecolor{fc}{rgb}{0.0000,0.0000,0.0000}
    \pgfsetfillcolor{fc}
    \pgftransformshift{\pgfqpoint{40.0000bp}{86.6667bp}}
    \pgftransformscale{1.5000}
    \pgftext[]{$0$}
  \end{pgfscope}
  \begin{pgfscope}
    \definecolor{fc}{rgb}{0.0000,0.0000,0.0000}
    \pgfsetfillcolor{fc}
    \pgfsetfillopacity{0.0000}
    \pgfsetlinewidth{0.7303bp}
    \definecolor{sc}{rgb}{0.0000,0.0000,0.0000}
    \pgfsetstrokecolor{sc}
    \pgfsetmiterjoin
    \pgfsetbuttcap
    \pgfpathqmoveto{33.3333bp}{80.0000bp}
    \pgfpathqlineto{33.3333bp}{93.3333bp}
    \pgfpathqlineto{20.0000bp}{93.3333bp}
    \pgfpathqlineto{20.0000bp}{80.0000bp}
    \pgfpathqlineto{33.3333bp}{80.0000bp}
    \pgfpathclose
    \pgfusepathqfillstroke
  \end{pgfscope}
  \begin{pgfscope}
    \definecolor{fc}{rgb}{0.0000,0.0000,0.0000}
    \pgfsetfillcolor{fc}
    \pgftransformshift{\pgfqpoint{26.6667bp}{86.6667bp}}
    \pgftransformscale{1.5000}
    \pgftext[]{$0$}
  \end{pgfscope}
  \begin{pgfscope}
    \definecolor{fc}{rgb}{0.0000,0.0000,0.0000}
    \pgfsetfillcolor{fc}
    \pgfsetfillopacity{0.0000}
    \pgfpathqmoveto{13.3333bp}{80.0000bp}
    \pgfpathqlineto{13.3333bp}{93.3333bp}
    \pgfpathqlineto{-0.0000bp}{93.3333bp}
    \pgfpathqlineto{-0.0000bp}{80.0000bp}
    \pgfpathqlineto{13.3333bp}{80.0000bp}
    \pgfpathclose
    \pgfusepathqfill
  \end{pgfscope}
  \begin{pgfscope}
    \definecolor{fc}{rgb}{0.0000,0.0000,0.0000}
    \pgfsetfillcolor{fc}
    \pgftransformshift{\pgfqpoint{6.6667bp}{86.6667bp}}
    \pgftransformscale{0.8333}
    \pgftext[]{$11$}
  \end{pgfscope}
  \begin{pgfscope}
    \definecolor{fc}{rgb}{0.0000,0.0000,0.0000}
    \pgfsetfillcolor{fc}
    \pgfsetfillopacity{0.0000}
    \pgfsetlinewidth{0.7303bp}
    \definecolor{sc}{rgb}{0.0000,0.0000,0.0000}
    \pgfsetstrokecolor{sc}
    \pgfsetmiterjoin
    \pgfsetbuttcap
    \pgfpathqmoveto{100.0000bp}{100.0000bp}
    \pgfpathqlineto{100.0000bp}{113.3333bp}
    \pgfpathqlineto{86.6667bp}{113.3333bp}
    \pgfpathqlineto{86.6667bp}{100.0000bp}
    \pgfpathqlineto{100.0000bp}{100.0000bp}
    \pgfpathclose
    \pgfusepathqfillstroke
  \end{pgfscope}
  \begin{pgfscope}
    \definecolor{fc}{rgb}{0.0000,0.0000,0.0000}
    \pgfsetfillcolor{fc}
    \pgftransformshift{\pgfqpoint{93.3333bp}{106.6667bp}}
    \pgftransformscale{1.5000}
    \pgftext[]{$0$}
  \end{pgfscope}
  \begin{pgfscope}
    \definecolor{fc}{rgb}{0.0000,0.0000,0.0000}
    \pgfsetfillcolor{fc}
    \pgfsetfillopacity{0.0000}
    \pgfsetlinewidth{0.7303bp}
    \definecolor{sc}{rgb}{0.0000,0.0000,0.0000}
    \pgfsetstrokecolor{sc}
    \pgfsetmiterjoin
    \pgfsetbuttcap
    \pgfpathqmoveto{86.6667bp}{100.0000bp}
    \pgfpathqlineto{86.6667bp}{113.3333bp}
    \pgfpathqlineto{73.3333bp}{113.3333bp}
    \pgfpathqlineto{73.3333bp}{100.0000bp}
    \pgfpathqlineto{86.6667bp}{100.0000bp}
    \pgfpathclose
    \pgfusepathqfillstroke
  \end{pgfscope}
  \begin{pgfscope}
    \definecolor{fc}{rgb}{0.0000,0.0000,0.0000}
    \pgfsetfillcolor{fc}
    \pgftransformshift{\pgfqpoint{80.0000bp}{106.6667bp}}
    \pgftransformscale{1.5000}
    \pgftext[]{$1$}
  \end{pgfscope}
  \begin{pgfscope}
    \definecolor{fc}{rgb}{0.0000,0.0000,0.0000}
    \pgfsetfillcolor{fc}
    \pgfsetfillopacity{0.0000}
    \pgfsetlinewidth{0.7303bp}
    \definecolor{sc}{rgb}{0.0000,0.0000,0.0000}
    \pgfsetstrokecolor{sc}
    \pgfsetmiterjoin
    \pgfsetbuttcap
    \pgfpathqmoveto{73.3333bp}{100.0000bp}
    \pgfpathqlineto{73.3333bp}{113.3333bp}
    \pgfpathqlineto{60.0000bp}{113.3333bp}
    \pgfpathqlineto{60.0000bp}{100.0000bp}
    \pgfpathqlineto{73.3333bp}{100.0000bp}
    \pgfpathclose
    \pgfusepathqfillstroke
  \end{pgfscope}
  \begin{pgfscope}
    \definecolor{fc}{rgb}{0.0000,0.0000,0.0000}
    \pgfsetfillcolor{fc}
    \pgftransformshift{\pgfqpoint{66.6667bp}{106.6667bp}}
    \pgftransformscale{1.5000}
    \pgftext[]{$0$}
  \end{pgfscope}
  \begin{pgfscope}
    \definecolor{fc}{rgb}{0.0000,0.0000,0.0000}
    \pgfsetfillcolor{fc}
    \pgfsetfillopacity{0.0000}
    \pgfsetlinewidth{0.7303bp}
    \definecolor{sc}{rgb}{0.0000,0.0000,0.0000}
    \pgfsetstrokecolor{sc}
    \pgfsetmiterjoin
    \pgfsetbuttcap
    \pgfpathqmoveto{60.0000bp}{100.0000bp}
    \pgfpathqlineto{60.0000bp}{113.3333bp}
    \pgfpathqlineto{46.6667bp}{113.3333bp}
    \pgfpathqlineto{46.6667bp}{100.0000bp}
    \pgfpathqlineto{60.0000bp}{100.0000bp}
    \pgfpathclose
    \pgfusepathqfillstroke
  \end{pgfscope}
  \begin{pgfscope}
    \definecolor{fc}{rgb}{0.0000,0.0000,0.0000}
    \pgfsetfillcolor{fc}
    \pgftransformshift{\pgfqpoint{53.3333bp}{106.6667bp}}
    \pgftransformscale{1.5000}
    \pgftext[]{$1$}
  \end{pgfscope}
  \begin{pgfscope}
    \definecolor{fc}{rgb}{0.0000,0.0000,0.0000}
    \pgfsetfillcolor{fc}
    \pgfsetfillopacity{0.0000}
    \pgfsetlinewidth{0.7303bp}
    \definecolor{sc}{rgb}{0.0000,0.0000,0.0000}
    \pgfsetstrokecolor{sc}
    \pgfsetmiterjoin
    \pgfsetbuttcap
    \pgfpathqmoveto{46.6667bp}{100.0000bp}
    \pgfpathqlineto{46.6667bp}{113.3333bp}
    \pgfpathqlineto{33.3333bp}{113.3333bp}
    \pgfpathqlineto{33.3333bp}{100.0000bp}
    \pgfpathqlineto{46.6667bp}{100.0000bp}
    \pgfpathclose
    \pgfusepathqfillstroke
  \end{pgfscope}
  \begin{pgfscope}
    \definecolor{fc}{rgb}{0.0000,0.0000,0.0000}
    \pgfsetfillcolor{fc}
    \pgftransformshift{\pgfqpoint{40.0000bp}{106.6667bp}}
    \pgftransformscale{1.5000}
    \pgftext[]{$0$}
  \end{pgfscope}
  \begin{pgfscope}
    \definecolor{fc}{rgb}{0.0000,0.0000,0.0000}
    \pgfsetfillcolor{fc}
    \pgfsetfillopacity{0.0000}
    \pgfsetlinewidth{0.7303bp}
    \definecolor{sc}{rgb}{0.0000,0.0000,0.0000}
    \pgfsetstrokecolor{sc}
    \pgfsetmiterjoin
    \pgfsetbuttcap
    \pgfpathqmoveto{33.3333bp}{100.0000bp}
    \pgfpathqlineto{33.3333bp}{113.3333bp}
    \pgfpathqlineto{20.0000bp}{113.3333bp}
    \pgfpathqlineto{20.0000bp}{100.0000bp}
    \pgfpathqlineto{33.3333bp}{100.0000bp}
    \pgfpathclose
    \pgfusepathqfillstroke
  \end{pgfscope}
  \begin{pgfscope}
    \definecolor{fc}{rgb}{0.0000,0.0000,0.0000}
    \pgfsetfillcolor{fc}
    \pgftransformshift{\pgfqpoint{26.6667bp}{106.6667bp}}
    \pgftransformscale{1.5000}
    \pgftext[]{$0$}
  \end{pgfscope}
  \begin{pgfscope}
    \definecolor{fc}{rgb}{0.0000,0.0000,0.0000}
    \pgfsetfillcolor{fc}
    \pgfsetfillopacity{0.0000}
    \pgfpathqmoveto{13.3333bp}{100.0000bp}
    \pgfpathqlineto{13.3333bp}{113.3333bp}
    \pgfpathqlineto{-0.0000bp}{113.3333bp}
    \pgfpathqlineto{-0.0000bp}{100.0000bp}
    \pgfpathqlineto{13.3333bp}{100.0000bp}
    \pgfpathclose
    \pgfusepathqfill
  \end{pgfscope}
  \begin{pgfscope}
    \definecolor{fc}{rgb}{0.0000,0.0000,0.0000}
    \pgfsetfillcolor{fc}
    \pgftransformshift{\pgfqpoint{6.6667bp}{106.6667bp}}
    \pgftransformscale{0.8333}
    \pgftext[]{$10$}
  \end{pgfscope}
  \begin{pgfscope}
    \definecolor{fc}{rgb}{0.0000,0.0000,0.0000}
    \pgfsetfillcolor{fc}
    \pgfsetfillopacity{0.0000}
    \pgfsetlinewidth{0.7303bp}
    \definecolor{sc}{rgb}{0.0000,0.0000,0.0000}
    \pgfsetstrokecolor{sc}
    \pgfsetmiterjoin
    \pgfsetbuttcap
    \pgfpathqmoveto{100.0000bp}{120.0000bp}
    \pgfpathqlineto{100.0000bp}{133.3333bp}
    \pgfpathqlineto{86.6667bp}{133.3333bp}
    \pgfpathqlineto{86.6667bp}{120.0000bp}
    \pgfpathqlineto{100.0000bp}{120.0000bp}
    \pgfpathclose
    \pgfusepathqfillstroke
  \end{pgfscope}
  \begin{pgfscope}
    \definecolor{fc}{rgb}{0.0000,0.0000,0.0000}
    \pgfsetfillcolor{fc}
    \pgftransformshift{\pgfqpoint{93.3333bp}{126.6667bp}}
    \pgftransformscale{1.5000}
    \pgftext[]{$1$}
  \end{pgfscope}
  \begin{pgfscope}
    \definecolor{fc}{rgb}{0.0000,0.0000,0.0000}
    \pgfsetfillcolor{fc}
    \pgfsetfillopacity{0.0000}
    \pgfsetlinewidth{0.7303bp}
    \definecolor{sc}{rgb}{0.0000,0.0000,0.0000}
    \pgfsetstrokecolor{sc}
    \pgfsetmiterjoin
    \pgfsetbuttcap
    \pgfpathqmoveto{86.6667bp}{120.0000bp}
    \pgfpathqlineto{86.6667bp}{133.3333bp}
    \pgfpathqlineto{73.3333bp}{133.3333bp}
    \pgfpathqlineto{73.3333bp}{120.0000bp}
    \pgfpathqlineto{86.6667bp}{120.0000bp}
    \pgfpathclose
    \pgfusepathqfillstroke
  \end{pgfscope}
  \begin{pgfscope}
    \definecolor{fc}{rgb}{0.0000,0.0000,0.0000}
    \pgfsetfillcolor{fc}
    \pgftransformshift{\pgfqpoint{80.0000bp}{126.6667bp}}
    \pgftransformscale{1.5000}
    \pgftext[]{$0$}
  \end{pgfscope}
  \begin{pgfscope}
    \definecolor{fc}{rgb}{0.0000,0.0000,0.0000}
    \pgfsetfillcolor{fc}
    \pgfsetfillopacity{0.0000}
    \pgfsetlinewidth{0.7303bp}
    \definecolor{sc}{rgb}{0.0000,0.0000,0.0000}
    \pgfsetstrokecolor{sc}
    \pgfsetmiterjoin
    \pgfsetbuttcap
    \pgfpathqmoveto{73.3333bp}{120.0000bp}
    \pgfpathqlineto{73.3333bp}{133.3333bp}
    \pgfpathqlineto{60.0000bp}{133.3333bp}
    \pgfpathqlineto{60.0000bp}{120.0000bp}
    \pgfpathqlineto{73.3333bp}{120.0000bp}
    \pgfpathclose
    \pgfusepathqfillstroke
  \end{pgfscope}
  \begin{pgfscope}
    \definecolor{fc}{rgb}{0.0000,0.0000,0.0000}
    \pgfsetfillcolor{fc}
    \pgftransformshift{\pgfqpoint{66.6667bp}{126.6667bp}}
    \pgftransformscale{1.5000}
    \pgftext[]{$0$}
  \end{pgfscope}
  \begin{pgfscope}
    \definecolor{fc}{rgb}{0.0000,0.0000,0.0000}
    \pgfsetfillcolor{fc}
    \pgfsetfillopacity{0.0000}
    \pgfsetlinewidth{0.7303bp}
    \definecolor{sc}{rgb}{0.0000,0.0000,0.0000}
    \pgfsetstrokecolor{sc}
    \pgfsetmiterjoin
    \pgfsetbuttcap
    \pgfpathqmoveto{60.0000bp}{120.0000bp}
    \pgfpathqlineto{60.0000bp}{133.3333bp}
    \pgfpathqlineto{46.6667bp}{133.3333bp}
    \pgfpathqlineto{46.6667bp}{120.0000bp}
    \pgfpathqlineto{60.0000bp}{120.0000bp}
    \pgfpathclose
    \pgfusepathqfillstroke
  \end{pgfscope}
  \begin{pgfscope}
    \definecolor{fc}{rgb}{0.0000,0.0000,0.0000}
    \pgfsetfillcolor{fc}
    \pgftransformshift{\pgfqpoint{53.3333bp}{126.6667bp}}
    \pgftransformscale{1.5000}
    \pgftext[]{$1$}
  \end{pgfscope}
  \begin{pgfscope}
    \definecolor{fc}{rgb}{0.0000,0.0000,0.0000}
    \pgfsetfillcolor{fc}
    \pgfsetfillopacity{0.0000}
    \pgfsetlinewidth{0.7303bp}
    \definecolor{sc}{rgb}{0.0000,0.0000,0.0000}
    \pgfsetstrokecolor{sc}
    \pgfsetmiterjoin
    \pgfsetbuttcap
    \pgfpathqmoveto{46.6667bp}{120.0000bp}
    \pgfpathqlineto{46.6667bp}{133.3333bp}
    \pgfpathqlineto{33.3333bp}{133.3333bp}
    \pgfpathqlineto{33.3333bp}{120.0000bp}
    \pgfpathqlineto{46.6667bp}{120.0000bp}
    \pgfpathclose
    \pgfusepathqfillstroke
  \end{pgfscope}
  \begin{pgfscope}
    \definecolor{fc}{rgb}{0.0000,0.0000,0.0000}
    \pgfsetfillcolor{fc}
    \pgftransformshift{\pgfqpoint{40.0000bp}{126.6667bp}}
    \pgftransformscale{1.5000}
    \pgftext[]{$0$}
  \end{pgfscope}
  \begin{pgfscope}
    \definecolor{fc}{rgb}{0.0000,0.0000,0.0000}
    \pgfsetfillcolor{fc}
    \pgfsetfillopacity{0.0000}
    \pgfsetlinewidth{0.7303bp}
    \definecolor{sc}{rgb}{0.0000,0.0000,0.0000}
    \pgfsetstrokecolor{sc}
    \pgfsetmiterjoin
    \pgfsetbuttcap
    \pgfpathqmoveto{33.3333bp}{120.0000bp}
    \pgfpathqlineto{33.3333bp}{133.3333bp}
    \pgfpathqlineto{20.0000bp}{133.3333bp}
    \pgfpathqlineto{20.0000bp}{120.0000bp}
    \pgfpathqlineto{33.3333bp}{120.0000bp}
    \pgfpathclose
    \pgfusepathqfillstroke
  \end{pgfscope}
  \begin{pgfscope}
    \definecolor{fc}{rgb}{0.0000,0.0000,0.0000}
    \pgfsetfillcolor{fc}
    \pgftransformshift{\pgfqpoint{26.6667bp}{126.6667bp}}
    \pgftransformscale{1.5000}
    \pgftext[]{$0$}
  \end{pgfscope}
  \begin{pgfscope}
    \definecolor{fc}{rgb}{0.0000,0.0000,0.0000}
    \pgfsetfillcolor{fc}
    \pgfsetfillopacity{0.0000}
    \pgfpathqmoveto{13.3333bp}{120.0000bp}
    \pgfpathqlineto{13.3333bp}{133.3333bp}
    \pgfpathqlineto{-0.0000bp}{133.3333bp}
    \pgfpathqlineto{-0.0000bp}{120.0000bp}
    \pgfpathqlineto{13.3333bp}{120.0000bp}
    \pgfpathclose
    \pgfusepathqfill
  \end{pgfscope}
  \begin{pgfscope}
    \definecolor{fc}{rgb}{0.0000,0.0000,0.0000}
    \pgfsetfillcolor{fc}
    \pgftransformshift{\pgfqpoint{6.6667bp}{126.6667bp}}
    \pgftransformscale{0.8333}
    \pgftext[]{$9$}
  \end{pgfscope}
  \begin{pgfscope}
    \definecolor{fc}{rgb}{0.0000,0.0000,0.0000}
    \pgfsetfillcolor{fc}
    \pgfsetfillopacity{0.0000}
    \pgfsetlinewidth{0.7303bp}
    \definecolor{sc}{rgb}{0.0000,0.0000,0.0000}
    \pgfsetstrokecolor{sc}
    \pgfsetmiterjoin
    \pgfsetbuttcap
    \pgfpathqmoveto{100.0000bp}{140.0000bp}
    \pgfpathqlineto{100.0000bp}{153.3333bp}
    \pgfpathqlineto{86.6667bp}{153.3333bp}
    \pgfpathqlineto{86.6667bp}{140.0000bp}
    \pgfpathqlineto{100.0000bp}{140.0000bp}
    \pgfpathclose
    \pgfusepathqfillstroke
  \end{pgfscope}
  \begin{pgfscope}
    \definecolor{fc}{rgb}{0.0000,0.0000,0.0000}
    \pgfsetfillcolor{fc}
    \pgftransformshift{\pgfqpoint{93.3333bp}{146.6667bp}}
    \pgftransformscale{1.5000}
    \pgftext[]{$0$}
  \end{pgfscope}
  \begin{pgfscope}
    \definecolor{fc}{rgb}{0.0000,0.0000,0.0000}
    \pgfsetfillcolor{fc}
    \pgfsetfillopacity{0.0000}
    \pgfsetlinewidth{0.7303bp}
    \definecolor{sc}{rgb}{0.0000,0.0000,0.0000}
    \pgfsetstrokecolor{sc}
    \pgfsetmiterjoin
    \pgfsetbuttcap
    \pgfpathqmoveto{86.6667bp}{140.0000bp}
    \pgfpathqlineto{86.6667bp}{153.3333bp}
    \pgfpathqlineto{73.3333bp}{153.3333bp}
    \pgfpathqlineto{73.3333bp}{140.0000bp}
    \pgfpathqlineto{86.6667bp}{140.0000bp}
    \pgfpathclose
    \pgfusepathqfillstroke
  \end{pgfscope}
  \begin{pgfscope}
    \definecolor{fc}{rgb}{0.0000,0.0000,0.0000}
    \pgfsetfillcolor{fc}
    \pgftransformshift{\pgfqpoint{80.0000bp}{146.6667bp}}
    \pgftransformscale{1.5000}
    \pgftext[]{$0$}
  \end{pgfscope}
  \begin{pgfscope}
    \definecolor{fc}{rgb}{0.0000,0.0000,0.0000}
    \pgfsetfillcolor{fc}
    \pgfsetfillopacity{0.0000}
    \pgfsetlinewidth{0.7303bp}
    \definecolor{sc}{rgb}{0.0000,0.0000,0.0000}
    \pgfsetstrokecolor{sc}
    \pgfsetmiterjoin
    \pgfsetbuttcap
    \pgfpathqmoveto{73.3333bp}{140.0000bp}
    \pgfpathqlineto{73.3333bp}{153.3333bp}
    \pgfpathqlineto{60.0000bp}{153.3333bp}
    \pgfpathqlineto{60.0000bp}{140.0000bp}
    \pgfpathqlineto{73.3333bp}{140.0000bp}
    \pgfpathclose
    \pgfusepathqfillstroke
  \end{pgfscope}
  \begin{pgfscope}
    \definecolor{fc}{rgb}{0.0000,0.0000,0.0000}
    \pgfsetfillcolor{fc}
    \pgftransformshift{\pgfqpoint{66.6667bp}{146.6667bp}}
    \pgftransformscale{1.5000}
    \pgftext[]{$0$}
  \end{pgfscope}
  \begin{pgfscope}
    \definecolor{fc}{rgb}{0.0000,0.0000,0.0000}
    \pgfsetfillcolor{fc}
    \pgfsetfillopacity{0.0000}
    \pgfsetlinewidth{0.7303bp}
    \definecolor{sc}{rgb}{0.0000,0.0000,0.0000}
    \pgfsetstrokecolor{sc}
    \pgfsetmiterjoin
    \pgfsetbuttcap
    \pgfpathqmoveto{60.0000bp}{140.0000bp}
    \pgfpathqlineto{60.0000bp}{153.3333bp}
    \pgfpathqlineto{46.6667bp}{153.3333bp}
    \pgfpathqlineto{46.6667bp}{140.0000bp}
    \pgfpathqlineto{60.0000bp}{140.0000bp}
    \pgfpathclose
    \pgfusepathqfillstroke
  \end{pgfscope}
  \begin{pgfscope}
    \definecolor{fc}{rgb}{0.0000,0.0000,0.0000}
    \pgfsetfillcolor{fc}
    \pgftransformshift{\pgfqpoint{53.3333bp}{146.6667bp}}
    \pgftransformscale{1.5000}
    \pgftext[]{$1$}
  \end{pgfscope}
  \begin{pgfscope}
    \definecolor{fc}{rgb}{0.0000,0.0000,0.0000}
    \pgfsetfillcolor{fc}
    \pgfsetfillopacity{0.0000}
    \pgfsetlinewidth{0.7303bp}
    \definecolor{sc}{rgb}{0.0000,0.0000,0.0000}
    \pgfsetstrokecolor{sc}
    \pgfsetmiterjoin
    \pgfsetbuttcap
    \pgfpathqmoveto{46.6667bp}{140.0000bp}
    \pgfpathqlineto{46.6667bp}{153.3333bp}
    \pgfpathqlineto{33.3333bp}{153.3333bp}
    \pgfpathqlineto{33.3333bp}{140.0000bp}
    \pgfpathqlineto{46.6667bp}{140.0000bp}
    \pgfpathclose
    \pgfusepathqfillstroke
  \end{pgfscope}
  \begin{pgfscope}
    \definecolor{fc}{rgb}{0.0000,0.0000,0.0000}
    \pgfsetfillcolor{fc}
    \pgftransformshift{\pgfqpoint{40.0000bp}{146.6667bp}}
    \pgftransformscale{1.5000}
    \pgftext[]{$0$}
  \end{pgfscope}
  \begin{pgfscope}
    \definecolor{fc}{rgb}{0.0000,0.0000,0.0000}
    \pgfsetfillcolor{fc}
    \pgfsetfillopacity{0.0000}
    \pgfsetlinewidth{0.7303bp}
    \definecolor{sc}{rgb}{0.0000,0.0000,0.0000}
    \pgfsetstrokecolor{sc}
    \pgfsetmiterjoin
    \pgfsetbuttcap
    \pgfpathqmoveto{33.3333bp}{140.0000bp}
    \pgfpathqlineto{33.3333bp}{153.3333bp}
    \pgfpathqlineto{20.0000bp}{153.3333bp}
    \pgfpathqlineto{20.0000bp}{140.0000bp}
    \pgfpathqlineto{33.3333bp}{140.0000bp}
    \pgfpathclose
    \pgfusepathqfillstroke
  \end{pgfscope}
  \begin{pgfscope}
    \definecolor{fc}{rgb}{0.0000,0.0000,0.0000}
    \pgfsetfillcolor{fc}
    \pgftransformshift{\pgfqpoint{26.6667bp}{146.6667bp}}
    \pgftransformscale{1.5000}
    \pgftext[]{$0$}
  \end{pgfscope}
  \begin{pgfscope}
    \definecolor{fc}{rgb}{0.0000,0.0000,0.0000}
    \pgfsetfillcolor{fc}
    \pgfsetfillopacity{0.0000}
    \pgfpathqmoveto{13.3333bp}{140.0000bp}
    \pgfpathqlineto{13.3333bp}{153.3333bp}
    \pgfpathqlineto{-0.0000bp}{153.3333bp}
    \pgfpathqlineto{-0.0000bp}{140.0000bp}
    \pgfpathqlineto{13.3333bp}{140.0000bp}
    \pgfpathclose
    \pgfusepathqfill
  \end{pgfscope}
  \begin{pgfscope}
    \definecolor{fc}{rgb}{0.0000,0.0000,0.0000}
    \pgfsetfillcolor{fc}
    \pgftransformshift{\pgfqpoint{6.6667bp}{146.6667bp}}
    \pgftransformscale{0.8333}
    \pgftext[]{$8$}
  \end{pgfscope}
  \begin{pgfscope}
    \definecolor{fc}{rgb}{0.0000,0.0000,0.0000}
    \pgfsetfillcolor{fc}
    \pgfsetfillopacity{0.0000}
    \pgfsetlinewidth{0.7303bp}
    \definecolor{sc}{rgb}{0.0000,0.0000,0.0000}
    \pgfsetstrokecolor{sc}
    \pgfsetmiterjoin
    \pgfsetbuttcap
    \pgfpathqmoveto{100.0000bp}{160.0000bp}
    \pgfpathqlineto{100.0000bp}{173.3333bp}
    \pgfpathqlineto{86.6667bp}{173.3333bp}
    \pgfpathqlineto{86.6667bp}{160.0000bp}
    \pgfpathqlineto{100.0000bp}{160.0000bp}
    \pgfpathclose
    \pgfusepathqfillstroke
  \end{pgfscope}
  \begin{pgfscope}
    \definecolor{fc}{rgb}{0.0000,0.0000,0.0000}
    \pgfsetfillcolor{fc}
    \pgftransformshift{\pgfqpoint{93.3333bp}{166.6667bp}}
    \pgftransformscale{1.5000}
    \pgftext[]{$1$}
  \end{pgfscope}
  \begin{pgfscope}
    \definecolor{fc}{rgb}{0.0000,0.0000,0.0000}
    \pgfsetfillcolor{fc}
    \pgfsetfillopacity{0.0000}
    \pgfsetlinewidth{0.7303bp}
    \definecolor{sc}{rgb}{0.0000,0.0000,0.0000}
    \pgfsetstrokecolor{sc}
    \pgfsetmiterjoin
    \pgfsetbuttcap
    \pgfpathqmoveto{86.6667bp}{160.0000bp}
    \pgfpathqlineto{86.6667bp}{173.3333bp}
    \pgfpathqlineto{73.3333bp}{173.3333bp}
    \pgfpathqlineto{73.3333bp}{160.0000bp}
    \pgfpathqlineto{86.6667bp}{160.0000bp}
    \pgfpathclose
    \pgfusepathqfillstroke
  \end{pgfscope}
  \begin{pgfscope}
    \definecolor{fc}{rgb}{0.0000,0.0000,0.0000}
    \pgfsetfillcolor{fc}
    \pgftransformshift{\pgfqpoint{80.0000bp}{166.6667bp}}
    \pgftransformscale{1.5000}
    \pgftext[]{$1$}
  \end{pgfscope}
  \begin{pgfscope}
    \definecolor{fc}{rgb}{0.0000,0.0000,0.0000}
    \pgfsetfillcolor{fc}
    \pgfsetfillopacity{0.0000}
    \pgfsetlinewidth{0.7303bp}
    \definecolor{sc}{rgb}{0.0000,0.0000,0.0000}
    \pgfsetstrokecolor{sc}
    \pgfsetmiterjoin
    \pgfsetbuttcap
    \pgfpathqmoveto{73.3333bp}{160.0000bp}
    \pgfpathqlineto{73.3333bp}{173.3333bp}
    \pgfpathqlineto{60.0000bp}{173.3333bp}
    \pgfpathqlineto{60.0000bp}{160.0000bp}
    \pgfpathqlineto{73.3333bp}{160.0000bp}
    \pgfpathclose
    \pgfusepathqfillstroke
  \end{pgfscope}
  \begin{pgfscope}
    \definecolor{fc}{rgb}{0.0000,0.0000,0.0000}
    \pgfsetfillcolor{fc}
    \pgftransformshift{\pgfqpoint{66.6667bp}{166.6667bp}}
    \pgftransformscale{1.5000}
    \pgftext[]{$1$}
  \end{pgfscope}
  \begin{pgfscope}
    \definecolor{fc}{rgb}{0.0000,0.0000,0.0000}
    \pgfsetfillcolor{fc}
    \pgfsetfillopacity{0.0000}
    \pgfsetlinewidth{0.7303bp}
    \definecolor{sc}{rgb}{0.0000,0.0000,0.0000}
    \pgfsetstrokecolor{sc}
    \pgfsetmiterjoin
    \pgfsetbuttcap
    \pgfpathqmoveto{60.0000bp}{160.0000bp}
    \pgfpathqlineto{60.0000bp}{173.3333bp}
    \pgfpathqlineto{46.6667bp}{173.3333bp}
    \pgfpathqlineto{46.6667bp}{160.0000bp}
    \pgfpathqlineto{60.0000bp}{160.0000bp}
    \pgfpathclose
    \pgfusepathqfillstroke
  \end{pgfscope}
  \begin{pgfscope}
    \definecolor{fc}{rgb}{0.0000,0.0000,0.0000}
    \pgfsetfillcolor{fc}
    \pgftransformshift{\pgfqpoint{53.3333bp}{166.6667bp}}
    \pgftransformscale{1.5000}
    \pgftext[]{$0$}
  \end{pgfscope}
  \begin{pgfscope}
    \definecolor{fc}{rgb}{0.0000,0.0000,0.0000}
    \pgfsetfillcolor{fc}
    \pgfsetfillopacity{0.0000}
    \pgfsetlinewidth{0.7303bp}
    \definecolor{sc}{rgb}{0.0000,0.0000,0.0000}
    \pgfsetstrokecolor{sc}
    \pgfsetmiterjoin
    \pgfsetbuttcap
    \pgfpathqmoveto{46.6667bp}{160.0000bp}
    \pgfpathqlineto{46.6667bp}{173.3333bp}
    \pgfpathqlineto{33.3333bp}{173.3333bp}
    \pgfpathqlineto{33.3333bp}{160.0000bp}
    \pgfpathqlineto{46.6667bp}{160.0000bp}
    \pgfpathclose
    \pgfusepathqfillstroke
  \end{pgfscope}
  \begin{pgfscope}
    \definecolor{fc}{rgb}{0.0000,0.0000,0.0000}
    \pgfsetfillcolor{fc}
    \pgftransformshift{\pgfqpoint{40.0000bp}{166.6667bp}}
    \pgftransformscale{1.5000}
    \pgftext[]{$0$}
  \end{pgfscope}
  \begin{pgfscope}
    \definecolor{fc}{rgb}{0.0000,0.0000,0.0000}
    \pgfsetfillcolor{fc}
    \pgfsetfillopacity{0.0000}
    \pgfsetlinewidth{0.7303bp}
    \definecolor{sc}{rgb}{0.0000,0.0000,0.0000}
    \pgfsetstrokecolor{sc}
    \pgfsetmiterjoin
    \pgfsetbuttcap
    \pgfpathqmoveto{33.3333bp}{160.0000bp}
    \pgfpathqlineto{33.3333bp}{173.3333bp}
    \pgfpathqlineto{20.0000bp}{173.3333bp}
    \pgfpathqlineto{20.0000bp}{160.0000bp}
    \pgfpathqlineto{33.3333bp}{160.0000bp}
    \pgfpathclose
    \pgfusepathqfillstroke
  \end{pgfscope}
  \begin{pgfscope}
    \definecolor{fc}{rgb}{0.0000,0.0000,0.0000}
    \pgfsetfillcolor{fc}
    \pgftransformshift{\pgfqpoint{26.6667bp}{166.6667bp}}
    \pgftransformscale{1.5000}
    \pgftext[]{$0$}
  \end{pgfscope}
  \begin{pgfscope}
    \definecolor{fc}{rgb}{0.0000,0.0000,0.0000}
    \pgfsetfillcolor{fc}
    \pgfsetfillopacity{0.0000}
    \pgfpathqmoveto{13.3333bp}{160.0000bp}
    \pgfpathqlineto{13.3333bp}{173.3333bp}
    \pgfpathqlineto{-0.0000bp}{173.3333bp}
    \pgfpathqlineto{-0.0000bp}{160.0000bp}
    \pgfpathqlineto{13.3333bp}{160.0000bp}
    \pgfpathclose
    \pgfusepathqfill
  \end{pgfscope}
  \begin{pgfscope}
    \definecolor{fc}{rgb}{0.0000,0.0000,0.0000}
    \pgfsetfillcolor{fc}
    \pgftransformshift{\pgfqpoint{6.6667bp}{166.6667bp}}
    \pgftransformscale{0.8333}
    \pgftext[]{$7$}
  \end{pgfscope}
  \begin{pgfscope}
    \definecolor{fc}{rgb}{0.0000,0.0000,0.0000}
    \pgfsetfillcolor{fc}
    \pgfsetfillopacity{0.0000}
    \pgfsetlinewidth{0.7303bp}
    \definecolor{sc}{rgb}{0.0000,0.0000,0.0000}
    \pgfsetstrokecolor{sc}
    \pgfsetmiterjoin
    \pgfsetbuttcap
    \pgfpathqmoveto{100.0000bp}{180.0000bp}
    \pgfpathqlineto{100.0000bp}{193.3333bp}
    \pgfpathqlineto{86.6667bp}{193.3333bp}
    \pgfpathqlineto{86.6667bp}{180.0000bp}
    \pgfpathqlineto{100.0000bp}{180.0000bp}
    \pgfpathclose
    \pgfusepathqfillstroke
  \end{pgfscope}
  \begin{pgfscope}
    \definecolor{fc}{rgb}{0.0000,0.0000,0.0000}
    \pgfsetfillcolor{fc}
    \pgftransformshift{\pgfqpoint{93.3333bp}{186.6667bp}}
    \pgftransformscale{1.5000}
    \pgftext[]{$0$}
  \end{pgfscope}
  \begin{pgfscope}
    \definecolor{fc}{rgb}{0.0000,0.0000,0.0000}
    \pgfsetfillcolor{fc}
    \pgfsetfillopacity{0.0000}
    \pgfsetlinewidth{0.7303bp}
    \definecolor{sc}{rgb}{0.0000,0.0000,0.0000}
    \pgfsetstrokecolor{sc}
    \pgfsetmiterjoin
    \pgfsetbuttcap
    \pgfpathqmoveto{86.6667bp}{180.0000bp}
    \pgfpathqlineto{86.6667bp}{193.3333bp}
    \pgfpathqlineto{73.3333bp}{193.3333bp}
    \pgfpathqlineto{73.3333bp}{180.0000bp}
    \pgfpathqlineto{86.6667bp}{180.0000bp}
    \pgfpathclose
    \pgfusepathqfillstroke
  \end{pgfscope}
  \begin{pgfscope}
    \definecolor{fc}{rgb}{0.0000,0.0000,0.0000}
    \pgfsetfillcolor{fc}
    \pgftransformshift{\pgfqpoint{80.0000bp}{186.6667bp}}
    \pgftransformscale{1.5000}
    \pgftext[]{$1$}
  \end{pgfscope}
  \begin{pgfscope}
    \definecolor{fc}{rgb}{0.0000,0.0000,0.0000}
    \pgfsetfillcolor{fc}
    \pgfsetfillopacity{0.0000}
    \pgfsetlinewidth{0.7303bp}
    \definecolor{sc}{rgb}{0.0000,0.0000,0.0000}
    \pgfsetstrokecolor{sc}
    \pgfsetmiterjoin
    \pgfsetbuttcap
    \pgfpathqmoveto{73.3333bp}{180.0000bp}
    \pgfpathqlineto{73.3333bp}{193.3333bp}
    \pgfpathqlineto{60.0000bp}{193.3333bp}
    \pgfpathqlineto{60.0000bp}{180.0000bp}
    \pgfpathqlineto{73.3333bp}{180.0000bp}
    \pgfpathclose
    \pgfusepathqfillstroke
  \end{pgfscope}
  \begin{pgfscope}
    \definecolor{fc}{rgb}{0.0000,0.0000,0.0000}
    \pgfsetfillcolor{fc}
    \pgftransformshift{\pgfqpoint{66.6667bp}{186.6667bp}}
    \pgftransformscale{1.5000}
    \pgftext[]{$1$}
  \end{pgfscope}
  \begin{pgfscope}
    \definecolor{fc}{rgb}{0.0000,0.0000,0.0000}
    \pgfsetfillcolor{fc}
    \pgfsetfillopacity{0.0000}
    \pgfsetlinewidth{0.7303bp}
    \definecolor{sc}{rgb}{0.0000,0.0000,0.0000}
    \pgfsetstrokecolor{sc}
    \pgfsetmiterjoin
    \pgfsetbuttcap
    \pgfpathqmoveto{60.0000bp}{180.0000bp}
    \pgfpathqlineto{60.0000bp}{193.3333bp}
    \pgfpathqlineto{46.6667bp}{193.3333bp}
    \pgfpathqlineto{46.6667bp}{180.0000bp}
    \pgfpathqlineto{60.0000bp}{180.0000bp}
    \pgfpathclose
    \pgfusepathqfillstroke
  \end{pgfscope}
  \begin{pgfscope}
    \definecolor{fc}{rgb}{0.0000,0.0000,0.0000}
    \pgfsetfillcolor{fc}
    \pgftransformshift{\pgfqpoint{53.3333bp}{186.6667bp}}
    \pgftransformscale{1.5000}
    \pgftext[]{$0$}
  \end{pgfscope}
  \begin{pgfscope}
    \definecolor{fc}{rgb}{0.0000,0.0000,0.0000}
    \pgfsetfillcolor{fc}
    \pgfsetfillopacity{0.0000}
    \pgfsetlinewidth{0.7303bp}
    \definecolor{sc}{rgb}{0.0000,0.0000,0.0000}
    \pgfsetstrokecolor{sc}
    \pgfsetmiterjoin
    \pgfsetbuttcap
    \pgfpathqmoveto{46.6667bp}{180.0000bp}
    \pgfpathqlineto{46.6667bp}{193.3333bp}
    \pgfpathqlineto{33.3333bp}{193.3333bp}
    \pgfpathqlineto{33.3333bp}{180.0000bp}
    \pgfpathqlineto{46.6667bp}{180.0000bp}
    \pgfpathclose
    \pgfusepathqfillstroke
  \end{pgfscope}
  \begin{pgfscope}
    \definecolor{fc}{rgb}{0.0000,0.0000,0.0000}
    \pgfsetfillcolor{fc}
    \pgftransformshift{\pgfqpoint{40.0000bp}{186.6667bp}}
    \pgftransformscale{1.5000}
    \pgftext[]{$0$}
  \end{pgfscope}
  \begin{pgfscope}
    \definecolor{fc}{rgb}{0.0000,0.0000,0.0000}
    \pgfsetfillcolor{fc}
    \pgfsetfillopacity{0.0000}
    \pgfsetlinewidth{0.7303bp}
    \definecolor{sc}{rgb}{0.0000,0.0000,0.0000}
    \pgfsetstrokecolor{sc}
    \pgfsetmiterjoin
    \pgfsetbuttcap
    \pgfpathqmoveto{33.3333bp}{180.0000bp}
    \pgfpathqlineto{33.3333bp}{193.3333bp}
    \pgfpathqlineto{20.0000bp}{193.3333bp}
    \pgfpathqlineto{20.0000bp}{180.0000bp}
    \pgfpathqlineto{33.3333bp}{180.0000bp}
    \pgfpathclose
    \pgfusepathqfillstroke
  \end{pgfscope}
  \begin{pgfscope}
    \definecolor{fc}{rgb}{0.0000,0.0000,0.0000}
    \pgfsetfillcolor{fc}
    \pgftransformshift{\pgfqpoint{26.6667bp}{186.6667bp}}
    \pgftransformscale{1.5000}
    \pgftext[]{$0$}
  \end{pgfscope}
  \begin{pgfscope}
    \definecolor{fc}{rgb}{0.0000,0.0000,0.0000}
    \pgfsetfillcolor{fc}
    \pgfsetfillopacity{0.0000}
    \pgfpathqmoveto{13.3333bp}{180.0000bp}
    \pgfpathqlineto{13.3333bp}{193.3333bp}
    \pgfpathqlineto{-0.0000bp}{193.3333bp}
    \pgfpathqlineto{-0.0000bp}{180.0000bp}
    \pgfpathqlineto{13.3333bp}{180.0000bp}
    \pgfpathclose
    \pgfusepathqfill
  \end{pgfscope}
  \begin{pgfscope}
    \definecolor{fc}{rgb}{0.0000,0.0000,0.0000}
    \pgfsetfillcolor{fc}
    \pgftransformshift{\pgfqpoint{6.6667bp}{186.6667bp}}
    \pgftransformscale{0.8333}
    \pgftext[]{$6$}
  \end{pgfscope}
  \begin{pgfscope}
    \definecolor{fc}{rgb}{0.0000,0.0000,0.0000}
    \pgfsetfillcolor{fc}
    \pgfsetfillopacity{0.0000}
    \pgfsetlinewidth{0.7303bp}
    \definecolor{sc}{rgb}{0.0000,0.0000,0.0000}
    \pgfsetstrokecolor{sc}
    \pgfsetmiterjoin
    \pgfsetbuttcap
    \pgfpathqmoveto{100.0000bp}{200.0000bp}
    \pgfpathqlineto{100.0000bp}{213.3333bp}
    \pgfpathqlineto{86.6667bp}{213.3333bp}
    \pgfpathqlineto{86.6667bp}{200.0000bp}
    \pgfpathqlineto{100.0000bp}{200.0000bp}
    \pgfpathclose
    \pgfusepathqfillstroke
  \end{pgfscope}
  \begin{pgfscope}
    \definecolor{fc}{rgb}{0.0000,0.0000,0.0000}
    \pgfsetfillcolor{fc}
    \pgftransformshift{\pgfqpoint{93.3333bp}{206.6667bp}}
    \pgftransformscale{1.5000}
    \pgftext[]{$1$}
  \end{pgfscope}
  \begin{pgfscope}
    \definecolor{fc}{rgb}{0.0000,0.0000,0.0000}
    \pgfsetfillcolor{fc}
    \pgfsetfillopacity{0.0000}
    \pgfsetlinewidth{0.7303bp}
    \definecolor{sc}{rgb}{0.0000,0.0000,0.0000}
    \pgfsetstrokecolor{sc}
    \pgfsetmiterjoin
    \pgfsetbuttcap
    \pgfpathqmoveto{86.6667bp}{200.0000bp}
    \pgfpathqlineto{86.6667bp}{213.3333bp}
    \pgfpathqlineto{73.3333bp}{213.3333bp}
    \pgfpathqlineto{73.3333bp}{200.0000bp}
    \pgfpathqlineto{86.6667bp}{200.0000bp}
    \pgfpathclose
    \pgfusepathqfillstroke
  \end{pgfscope}
  \begin{pgfscope}
    \definecolor{fc}{rgb}{0.0000,0.0000,0.0000}
    \pgfsetfillcolor{fc}
    \pgftransformshift{\pgfqpoint{80.0000bp}{206.6667bp}}
    \pgftransformscale{1.5000}
    \pgftext[]{$0$}
  \end{pgfscope}
  \begin{pgfscope}
    \definecolor{fc}{rgb}{0.0000,0.0000,0.0000}
    \pgfsetfillcolor{fc}
    \pgfsetfillopacity{0.0000}
    \pgfsetlinewidth{0.7303bp}
    \definecolor{sc}{rgb}{0.0000,0.0000,0.0000}
    \pgfsetstrokecolor{sc}
    \pgfsetmiterjoin
    \pgfsetbuttcap
    \pgfpathqmoveto{73.3333bp}{200.0000bp}
    \pgfpathqlineto{73.3333bp}{213.3333bp}
    \pgfpathqlineto{60.0000bp}{213.3333bp}
    \pgfpathqlineto{60.0000bp}{200.0000bp}
    \pgfpathqlineto{73.3333bp}{200.0000bp}
    \pgfpathclose
    \pgfusepathqfillstroke
  \end{pgfscope}
  \begin{pgfscope}
    \definecolor{fc}{rgb}{0.0000,0.0000,0.0000}
    \pgfsetfillcolor{fc}
    \pgftransformshift{\pgfqpoint{66.6667bp}{206.6667bp}}
    \pgftransformscale{1.5000}
    \pgftext[]{$1$}
  \end{pgfscope}
  \begin{pgfscope}
    \definecolor{fc}{rgb}{0.0000,0.0000,0.0000}
    \pgfsetfillcolor{fc}
    \pgfsetfillopacity{0.0000}
    \pgfsetlinewidth{0.7303bp}
    \definecolor{sc}{rgb}{0.0000,0.0000,0.0000}
    \pgfsetstrokecolor{sc}
    \pgfsetmiterjoin
    \pgfsetbuttcap
    \pgfpathqmoveto{60.0000bp}{200.0000bp}
    \pgfpathqlineto{60.0000bp}{213.3333bp}
    \pgfpathqlineto{46.6667bp}{213.3333bp}
    \pgfpathqlineto{46.6667bp}{200.0000bp}
    \pgfpathqlineto{60.0000bp}{200.0000bp}
    \pgfpathclose
    \pgfusepathqfillstroke
  \end{pgfscope}
  \begin{pgfscope}
    \definecolor{fc}{rgb}{0.0000,0.0000,0.0000}
    \pgfsetfillcolor{fc}
    \pgftransformshift{\pgfqpoint{53.3333bp}{206.6667bp}}
    \pgftransformscale{1.5000}
    \pgftext[]{$0$}
  \end{pgfscope}
  \begin{pgfscope}
    \definecolor{fc}{rgb}{0.0000,0.0000,0.0000}
    \pgfsetfillcolor{fc}
    \pgfsetfillopacity{0.0000}
    \pgfsetlinewidth{0.7303bp}
    \definecolor{sc}{rgb}{0.0000,0.0000,0.0000}
    \pgfsetstrokecolor{sc}
    \pgfsetmiterjoin
    \pgfsetbuttcap
    \pgfpathqmoveto{46.6667bp}{200.0000bp}
    \pgfpathqlineto{46.6667bp}{213.3333bp}
    \pgfpathqlineto{33.3333bp}{213.3333bp}
    \pgfpathqlineto{33.3333bp}{200.0000bp}
    \pgfpathqlineto{46.6667bp}{200.0000bp}
    \pgfpathclose
    \pgfusepathqfillstroke
  \end{pgfscope}
  \begin{pgfscope}
    \definecolor{fc}{rgb}{0.0000,0.0000,0.0000}
    \pgfsetfillcolor{fc}
    \pgftransformshift{\pgfqpoint{40.0000bp}{206.6667bp}}
    \pgftransformscale{1.5000}
    \pgftext[]{$0$}
  \end{pgfscope}
  \begin{pgfscope}
    \definecolor{fc}{rgb}{0.0000,0.0000,0.0000}
    \pgfsetfillcolor{fc}
    \pgfsetfillopacity{0.0000}
    \pgfsetlinewidth{0.7303bp}
    \definecolor{sc}{rgb}{0.0000,0.0000,0.0000}
    \pgfsetstrokecolor{sc}
    \pgfsetmiterjoin
    \pgfsetbuttcap
    \pgfpathqmoveto{33.3333bp}{200.0000bp}
    \pgfpathqlineto{33.3333bp}{213.3333bp}
    \pgfpathqlineto{20.0000bp}{213.3333bp}
    \pgfpathqlineto{20.0000bp}{200.0000bp}
    \pgfpathqlineto{33.3333bp}{200.0000bp}
    \pgfpathclose
    \pgfusepathqfillstroke
  \end{pgfscope}
  \begin{pgfscope}
    \definecolor{fc}{rgb}{0.0000,0.0000,0.0000}
    \pgfsetfillcolor{fc}
    \pgftransformshift{\pgfqpoint{26.6667bp}{206.6667bp}}
    \pgftransformscale{1.5000}
    \pgftext[]{$0$}
  \end{pgfscope}
  \begin{pgfscope}
    \definecolor{fc}{rgb}{0.0000,0.0000,0.0000}
    \pgfsetfillcolor{fc}
    \pgfsetfillopacity{0.0000}
    \pgfpathqmoveto{13.3333bp}{200.0000bp}
    \pgfpathqlineto{13.3333bp}{213.3333bp}
    \pgfpathqlineto{-0.0000bp}{213.3333bp}
    \pgfpathqlineto{-0.0000bp}{200.0000bp}
    \pgfpathqlineto{13.3333bp}{200.0000bp}
    \pgfpathclose
    \pgfusepathqfill
  \end{pgfscope}
  \begin{pgfscope}
    \definecolor{fc}{rgb}{0.0000,0.0000,0.0000}
    \pgfsetfillcolor{fc}
    \pgftransformshift{\pgfqpoint{6.6667bp}{206.6667bp}}
    \pgftransformscale{0.8333}
    \pgftext[]{$5$}
  \end{pgfscope}
  \begin{pgfscope}
    \definecolor{fc}{rgb}{0.0000,0.0000,0.0000}
    \pgfsetfillcolor{fc}
    \pgfsetfillopacity{0.0000}
    \pgfsetlinewidth{0.7303bp}
    \definecolor{sc}{rgb}{0.0000,0.0000,0.0000}
    \pgfsetstrokecolor{sc}
    \pgfsetmiterjoin
    \pgfsetbuttcap
    \pgfpathqmoveto{100.0000bp}{220.0000bp}
    \pgfpathqlineto{100.0000bp}{233.3333bp}
    \pgfpathqlineto{86.6667bp}{233.3333bp}
    \pgfpathqlineto{86.6667bp}{220.0000bp}
    \pgfpathqlineto{100.0000bp}{220.0000bp}
    \pgfpathclose
    \pgfusepathqfillstroke
  \end{pgfscope}
  \begin{pgfscope}
    \definecolor{fc}{rgb}{0.0000,0.0000,0.0000}
    \pgfsetfillcolor{fc}
    \pgftransformshift{\pgfqpoint{93.3333bp}{226.6667bp}}
    \pgftransformscale{1.5000}
    \pgftext[]{$0$}
  \end{pgfscope}
  \begin{pgfscope}
    \definecolor{fc}{rgb}{0.0000,0.0000,0.0000}
    \pgfsetfillcolor{fc}
    \pgfsetfillopacity{0.0000}
    \pgfsetlinewidth{0.7303bp}
    \definecolor{sc}{rgb}{0.0000,0.0000,0.0000}
    \pgfsetstrokecolor{sc}
    \pgfsetmiterjoin
    \pgfsetbuttcap
    \pgfpathqmoveto{86.6667bp}{220.0000bp}
    \pgfpathqlineto{86.6667bp}{233.3333bp}
    \pgfpathqlineto{73.3333bp}{233.3333bp}
    \pgfpathqlineto{73.3333bp}{220.0000bp}
    \pgfpathqlineto{86.6667bp}{220.0000bp}
    \pgfpathclose
    \pgfusepathqfillstroke
  \end{pgfscope}
  \begin{pgfscope}
    \definecolor{fc}{rgb}{0.0000,0.0000,0.0000}
    \pgfsetfillcolor{fc}
    \pgftransformshift{\pgfqpoint{80.0000bp}{226.6667bp}}
    \pgftransformscale{1.5000}
    \pgftext[]{$0$}
  \end{pgfscope}
  \begin{pgfscope}
    \definecolor{fc}{rgb}{0.0000,0.0000,0.0000}
    \pgfsetfillcolor{fc}
    \pgfsetfillopacity{0.0000}
    \pgfsetlinewidth{0.7303bp}
    \definecolor{sc}{rgb}{0.0000,0.0000,0.0000}
    \pgfsetstrokecolor{sc}
    \pgfsetmiterjoin
    \pgfsetbuttcap
    \pgfpathqmoveto{73.3333bp}{220.0000bp}
    \pgfpathqlineto{73.3333bp}{233.3333bp}
    \pgfpathqlineto{60.0000bp}{233.3333bp}
    \pgfpathqlineto{60.0000bp}{220.0000bp}
    \pgfpathqlineto{73.3333bp}{220.0000bp}
    \pgfpathclose
    \pgfusepathqfillstroke
  \end{pgfscope}
  \begin{pgfscope}
    \definecolor{fc}{rgb}{0.0000,0.0000,0.0000}
    \pgfsetfillcolor{fc}
    \pgftransformshift{\pgfqpoint{66.6667bp}{226.6667bp}}
    \pgftransformscale{1.5000}
    \pgftext[]{$1$}
  \end{pgfscope}
  \begin{pgfscope}
    \definecolor{fc}{rgb}{0.0000,0.0000,0.0000}
    \pgfsetfillcolor{fc}
    \pgfsetfillopacity{0.0000}
    \pgfsetlinewidth{0.7303bp}
    \definecolor{sc}{rgb}{0.0000,0.0000,0.0000}
    \pgfsetstrokecolor{sc}
    \pgfsetmiterjoin
    \pgfsetbuttcap
    \pgfpathqmoveto{60.0000bp}{220.0000bp}
    \pgfpathqlineto{60.0000bp}{233.3333bp}
    \pgfpathqlineto{46.6667bp}{233.3333bp}
    \pgfpathqlineto{46.6667bp}{220.0000bp}
    \pgfpathqlineto{60.0000bp}{220.0000bp}
    \pgfpathclose
    \pgfusepathqfillstroke
  \end{pgfscope}
  \begin{pgfscope}
    \definecolor{fc}{rgb}{0.0000,0.0000,0.0000}
    \pgfsetfillcolor{fc}
    \pgftransformshift{\pgfqpoint{53.3333bp}{226.6667bp}}
    \pgftransformscale{1.5000}
    \pgftext[]{$0$}
  \end{pgfscope}
  \begin{pgfscope}
    \definecolor{fc}{rgb}{0.0000,0.0000,0.0000}
    \pgfsetfillcolor{fc}
    \pgfsetfillopacity{0.0000}
    \pgfsetlinewidth{0.7303bp}
    \definecolor{sc}{rgb}{0.0000,0.0000,0.0000}
    \pgfsetstrokecolor{sc}
    \pgfsetmiterjoin
    \pgfsetbuttcap
    \pgfpathqmoveto{46.6667bp}{220.0000bp}
    \pgfpathqlineto{46.6667bp}{233.3333bp}
    \pgfpathqlineto{33.3333bp}{233.3333bp}
    \pgfpathqlineto{33.3333bp}{220.0000bp}
    \pgfpathqlineto{46.6667bp}{220.0000bp}
    \pgfpathclose
    \pgfusepathqfillstroke
  \end{pgfscope}
  \begin{pgfscope}
    \definecolor{fc}{rgb}{0.0000,0.0000,0.0000}
    \pgfsetfillcolor{fc}
    \pgftransformshift{\pgfqpoint{40.0000bp}{226.6667bp}}
    \pgftransformscale{1.5000}
    \pgftext[]{$0$}
  \end{pgfscope}
  \begin{pgfscope}
    \definecolor{fc}{rgb}{0.0000,0.0000,0.0000}
    \pgfsetfillcolor{fc}
    \pgfsetfillopacity{0.0000}
    \pgfsetlinewidth{0.7303bp}
    \definecolor{sc}{rgb}{0.0000,0.0000,0.0000}
    \pgfsetstrokecolor{sc}
    \pgfsetmiterjoin
    \pgfsetbuttcap
    \pgfpathqmoveto{33.3333bp}{220.0000bp}
    \pgfpathqlineto{33.3333bp}{233.3333bp}
    \pgfpathqlineto{20.0000bp}{233.3333bp}
    \pgfpathqlineto{20.0000bp}{220.0000bp}
    \pgfpathqlineto{33.3333bp}{220.0000bp}
    \pgfpathclose
    \pgfusepathqfillstroke
  \end{pgfscope}
  \begin{pgfscope}
    \definecolor{fc}{rgb}{0.0000,0.0000,0.0000}
    \pgfsetfillcolor{fc}
    \pgftransformshift{\pgfqpoint{26.6667bp}{226.6667bp}}
    \pgftransformscale{1.5000}
    \pgftext[]{$0$}
  \end{pgfscope}
  \begin{pgfscope}
    \definecolor{fc}{rgb}{0.0000,0.0000,0.0000}
    \pgfsetfillcolor{fc}
    \pgfsetfillopacity{0.0000}
    \pgfpathqmoveto{13.3333bp}{220.0000bp}
    \pgfpathqlineto{13.3333bp}{233.3333bp}
    \pgfpathqlineto{-0.0000bp}{233.3333bp}
    \pgfpathqlineto{-0.0000bp}{220.0000bp}
    \pgfpathqlineto{13.3333bp}{220.0000bp}
    \pgfpathclose
    \pgfusepathqfill
  \end{pgfscope}
  \begin{pgfscope}
    \definecolor{fc}{rgb}{0.0000,0.0000,0.0000}
    \pgfsetfillcolor{fc}
    \pgftransformshift{\pgfqpoint{6.6667bp}{226.6667bp}}
    \pgftransformscale{0.8333}
    \pgftext[]{$4$}
  \end{pgfscope}
  \begin{pgfscope}
    \definecolor{fc}{rgb}{0.0000,0.0000,0.0000}
    \pgfsetfillcolor{fc}
    \pgfsetfillopacity{0.0000}
    \pgfsetlinewidth{0.7303bp}
    \definecolor{sc}{rgb}{0.0000,0.0000,0.0000}
    \pgfsetstrokecolor{sc}
    \pgfsetmiterjoin
    \pgfsetbuttcap
    \pgfpathqmoveto{100.0000bp}{240.0000bp}
    \pgfpathqlineto{100.0000bp}{253.3333bp}
    \pgfpathqlineto{86.6667bp}{253.3333bp}
    \pgfpathqlineto{86.6667bp}{240.0000bp}
    \pgfpathqlineto{100.0000bp}{240.0000bp}
    \pgfpathclose
    \pgfusepathqfillstroke
  \end{pgfscope}
  \begin{pgfscope}
    \definecolor{fc}{rgb}{0.0000,0.0000,0.0000}
    \pgfsetfillcolor{fc}
    \pgftransformshift{\pgfqpoint{93.3333bp}{246.6667bp}}
    \pgftransformscale{1.5000}
    \pgftext[]{$1$}
  \end{pgfscope}
  \begin{pgfscope}
    \definecolor{fc}{rgb}{0.0000,0.0000,0.0000}
    \pgfsetfillcolor{fc}
    \pgfsetfillopacity{0.0000}
    \pgfsetlinewidth{0.7303bp}
    \definecolor{sc}{rgb}{0.0000,0.0000,0.0000}
    \pgfsetstrokecolor{sc}
    \pgfsetmiterjoin
    \pgfsetbuttcap
    \pgfpathqmoveto{86.6667bp}{240.0000bp}
    \pgfpathqlineto{86.6667bp}{253.3333bp}
    \pgfpathqlineto{73.3333bp}{253.3333bp}
    \pgfpathqlineto{73.3333bp}{240.0000bp}
    \pgfpathqlineto{86.6667bp}{240.0000bp}
    \pgfpathclose
    \pgfusepathqfillstroke
  \end{pgfscope}
  \begin{pgfscope}
    \definecolor{fc}{rgb}{0.0000,0.0000,0.0000}
    \pgfsetfillcolor{fc}
    \pgftransformshift{\pgfqpoint{80.0000bp}{246.6667bp}}
    \pgftransformscale{1.5000}
    \pgftext[]{$1$}
  \end{pgfscope}
  \begin{pgfscope}
    \definecolor{fc}{rgb}{0.0000,0.0000,0.0000}
    \pgfsetfillcolor{fc}
    \pgfsetfillopacity{0.0000}
    \pgfsetlinewidth{0.7303bp}
    \definecolor{sc}{rgb}{0.0000,0.0000,0.0000}
    \pgfsetstrokecolor{sc}
    \pgfsetmiterjoin
    \pgfsetbuttcap
    \pgfpathqmoveto{73.3333bp}{240.0000bp}
    \pgfpathqlineto{73.3333bp}{253.3333bp}
    \pgfpathqlineto{60.0000bp}{253.3333bp}
    \pgfpathqlineto{60.0000bp}{240.0000bp}
    \pgfpathqlineto{73.3333bp}{240.0000bp}
    \pgfpathclose
    \pgfusepathqfillstroke
  \end{pgfscope}
  \begin{pgfscope}
    \definecolor{fc}{rgb}{0.0000,0.0000,0.0000}
    \pgfsetfillcolor{fc}
    \pgftransformshift{\pgfqpoint{66.6667bp}{246.6667bp}}
    \pgftransformscale{1.5000}
    \pgftext[]{$0$}
  \end{pgfscope}
  \begin{pgfscope}
    \definecolor{fc}{rgb}{0.0000,0.0000,0.0000}
    \pgfsetfillcolor{fc}
    \pgfsetfillopacity{0.0000}
    \pgfsetlinewidth{0.7303bp}
    \definecolor{sc}{rgb}{0.0000,0.0000,0.0000}
    \pgfsetstrokecolor{sc}
    \pgfsetmiterjoin
    \pgfsetbuttcap
    \pgfpathqmoveto{60.0000bp}{240.0000bp}
    \pgfpathqlineto{60.0000bp}{253.3333bp}
    \pgfpathqlineto{46.6667bp}{253.3333bp}
    \pgfpathqlineto{46.6667bp}{240.0000bp}
    \pgfpathqlineto{60.0000bp}{240.0000bp}
    \pgfpathclose
    \pgfusepathqfillstroke
  \end{pgfscope}
  \begin{pgfscope}
    \definecolor{fc}{rgb}{0.0000,0.0000,0.0000}
    \pgfsetfillcolor{fc}
    \pgftransformshift{\pgfqpoint{53.3333bp}{246.6667bp}}
    \pgftransformscale{1.5000}
    \pgftext[]{$0$}
  \end{pgfscope}
  \begin{pgfscope}
    \definecolor{fc}{rgb}{0.0000,0.0000,0.0000}
    \pgfsetfillcolor{fc}
    \pgfsetfillopacity{0.0000}
    \pgfsetlinewidth{0.7303bp}
    \definecolor{sc}{rgb}{0.0000,0.0000,0.0000}
    \pgfsetstrokecolor{sc}
    \pgfsetmiterjoin
    \pgfsetbuttcap
    \pgfpathqmoveto{46.6667bp}{240.0000bp}
    \pgfpathqlineto{46.6667bp}{253.3333bp}
    \pgfpathqlineto{33.3333bp}{253.3333bp}
    \pgfpathqlineto{33.3333bp}{240.0000bp}
    \pgfpathqlineto{46.6667bp}{240.0000bp}
    \pgfpathclose
    \pgfusepathqfillstroke
  \end{pgfscope}
  \begin{pgfscope}
    \definecolor{fc}{rgb}{0.0000,0.0000,0.0000}
    \pgfsetfillcolor{fc}
    \pgftransformshift{\pgfqpoint{40.0000bp}{246.6667bp}}
    \pgftransformscale{1.5000}
    \pgftext[]{$0$}
  \end{pgfscope}
  \begin{pgfscope}
    \definecolor{fc}{rgb}{0.0000,0.0000,0.0000}
    \pgfsetfillcolor{fc}
    \pgfsetfillopacity{0.0000}
    \pgfsetlinewidth{0.7303bp}
    \definecolor{sc}{rgb}{0.0000,0.0000,0.0000}
    \pgfsetstrokecolor{sc}
    \pgfsetmiterjoin
    \pgfsetbuttcap
    \pgfpathqmoveto{33.3333bp}{240.0000bp}
    \pgfpathqlineto{33.3333bp}{253.3333bp}
    \pgfpathqlineto{20.0000bp}{253.3333bp}
    \pgfpathqlineto{20.0000bp}{240.0000bp}
    \pgfpathqlineto{33.3333bp}{240.0000bp}
    \pgfpathclose
    \pgfusepathqfillstroke
  \end{pgfscope}
  \begin{pgfscope}
    \definecolor{fc}{rgb}{0.0000,0.0000,0.0000}
    \pgfsetfillcolor{fc}
    \pgftransformshift{\pgfqpoint{26.6667bp}{246.6667bp}}
    \pgftransformscale{1.5000}
    \pgftext[]{$0$}
  \end{pgfscope}
  \begin{pgfscope}
    \definecolor{fc}{rgb}{0.0000,0.0000,0.0000}
    \pgfsetfillcolor{fc}
    \pgfsetfillopacity{0.0000}
    \pgfpathqmoveto{13.3333bp}{240.0000bp}
    \pgfpathqlineto{13.3333bp}{253.3333bp}
    \pgfpathqlineto{-0.0000bp}{253.3333bp}
    \pgfpathqlineto{-0.0000bp}{240.0000bp}
    \pgfpathqlineto{13.3333bp}{240.0000bp}
    \pgfpathclose
    \pgfusepathqfill
  \end{pgfscope}
  \begin{pgfscope}
    \definecolor{fc}{rgb}{0.0000,0.0000,0.0000}
    \pgfsetfillcolor{fc}
    \pgftransformshift{\pgfqpoint{6.6667bp}{246.6667bp}}
    \pgftransformscale{0.8333}
    \pgftext[]{$3$}
  \end{pgfscope}
  \begin{pgfscope}
    \definecolor{fc}{rgb}{0.0000,0.0000,0.0000}
    \pgfsetfillcolor{fc}
    \pgfsetfillopacity{0.0000}
    \pgfsetlinewidth{0.7303bp}
    \definecolor{sc}{rgb}{0.0000,0.0000,0.0000}
    \pgfsetstrokecolor{sc}
    \pgfsetmiterjoin
    \pgfsetbuttcap
    \pgfpathqmoveto{100.0000bp}{260.0000bp}
    \pgfpathqlineto{100.0000bp}{273.3333bp}
    \pgfpathqlineto{86.6667bp}{273.3333bp}
    \pgfpathqlineto{86.6667bp}{260.0000bp}
    \pgfpathqlineto{100.0000bp}{260.0000bp}
    \pgfpathclose
    \pgfusepathqfillstroke
  \end{pgfscope}
  \begin{pgfscope}
    \definecolor{fc}{rgb}{0.0000,0.0000,0.0000}
    \pgfsetfillcolor{fc}
    \pgftransformshift{\pgfqpoint{93.3333bp}{266.6667bp}}
    \pgftransformscale{1.5000}
    \pgftext[]{$0$}
  \end{pgfscope}
  \begin{pgfscope}
    \definecolor{fc}{rgb}{0.0000,0.0000,0.0000}
    \pgfsetfillcolor{fc}
    \pgfsetfillopacity{0.0000}
    \pgfsetlinewidth{0.7303bp}
    \definecolor{sc}{rgb}{0.0000,0.0000,0.0000}
    \pgfsetstrokecolor{sc}
    \pgfsetmiterjoin
    \pgfsetbuttcap
    \pgfpathqmoveto{86.6667bp}{260.0000bp}
    \pgfpathqlineto{86.6667bp}{273.3333bp}
    \pgfpathqlineto{73.3333bp}{273.3333bp}
    \pgfpathqlineto{73.3333bp}{260.0000bp}
    \pgfpathqlineto{86.6667bp}{260.0000bp}
    \pgfpathclose
    \pgfusepathqfillstroke
  \end{pgfscope}
  \begin{pgfscope}
    \definecolor{fc}{rgb}{0.0000,0.0000,0.0000}
    \pgfsetfillcolor{fc}
    \pgftransformshift{\pgfqpoint{80.0000bp}{266.6667bp}}
    \pgftransformscale{1.5000}
    \pgftext[]{$1$}
  \end{pgfscope}
  \begin{pgfscope}
    \definecolor{fc}{rgb}{0.0000,0.0000,0.0000}
    \pgfsetfillcolor{fc}
    \pgfsetfillopacity{0.0000}
    \pgfsetlinewidth{0.7303bp}
    \definecolor{sc}{rgb}{0.0000,0.0000,0.0000}
    \pgfsetstrokecolor{sc}
    \pgfsetmiterjoin
    \pgfsetbuttcap
    \pgfpathqmoveto{73.3333bp}{260.0000bp}
    \pgfpathqlineto{73.3333bp}{273.3333bp}
    \pgfpathqlineto{60.0000bp}{273.3333bp}
    \pgfpathqlineto{60.0000bp}{260.0000bp}
    \pgfpathqlineto{73.3333bp}{260.0000bp}
    \pgfpathclose
    \pgfusepathqfillstroke
  \end{pgfscope}
  \begin{pgfscope}
    \definecolor{fc}{rgb}{0.0000,0.0000,0.0000}
    \pgfsetfillcolor{fc}
    \pgftransformshift{\pgfqpoint{66.6667bp}{266.6667bp}}
    \pgftransformscale{1.5000}
    \pgftext[]{$0$}
  \end{pgfscope}
  \begin{pgfscope}
    \definecolor{fc}{rgb}{0.0000,0.0000,0.0000}
    \pgfsetfillcolor{fc}
    \pgfsetfillopacity{0.0000}
    \pgfsetlinewidth{0.7303bp}
    \definecolor{sc}{rgb}{0.0000,0.0000,0.0000}
    \pgfsetstrokecolor{sc}
    \pgfsetmiterjoin
    \pgfsetbuttcap
    \pgfpathqmoveto{60.0000bp}{260.0000bp}
    \pgfpathqlineto{60.0000bp}{273.3333bp}
    \pgfpathqlineto{46.6667bp}{273.3333bp}
    \pgfpathqlineto{46.6667bp}{260.0000bp}
    \pgfpathqlineto{60.0000bp}{260.0000bp}
    \pgfpathclose
    \pgfusepathqfillstroke
  \end{pgfscope}
  \begin{pgfscope}
    \definecolor{fc}{rgb}{0.0000,0.0000,0.0000}
    \pgfsetfillcolor{fc}
    \pgftransformshift{\pgfqpoint{53.3333bp}{266.6667bp}}
    \pgftransformscale{1.5000}
    \pgftext[]{$0$}
  \end{pgfscope}
  \begin{pgfscope}
    \definecolor{fc}{rgb}{0.0000,0.0000,0.0000}
    \pgfsetfillcolor{fc}
    \pgfsetfillopacity{0.0000}
    \pgfsetlinewidth{0.7303bp}
    \definecolor{sc}{rgb}{0.0000,0.0000,0.0000}
    \pgfsetstrokecolor{sc}
    \pgfsetmiterjoin
    \pgfsetbuttcap
    \pgfpathqmoveto{46.6667bp}{260.0000bp}
    \pgfpathqlineto{46.6667bp}{273.3333bp}
    \pgfpathqlineto{33.3333bp}{273.3333bp}
    \pgfpathqlineto{33.3333bp}{260.0000bp}
    \pgfpathqlineto{46.6667bp}{260.0000bp}
    \pgfpathclose
    \pgfusepathqfillstroke
  \end{pgfscope}
  \begin{pgfscope}
    \definecolor{fc}{rgb}{0.0000,0.0000,0.0000}
    \pgfsetfillcolor{fc}
    \pgftransformshift{\pgfqpoint{40.0000bp}{266.6667bp}}
    \pgftransformscale{1.5000}
    \pgftext[]{$0$}
  \end{pgfscope}
  \begin{pgfscope}
    \definecolor{fc}{rgb}{0.0000,0.0000,0.0000}
    \pgfsetfillcolor{fc}
    \pgfsetfillopacity{0.0000}
    \pgfsetlinewidth{0.7303bp}
    \definecolor{sc}{rgb}{0.0000,0.0000,0.0000}
    \pgfsetstrokecolor{sc}
    \pgfsetmiterjoin
    \pgfsetbuttcap
    \pgfpathqmoveto{33.3333bp}{260.0000bp}
    \pgfpathqlineto{33.3333bp}{273.3333bp}
    \pgfpathqlineto{20.0000bp}{273.3333bp}
    \pgfpathqlineto{20.0000bp}{260.0000bp}
    \pgfpathqlineto{33.3333bp}{260.0000bp}
    \pgfpathclose
    \pgfusepathqfillstroke
  \end{pgfscope}
  \begin{pgfscope}
    \definecolor{fc}{rgb}{0.0000,0.0000,0.0000}
    \pgfsetfillcolor{fc}
    \pgftransformshift{\pgfqpoint{26.6667bp}{266.6667bp}}
    \pgftransformscale{1.5000}
    \pgftext[]{$0$}
  \end{pgfscope}
  \begin{pgfscope}
    \definecolor{fc}{rgb}{0.0000,0.0000,0.0000}
    \pgfsetfillcolor{fc}
    \pgfsetfillopacity{0.0000}
    \pgfpathqmoveto{13.3333bp}{260.0000bp}
    \pgfpathqlineto{13.3333bp}{273.3333bp}
    \pgfpathqlineto{-0.0000bp}{273.3333bp}
    \pgfpathqlineto{-0.0000bp}{260.0000bp}
    \pgfpathqlineto{13.3333bp}{260.0000bp}
    \pgfpathclose
    \pgfusepathqfill
  \end{pgfscope}
  \begin{pgfscope}
    \definecolor{fc}{rgb}{0.0000,0.0000,0.0000}
    \pgfsetfillcolor{fc}
    \pgftransformshift{\pgfqpoint{6.6667bp}{266.6667bp}}
    \pgftransformscale{0.8333}
    \pgftext[]{$2$}
  \end{pgfscope}
  \begin{pgfscope}
    \definecolor{fc}{rgb}{0.0000,0.0000,0.0000}
    \pgfsetfillcolor{fc}
    \pgfsetfillopacity{0.0000}
    \pgfsetlinewidth{0.7303bp}
    \definecolor{sc}{rgb}{0.0000,0.0000,0.0000}
    \pgfsetstrokecolor{sc}
    \pgfsetmiterjoin
    \pgfsetbuttcap
    \pgfpathqmoveto{100.0000bp}{280.0000bp}
    \pgfpathqlineto{100.0000bp}{293.3333bp}
    \pgfpathqlineto{86.6667bp}{293.3333bp}
    \pgfpathqlineto{86.6667bp}{280.0000bp}
    \pgfpathqlineto{100.0000bp}{280.0000bp}
    \pgfpathclose
    \pgfusepathqfillstroke
  \end{pgfscope}
  \begin{pgfscope}
    \definecolor{fc}{rgb}{0.0000,0.0000,0.0000}
    \pgfsetfillcolor{fc}
    \pgftransformshift{\pgfqpoint{93.3333bp}{286.6667bp}}
    \pgftransformscale{1.5000}
    \pgftext[]{$1$}
  \end{pgfscope}
  \begin{pgfscope}
    \definecolor{fc}{rgb}{0.0000,0.0000,0.0000}
    \pgfsetfillcolor{fc}
    \pgfsetfillopacity{0.0000}
    \pgfsetlinewidth{0.7303bp}
    \definecolor{sc}{rgb}{0.0000,0.0000,0.0000}
    \pgfsetstrokecolor{sc}
    \pgfsetmiterjoin
    \pgfsetbuttcap
    \pgfpathqmoveto{86.6667bp}{280.0000bp}
    \pgfpathqlineto{86.6667bp}{293.3333bp}
    \pgfpathqlineto{73.3333bp}{293.3333bp}
    \pgfpathqlineto{73.3333bp}{280.0000bp}
    \pgfpathqlineto{86.6667bp}{280.0000bp}
    \pgfpathclose
    \pgfusepathqfillstroke
  \end{pgfscope}
  \begin{pgfscope}
    \definecolor{fc}{rgb}{0.0000,0.0000,0.0000}
    \pgfsetfillcolor{fc}
    \pgftransformshift{\pgfqpoint{80.0000bp}{286.6667bp}}
    \pgftransformscale{1.5000}
    \pgftext[]{$0$}
  \end{pgfscope}
  \begin{pgfscope}
    \definecolor{fc}{rgb}{0.0000,0.0000,0.0000}
    \pgfsetfillcolor{fc}
    \pgfsetfillopacity{0.0000}
    \pgfsetlinewidth{0.7303bp}
    \definecolor{sc}{rgb}{0.0000,0.0000,0.0000}
    \pgfsetstrokecolor{sc}
    \pgfsetmiterjoin
    \pgfsetbuttcap
    \pgfpathqmoveto{73.3333bp}{280.0000bp}
    \pgfpathqlineto{73.3333bp}{293.3333bp}
    \pgfpathqlineto{60.0000bp}{293.3333bp}
    \pgfpathqlineto{60.0000bp}{280.0000bp}
    \pgfpathqlineto{73.3333bp}{280.0000bp}
    \pgfpathclose
    \pgfusepathqfillstroke
  \end{pgfscope}
  \begin{pgfscope}
    \definecolor{fc}{rgb}{0.0000,0.0000,0.0000}
    \pgfsetfillcolor{fc}
    \pgftransformshift{\pgfqpoint{66.6667bp}{286.6667bp}}
    \pgftransformscale{1.5000}
    \pgftext[]{$0$}
  \end{pgfscope}
  \begin{pgfscope}
    \definecolor{fc}{rgb}{0.0000,0.0000,0.0000}
    \pgfsetfillcolor{fc}
    \pgfsetfillopacity{0.0000}
    \pgfsetlinewidth{0.7303bp}
    \definecolor{sc}{rgb}{0.0000,0.0000,0.0000}
    \pgfsetstrokecolor{sc}
    \pgfsetmiterjoin
    \pgfsetbuttcap
    \pgfpathqmoveto{60.0000bp}{280.0000bp}
    \pgfpathqlineto{60.0000bp}{293.3333bp}
    \pgfpathqlineto{46.6667bp}{293.3333bp}
    \pgfpathqlineto{46.6667bp}{280.0000bp}
    \pgfpathqlineto{60.0000bp}{280.0000bp}
    \pgfpathclose
    \pgfusepathqfillstroke
  \end{pgfscope}
  \begin{pgfscope}
    \definecolor{fc}{rgb}{0.0000,0.0000,0.0000}
    \pgfsetfillcolor{fc}
    \pgftransformshift{\pgfqpoint{53.3333bp}{286.6667bp}}
    \pgftransformscale{1.5000}
    \pgftext[]{$0$}
  \end{pgfscope}
  \begin{pgfscope}
    \definecolor{fc}{rgb}{0.0000,0.0000,0.0000}
    \pgfsetfillcolor{fc}
    \pgfsetfillopacity{0.0000}
    \pgfsetlinewidth{0.7303bp}
    \definecolor{sc}{rgb}{0.0000,0.0000,0.0000}
    \pgfsetstrokecolor{sc}
    \pgfsetmiterjoin
    \pgfsetbuttcap
    \pgfpathqmoveto{46.6667bp}{280.0000bp}
    \pgfpathqlineto{46.6667bp}{293.3333bp}
    \pgfpathqlineto{33.3333bp}{293.3333bp}
    \pgfpathqlineto{33.3333bp}{280.0000bp}
    \pgfpathqlineto{46.6667bp}{280.0000bp}
    \pgfpathclose
    \pgfusepathqfillstroke
  \end{pgfscope}
  \begin{pgfscope}
    \definecolor{fc}{rgb}{0.0000,0.0000,0.0000}
    \pgfsetfillcolor{fc}
    \pgftransformshift{\pgfqpoint{40.0000bp}{286.6667bp}}
    \pgftransformscale{1.5000}
    \pgftext[]{$0$}
  \end{pgfscope}
  \begin{pgfscope}
    \definecolor{fc}{rgb}{0.0000,0.0000,0.0000}
    \pgfsetfillcolor{fc}
    \pgfsetfillopacity{0.0000}
    \pgfsetlinewidth{0.7303bp}
    \definecolor{sc}{rgb}{0.0000,0.0000,0.0000}
    \pgfsetstrokecolor{sc}
    \pgfsetmiterjoin
    \pgfsetbuttcap
    \pgfpathqmoveto{33.3333bp}{280.0000bp}
    \pgfpathqlineto{33.3333bp}{293.3333bp}
    \pgfpathqlineto{20.0000bp}{293.3333bp}
    \pgfpathqlineto{20.0000bp}{280.0000bp}
    \pgfpathqlineto{33.3333bp}{280.0000bp}
    \pgfpathclose
    \pgfusepathqfillstroke
  \end{pgfscope}
  \begin{pgfscope}
    \definecolor{fc}{rgb}{0.0000,0.0000,0.0000}
    \pgfsetfillcolor{fc}
    \pgftransformshift{\pgfqpoint{26.6667bp}{286.6667bp}}
    \pgftransformscale{1.5000}
    \pgftext[]{$0$}
  \end{pgfscope}
  \begin{pgfscope}
    \definecolor{fc}{rgb}{0.0000,0.0000,0.0000}
    \pgfsetfillcolor{fc}
    \pgfsetfillopacity{0.0000}
    \pgfpathqmoveto{13.3333bp}{280.0000bp}
    \pgfpathqlineto{13.3333bp}{293.3333bp}
    \pgfpathqlineto{-0.0000bp}{293.3333bp}
    \pgfpathqlineto{-0.0000bp}{280.0000bp}
    \pgfpathqlineto{13.3333bp}{280.0000bp}
    \pgfpathclose
    \pgfusepathqfill
  \end{pgfscope}
  \begin{pgfscope}
    \definecolor{fc}{rgb}{0.0000,0.0000,0.0000}
    \pgfsetfillcolor{fc}
    \pgftransformshift{\pgfqpoint{6.6667bp}{286.6667bp}}
    \pgftransformscale{0.8333}
    \pgftext[]{$1$}
  \end{pgfscope}
  \begin{pgfscope}
    \definecolor{fc}{rgb}{0.0000,0.0000,0.0000}
    \pgfsetfillcolor{fc}
    \pgfsetfillopacity{0.0000}
    \pgfsetlinewidth{0.7303bp}
    \definecolor{sc}{rgb}{0.0000,0.0000,0.0000}
    \pgfsetstrokecolor{sc}
    \pgfsetmiterjoin
    \pgfsetbuttcap
    \pgfpathqmoveto{100.0000bp}{300.0000bp}
    \pgfpathqlineto{100.0000bp}{313.3333bp}
    \pgfpathqlineto{86.6667bp}{313.3333bp}
    \pgfpathqlineto{86.6667bp}{300.0000bp}
    \pgfpathqlineto{100.0000bp}{300.0000bp}
    \pgfpathclose
    \pgfusepathqfillstroke
  \end{pgfscope}
  \begin{pgfscope}
    \definecolor{fc}{rgb}{0.0000,0.0000,0.0000}
    \pgfsetfillcolor{fc}
    \pgftransformshift{\pgfqpoint{93.3333bp}{306.6667bp}}
    \pgftransformscale{1.5000}
    \pgftext[]{$0$}
  \end{pgfscope}
  \begin{pgfscope}
    \definecolor{fc}{rgb}{0.0000,0.0000,0.0000}
    \pgfsetfillcolor{fc}
    \pgfsetfillopacity{0.0000}
    \pgfsetlinewidth{0.7303bp}
    \definecolor{sc}{rgb}{0.0000,0.0000,0.0000}
    \pgfsetstrokecolor{sc}
    \pgfsetmiterjoin
    \pgfsetbuttcap
    \pgfpathqmoveto{86.6667bp}{300.0000bp}
    \pgfpathqlineto{86.6667bp}{313.3333bp}
    \pgfpathqlineto{73.3333bp}{313.3333bp}
    \pgfpathqlineto{73.3333bp}{300.0000bp}
    \pgfpathqlineto{86.6667bp}{300.0000bp}
    \pgfpathclose
    \pgfusepathqfillstroke
  \end{pgfscope}
  \begin{pgfscope}
    \definecolor{fc}{rgb}{0.0000,0.0000,0.0000}
    \pgfsetfillcolor{fc}
    \pgftransformshift{\pgfqpoint{80.0000bp}{306.6667bp}}
    \pgftransformscale{1.5000}
    \pgftext[]{$0$}
  \end{pgfscope}
  \begin{pgfscope}
    \definecolor{fc}{rgb}{0.0000,0.0000,0.0000}
    \pgfsetfillcolor{fc}
    \pgfsetfillopacity{0.0000}
    \pgfsetlinewidth{0.7303bp}
    \definecolor{sc}{rgb}{0.0000,0.0000,0.0000}
    \pgfsetstrokecolor{sc}
    \pgfsetmiterjoin
    \pgfsetbuttcap
    \pgfpathqmoveto{73.3333bp}{300.0000bp}
    \pgfpathqlineto{73.3333bp}{313.3333bp}
    \pgfpathqlineto{60.0000bp}{313.3333bp}
    \pgfpathqlineto{60.0000bp}{300.0000bp}
    \pgfpathqlineto{73.3333bp}{300.0000bp}
    \pgfpathclose
    \pgfusepathqfillstroke
  \end{pgfscope}
  \begin{pgfscope}
    \definecolor{fc}{rgb}{0.0000,0.0000,0.0000}
    \pgfsetfillcolor{fc}
    \pgftransformshift{\pgfqpoint{66.6667bp}{306.6667bp}}
    \pgftransformscale{1.5000}
    \pgftext[]{$0$}
  \end{pgfscope}
  \begin{pgfscope}
    \definecolor{fc}{rgb}{0.0000,0.0000,0.0000}
    \pgfsetfillcolor{fc}
    \pgfsetfillopacity{0.0000}
    \pgfsetlinewidth{0.7303bp}
    \definecolor{sc}{rgb}{0.0000,0.0000,0.0000}
    \pgfsetstrokecolor{sc}
    \pgfsetmiterjoin
    \pgfsetbuttcap
    \pgfpathqmoveto{60.0000bp}{300.0000bp}
    \pgfpathqlineto{60.0000bp}{313.3333bp}
    \pgfpathqlineto{46.6667bp}{313.3333bp}
    \pgfpathqlineto{46.6667bp}{300.0000bp}
    \pgfpathqlineto{60.0000bp}{300.0000bp}
    \pgfpathclose
    \pgfusepathqfillstroke
  \end{pgfscope}
  \begin{pgfscope}
    \definecolor{fc}{rgb}{0.0000,0.0000,0.0000}
    \pgfsetfillcolor{fc}
    \pgftransformshift{\pgfqpoint{53.3333bp}{306.6667bp}}
    \pgftransformscale{1.5000}
    \pgftext[]{$0$}
  \end{pgfscope}
  \begin{pgfscope}
    \definecolor{fc}{rgb}{0.0000,0.0000,0.0000}
    \pgfsetfillcolor{fc}
    \pgfsetfillopacity{0.0000}
    \pgfsetlinewidth{0.7303bp}
    \definecolor{sc}{rgb}{0.0000,0.0000,0.0000}
    \pgfsetstrokecolor{sc}
    \pgfsetmiterjoin
    \pgfsetbuttcap
    \pgfpathqmoveto{46.6667bp}{300.0000bp}
    \pgfpathqlineto{46.6667bp}{313.3333bp}
    \pgfpathqlineto{33.3333bp}{313.3333bp}
    \pgfpathqlineto{33.3333bp}{300.0000bp}
    \pgfpathqlineto{46.6667bp}{300.0000bp}
    \pgfpathclose
    \pgfusepathqfillstroke
  \end{pgfscope}
  \begin{pgfscope}
    \definecolor{fc}{rgb}{0.0000,0.0000,0.0000}
    \pgfsetfillcolor{fc}
    \pgftransformshift{\pgfqpoint{40.0000bp}{306.6667bp}}
    \pgftransformscale{1.5000}
    \pgftext[]{$0$}
  \end{pgfscope}
  \begin{pgfscope}
    \definecolor{fc}{rgb}{0.0000,0.0000,0.0000}
    \pgfsetfillcolor{fc}
    \pgfsetfillopacity{0.0000}
    \pgfsetlinewidth{0.7303bp}
    \definecolor{sc}{rgb}{0.0000,0.0000,0.0000}
    \pgfsetstrokecolor{sc}
    \pgfsetmiterjoin
    \pgfsetbuttcap
    \pgfpathqmoveto{33.3333bp}{300.0000bp}
    \pgfpathqlineto{33.3333bp}{313.3333bp}
    \pgfpathqlineto{20.0000bp}{313.3333bp}
    \pgfpathqlineto{20.0000bp}{300.0000bp}
    \pgfpathqlineto{33.3333bp}{300.0000bp}
    \pgfpathclose
    \pgfusepathqfillstroke
  \end{pgfscope}
  \begin{pgfscope}
    \definecolor{fc}{rgb}{0.0000,0.0000,0.0000}
    \pgfsetfillcolor{fc}
    \pgftransformshift{\pgfqpoint{26.6667bp}{306.6667bp}}
    \pgftransformscale{1.5000}
    \pgftext[]{$0$}
  \end{pgfscope}
  \begin{pgfscope}
    \definecolor{fc}{rgb}{0.0000,0.0000,0.0000}
    \pgfsetfillcolor{fc}
    \pgfsetfillopacity{0.0000}
    \pgfpathqmoveto{13.3333bp}{300.0000bp}
    \pgfpathqlineto{13.3333bp}{313.3333bp}
    \pgfpathqlineto{-0.0000bp}{313.3333bp}
    \pgfpathqlineto{-0.0000bp}{300.0000bp}
    \pgfpathqlineto{13.3333bp}{300.0000bp}
    \pgfpathclose
    \pgfusepathqfill
  \end{pgfscope}
  \begin{pgfscope}
    \definecolor{fc}{rgb}{0.0000,0.0000,0.0000}
    \pgfsetfillcolor{fc}
    \pgftransformshift{\pgfqpoint{6.6667bp}{306.6667bp}}
    \pgftransformscale{0.8333}
    \pgftext[]{$0$}
  \end{pgfscope}
  \begin{pgfscope}
    \definecolor{fc}{rgb}{0.0000,0.0000,0.0000}
    \pgfsetfillcolor{fc}
    \pgfsetfillopacity{0.0000}
    \pgfpathqmoveto{100.0000bp}{320.0000bp}
    \pgfpathqlineto{100.0000bp}{333.3333bp}
    \pgfpathqlineto{86.6667bp}{333.3333bp}
    \pgfpathqlineto{86.6667bp}{320.0000bp}
    \pgfpathqlineto{100.0000bp}{320.0000bp}
    \pgfpathclose
    \pgfusepathqfill
  \end{pgfscope}
  \begin{pgfscope}
    \definecolor{fc}{rgb}{0.0000,0.0000,0.0000}
    \pgfsetfillcolor{fc}
    \pgftransformshift{\pgfqpoint{93.3333bp}{326.6667bp}}
    \pgftransformscale{0.8333}
    \pgftext[]{\texttt{0}}
  \end{pgfscope}
  \begin{pgfscope}
    \definecolor{fc}{rgb}{0.0000,0.0000,0.0000}
    \pgfsetfillcolor{fc}
    \pgfsetfillopacity{0.0000}
    \pgfpathqmoveto{86.6667bp}{320.0000bp}
    \pgfpathqlineto{86.6667bp}{333.3333bp}
    \pgfpathqlineto{73.3333bp}{333.3333bp}
    \pgfpathqlineto{73.3333bp}{320.0000bp}
    \pgfpathqlineto{86.6667bp}{320.0000bp}
    \pgfpathclose
    \pgfusepathqfill
  \end{pgfscope}
  \begin{pgfscope}
    \definecolor{fc}{rgb}{0.0000,0.0000,0.0000}
    \pgfsetfillcolor{fc}
    \pgftransformshift{\pgfqpoint{80.0000bp}{326.6667bp}}
    \pgftransformscale{0.8333}
    \pgftext[]{\texttt{1}}
  \end{pgfscope}
  \begin{pgfscope}
    \definecolor{fc}{rgb}{0.0000,0.0000,0.0000}
    \pgfsetfillcolor{fc}
    \pgfsetfillopacity{0.0000}
    \pgfpathqmoveto{73.3333bp}{320.0000bp}
    \pgfpathqlineto{73.3333bp}{333.3333bp}
    \pgfpathqlineto{60.0000bp}{333.3333bp}
    \pgfpathqlineto{60.0000bp}{320.0000bp}
    \pgfpathqlineto{73.3333bp}{320.0000bp}
    \pgfpathclose
    \pgfusepathqfill
  \end{pgfscope}
  \begin{pgfscope}
    \definecolor{fc}{rgb}{0.0000,0.0000,0.0000}
    \pgfsetfillcolor{fc}
    \pgftransformshift{\pgfqpoint{66.6667bp}{326.6667bp}}
    \pgftransformscale{0.8333}
    \pgftext[]{\texttt{2}}
  \end{pgfscope}
  \begin{pgfscope}
    \definecolor{fc}{rgb}{0.0000,0.0000,0.0000}
    \pgfsetfillcolor{fc}
    \pgfsetfillopacity{0.0000}
    \pgfpathqmoveto{60.0000bp}{320.0000bp}
    \pgfpathqlineto{60.0000bp}{333.3333bp}
    \pgfpathqlineto{46.6667bp}{333.3333bp}
    \pgfpathqlineto{46.6667bp}{320.0000bp}
    \pgfpathqlineto{60.0000bp}{320.0000bp}
    \pgfpathclose
    \pgfusepathqfill
  \end{pgfscope}
  \begin{pgfscope}
    \definecolor{fc}{rgb}{0.0000,0.0000,0.0000}
    \pgfsetfillcolor{fc}
    \pgftransformshift{\pgfqpoint{53.3333bp}{326.6667bp}}
    \pgftransformscale{0.8333}
    \pgftext[]{\texttt{3}}
  \end{pgfscope}
  \begin{pgfscope}
    \definecolor{fc}{rgb}{0.0000,0.0000,0.0000}
    \pgfsetfillcolor{fc}
    \pgfsetfillopacity{0.0000}
    \pgfpathqmoveto{46.6667bp}{320.0000bp}
    \pgfpathqlineto{46.6667bp}{333.3333bp}
    \pgfpathqlineto{33.3333bp}{333.3333bp}
    \pgfpathqlineto{33.3333bp}{320.0000bp}
    \pgfpathqlineto{46.6667bp}{320.0000bp}
    \pgfpathclose
    \pgfusepathqfill
  \end{pgfscope}
  \begin{pgfscope}
    \definecolor{fc}{rgb}{0.0000,0.0000,0.0000}
    \pgfsetfillcolor{fc}
    \pgftransformshift{\pgfqpoint{40.0000bp}{326.6667bp}}
    \pgftransformscale{0.8333}
    \pgftext[]{\texttt{4}}
  \end{pgfscope}
  \begin{pgfscope}
    \definecolor{fc}{rgb}{0.0000,0.0000,0.0000}
    \pgfsetfillcolor{fc}
    \pgfsetfillopacity{0.0000}
    \pgfpathqmoveto{33.3333bp}{320.0000bp}
    \pgfpathqlineto{33.3333bp}{333.3333bp}
    \pgfpathqlineto{20.0000bp}{333.3333bp}
    \pgfpathqlineto{20.0000bp}{320.0000bp}
    \pgfpathqlineto{33.3333bp}{320.0000bp}
    \pgfpathclose
    \pgfusepathqfill
  \end{pgfscope}
  \begin{pgfscope}
    \definecolor{fc}{rgb}{0.0000,0.0000,0.0000}
    \pgfsetfillcolor{fc}
    \pgftransformshift{\pgfqpoint{26.6667bp}{326.6667bp}}
    \pgftransformscale{0.8333}
    \pgftext[]{\texttt{5}}
  \end{pgfscope}
\end{pgfpicture}

  \end{center}
\end{model*}

\pref{model:bin-inc} shows a binary counter, stored as an array of
bits with the $2^i$ place stored at index $i$,\marginnote{Note that
  the array is drawn with index $0$ at the right side instead of the
  left!} undergoing a sequence of increment operations. The indices
are shown at the top, and the number represented by each state of the
binary counter is shown at the left.

\begin{questions}
  \item How many bits differ between the counter in state $0$ and
    state $1$?
  \item How many bits differ between states $1$ and $2$?  Between $2$
    and $3$?  Between $3$ and $4$?
  \item Next to each counter state in the model, write the number of
    bits that changed from the previous state.  Circle the bits that
    changed.
  \item Now, highlight the bits that changed \emph{from zero to one}.
  \item What patterns do you notice?
  \item How many bits are there that change from zero to one each
    time?
  \item How do the bits that change from zero to one relate to the
    bits that change from one to zero?
  \item Write pseudocode to perform an increment operation, given an
    array of bits $b$ as an input.\marginnote{You do not need to worry
      about overflowing the array.} \vspace{1in}
  \item \label{q:inc-best} If we assume that changing the value of a
    bit takes 1 time step, what is the best-case runtime of your
    algorithm when given a counter representing some number $n$?
    Express your answer using big-$\Theta$ notation.
  \item Give an example of a best-case input for your algorithm.
  \item \label{q:inc-worst} What is the worst-case runtime of your
    algorithm when given a counter representing some number $n$?
    Express your answer in terms of $n$, using big-$\Theta$
    notation. (Careful!  $n$ is the \emph{number represented by} the
    bits, not the \emph{number of bits}.)
  \item Give an example of a worst-case input for your algorithm.
  \item Based on your answer to \pref{q:inc-best}, what is the best
    total running time we could possibly hope for a sequence of $n$
    increment operations?
  \item \label{q:worst-total} Based on your answer to
    \pref{q:inc-worst}, what is the worst possible total running time
    for a sequence of $n$ increment operations?
  \end{questions}

\pause

\begin{model*}{Total cost of repeated increments}{bin-inc-total}
  \centering
  % \tabcolsep=0.1cm
  \begin{tabular}{c|ccccccccccccccccc}
    $n$ & $0$ & $1$ & $2$ & $3$ & $4$ & $5$ & $6$ & $7$ & $8$ & $9$ & $10$
    & $11$ & $12$ & $13$ & $14$ & $15$ & $16$ \\[8pt]
    cost of $(n-1) \to n$ &  & $1$ & $2$ & $1$ & $3$ & &  &  &  &  &
    &  &  &  &  &  &  \\[8pt]
    cumulative cost & $0$ & $1$ & $3$ & $4$ & $7$ &  &  &  &
    &  &  &  &  &  &  &  &
  \end{tabular}
\end{model*}

\begin{questions}
\item Start by filling in the missing values in the table above. Each
  value in the second row counts the number of bit flips needed to
  increment a binary counter from $(n-1)$ to $n$, and each value in
  the third row is the sum of all the values in the second row so far.
  \item How many bit flips are needed, in total, to start at $0$ and
    repeatedly increment a binary counter until reaching $16$?
  \item Look at the third row and compare it to the first row.  What
    patterns do you notice? \marginnote{\emph{Hint}: look at powers of
      two.  There's no one right answer to this question.}
  \item Make a conjecture: how many total bit flips will be needed to
    increment from $0$ to $32$?
  \item In general, how many bit flips do you think will be needed to
    increment up to $2^k$?
  \item \label{q:upper-bound} Generalize your conjecture to give an
    \emph{upper bound} on the total number of bit flips needed to
    increment from $0$ to any $n$ (not necessarily a power of $2$).
    That is, can you say anything about how big the entries in the
    third row can get, relative to $n$?
  \item Based on your conjecture, if we repeatedly increment a binary
    counter from $0$ up to $n$, how long does each increment take
    \emph{on average}?  Express your answer using big-$O$ notation.
  \item Why is this an interesting result?\marginnote{\emph{Hint}:
      look at your answers to Questions~\ref{q:inc-worst}--\ref{q:worst-total}.}
\end{questions}

% \pause

% \newcommand{\one}{\underset{\$}{1}}
% \newcommand{\pay}{\overset{\$\$}{\longrightarrow}}

% \begin{model*}{The accounting method}{accounting}
%   \[ 0000 \pay 000\one \pay 00\one 0 \pay 00\one\one \pay 0 \one 00 \pay
%      0\one 0\one \pay 0\one \one 0 \pay 0 \one\one\one \pay \one 000
%   \]
% \end{model*}

% Now, let's actually prove your conjecture from \pref{q:upper-bound}.
% Imagine that we are providing a binary counter
% service.\marginnote{Perhaps it should be called \url{countr.com}\dots
%   drat, that domain is already taken.}  Anyone can ask us to create a
% binary counter for them, which starts out with the value zero. They
% can then ask us to increment their counter whenever they want (they
% can also ask us to tell them the current value).  Every time we flip a
% bit, it costs us \$1.  How much should we charge our customers for an
% increment operation to make sure that we can at least break even?

% \begin{questions}
%   \item Explain why charging \$1 for an increment operation is not
%     enough.  That is, if we charge only \$1 for every increment
%     operation, it will not be enough to cover our costs and we will
%     eventually go bankrupt.
% \end{questions}

% The model shows what happens if we charge \$2 per increment
% operation. The two \$\$ signs over each arrow represent the \$2 paid
% by a customer every time they want to increment their counter.  In
% between the arrows are various states of the counter \emph{along with
%   extra money we have saved}.

% \begin{questions}
%   \item Consider the first arrow in the diagram, \[ 0000 \pay
%       000\one. \] Explain what is happening with the money.  Why is
%     there \$1 left over? Why do you think we save it under the $1$?

%   \item Now explain the second step, \[ 000\one \pay 00\one 0. \] Why
%     is there \$1 left over again?  What do you think we did with the
%     \$1 that was stored under the rightmost $1$?  What did we do with
%     the \$2 paid by the customer?

%   \item Now explain \[ 00\one 0 \pay 00\one\one \pay 0\one 00. \]

%   \item Can you explain why charging \$2 per increment operation will
%     ensure that we always have enough money to cover our costs?

%   \item This actually constitutes a proof of your conjecture from
%     \pref{q:upper-bound}.  Explain why.
% \end{questions}

\end{document}
