% -*- compile-command: "rubber -d divide-and-conquer-intro.tex" -*-
\documentclass{tufte-handout}

\usepackage{algo-activity}
\usepackage{../acode}

\title{\thecourse: Introduction to Divide \& Conquer}
\date{}

%%% NOTE: In Fall 2017 this had another part but we didn't get to it;
%%% I just spent another class doing the second part.  So I have now
%%% moved that part into a second activity, #10: D&C arithmetic.

\begin{document}

\maketitle

\begin{model*}{Merge sort}{mergesort}
  \begin{acode}
\> mergesort($xs$) = \\
\> \tb \> !if $\mathit{len}(xs) \leq 1$ !then !return $xs$ \\
\>     \> split $xs$ into halves $(xs_1, xs_2)$ \\
\>     \> $xs_1' \gets$ mergesort($xs_1$) \\
\>     \> $xs_2' \gets$ mergesort($xs_2$) \\
\>     \> $xs' \gets$ merge($xs_1'$, $xs_2'$) \\
\>     \> !return $xs'$
  \end{acode}

  \begin{align*}
    T(1) &= \Theta(1) \\
    T(n) &= 2 T(n/2) + \Theta(n)
  \end{align*}

  \begin{center}
    \begin{pgfpicture}
  \pgfpathrectangle{\pgfpointorigin}{\pgfqpoint{300.0000bp}{138.0000bp}}
  \pgfusepath{use as bounding box}
  \begin{pgfscope}
    \definecolor{fc}{rgb}{0.0000,0.0000,0.0000}
    \pgfsetfillcolor{fc}
    \pgfsetlinewidth{0.8139bp}
    \definecolor{sc}{rgb}{0.0000,0.0000,0.0000}
    \pgfsetstrokecolor{sc}
    \pgfsetmiterjoin
    \pgfsetbuttcap
    \pgfpathqmoveto{152.0000bp}{2.0000bp}
    \pgfpathqcurveto{152.0000bp}{3.1046bp}{151.1046bp}{4.0000bp}{150.0000bp}{4.0000bp}
    \pgfpathqcurveto{148.8954bp}{4.0000bp}{148.0000bp}{3.1046bp}{148.0000bp}{2.0000bp}
    \pgfpathqcurveto{148.0000bp}{0.8954bp}{148.8954bp}{0.0000bp}{150.0000bp}{0.0000bp}
    \pgfpathqcurveto{151.1046bp}{0.0000bp}{152.0000bp}{0.8954bp}{152.0000bp}{2.0000bp}
    \pgfpathclose
    \pgfusepathqfillstroke
  \end{pgfscope}
  \begin{pgfscope}
    \definecolor{fc}{rgb}{0.0000,0.0000,0.0000}
    \pgfsetfillcolor{fc}
    \pgfsetlinewidth{0.8139bp}
    \definecolor{sc}{rgb}{0.0000,0.0000,0.0000}
    \pgfsetstrokecolor{sc}
    \pgfsetmiterjoin
    \pgfsetbuttcap
    \pgfpathqmoveto{152.0000bp}{9.0000bp}
    \pgfpathqcurveto{152.0000bp}{10.1046bp}{151.1046bp}{11.0000bp}{150.0000bp}{11.0000bp}
    \pgfpathqcurveto{148.8954bp}{11.0000bp}{148.0000bp}{10.1046bp}{148.0000bp}{9.0000bp}
    \pgfpathqcurveto{148.0000bp}{7.8954bp}{148.8954bp}{7.0000bp}{150.0000bp}{7.0000bp}
    \pgfpathqcurveto{151.1046bp}{7.0000bp}{152.0000bp}{7.8954bp}{152.0000bp}{9.0000bp}
    \pgfpathclose
    \pgfusepathqfillstroke
  \end{pgfscope}
  \begin{pgfscope}
    \definecolor{fc}{rgb}{0.0000,0.0000,0.0000}
    \pgfsetfillcolor{fc}
    \pgfsetlinewidth{0.8139bp}
    \definecolor{sc}{rgb}{0.0000,0.0000,0.0000}
    \pgfsetstrokecolor{sc}
    \pgfsetmiterjoin
    \pgfsetbuttcap
    \pgfpathqmoveto{152.0000bp}{16.0000bp}
    \pgfpathqcurveto{152.0000bp}{17.1046bp}{151.1046bp}{18.0000bp}{150.0000bp}{18.0000bp}
    \pgfpathqcurveto{148.8954bp}{18.0000bp}{148.0000bp}{17.1046bp}{148.0000bp}{16.0000bp}
    \pgfpathqcurveto{148.0000bp}{14.8954bp}{148.8954bp}{14.0000bp}{150.0000bp}{14.0000bp}
    \pgfpathqcurveto{151.1046bp}{14.0000bp}{152.0000bp}{14.8954bp}{152.0000bp}{16.0000bp}
    \pgfpathclose
    \pgfusepathqfillstroke
  \end{pgfscope}
  \begin{pgfscope}
    \pgfsetlinewidth{0.8139bp}
    \definecolor{sc}{rgb}{0.0000,0.0000,0.0000}
    \pgfsetstrokecolor{sc}
    \pgfsetmiterjoin
    \pgfsetbuttcap
    \pgfpathqmoveto{150.0000bp}{128.0000bp}
    \pgfpathqlineto{230.0000bp}{98.0000bp}
    \pgfusepathqstroke
  \end{pgfscope}
  \begin{pgfscope}
    \pgfsetlinewidth{0.8139bp}
    \definecolor{sc}{rgb}{0.0000,0.0000,0.0000}
    \pgfsetstrokecolor{sc}
    \pgfsetmiterjoin
    \pgfsetbuttcap
    \pgfpathqmoveto{150.0000bp}{128.0000bp}
    \pgfpathqlineto{70.0000bp}{98.0000bp}
    \pgfusepathqstroke
  \end{pgfscope}
  \begin{pgfscope}
    \pgfsetlinewidth{0.8139bp}
    \definecolor{sc}{rgb}{0.0000,0.0000,0.0000}
    \pgfsetstrokecolor{sc}
    \pgfsetmiterjoin
    \pgfsetbuttcap
    \pgfpathqmoveto{230.0000bp}{98.0000bp}
    \pgfpathqlineto{270.0000bp}{68.0000bp}
    \pgfusepathqstroke
  \end{pgfscope}
  \begin{pgfscope}
    \pgfsetlinewidth{0.8139bp}
    \definecolor{sc}{rgb}{0.0000,0.0000,0.0000}
    \pgfsetstrokecolor{sc}
    \pgfsetmiterjoin
    \pgfsetbuttcap
    \pgfpathqmoveto{230.0000bp}{98.0000bp}
    \pgfpathqlineto{190.0000bp}{68.0000bp}
    \pgfusepathqstroke
  \end{pgfscope}
  \begin{pgfscope}
    \pgfsetlinewidth{0.8139bp}
    \definecolor{sc}{rgb}{0.0000,0.0000,0.0000}
    \pgfsetstrokecolor{sc}
    \pgfsetmiterjoin
    \pgfsetbuttcap
    \pgfpathqmoveto{270.0000bp}{68.0000bp}
    \pgfpathqlineto{290.0000bp}{38.0000bp}
    \pgfusepathqstroke
  \end{pgfscope}
  \begin{pgfscope}
    \pgfsetlinewidth{0.8139bp}
    \definecolor{sc}{rgb}{0.0000,0.0000,0.0000}
    \pgfsetstrokecolor{sc}
    \pgfsetmiterjoin
    \pgfsetbuttcap
    \pgfpathqmoveto{270.0000bp}{68.0000bp}
    \pgfpathqlineto{250.0000bp}{38.0000bp}
    \pgfusepathqstroke
  \end{pgfscope}
  \begin{pgfscope}
    \definecolor{fc}{rgb}{1.0000,1.0000,1.0000}
    \pgfsetfillcolor{fc}
    \pgfpathqmoveto{300.0000bp}{28.0000bp}
    \pgfpathqlineto{300.0000bp}{48.0000bp}
    \pgfpathqlineto{280.0000bp}{48.0000bp}
    \pgfpathqlineto{280.0000bp}{28.0000bp}
    \pgfpathqlineto{300.0000bp}{28.0000bp}
    \pgfpathclose
    \pgfusepathqfill
  \end{pgfscope}
  \begin{pgfscope}
    \definecolor{fc}{rgb}{0.0000,0.0000,0.0000}
    \pgfsetfillcolor{fc}
    \pgftransformshift{\pgfqpoint{290.0000bp}{38.0000bp}}
    \pgftransformscale{1.2500}
    \pgftext[]{$n/8$}
  \end{pgfscope}
  \begin{pgfscope}
    \definecolor{fc}{rgb}{1.0000,1.0000,1.0000}
    \pgfsetfillcolor{fc}
    \pgfpathqmoveto{260.0000bp}{28.0000bp}
    \pgfpathqlineto{260.0000bp}{48.0000bp}
    \pgfpathqlineto{240.0000bp}{48.0000bp}
    \pgfpathqlineto{240.0000bp}{28.0000bp}
    \pgfpathqlineto{260.0000bp}{28.0000bp}
    \pgfpathclose
    \pgfusepathqfill
  \end{pgfscope}
  \begin{pgfscope}
    \definecolor{fc}{rgb}{0.0000,0.0000,0.0000}
    \pgfsetfillcolor{fc}
    \pgftransformshift{\pgfqpoint{250.0000bp}{38.0000bp}}
    \pgftransformscale{1.2500}
    \pgftext[]{$n/8$}
  \end{pgfscope}
  \begin{pgfscope}
    \definecolor{fc}{rgb}{1.0000,1.0000,1.0000}
    \pgfsetfillcolor{fc}
    \pgfpathqmoveto{280.0000bp}{58.0000bp}
    \pgfpathqlineto{280.0000bp}{78.0000bp}
    \pgfpathqlineto{260.0000bp}{78.0000bp}
    \pgfpathqlineto{260.0000bp}{58.0000bp}
    \pgfpathqlineto{280.0000bp}{58.0000bp}
    \pgfpathclose
    \pgfusepathqfill
  \end{pgfscope}
  \begin{pgfscope}
    \definecolor{fc}{rgb}{0.0000,0.0000,0.0000}
    \pgfsetfillcolor{fc}
    \pgftransformshift{\pgfqpoint{270.0000bp}{68.0000bp}}
    \pgftransformscale{1.2500}
    \pgftext[]{$n/4$}
  \end{pgfscope}
  \begin{pgfscope}
    \pgfsetlinewidth{0.8139bp}
    \definecolor{sc}{rgb}{0.0000,0.0000,0.0000}
    \pgfsetstrokecolor{sc}
    \pgfsetmiterjoin
    \pgfsetbuttcap
    \pgfpathqmoveto{190.0000bp}{68.0000bp}
    \pgfpathqlineto{210.0000bp}{38.0000bp}
    \pgfusepathqstroke
  \end{pgfscope}
  \begin{pgfscope}
    \pgfsetlinewidth{0.8139bp}
    \definecolor{sc}{rgb}{0.0000,0.0000,0.0000}
    \pgfsetstrokecolor{sc}
    \pgfsetmiterjoin
    \pgfsetbuttcap
    \pgfpathqmoveto{190.0000bp}{68.0000bp}
    \pgfpathqlineto{170.0000bp}{38.0000bp}
    \pgfusepathqstroke
  \end{pgfscope}
  \begin{pgfscope}
    \definecolor{fc}{rgb}{1.0000,1.0000,1.0000}
    \pgfsetfillcolor{fc}
    \pgfpathqmoveto{220.0000bp}{28.0000bp}
    \pgfpathqlineto{220.0000bp}{48.0000bp}
    \pgfpathqlineto{200.0000bp}{48.0000bp}
    \pgfpathqlineto{200.0000bp}{28.0000bp}
    \pgfpathqlineto{220.0000bp}{28.0000bp}
    \pgfpathclose
    \pgfusepathqfill
  \end{pgfscope}
  \begin{pgfscope}
    \definecolor{fc}{rgb}{0.0000,0.0000,0.0000}
    \pgfsetfillcolor{fc}
    \pgftransformshift{\pgfqpoint{210.0000bp}{38.0000bp}}
    \pgftransformscale{1.2500}
    \pgftext[]{$n/8$}
  \end{pgfscope}
  \begin{pgfscope}
    \definecolor{fc}{rgb}{1.0000,1.0000,1.0000}
    \pgfsetfillcolor{fc}
    \pgfpathqmoveto{180.0000bp}{28.0000bp}
    \pgfpathqlineto{180.0000bp}{48.0000bp}
    \pgfpathqlineto{160.0000bp}{48.0000bp}
    \pgfpathqlineto{160.0000bp}{28.0000bp}
    \pgfpathqlineto{180.0000bp}{28.0000bp}
    \pgfpathclose
    \pgfusepathqfill
  \end{pgfscope}
  \begin{pgfscope}
    \definecolor{fc}{rgb}{0.0000,0.0000,0.0000}
    \pgfsetfillcolor{fc}
    \pgftransformshift{\pgfqpoint{170.0000bp}{38.0000bp}}
    \pgftransformscale{1.2500}
    \pgftext[]{$n/8$}
  \end{pgfscope}
  \begin{pgfscope}
    \definecolor{fc}{rgb}{1.0000,1.0000,1.0000}
    \pgfsetfillcolor{fc}
    \pgfpathqmoveto{200.0000bp}{58.0000bp}
    \pgfpathqlineto{200.0000bp}{78.0000bp}
    \pgfpathqlineto{180.0000bp}{78.0000bp}
    \pgfpathqlineto{180.0000bp}{58.0000bp}
    \pgfpathqlineto{200.0000bp}{58.0000bp}
    \pgfpathclose
    \pgfusepathqfill
  \end{pgfscope}
  \begin{pgfscope}
    \definecolor{fc}{rgb}{0.0000,0.0000,0.0000}
    \pgfsetfillcolor{fc}
    \pgftransformshift{\pgfqpoint{190.0000bp}{68.0000bp}}
    \pgftransformscale{1.2500}
    \pgftext[]{$n/4$}
  \end{pgfscope}
  \begin{pgfscope}
    \definecolor{fc}{rgb}{1.0000,1.0000,1.0000}
    \pgfsetfillcolor{fc}
    \pgfpathqmoveto{240.0000bp}{88.0000bp}
    \pgfpathqlineto{240.0000bp}{108.0000bp}
    \pgfpathqlineto{220.0000bp}{108.0000bp}
    \pgfpathqlineto{220.0000bp}{88.0000bp}
    \pgfpathqlineto{240.0000bp}{88.0000bp}
    \pgfpathclose
    \pgfusepathqfill
  \end{pgfscope}
  \begin{pgfscope}
    \definecolor{fc}{rgb}{0.0000,0.0000,0.0000}
    \pgfsetfillcolor{fc}
    \pgftransformshift{\pgfqpoint{230.0000bp}{98.0000bp}}
    \pgftransformscale{1.2500}
    \pgftext[]{$n/2$}
  \end{pgfscope}
  \begin{pgfscope}
    \pgfsetlinewidth{0.8139bp}
    \definecolor{sc}{rgb}{0.0000,0.0000,0.0000}
    \pgfsetstrokecolor{sc}
    \pgfsetmiterjoin
    \pgfsetbuttcap
    \pgfpathqmoveto{70.0000bp}{98.0000bp}
    \pgfpathqlineto{110.0000bp}{68.0000bp}
    \pgfusepathqstroke
  \end{pgfscope}
  \begin{pgfscope}
    \pgfsetlinewidth{0.8139bp}
    \definecolor{sc}{rgb}{0.0000,0.0000,0.0000}
    \pgfsetstrokecolor{sc}
    \pgfsetmiterjoin
    \pgfsetbuttcap
    \pgfpathqmoveto{70.0000bp}{98.0000bp}
    \pgfpathqlineto{30.0000bp}{68.0000bp}
    \pgfusepathqstroke
  \end{pgfscope}
  \begin{pgfscope}
    \pgfsetlinewidth{0.8139bp}
    \definecolor{sc}{rgb}{0.0000,0.0000,0.0000}
    \pgfsetstrokecolor{sc}
    \pgfsetmiterjoin
    \pgfsetbuttcap
    \pgfpathqmoveto{110.0000bp}{68.0000bp}
    \pgfpathqlineto{130.0000bp}{38.0000bp}
    \pgfusepathqstroke
  \end{pgfscope}
  \begin{pgfscope}
    \pgfsetlinewidth{0.8139bp}
    \definecolor{sc}{rgb}{0.0000,0.0000,0.0000}
    \pgfsetstrokecolor{sc}
    \pgfsetmiterjoin
    \pgfsetbuttcap
    \pgfpathqmoveto{110.0000bp}{68.0000bp}
    \pgfpathqlineto{90.0000bp}{38.0000bp}
    \pgfusepathqstroke
  \end{pgfscope}
  \begin{pgfscope}
    \definecolor{fc}{rgb}{1.0000,1.0000,1.0000}
    \pgfsetfillcolor{fc}
    \pgfpathqmoveto{140.0000bp}{28.0000bp}
    \pgfpathqlineto{140.0000bp}{48.0000bp}
    \pgfpathqlineto{120.0000bp}{48.0000bp}
    \pgfpathqlineto{120.0000bp}{28.0000bp}
    \pgfpathqlineto{140.0000bp}{28.0000bp}
    \pgfpathclose
    \pgfusepathqfill
  \end{pgfscope}
  \begin{pgfscope}
    \definecolor{fc}{rgb}{0.0000,0.0000,0.0000}
    \pgfsetfillcolor{fc}
    \pgftransformshift{\pgfqpoint{130.0000bp}{38.0000bp}}
    \pgftransformscale{1.2500}
    \pgftext[]{$n/8$}
  \end{pgfscope}
  \begin{pgfscope}
    \definecolor{fc}{rgb}{1.0000,1.0000,1.0000}
    \pgfsetfillcolor{fc}
    \pgfpathqmoveto{100.0000bp}{28.0000bp}
    \pgfpathqlineto{100.0000bp}{48.0000bp}
    \pgfpathqlineto{80.0000bp}{48.0000bp}
    \pgfpathqlineto{80.0000bp}{28.0000bp}
    \pgfpathqlineto{100.0000bp}{28.0000bp}
    \pgfpathclose
    \pgfusepathqfill
  \end{pgfscope}
  \begin{pgfscope}
    \definecolor{fc}{rgb}{0.0000,0.0000,0.0000}
    \pgfsetfillcolor{fc}
    \pgftransformshift{\pgfqpoint{90.0000bp}{38.0000bp}}
    \pgftransformscale{1.2500}
    \pgftext[]{$n/8$}
  \end{pgfscope}
  \begin{pgfscope}
    \definecolor{fc}{rgb}{1.0000,1.0000,1.0000}
    \pgfsetfillcolor{fc}
    \pgfpathqmoveto{120.0000bp}{58.0000bp}
    \pgfpathqlineto{120.0000bp}{78.0000bp}
    \pgfpathqlineto{100.0000bp}{78.0000bp}
    \pgfpathqlineto{100.0000bp}{58.0000bp}
    \pgfpathqlineto{120.0000bp}{58.0000bp}
    \pgfpathclose
    \pgfusepathqfill
  \end{pgfscope}
  \begin{pgfscope}
    \definecolor{fc}{rgb}{0.0000,0.0000,0.0000}
    \pgfsetfillcolor{fc}
    \pgftransformshift{\pgfqpoint{110.0000bp}{68.0000bp}}
    \pgftransformscale{1.2500}
    \pgftext[]{$n/4$}
  \end{pgfscope}
  \begin{pgfscope}
    \pgfsetlinewidth{0.8139bp}
    \definecolor{sc}{rgb}{0.0000,0.0000,0.0000}
    \pgfsetstrokecolor{sc}
    \pgfsetmiterjoin
    \pgfsetbuttcap
    \pgfpathqmoveto{30.0000bp}{68.0000bp}
    \pgfpathqlineto{50.0000bp}{38.0000bp}
    \pgfusepathqstroke
  \end{pgfscope}
  \begin{pgfscope}
    \pgfsetlinewidth{0.8139bp}
    \definecolor{sc}{rgb}{0.0000,0.0000,0.0000}
    \pgfsetstrokecolor{sc}
    \pgfsetmiterjoin
    \pgfsetbuttcap
    \pgfpathqmoveto{30.0000bp}{68.0000bp}
    \pgfpathqlineto{10.0000bp}{38.0000bp}
    \pgfusepathqstroke
  \end{pgfscope}
  \begin{pgfscope}
    \definecolor{fc}{rgb}{1.0000,1.0000,1.0000}
    \pgfsetfillcolor{fc}
    \pgfpathqmoveto{60.0000bp}{28.0000bp}
    \pgfpathqlineto{60.0000bp}{48.0000bp}
    \pgfpathqlineto{40.0000bp}{48.0000bp}
    \pgfpathqlineto{40.0000bp}{28.0000bp}
    \pgfpathqlineto{60.0000bp}{28.0000bp}
    \pgfpathclose
    \pgfusepathqfill
  \end{pgfscope}
  \begin{pgfscope}
    \definecolor{fc}{rgb}{0.0000,0.0000,0.0000}
    \pgfsetfillcolor{fc}
    \pgftransformshift{\pgfqpoint{50.0000bp}{38.0000bp}}
    \pgftransformscale{1.2500}
    \pgftext[]{$n/8$}
  \end{pgfscope}
  \begin{pgfscope}
    \definecolor{fc}{rgb}{1.0000,1.0000,1.0000}
    \pgfsetfillcolor{fc}
    \pgfpathqmoveto{20.0000bp}{28.0000bp}
    \pgfpathqlineto{20.0000bp}{48.0000bp}
    \pgfpathqlineto{-0.0000bp}{48.0000bp}
    \pgfpathqlineto{-0.0000bp}{28.0000bp}
    \pgfpathqlineto{20.0000bp}{28.0000bp}
    \pgfpathclose
    \pgfusepathqfill
  \end{pgfscope}
  \begin{pgfscope}
    \definecolor{fc}{rgb}{0.0000,0.0000,0.0000}
    \pgfsetfillcolor{fc}
    \pgftransformshift{\pgfqpoint{10.0000bp}{38.0000bp}}
    \pgftransformscale{1.2500}
    \pgftext[]{$n/8$}
  \end{pgfscope}
  \begin{pgfscope}
    \definecolor{fc}{rgb}{1.0000,1.0000,1.0000}
    \pgfsetfillcolor{fc}
    \pgfpathqmoveto{40.0000bp}{58.0000bp}
    \pgfpathqlineto{40.0000bp}{78.0000bp}
    \pgfpathqlineto{20.0000bp}{78.0000bp}
    \pgfpathqlineto{20.0000bp}{58.0000bp}
    \pgfpathqlineto{40.0000bp}{58.0000bp}
    \pgfpathclose
    \pgfusepathqfill
  \end{pgfscope}
  \begin{pgfscope}
    \definecolor{fc}{rgb}{0.0000,0.0000,0.0000}
    \pgfsetfillcolor{fc}
    \pgftransformshift{\pgfqpoint{30.0000bp}{68.0000bp}}
    \pgftransformscale{1.2500}
    \pgftext[]{$n/4$}
  \end{pgfscope}
  \begin{pgfscope}
    \definecolor{fc}{rgb}{1.0000,1.0000,1.0000}
    \pgfsetfillcolor{fc}
    \pgfpathqmoveto{80.0000bp}{88.0000bp}
    \pgfpathqlineto{80.0000bp}{108.0000bp}
    \pgfpathqlineto{60.0000bp}{108.0000bp}
    \pgfpathqlineto{60.0000bp}{88.0000bp}
    \pgfpathqlineto{80.0000bp}{88.0000bp}
    \pgfpathclose
    \pgfusepathqfill
  \end{pgfscope}
  \begin{pgfscope}
    \definecolor{fc}{rgb}{0.0000,0.0000,0.0000}
    \pgfsetfillcolor{fc}
    \pgftransformshift{\pgfqpoint{70.0000bp}{98.0000bp}}
    \pgftransformscale{1.2500}
    \pgftext[]{$n/2$}
  \end{pgfscope}
  \begin{pgfscope}
    \definecolor{fc}{rgb}{1.0000,1.0000,1.0000}
    \pgfsetfillcolor{fc}
    \pgfpathqmoveto{160.0000bp}{118.0000bp}
    \pgfpathqlineto{160.0000bp}{138.0000bp}
    \pgfpathqlineto{140.0000bp}{138.0000bp}
    \pgfpathqlineto{140.0000bp}{118.0000bp}
    \pgfpathqlineto{160.0000bp}{118.0000bp}
    \pgfpathclose
    \pgfusepathqfill
  \end{pgfscope}
  \begin{pgfscope}
    \definecolor{fc}{rgb}{0.0000,0.0000,0.0000}
    \pgfsetfillcolor{fc}
    \pgftransformshift{\pgfqpoint{150.0000bp}{128.0000bp}}
    \pgftransformscale{1.2500}
    \pgftext[]{$n$}
  \end{pgfscope}
\end{pgfpicture}

  \end{center}
\end{model*}

Recall the \emph{merge sort} algorithm, which works by splitting the
input list into halves, recursively sorting the two halves, and then
merging the two sorted halves back together.

\begin{objective}
  Students will use recurrence relations and recursion trees to
  describe and analyze divide and conquer algorithms.
\end{objective}

\begin{questions}
  \item How long does \textsf{mergesort} take on a list of length $1$?
  \item \emph{Just by looking at the code}, how many recursive calls does
  \textsf{mergesort} make at each step?\marginnote{Don't overthink
  this one. Yes, it's really that easy.}
  \item If $xs$ has size $n$, what is the size of the inputs to the
    recursive calls to \textsf{mergesort}?
  \item (Review) How long does it take (in big-$\Theta$ terms) to
    merge $xs_1$ and $xs_2$ after they are sorted?
  \item Let $T(n)$ denote the total amount of time taken by
    \textsf{mergesort} on an input list of length $n$.  Use your
    answers to the previous questions to explain the equations for
    $T(n)$ given in the model.  This is called a \emph{recurrence
      relation} because it defines $T(n)$ via recursion.
  \item Suppose algorithm $X$ takes an input of size $n$, splits it
    into three equal-sized pieces, and makes a recursive call on each
    piece.  Deciding how to split up the input into pieces takes
    $\Theta(n^2)$ time; combining the results of the recursive calls
    takes additional $\Theta(n)$ time.  In the base case, algorithm
    $X$ takes constant time on an input of size $1$. Write a
    recurrence relation $X(n)$ describing the time taken by algorithm
    $X$, similar to the one given in the model.
  \item Now suppose algorithm $X$ makes only two recursive calls instead
    of three, but each recursive call is still on an input one-third
    the size of the original input.  How does your recurrence relation
    for $X$ change?
  \item Write a recurrence relation for binary search.
\end{questions}

Now let's return to considering merge sort.  The tree shown in
the model represents the call tree of merge sort on an input of
size $n$, that is, each node in the tree represents one recursive call
to merge sort.  The expression at each node shows how much work
happens at that node (from merging).

\begin{questions}
  \item Notice that the entire tree is not shown; the dots indicate
    that the tree continues further with the same pattern.  What is
    the depth (number of levels) of the tree, in terms of $n$?
  \item How much total work happens on each individual level of the tree?
  \item How much total work happens in the entire tree?
  \item Draw a similar tree for the second version of algorithm
    $X$. Be careful to distinguish between the \emph{size of the
      input} and the \emph{amount of work} done at each node.
\end{questions}

\end{document}
