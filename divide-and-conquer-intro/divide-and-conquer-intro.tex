% -*- compile-command: "./build.sh" -*-
\documentclass{tufte-handout}

\usepackage{algo-activity}
\usepackage{acode}

\title{\thecourse: Introduction to Divide \& Conquer}
\date{}

%%% NOTE: In Fall 2017 this had another part but we didn't get to it;
%%% I just spent another class doing the second part.  So I have now
%%% moved that part into a second activity, #10: D&C arithmetic.

\begin{document}

\maketitle

\begin{model*}{Merge sort}{mergesort}
  \begin{acode}
\> mergesort($xs$) = \\
\> \tb \> !if $\mathit{len}(xs) \leq 1$ !then !return $xs$ \\
\>     \> split $xs$ into halves $(xs_1, xs_2)$ \\
\>     \> $xs_1' \gets$ mergesort($xs_1$) \\
\>     \> $xs_2' \gets$ mergesort($xs_2$) \\
\>     \> $xs' \gets$ merge($xs_1'$, $xs_2'$) \\
\>     \> !return $xs'$
  \end{acode}

  \begin{align*}
    T(1) &= \Theta(1) \\
    T(n) &= 2 T(n/2) + \Theta(n)
  \end{align*}

  \begin{center}
  \begin{diagram}[width=300]
    import Diagrams.TwoD.Layout.Tree
    import Diagrams.Prelude hiding (Empty)
    import Data.Maybe

    mergeSortTree :: Int -> BTree String
    mergeSortTree = go 1
      where
        go _ 0 = Empty
        go k n = BNode ("n" ++ if (k > 1) then ("/" ++ show k) else "") s s
          where s = go (2*k) (n-1)

    tree :: Diagram B
    tree = maybe mempty
            (renderTree (\l -> text ("$" ++ l ++ "$") <> square 2 # lw none # fc white ) (~~))
        . symmLayoutBin' (with & slHSep .~ 4 & slVSep .~ 3)
        $ mergeSortTree 4   -- $

    dia :: Diagram B
    dia = vsep 1 [tree, vsep 0.3 (replicate 3 (circle 0.2 # fc black))]
  \end{diagram}
  \end{center}
\end{model*}

Recall the \emph{merge sort} algorithm, which works by splitting the
input list into halves, recursively sorting the two halves, and then
merging the two sorted halves back together.

\begin{objective}
  Students will use recurrence relations and recursion trees to
  describe and analyze divide and conquer algorithms.
\end{objective}

\begin{questions}
  \item How long does \textsf{mergesort} take on a list of length $1$?
  \item Just by looking at the code, how many recursive calls does
  \textsf{mergesort} make at each step?\marginnote{\emph{Hint}: don't overthink
  this one; yes, it's really that easy.}
  \item If $xs$ has size $n$, what is the size of the inputs to the
    recursive calls to \textsf{mergesort}?
  \item (Review) How long does it take (in big-$\Theta$ terms) to
    merge $xs_1$ and $xs_2$ after they are sorted?
  \item Let $T(n)$ denote the total amount of time taken by
    \textsf{mergesort} on an input list of length $n$.  Use your
    answers to the previous questions to explain the equations for
    $T(n)$ given in the model.  This is called a \emph{recurrence
      relation} because it defines $T(n)$ via recursion.
  \item Suppose algorithm $X$ takes an input of size $n$, splits it
    into three equal-sized pieces, and makes a recursive call on each
    piece.  Deciding how to split up the input into pieces takes
    $\Theta(n^2)$ time; combining the results of the recursive calls
    takes additional $\Theta(n)$ time.  In the base case, algorithm
    $X$ takes constant time on an input of size $1$. Write a
    recurrence relation $X(n)$ describing the time taken by algorithm
    $X$, similar to the one given in the model.
  \item Now suppose algorithm $X$ makes only two recursive calls instead
    of three, but each recursive call is still on an input one-third
    the size of the original input.  How does your recurrence relation
    for $X$ change?
  \item Write a recurrence relation for binary search.
\end{questions}

Now let's return to considering merge sort.  The tree shown in
the model represents the call tree of merge sort on an input of
size $n$, that is, each node in the tree represents one recursive call
to merge sort.  The expression at each node shows how much work
happens at that node (from merging).

\begin{questions}
  \item Notice that the entire tree is not shown; the dots indicate
    that the tree continues further with the same pattern.  What is
    the depth (number of levels) of the tree, in terms of $n$?
  \item How much total work happens on each individual level of the tree?
  \item How much total work happens in the entire tree?
  \item Draw a similar tree for the second version of algorithm $X$.
\end{questions}

\end{document}
