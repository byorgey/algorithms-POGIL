% -*- compile-command: "./build.sh" -*-
\documentclass{tufte-handout}

\usepackage{algo-activity}

\title{\thecourse: Introduction to Flow Networks}
\date{}

\begin{document}

\maketitle

% \begin{objective}
%   Students will XXX.
% \end{objective}

\begin{model}{Networks and flows}{networksandflows}
  \begin{center}
  {\huge A}
  \begin{diagram}[width=300]
    import FlowIntro
    import qualified Data.Map as M
    dia = drawFlow False flowExample
  \end{diagram}
  \end{center}
  \bigskip \bigskip

  \hrule \bigskip

  \begin{center}
  {\huge B}
  \begin{diagram}[width=300]
    import FlowIntro
    import qualified Data.Map as M

    flow = M.fromList $  -- $
      [ (('s','a'), 2)
      , (('s','b'), 3)
      , (('a','d'), 1)
      , (('d','t'), 1)
      , (('a','b'), 1)
      , (('b','e'), 4)
      , (('e','f'), 2)
      , (('e','t'), 2)
      , (('f','t'), 2)
      ]
    dia = drawFlow True (flowExample # withFlow flow)
  \end{diagram}
  \end{center}
\end{model}

\setcounter{modelcounter}{0}
\begin{model}{(continued)}{networksandflows}
  \begin{center}
  {\huge C}
  \begin{diagram}[width=300]
    import FlowIntro
    import qualified Data.Map as M

    flow = M.fromList $  -- $
      [ (('s','a'), 10)
      , (('a','d'), 9)
      , (('a','e'), 1)
      , (('d','t'), 9)
      , (('e','t'), 1)
      ]
    dia = drawFlow True (flowExample # withFlow flow)
  \end{diagram}
  \end{center}
  \bigskip \bigskip

  \hrule \bigskip

  \begin{center}
  {\huge D}
  \begin{diagram}[width=300]
    import FlowIntro
    import qualified Data.Map as M

    flow = M.fromList $  -- $
      [ (('s','a'), 10)
      , (('s','b'), 5)
      , (('b','e'), 7)
      , (('a','d'), 10)
      , (('d','e'), 1)
      , (('d','t'), 9)
      , (('e','t'), 8)
      ]
    dia = drawFlow True (flowExample # withFlow flow)
  \end{diagram}
  \end{center}
\end{model}

Consider graph $A$. Once again we have a directed graph with weighted
edges.  However, instead of thinking of the weights as some sort of
length, we will now think of them as a \emph{capacity}: the ``maximum
amount of stuff'' that the edge can carry.  For example, the capacity
might be used to model things like:
\begin{itemize}
\item maximum gallons of water per minute that can flow through a pipe;
\item maximum number of trucks per hour that can drive along a road; or
\item maximum number of times a certain resource can be used before it
  is all used up.
\end{itemize}

\begin{questions}
\item Consider graph $B$.  How is it related to graph $A$?
\item What do the blue edges in graph $B$ all have in common?
\item What do you think the labels on the edges of graph $B$
  represent?
\item Now consider graph $C$.  Why do you think some of the edges are
  green?
\item Graph $D$ is invalid!  In fact, there are two things wrong with
  it.  What are they?
\end{questions}

\newcommand{\R}{\mathbb{R}}

\pause
\begin{defn}
  A \term{flow network} is a directed graph $G = (V,E)$ with
  \begin{itemize}
  \item a distinguished \term{source} vertex $s \in V$, with only
    outgoing edges;
  \item a distinguished \term{sink} or \term{target} vertex $t \in V$,
    with only incoming edges;
  \item a \term{capacity function} $c : E \to \R^+$ assigning a
    non-negative real number capacity $c(e)$ to each edge $e \in E$.
  \end{itemize}
\end{defn}

\begin{questions}
  \item Is graph $A$ a flow network?  Why or why not?
\end{questions}

Now let's define a \term{flow}.  Both graphs $B$ and $C$ depict valid
flows on $A$; graph $D$ does not.

\begin{defn}
  A \term{flow} on a flow network $G$ is a function $f : E \to \R^+$
  assigning a non-negative flow $f(e)$ to each edge, such that
  \begin{enumerate}
  \item $\underline{\phantom{XXXXXXXX}} \leq f(e) \leq
    \underline{\phantom{XXXXXXXX}}$ for every $e \in E$
  \item At each vertex $v \in V$ other than $s$ and $t$, \blank
    \newline \blank.
  \end{enumerate}
\end{defn}

\begin{defn}
  The \term{value} of a flow, $v(f)$, is the sum of the flow on all
  edges leaving $s$.
\end{defn}

\begin{questions}
  \item What is the value of the flow on graph $B$?
  \item What is the value of the flow on graph $C$?
  \item Make a conjecture about the relationship between the value of
    a flow and the amount of flow entering $t$.
  \item For each amount, say whether you can construct a flow on graph
    $A$ with the given value.
    \begin{enumerate}[label=(\alph*)]
    \item $15$ \vspace{1in}
    \item $40$ \vspace{1in}
    \item $30$ \vspace{1in}
    \end{enumerate}
  \item What is the value of the biggest flow you can construct on
    graph $A$?
\end{questions}

\end{document}
