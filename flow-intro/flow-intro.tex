% -*- compile-command: "rubber -d --unsafe flow-intro.tex" -*-
\documentclass{tufte-handout}

\usepackage{algo-activity}

\title{\thecourse: Introduction to Flow Networks}
\date{}

\graphicspath{{images/}{../images/}}

\begin{document}

\maketitle

% \begin{objective}
%   Students will XXX.
% \end{objective}

\begin{model}{Networks and flows}{networksandflows}
  \begin{center}
    {\huge A}
    \begin{pgfpicture}
  \pgfpathrectangle{\pgfpointorigin}{\pgfqpoint{300.0000bp}{213.0000bp}}
  \pgfusepath{use as bounding box}
  \begin{pgfscope}
    \pgfsetlinewidth{1.0111bp}
    \definecolor{sc}{rgb}{0.0000,0.0000,0.0000}
    \pgfsetstrokecolor{sc}
    \pgfsetmiterjoin
    \pgfsetbuttcap
    \pgfpathqmoveto{12.3744bp}{90.6256bp}
    \pgfpathqlineto{87.6256bp}{15.3744bp}
    \pgfusepathqstroke
  \end{pgfscope}
  \begin{pgfscope}
    \definecolor{fc}{rgb}{0.0000,0.0000,0.0000}
    \pgfsetfillcolor{fc}
    \pgfpathqmoveto{83.4631bp}{15.2943bp}
    \pgfpathqlineto{91.7081bp}{11.2919bp}
    \pgfpathqlineto{87.7057bp}{19.5369bp}
    \pgfpathqlineto{83.4631bp}{15.2943bp}
    \pgfpathclose
    \pgfusepathqfill
  \end{pgfscope}
  \begin{pgfscope}
    \definecolor{fc}{rgb}{0.0000,0.0000,0.0000}
    \pgfsetfillcolor{fc}
    \pgftransformshift{\pgfqpoint{57.0711bp}{60.0711bp}}
    \pgftransformscale{1.2500}
    \pgftext[]{$15$}
  \end{pgfscope}
  \begin{pgfscope}
    \pgfsetlinewidth{1.0111bp}
    \definecolor{sc}{rgb}{0.0000,0.0000,0.0000}
    \pgfsetstrokecolor{sc}
    \pgfsetmiterjoin
    \pgfsetbuttcap
    \pgfpathqmoveto{17.5000bp}{103.0000bp}
    \pgfpathqlineto{82.5000bp}{103.0000bp}
    \pgfusepathqstroke
  \end{pgfscope}
  \begin{pgfscope}
    \definecolor{fc}{rgb}{0.0000,0.0000,0.0000}
    \pgfsetfillcolor{fc}
    \pgfpathqmoveto{79.6132bp}{100.0000bp}
    \pgfpathqlineto{88.2735bp}{103.0000bp}
    \pgfpathqlineto{79.6132bp}{106.0000bp}
    \pgfpathqlineto{79.6132bp}{100.0000bp}
    \pgfpathclose
    \pgfusepathqfill
  \end{pgfscope}
  \begin{pgfscope}
    \definecolor{fc}{rgb}{0.0000,0.0000,0.0000}
    \pgfsetfillcolor{fc}
    \pgftransformshift{\pgfqpoint{50.0000bp}{113.0000bp}}
    \pgftransformscale{1.2500}
    \pgftext[]{$5$}
  \end{pgfscope}
  \begin{pgfscope}
    \pgfsetlinewidth{1.0111bp}
    \definecolor{sc}{rgb}{0.0000,0.0000,0.0000}
    \pgfsetstrokecolor{sc}
    \pgfsetmiterjoin
    \pgfsetbuttcap
    \pgfpathqmoveto{12.3744bp}{115.3744bp}
    \pgfpathqlineto{87.6256bp}{190.6256bp}
    \pgfusepathqstroke
  \end{pgfscope}
  \begin{pgfscope}
    \definecolor{fc}{rgb}{0.0000,0.0000,0.0000}
    \pgfsetfillcolor{fc}
    \pgfpathqmoveto{87.7057bp}{186.4631bp}
    \pgfpathqlineto{91.7081bp}{194.7081bp}
    \pgfpathqlineto{83.4631bp}{190.7057bp}
    \pgfpathqlineto{87.7057bp}{186.4631bp}
    \pgfpathclose
    \pgfusepathqfill
  \end{pgfscope}
  \begin{pgfscope}
    \definecolor{fc}{rgb}{0.0000,0.0000,0.0000}
    \pgfsetfillcolor{fc}
    \pgftransformshift{\pgfqpoint{42.9289bp}{160.0711bp}}
    \pgftransformscale{1.2500}
    \pgftext[]{$10$}
  \end{pgfscope}
  \begin{pgfscope}
    \pgfsetlinewidth{1.0111bp}
    \definecolor{sc}{rgb}{0.0000,0.0000,0.0000}
    \pgfsetstrokecolor{sc}
    \pgfsetmiterjoin
    \pgfsetbuttcap
    \pgfpathqmoveto{212.3744bp}{15.3744bp}
    \pgfpathqlineto{287.6256bp}{90.6256bp}
    \pgfusepathqstroke
  \end{pgfscope}
  \begin{pgfscope}
    \definecolor{fc}{rgb}{0.0000,0.0000,0.0000}
    \pgfsetfillcolor{fc}
    \pgfpathqmoveto{287.7057bp}{86.4631bp}
    \pgfpathqlineto{291.7081bp}{94.7081bp}
    \pgfpathqlineto{283.4631bp}{90.7057bp}
    \pgfpathqlineto{287.7057bp}{86.4631bp}
    \pgfpathclose
    \pgfusepathqfill
  \end{pgfscope}
  \begin{pgfscope}
    \definecolor{fc}{rgb}{0.0000,0.0000,0.0000}
    \pgfsetfillcolor{fc}
    \pgftransformshift{\pgfqpoint{242.9289bp}{60.0711bp}}
    \pgftransformscale{1.2500}
    \pgftext[]{$10$}
  \end{pgfscope}
  \begin{pgfscope}
    \pgfsetlinewidth{1.0111bp}
    \definecolor{sc}{rgb}{0.0000,0.0000,0.0000}
    \pgfsetstrokecolor{sc}
    \pgfsetmiterjoin
    \pgfsetbuttcap
    \pgfpathqmoveto{187.6256bp}{15.3744bp}
    \pgfpathqlineto{112.3744bp}{90.6256bp}
    \pgfusepathqstroke
  \end{pgfscope}
  \begin{pgfscope}
    \definecolor{fc}{rgb}{0.0000,0.0000,0.0000}
    \pgfsetfillcolor{fc}
    \pgfpathqmoveto{116.5369bp}{90.7057bp}
    \pgfpathqlineto{108.2919bp}{94.7081bp}
    \pgfpathqlineto{112.2943bp}{86.4631bp}
    \pgfpathqlineto{116.5369bp}{90.7057bp}
    \pgfpathclose
    \pgfusepathqfill
  \end{pgfscope}
  \begin{pgfscope}
    \definecolor{fc}{rgb}{0.0000,0.0000,0.0000}
    \pgfsetfillcolor{fc}
    \pgftransformshift{\pgfqpoint{142.9289bp}{45.9289bp}}
    \pgftransformscale{1.2500}
    \pgftext[]{$6$}
  \end{pgfscope}
  \begin{pgfscope}
    \pgfsetlinewidth{1.0111bp}
    \definecolor{sc}{rgb}{0.0000,0.0000,0.0000}
    \pgfsetstrokecolor{sc}
    \pgfsetmiterjoin
    \pgfsetbuttcap
    \pgfpathqmoveto{217.5000bp}{103.0000bp}
    \pgfpathqlineto{282.5000bp}{103.0000bp}
    \pgfusepathqstroke
  \end{pgfscope}
  \begin{pgfscope}
    \definecolor{fc}{rgb}{0.0000,0.0000,0.0000}
    \pgfsetfillcolor{fc}
    \pgfpathqmoveto{279.6132bp}{100.0000bp}
    \pgfpathqlineto{288.2735bp}{103.0000bp}
    \pgfpathqlineto{279.6132bp}{106.0000bp}
    \pgfpathqlineto{279.6132bp}{100.0000bp}
    \pgfpathclose
    \pgfusepathqfill
  \end{pgfscope}
  \begin{pgfscope}
    \definecolor{fc}{rgb}{0.0000,0.0000,0.0000}
    \pgfsetfillcolor{fc}
    \pgftransformshift{\pgfqpoint{250.0000bp}{113.0000bp}}
    \pgftransformscale{1.2500}
    \pgftext[]{$10$}
  \end{pgfscope}
  \begin{pgfscope}
    \pgfsetlinewidth{1.0111bp}
    \definecolor{sc}{rgb}{0.0000,0.0000,0.0000}
    \pgfsetstrokecolor{sc}
    \pgfsetmiterjoin
    \pgfsetbuttcap
    \pgfpathqmoveto{200.0000bp}{85.5000bp}
    \pgfpathqlineto{200.0000bp}{20.5000bp}
    \pgfusepathqstroke
  \end{pgfscope}
  \begin{pgfscope}
    \definecolor{fc}{rgb}{0.0000,0.0000,0.0000}
    \pgfsetfillcolor{fc}
    \pgfpathqmoveto{197.0000bp}{23.3868bp}
    \pgfpathqlineto{200.0000bp}{14.7265bp}
    \pgfpathqlineto{203.0000bp}{23.3868bp}
    \pgfpathqlineto{197.0000bp}{23.3868bp}
    \pgfpathclose
    \pgfusepathqfill
  \end{pgfscope}
  \begin{pgfscope}
    \definecolor{fc}{rgb}{0.0000,0.0000,0.0000}
    \pgfsetfillcolor{fc}
    \pgftransformshift{\pgfqpoint{210.0000bp}{53.0000bp}}
    \pgftransformscale{1.2500}
    \pgftext[]{$15$}
  \end{pgfscope}
  \begin{pgfscope}
    \pgfsetlinewidth{1.0111bp}
    \definecolor{sc}{rgb}{0.0000,0.0000,0.0000}
    \pgfsetstrokecolor{sc}
    \pgfsetmiterjoin
    \pgfsetbuttcap
    \pgfpathqmoveto{212.3744bp}{190.6256bp}
    \pgfpathqlineto{287.6256bp}{115.3744bp}
    \pgfusepathqstroke
  \end{pgfscope}
  \begin{pgfscope}
    \definecolor{fc}{rgb}{0.0000,0.0000,0.0000}
    \pgfsetfillcolor{fc}
    \pgfpathqmoveto{283.4631bp}{115.2943bp}
    \pgfpathqlineto{291.7081bp}{111.2919bp}
    \pgfpathqlineto{287.7057bp}{119.5369bp}
    \pgfpathqlineto{283.4631bp}{115.2943bp}
    \pgfpathclose
    \pgfusepathqfill
  \end{pgfscope}
  \begin{pgfscope}
    \definecolor{fc}{rgb}{0.0000,0.0000,0.0000}
    \pgfsetfillcolor{fc}
    \pgftransformshift{\pgfqpoint{257.0711bp}{160.0711bp}}
    \pgftransformscale{1.2500}
    \pgftext[]{$10$}
  \end{pgfscope}
  \begin{pgfscope}
    \pgfsetlinewidth{1.0111bp}
    \definecolor{sc}{rgb}{0.0000,0.0000,0.0000}
    \pgfsetstrokecolor{sc}
    \pgfsetmiterjoin
    \pgfsetbuttcap
    \pgfpathqmoveto{200.0000bp}{185.5000bp}
    \pgfpathqlineto{200.0000bp}{120.5000bp}
    \pgfusepathqstroke
  \end{pgfscope}
  \begin{pgfscope}
    \definecolor{fc}{rgb}{0.0000,0.0000,0.0000}
    \pgfsetfillcolor{fc}
    \pgfpathqmoveto{197.0000bp}{123.3868bp}
    \pgfpathqlineto{200.0000bp}{114.7265bp}
    \pgfpathqlineto{203.0000bp}{123.3868bp}
    \pgfpathqlineto{197.0000bp}{123.3868bp}
    \pgfpathclose
    \pgfusepathqfill
  \end{pgfscope}
  \begin{pgfscope}
    \definecolor{fc}{rgb}{0.0000,0.0000,0.0000}
    \pgfsetfillcolor{fc}
    \pgftransformshift{\pgfqpoint{210.0000bp}{153.0000bp}}
    \pgftransformscale{1.2500}
    \pgftext[]{$15$}
  \end{pgfscope}
  \begin{pgfscope}
    \pgfsetlinewidth{1.0111bp}
    \definecolor{sc}{rgb}{0.0000,0.0000,0.0000}
    \pgfsetstrokecolor{sc}
    \pgfsetmiterjoin
    \pgfsetbuttcap
    \pgfpathqmoveto{117.5000bp}{3.0000bp}
    \pgfpathqlineto{182.5000bp}{3.0000bp}
    \pgfusepathqstroke
  \end{pgfscope}
  \begin{pgfscope}
    \definecolor{fc}{rgb}{0.0000,0.0000,0.0000}
    \pgfsetfillcolor{fc}
    \pgfpathqmoveto{179.6132bp}{-0.0000bp}
    \pgfpathqlineto{188.2735bp}{3.0000bp}
    \pgfpathqlineto{179.6132bp}{6.0000bp}
    \pgfpathqlineto{179.6132bp}{-0.0000bp}
    \pgfpathclose
    \pgfusepathqfill
  \end{pgfscope}
  \begin{pgfscope}
    \definecolor{fc}{rgb}{0.0000,0.0000,0.0000}
    \pgfsetfillcolor{fc}
    \pgftransformshift{\pgfqpoint{150.0000bp}{13.0000bp}}
    \pgftransformscale{1.2500}
    \pgftext[]{$30$}
  \end{pgfscope}
  \begin{pgfscope}
    \pgfsetlinewidth{1.0111bp}
    \definecolor{sc}{rgb}{0.0000,0.0000,0.0000}
    \pgfsetstrokecolor{sc}
    \pgfsetmiterjoin
    \pgfsetbuttcap
    \pgfpathqmoveto{117.5000bp}{103.0000bp}
    \pgfpathqlineto{182.5000bp}{103.0000bp}
    \pgfusepathqstroke
  \end{pgfscope}
  \begin{pgfscope}
    \definecolor{fc}{rgb}{0.0000,0.0000,0.0000}
    \pgfsetfillcolor{fc}
    \pgfpathqmoveto{179.6132bp}{100.0000bp}
    \pgfpathqlineto{188.2735bp}{103.0000bp}
    \pgfpathqlineto{179.6132bp}{106.0000bp}
    \pgfpathqlineto{179.6132bp}{100.0000bp}
    \pgfpathclose
    \pgfusepathqfill
  \end{pgfscope}
  \begin{pgfscope}
    \definecolor{fc}{rgb}{0.0000,0.0000,0.0000}
    \pgfsetfillcolor{fc}
    \pgftransformshift{\pgfqpoint{150.0000bp}{113.0000bp}}
    \pgftransformscale{1.2500}
    \pgftext[]{$8$}
  \end{pgfscope}
  \begin{pgfscope}
    \pgfsetlinewidth{1.0111bp}
    \definecolor{sc}{rgb}{0.0000,0.0000,0.0000}
    \pgfsetstrokecolor{sc}
    \pgfsetmiterjoin
    \pgfsetbuttcap
    \pgfpathqmoveto{100.0000bp}{85.5000bp}
    \pgfpathqlineto{100.0000bp}{20.5000bp}
    \pgfusepathqstroke
  \end{pgfscope}
  \begin{pgfscope}
    \definecolor{fc}{rgb}{0.0000,0.0000,0.0000}
    \pgfsetfillcolor{fc}
    \pgfpathqmoveto{97.0000bp}{23.3868bp}
    \pgfpathqlineto{100.0000bp}{14.7265bp}
    \pgfpathqlineto{103.0000bp}{23.3868bp}
    \pgfpathqlineto{97.0000bp}{23.3868bp}
    \pgfpathclose
    \pgfusepathqfill
  \end{pgfscope}
  \begin{pgfscope}
    \definecolor{fc}{rgb}{0.0000,0.0000,0.0000}
    \pgfsetfillcolor{fc}
    \pgftransformshift{\pgfqpoint{110.0000bp}{53.0000bp}}
    \pgftransformscale{1.2500}
    \pgftext[]{$4$}
  \end{pgfscope}
  \begin{pgfscope}
    \pgfsetlinewidth{1.0111bp}
    \definecolor{sc}{rgb}{0.0000,0.0000,0.0000}
    \pgfsetstrokecolor{sc}
    \pgfsetmiterjoin
    \pgfsetbuttcap
    \pgfpathqmoveto{112.3744bp}{190.6256bp}
    \pgfpathqlineto{187.6256bp}{115.3744bp}
    \pgfusepathqstroke
  \end{pgfscope}
  \begin{pgfscope}
    \definecolor{fc}{rgb}{0.0000,0.0000,0.0000}
    \pgfsetfillcolor{fc}
    \pgfpathqmoveto{183.4631bp}{115.2943bp}
    \pgfpathqlineto{191.7081bp}{111.2919bp}
    \pgfpathqlineto{187.7057bp}{119.5369bp}
    \pgfpathqlineto{183.4631bp}{115.2943bp}
    \pgfpathclose
    \pgfusepathqfill
  \end{pgfscope}
  \begin{pgfscope}
    \definecolor{fc}{rgb}{0.0000,0.0000,0.0000}
    \pgfsetfillcolor{fc}
    \pgftransformshift{\pgfqpoint{157.0711bp}{160.0711bp}}
    \pgftransformscale{1.2500}
    \pgftext[]{$15$}
  \end{pgfscope}
  \begin{pgfscope}
    \pgfsetlinewidth{1.0111bp}
    \definecolor{sc}{rgb}{0.0000,0.0000,0.0000}
    \pgfsetstrokecolor{sc}
    \pgfsetmiterjoin
    \pgfsetbuttcap
    \pgfpathqmoveto{117.5000bp}{203.0000bp}
    \pgfpathqlineto{182.5000bp}{203.0000bp}
    \pgfusepathqstroke
  \end{pgfscope}
  \begin{pgfscope}
    \definecolor{fc}{rgb}{0.0000,0.0000,0.0000}
    \pgfsetfillcolor{fc}
    \pgfpathqmoveto{179.6132bp}{200.0000bp}
    \pgfpathqlineto{188.2735bp}{203.0000bp}
    \pgfpathqlineto{179.6132bp}{206.0000bp}
    \pgfpathqlineto{179.6132bp}{200.0000bp}
    \pgfpathclose
    \pgfusepathqfill
  \end{pgfscope}
  \begin{pgfscope}
    \definecolor{fc}{rgb}{0.0000,0.0000,0.0000}
    \pgfsetfillcolor{fc}
    \pgftransformshift{\pgfqpoint{150.0000bp}{213.0000bp}}
    \pgftransformscale{1.2500}
    \pgftext[]{$9$}
  \end{pgfscope}
  \begin{pgfscope}
    \pgfsetlinewidth{1.0111bp}
    \definecolor{sc}{rgb}{0.0000,0.0000,0.0000}
    \pgfsetstrokecolor{sc}
    \pgfsetmiterjoin
    \pgfsetbuttcap
    \pgfpathqmoveto{100.0000bp}{185.5000bp}
    \pgfpathqlineto{100.0000bp}{120.5000bp}
    \pgfusepathqstroke
  \end{pgfscope}
  \begin{pgfscope}
    \definecolor{fc}{rgb}{0.0000,0.0000,0.0000}
    \pgfsetfillcolor{fc}
    \pgfpathqmoveto{97.0000bp}{123.3868bp}
    \pgfpathqlineto{100.0000bp}{114.7265bp}
    \pgfpathqlineto{103.0000bp}{123.3868bp}
    \pgfpathqlineto{97.0000bp}{123.3868bp}
    \pgfpathclose
    \pgfusepathqfill
  \end{pgfscope}
  \begin{pgfscope}
    \definecolor{fc}{rgb}{0.0000,0.0000,0.0000}
    \pgfsetfillcolor{fc}
    \pgftransformshift{\pgfqpoint{110.0000bp}{153.0000bp}}
    \pgftransformscale{1.2500}
    \pgftext[]{$4$}
  \end{pgfscope}
  \begin{pgfscope}
    \definecolor{fc}{rgb}{0.0000,0.0000,0.0000}
    \pgfsetfillcolor{fc}
    \pgftransformshift{\pgfqpoint{300.0000bp}{103.0000bp}}
    \pgftransformscale{2.5000}
    \pgftext[]{$t$}
  \end{pgfscope}
  \begin{pgfscope}
    \definecolor{fc}{rgb}{0.0000,0.0000,0.0000}
    \pgfsetfillcolor{fc}
    \pgftransformshift{\pgfqpoint{0.0000bp}{103.0000bp}}
    \pgftransformscale{2.5000}
    \pgftext[]{$s$}
  \end{pgfscope}
  \begin{pgfscope}
    \definecolor{fc}{rgb}{0.0000,0.0000,0.0000}
    \pgfsetfillcolor{fc}
    \pgftransformshift{\pgfqpoint{200.0000bp}{3.0000bp}}
    \pgftransformscale{2.5000}
    \pgftext[]{$f$}
  \end{pgfscope}
  \begin{pgfscope}
    \definecolor{fc}{rgb}{0.0000,0.0000,0.0000}
    \pgfsetfillcolor{fc}
    \pgftransformshift{\pgfqpoint{200.0000bp}{103.0000bp}}
    \pgftransformscale{2.5000}
    \pgftext[]{$e$}
  \end{pgfscope}
  \begin{pgfscope}
    \definecolor{fc}{rgb}{0.0000,0.0000,0.0000}
    \pgfsetfillcolor{fc}
    \pgftransformshift{\pgfqpoint{200.0000bp}{203.0000bp}}
    \pgftransformscale{2.5000}
    \pgftext[]{$d$}
  \end{pgfscope}
  \begin{pgfscope}
    \definecolor{fc}{rgb}{0.0000,0.0000,0.0000}
    \pgfsetfillcolor{fc}
    \pgftransformshift{\pgfqpoint{100.0000bp}{3.0000bp}}
    \pgftransformscale{2.5000}
    \pgftext[]{$c$}
  \end{pgfscope}
  \begin{pgfscope}
    \definecolor{fc}{rgb}{0.0000,0.0000,0.0000}
    \pgfsetfillcolor{fc}
    \pgftransformshift{\pgfqpoint{100.0000bp}{103.0000bp}}
    \pgftransformscale{2.5000}
    \pgftext[]{$b$}
  \end{pgfscope}
  \begin{pgfscope}
    \definecolor{fc}{rgb}{0.0000,0.0000,0.0000}
    \pgfsetfillcolor{fc}
    \pgftransformshift{\pgfqpoint{100.0000bp}{203.0000bp}}
    \pgftransformscale{2.5000}
    \pgftext[]{$a$}
  \end{pgfscope}
\end{pgfpicture}

  \end{center}
  \bigskip \bigskip

  \hrule \bigskip

  \begin{center}
    {\huge B}
    \begin{pgfpicture}
  \pgfpathrectangle{\pgfpointorigin}{\pgfqpoint{300.0000bp}{217.0000bp}}
  \pgfusepath{use as bounding box}
  \begin{pgfscope}
    \pgfsetlinewidth{1.0229bp}
    \definecolor{sc}{rgb}{0.0000,0.0000,0.0000}
    \pgfsetstrokecolor{sc}
    \pgfsetmiterjoin
    \pgfsetbuttcap
    \pgfpathqmoveto{12.3744bp}{90.6256bp}
    \pgfpathqlineto{87.6256bp}{15.3744bp}
    \pgfusepathqstroke
  \end{pgfscope}
  \begin{pgfscope}
    \definecolor{fc}{rgb}{0.0000,0.0000,0.0000}
    \pgfsetfillcolor{fc}
    \pgfpathqmoveto{83.4631bp}{15.2943bp}
    \pgfpathqlineto{91.7081bp}{11.2919bp}
    \pgfpathqlineto{87.7057bp}{19.5369bp}
    \pgfpathqlineto{83.4631bp}{15.2943bp}
    \pgfpathclose
    \pgfusepathqfill
  \end{pgfscope}
  \begin{pgfscope}
    \definecolor{fc}{rgb}{0.0000,0.0000,0.0000}
    \pgfsetfillcolor{fc}
    \pgftransformshift{\pgfqpoint{60.6066bp}{63.6066bp}}
    \pgftransformscale{1.0000}
    \pgftext[]{$0/15$}
  \end{pgfscope}
  \begin{pgfscope}
    \pgfsetlinewidth{2.5573bp}
    \definecolor{sc}{rgb}{0.0000,0.0000,1.0000}
    \pgfsetstrokecolor{sc}
    \pgfsetmiterjoin
    \pgfsetbuttcap
    \pgfpathqmoveto{17.5000bp}{103.0000bp}
    \pgfpathqlineto{82.5000bp}{103.0000bp}
    \pgfusepathqstroke
  \end{pgfscope}
  \begin{pgfscope}
    \definecolor{fc}{rgb}{0.0000,0.0000,1.0000}
    \pgfsetfillcolor{fc}
    \pgfpathqmoveto{79.6132bp}{100.0000bp}
    \pgfpathqlineto{88.2735bp}{103.0000bp}
    \pgfpathqlineto{79.6132bp}{106.0000bp}
    \pgfpathqlineto{79.6132bp}{100.0000bp}
    \pgfpathclose
    \pgfusepathqfill
  \end{pgfscope}
  \begin{pgfscope}
    \definecolor{fc}{rgb}{0.0000,0.0000,0.0000}
    \pgfsetfillcolor{fc}
    \pgftransformshift{\pgfqpoint{50.0000bp}{118.0000bp}}
    \pgftransformscale{1.0000}
    \pgftext[]{$3/5$}
  \end{pgfscope}
  \begin{pgfscope}
    \pgfsetlinewidth{2.5573bp}
    \definecolor{sc}{rgb}{0.0000,0.0000,1.0000}
    \pgfsetstrokecolor{sc}
    \pgfsetmiterjoin
    \pgfsetbuttcap
    \pgfpathqmoveto{12.3744bp}{115.3744bp}
    \pgfpathqlineto{87.6256bp}{190.6256bp}
    \pgfusepathqstroke
  \end{pgfscope}
  \begin{pgfscope}
    \definecolor{fc}{rgb}{0.0000,0.0000,1.0000}
    \pgfsetfillcolor{fc}
    \pgfpathqmoveto{87.7057bp}{186.4631bp}
    \pgfpathqlineto{91.7081bp}{194.7081bp}
    \pgfpathqlineto{83.4631bp}{190.7057bp}
    \pgfpathqlineto{87.7057bp}{186.4631bp}
    \pgfpathclose
    \pgfusepathqfill
  \end{pgfscope}
  \begin{pgfscope}
    \definecolor{fc}{rgb}{0.0000,0.0000,0.0000}
    \pgfsetfillcolor{fc}
    \pgftransformshift{\pgfqpoint{39.3934bp}{163.6066bp}}
    \pgftransformscale{1.0000}
    \pgftext[]{$2/10$}
  \end{pgfscope}
  \begin{pgfscope}
    \pgfsetlinewidth{2.5573bp}
    \definecolor{sc}{rgb}{0.0000,0.0000,1.0000}
    \pgfsetstrokecolor{sc}
    \pgfsetmiterjoin
    \pgfsetbuttcap
    \pgfpathqmoveto{212.3744bp}{15.3744bp}
    \pgfpathqlineto{287.6256bp}{90.6256bp}
    \pgfusepathqstroke
  \end{pgfscope}
  \begin{pgfscope}
    \definecolor{fc}{rgb}{0.0000,0.0000,1.0000}
    \pgfsetfillcolor{fc}
    \pgfpathqmoveto{287.7057bp}{86.4631bp}
    \pgfpathqlineto{291.7081bp}{94.7081bp}
    \pgfpathqlineto{283.4631bp}{90.7057bp}
    \pgfpathqlineto{287.7057bp}{86.4631bp}
    \pgfpathclose
    \pgfusepathqfill
  \end{pgfscope}
  \begin{pgfscope}
    \definecolor{fc}{rgb}{0.0000,0.0000,0.0000}
    \pgfsetfillcolor{fc}
    \pgftransformshift{\pgfqpoint{239.3934bp}{63.6066bp}}
    \pgftransformscale{1.0000}
    \pgftext[]{$2/10$}
  \end{pgfscope}
  \begin{pgfscope}
    \pgfsetlinewidth{1.0229bp}
    \definecolor{sc}{rgb}{0.0000,0.0000,0.0000}
    \pgfsetstrokecolor{sc}
    \pgfsetmiterjoin
    \pgfsetbuttcap
    \pgfpathqmoveto{187.6256bp}{15.3744bp}
    \pgfpathqlineto{112.3744bp}{90.6256bp}
    \pgfusepathqstroke
  \end{pgfscope}
  \begin{pgfscope}
    \definecolor{fc}{rgb}{0.0000,0.0000,0.0000}
    \pgfsetfillcolor{fc}
    \pgfpathqmoveto{116.5369bp}{90.7057bp}
    \pgfpathqlineto{108.2919bp}{94.7081bp}
    \pgfpathqlineto{112.2943bp}{86.4631bp}
    \pgfpathqlineto{116.5369bp}{90.7057bp}
    \pgfpathclose
    \pgfusepathqfill
  \end{pgfscope}
  \begin{pgfscope}
    \definecolor{fc}{rgb}{0.0000,0.0000,0.0000}
    \pgfsetfillcolor{fc}
    \pgftransformshift{\pgfqpoint{139.3934bp}{42.3934bp}}
    \pgftransformscale{1.0000}
    \pgftext[]{$0/6$}
  \end{pgfscope}
  \begin{pgfscope}
    \pgfsetlinewidth{2.5573bp}
    \definecolor{sc}{rgb}{0.0000,0.0000,1.0000}
    \pgfsetstrokecolor{sc}
    \pgfsetmiterjoin
    \pgfsetbuttcap
    \pgfpathqmoveto{217.5000bp}{103.0000bp}
    \pgfpathqlineto{282.5000bp}{103.0000bp}
    \pgfusepathqstroke
  \end{pgfscope}
  \begin{pgfscope}
    \definecolor{fc}{rgb}{0.0000,0.0000,1.0000}
    \pgfsetfillcolor{fc}
    \pgfpathqmoveto{279.6132bp}{100.0000bp}
    \pgfpathqlineto{288.2735bp}{103.0000bp}
    \pgfpathqlineto{279.6132bp}{106.0000bp}
    \pgfpathqlineto{279.6132bp}{100.0000bp}
    \pgfpathclose
    \pgfusepathqfill
  \end{pgfscope}
  \begin{pgfscope}
    \definecolor{fc}{rgb}{0.0000,0.0000,0.0000}
    \pgfsetfillcolor{fc}
    \pgftransformshift{\pgfqpoint{250.0000bp}{118.0000bp}}
    \pgftransformscale{1.0000}
    \pgftext[]{$2/10$}
  \end{pgfscope}
  \begin{pgfscope}
    \pgfsetlinewidth{2.5573bp}
    \definecolor{sc}{rgb}{0.0000,0.0000,1.0000}
    \pgfsetstrokecolor{sc}
    \pgfsetmiterjoin
    \pgfsetbuttcap
    \pgfpathqmoveto{200.0000bp}{85.5000bp}
    \pgfpathqlineto{200.0000bp}{20.5000bp}
    \pgfusepathqstroke
  \end{pgfscope}
  \begin{pgfscope}
    \definecolor{fc}{rgb}{0.0000,0.0000,1.0000}
    \pgfsetfillcolor{fc}
    \pgfpathqmoveto{197.0000bp}{23.3868bp}
    \pgfpathqlineto{200.0000bp}{14.7265bp}
    \pgfpathqlineto{203.0000bp}{23.3868bp}
    \pgfpathqlineto{197.0000bp}{23.3868bp}
    \pgfpathclose
    \pgfusepathqfill
  \end{pgfscope}
  \begin{pgfscope}
    \definecolor{fc}{rgb}{0.0000,0.0000,0.0000}
    \pgfsetfillcolor{fc}
    \pgftransformshift{\pgfqpoint{215.0000bp}{53.0000bp}}
    \pgftransformscale{1.0000}
    \pgftext[]{$2/15$}
  \end{pgfscope}
  \begin{pgfscope}
    \pgfsetlinewidth{2.5573bp}
    \definecolor{sc}{rgb}{0.0000,0.0000,1.0000}
    \pgfsetstrokecolor{sc}
    \pgfsetmiterjoin
    \pgfsetbuttcap
    \pgfpathqmoveto{212.3744bp}{190.6256bp}
    \pgfpathqlineto{287.6256bp}{115.3744bp}
    \pgfusepathqstroke
  \end{pgfscope}
  \begin{pgfscope}
    \definecolor{fc}{rgb}{0.0000,0.0000,1.0000}
    \pgfsetfillcolor{fc}
    \pgfpathqmoveto{283.4631bp}{115.2943bp}
    \pgfpathqlineto{291.7081bp}{111.2919bp}
    \pgfpathqlineto{287.7057bp}{119.5369bp}
    \pgfpathqlineto{283.4631bp}{115.2943bp}
    \pgfpathclose
    \pgfusepathqfill
  \end{pgfscope}
  \begin{pgfscope}
    \definecolor{fc}{rgb}{0.0000,0.0000,0.0000}
    \pgfsetfillcolor{fc}
    \pgftransformshift{\pgfqpoint{260.6066bp}{163.6066bp}}
    \pgftransformscale{1.0000}
    \pgftext[]{$1/10$}
  \end{pgfscope}
  \begin{pgfscope}
    \pgfsetlinewidth{1.0229bp}
    \definecolor{sc}{rgb}{0.0000,0.0000,0.0000}
    \pgfsetstrokecolor{sc}
    \pgfsetmiterjoin
    \pgfsetbuttcap
    \pgfpathqmoveto{200.0000bp}{185.5000bp}
    \pgfpathqlineto{200.0000bp}{120.5000bp}
    \pgfusepathqstroke
  \end{pgfscope}
  \begin{pgfscope}
    \definecolor{fc}{rgb}{0.0000,0.0000,0.0000}
    \pgfsetfillcolor{fc}
    \pgfpathqmoveto{197.0000bp}{123.3868bp}
    \pgfpathqlineto{200.0000bp}{114.7265bp}
    \pgfpathqlineto{203.0000bp}{123.3868bp}
    \pgfpathqlineto{197.0000bp}{123.3868bp}
    \pgfpathclose
    \pgfusepathqfill
  \end{pgfscope}
  \begin{pgfscope}
    \definecolor{fc}{rgb}{0.0000,0.0000,0.0000}
    \pgfsetfillcolor{fc}
    \pgftransformshift{\pgfqpoint{215.0000bp}{153.0000bp}}
    \pgftransformscale{1.0000}
    \pgftext[]{$0/15$}
  \end{pgfscope}
  \begin{pgfscope}
    \pgfsetlinewidth{1.0229bp}
    \definecolor{sc}{rgb}{0.0000,0.0000,0.0000}
    \pgfsetstrokecolor{sc}
    \pgfsetmiterjoin
    \pgfsetbuttcap
    \pgfpathqmoveto{117.5000bp}{3.0000bp}
    \pgfpathqlineto{182.5000bp}{3.0000bp}
    \pgfusepathqstroke
  \end{pgfscope}
  \begin{pgfscope}
    \definecolor{fc}{rgb}{0.0000,0.0000,0.0000}
    \pgfsetfillcolor{fc}
    \pgfpathqmoveto{179.6132bp}{-0.0000bp}
    \pgfpathqlineto{188.2735bp}{3.0000bp}
    \pgfpathqlineto{179.6132bp}{6.0000bp}
    \pgfpathqlineto{179.6132bp}{-0.0000bp}
    \pgfpathclose
    \pgfusepathqfill
  \end{pgfscope}
  \begin{pgfscope}
    \definecolor{fc}{rgb}{0.0000,0.0000,0.0000}
    \pgfsetfillcolor{fc}
    \pgftransformshift{\pgfqpoint{150.0000bp}{18.0000bp}}
    \pgftransformscale{1.0000}
    \pgftext[]{$0/30$}
  \end{pgfscope}
  \begin{pgfscope}
    \pgfsetlinewidth{2.5573bp}
    \definecolor{sc}{rgb}{0.0000,0.0000,1.0000}
    \pgfsetstrokecolor{sc}
    \pgfsetmiterjoin
    \pgfsetbuttcap
    \pgfpathqmoveto{117.5000bp}{103.0000bp}
    \pgfpathqlineto{182.5000bp}{103.0000bp}
    \pgfusepathqstroke
  \end{pgfscope}
  \begin{pgfscope}
    \definecolor{fc}{rgb}{0.0000,0.0000,1.0000}
    \pgfsetfillcolor{fc}
    \pgfpathqmoveto{179.6132bp}{100.0000bp}
    \pgfpathqlineto{188.2735bp}{103.0000bp}
    \pgfpathqlineto{179.6132bp}{106.0000bp}
    \pgfpathqlineto{179.6132bp}{100.0000bp}
    \pgfpathclose
    \pgfusepathqfill
  \end{pgfscope}
  \begin{pgfscope}
    \definecolor{fc}{rgb}{0.0000,0.0000,0.0000}
    \pgfsetfillcolor{fc}
    \pgftransformshift{\pgfqpoint{150.0000bp}{118.0000bp}}
    \pgftransformscale{1.0000}
    \pgftext[]{$4/8$}
  \end{pgfscope}
  \begin{pgfscope}
    \pgfsetlinewidth{1.0229bp}
    \definecolor{sc}{rgb}{0.0000,0.0000,0.0000}
    \pgfsetstrokecolor{sc}
    \pgfsetmiterjoin
    \pgfsetbuttcap
    \pgfpathqmoveto{100.0000bp}{85.5000bp}
    \pgfpathqlineto{100.0000bp}{20.5000bp}
    \pgfusepathqstroke
  \end{pgfscope}
  \begin{pgfscope}
    \definecolor{fc}{rgb}{0.0000,0.0000,0.0000}
    \pgfsetfillcolor{fc}
    \pgfpathqmoveto{97.0000bp}{23.3868bp}
    \pgfpathqlineto{100.0000bp}{14.7265bp}
    \pgfpathqlineto{103.0000bp}{23.3868bp}
    \pgfpathqlineto{97.0000bp}{23.3868bp}
    \pgfpathclose
    \pgfusepathqfill
  \end{pgfscope}
  \begin{pgfscope}
    \definecolor{fc}{rgb}{0.0000,0.0000,0.0000}
    \pgfsetfillcolor{fc}
    \pgftransformshift{\pgfqpoint{115.0000bp}{53.0000bp}}
    \pgftransformscale{1.0000}
    \pgftext[]{$0/4$}
  \end{pgfscope}
  \begin{pgfscope}
    \pgfsetlinewidth{1.0229bp}
    \definecolor{sc}{rgb}{0.0000,0.0000,0.0000}
    \pgfsetstrokecolor{sc}
    \pgfsetmiterjoin
    \pgfsetbuttcap
    \pgfpathqmoveto{112.3744bp}{190.6256bp}
    \pgfpathqlineto{187.6256bp}{115.3744bp}
    \pgfusepathqstroke
  \end{pgfscope}
  \begin{pgfscope}
    \definecolor{fc}{rgb}{0.0000,0.0000,0.0000}
    \pgfsetfillcolor{fc}
    \pgfpathqmoveto{183.4631bp}{115.2943bp}
    \pgfpathqlineto{191.7081bp}{111.2919bp}
    \pgfpathqlineto{187.7057bp}{119.5369bp}
    \pgfpathqlineto{183.4631bp}{115.2943bp}
    \pgfpathclose
    \pgfusepathqfill
  \end{pgfscope}
  \begin{pgfscope}
    \definecolor{fc}{rgb}{0.0000,0.0000,0.0000}
    \pgfsetfillcolor{fc}
    \pgftransformshift{\pgfqpoint{160.6066bp}{163.6066bp}}
    \pgftransformscale{1.0000}
    \pgftext[]{$0/15$}
  \end{pgfscope}
  \begin{pgfscope}
    \pgfsetlinewidth{2.5573bp}
    \definecolor{sc}{rgb}{0.0000,0.0000,1.0000}
    \pgfsetstrokecolor{sc}
    \pgfsetmiterjoin
    \pgfsetbuttcap
    \pgfpathqmoveto{117.5000bp}{203.0000bp}
    \pgfpathqlineto{182.5000bp}{203.0000bp}
    \pgfusepathqstroke
  \end{pgfscope}
  \begin{pgfscope}
    \definecolor{fc}{rgb}{0.0000,0.0000,1.0000}
    \pgfsetfillcolor{fc}
    \pgfpathqmoveto{179.6132bp}{200.0000bp}
    \pgfpathqlineto{188.2735bp}{203.0000bp}
    \pgfpathqlineto{179.6132bp}{206.0000bp}
    \pgfpathqlineto{179.6132bp}{200.0000bp}
    \pgfpathclose
    \pgfusepathqfill
  \end{pgfscope}
  \begin{pgfscope}
    \definecolor{fc}{rgb}{0.0000,0.0000,0.0000}
    \pgfsetfillcolor{fc}
    \pgftransformshift{\pgfqpoint{150.0000bp}{218.0000bp}}
    \pgftransformscale{1.0000}
    \pgftext[]{$1/9$}
  \end{pgfscope}
  \begin{pgfscope}
    \pgfsetlinewidth{2.5573bp}
    \definecolor{sc}{rgb}{0.0000,0.0000,1.0000}
    \pgfsetstrokecolor{sc}
    \pgfsetmiterjoin
    \pgfsetbuttcap
    \pgfpathqmoveto{100.0000bp}{185.5000bp}
    \pgfpathqlineto{100.0000bp}{120.5000bp}
    \pgfusepathqstroke
  \end{pgfscope}
  \begin{pgfscope}
    \definecolor{fc}{rgb}{0.0000,0.0000,1.0000}
    \pgfsetfillcolor{fc}
    \pgfpathqmoveto{97.0000bp}{123.3868bp}
    \pgfpathqlineto{100.0000bp}{114.7265bp}
    \pgfpathqlineto{103.0000bp}{123.3868bp}
    \pgfpathqlineto{97.0000bp}{123.3868bp}
    \pgfpathclose
    \pgfusepathqfill
  \end{pgfscope}
  \begin{pgfscope}
    \definecolor{fc}{rgb}{0.0000,0.0000,0.0000}
    \pgfsetfillcolor{fc}
    \pgftransformshift{\pgfqpoint{115.0000bp}{153.0000bp}}
    \pgftransformscale{1.0000}
    \pgftext[]{$1/4$}
  \end{pgfscope}
  \begin{pgfscope}
    \definecolor{fc}{rgb}{0.0000,0.0000,0.0000}
    \pgfsetfillcolor{fc}
    \pgftransformshift{\pgfqpoint{300.0000bp}{103.0000bp}}
    \pgftransformscale{2.5000}
    \pgftext[]{$t$}
  \end{pgfscope}
  \begin{pgfscope}
    \definecolor{fc}{rgb}{0.0000,0.0000,0.0000}
    \pgfsetfillcolor{fc}
    \pgftransformshift{\pgfqpoint{0.0000bp}{103.0000bp}}
    \pgftransformscale{2.5000}
    \pgftext[]{$s$}
  \end{pgfscope}
  \begin{pgfscope}
    \definecolor{fc}{rgb}{0.0000,0.0000,0.0000}
    \pgfsetfillcolor{fc}
    \pgftransformshift{\pgfqpoint{200.0000bp}{3.0000bp}}
    \pgftransformscale{2.5000}
    \pgftext[]{$f$}
  \end{pgfscope}
  \begin{pgfscope}
    \definecolor{fc}{rgb}{0.0000,0.0000,0.0000}
    \pgfsetfillcolor{fc}
    \pgftransformshift{\pgfqpoint{200.0000bp}{103.0000bp}}
    \pgftransformscale{2.5000}
    \pgftext[]{$e$}
  \end{pgfscope}
  \begin{pgfscope}
    \definecolor{fc}{rgb}{0.0000,0.0000,0.0000}
    \pgfsetfillcolor{fc}
    \pgftransformshift{\pgfqpoint{200.0000bp}{203.0000bp}}
    \pgftransformscale{2.5000}
    \pgftext[]{$d$}
  \end{pgfscope}
  \begin{pgfscope}
    \definecolor{fc}{rgb}{0.0000,0.0000,0.0000}
    \pgfsetfillcolor{fc}
    \pgftransformshift{\pgfqpoint{100.0000bp}{3.0000bp}}
    \pgftransformscale{2.5000}
    \pgftext[]{$c$}
  \end{pgfscope}
  \begin{pgfscope}
    \definecolor{fc}{rgb}{0.0000,0.0000,0.0000}
    \pgfsetfillcolor{fc}
    \pgftransformshift{\pgfqpoint{100.0000bp}{103.0000bp}}
    \pgftransformscale{2.5000}
    \pgftext[]{$b$}
  \end{pgfscope}
  \begin{pgfscope}
    \definecolor{fc}{rgb}{0.0000,0.0000,0.0000}
    \pgfsetfillcolor{fc}
    \pgftransformshift{\pgfqpoint{100.0000bp}{203.0000bp}}
    \pgftransformscale{2.5000}
    \pgftext[]{$a$}
  \end{pgfscope}
\end{pgfpicture}

  \end{center}
\end{model}

\setcounter{modelcounter}{0}
\begin{model}{(continued)}{networksandflows}
  \begin{center}
    {\huge C}
    \begin{pgfpicture}
  \pgfpathrectangle{\pgfpointorigin}{\pgfqpoint{300.0000bp}{217.0000bp}}
  \pgfusepath{use as bounding box}
  \begin{pgfscope}
    \pgfsetlinewidth{1.0229bp}
    \definecolor{sc}{rgb}{0.0000,0.0000,0.0000}
    \pgfsetstrokecolor{sc}
    \pgfsetmiterjoin
    \pgfsetbuttcap
    \pgfpathqmoveto{12.3744bp}{90.6256bp}
    \pgfpathqlineto{87.6256bp}{15.3744bp}
    \pgfusepathqstroke
  \end{pgfscope}
  \begin{pgfscope}
    \definecolor{fc}{rgb}{0.0000,0.0000,0.0000}
    \pgfsetfillcolor{fc}
    \pgfpathqmoveto{83.4631bp}{15.2943bp}
    \pgfpathqlineto{91.7081bp}{11.2919bp}
    \pgfpathqlineto{87.7057bp}{19.5369bp}
    \pgfpathqlineto{83.4631bp}{15.2943bp}
    \pgfpathclose
    \pgfusepathqfill
  \end{pgfscope}
  \begin{pgfscope}
    \definecolor{fc}{rgb}{0.0000,0.0000,0.0000}
    \pgfsetfillcolor{fc}
    \pgftransformshift{\pgfqpoint{60.6066bp}{63.6066bp}}
    \pgftransformscale{1.0000}
    \pgftext[]{$0/15$}
  \end{pgfscope}
  \begin{pgfscope}
    \pgfsetlinewidth{1.0229bp}
    \definecolor{sc}{rgb}{0.0000,0.0000,0.0000}
    \pgfsetstrokecolor{sc}
    \pgfsetmiterjoin
    \pgfsetbuttcap
    \pgfpathqmoveto{17.5000bp}{103.0000bp}
    \pgfpathqlineto{82.5000bp}{103.0000bp}
    \pgfusepathqstroke
  \end{pgfscope}
  \begin{pgfscope}
    \definecolor{fc}{rgb}{0.0000,0.0000,0.0000}
    \pgfsetfillcolor{fc}
    \pgfpathqmoveto{79.6132bp}{100.0000bp}
    \pgfpathqlineto{88.2735bp}{103.0000bp}
    \pgfpathqlineto{79.6132bp}{106.0000bp}
    \pgfpathqlineto{79.6132bp}{100.0000bp}
    \pgfpathclose
    \pgfusepathqfill
  \end{pgfscope}
  \begin{pgfscope}
    \definecolor{fc}{rgb}{0.0000,0.0000,0.0000}
    \pgfsetfillcolor{fc}
    \pgftransformshift{\pgfqpoint{50.0000bp}{118.0000bp}}
    \pgftransformscale{1.0000}
    \pgftext[]{$0/5$}
  \end{pgfscope}
  \begin{pgfscope}
    \pgfsetlinewidth{2.5573bp}
    \definecolor{sc}{rgb}{0.0000,0.5020,0.0000}
    \pgfsetstrokecolor{sc}
    \pgfsetmiterjoin
    \pgfsetbuttcap
    \pgfpathqmoveto{12.3744bp}{115.3744bp}
    \pgfpathqlineto{87.6256bp}{190.6256bp}
    \pgfusepathqstroke
  \end{pgfscope}
  \begin{pgfscope}
    \definecolor{fc}{rgb}{0.0000,0.5020,0.0000}
    \pgfsetfillcolor{fc}
    \pgfpathqmoveto{87.7057bp}{186.4631bp}
    \pgfpathqlineto{91.7081bp}{194.7081bp}
    \pgfpathqlineto{83.4631bp}{190.7057bp}
    \pgfpathqlineto{87.7057bp}{186.4631bp}
    \pgfpathclose
    \pgfusepathqfill
  \end{pgfscope}
  \begin{pgfscope}
    \definecolor{fc}{rgb}{0.0000,0.0000,0.0000}
    \pgfsetfillcolor{fc}
    \pgftransformshift{\pgfqpoint{39.3934bp}{163.6066bp}}
    \pgftransformscale{1.0000}
    \pgftext[]{$10/10$}
  \end{pgfscope}
  \begin{pgfscope}
    \pgfsetlinewidth{1.0229bp}
    \definecolor{sc}{rgb}{0.0000,0.0000,0.0000}
    \pgfsetstrokecolor{sc}
    \pgfsetmiterjoin
    \pgfsetbuttcap
    \pgfpathqmoveto{212.3744bp}{15.3744bp}
    \pgfpathqlineto{287.6256bp}{90.6256bp}
    \pgfusepathqstroke
  \end{pgfscope}
  \begin{pgfscope}
    \definecolor{fc}{rgb}{0.0000,0.0000,0.0000}
    \pgfsetfillcolor{fc}
    \pgfpathqmoveto{287.7057bp}{86.4631bp}
    \pgfpathqlineto{291.7081bp}{94.7081bp}
    \pgfpathqlineto{283.4631bp}{90.7057bp}
    \pgfpathqlineto{287.7057bp}{86.4631bp}
    \pgfpathclose
    \pgfusepathqfill
  \end{pgfscope}
  \begin{pgfscope}
    \definecolor{fc}{rgb}{0.0000,0.0000,0.0000}
    \pgfsetfillcolor{fc}
    \pgftransformshift{\pgfqpoint{239.3934bp}{63.6066bp}}
    \pgftransformscale{1.0000}
    \pgftext[]{$0/10$}
  \end{pgfscope}
  \begin{pgfscope}
    \pgfsetlinewidth{1.0229bp}
    \definecolor{sc}{rgb}{0.0000,0.0000,0.0000}
    \pgfsetstrokecolor{sc}
    \pgfsetmiterjoin
    \pgfsetbuttcap
    \pgfpathqmoveto{187.6256bp}{15.3744bp}
    \pgfpathqlineto{112.3744bp}{90.6256bp}
    \pgfusepathqstroke
  \end{pgfscope}
  \begin{pgfscope}
    \definecolor{fc}{rgb}{0.0000,0.0000,0.0000}
    \pgfsetfillcolor{fc}
    \pgfpathqmoveto{116.5369bp}{90.7057bp}
    \pgfpathqlineto{108.2919bp}{94.7081bp}
    \pgfpathqlineto{112.2943bp}{86.4631bp}
    \pgfpathqlineto{116.5369bp}{90.7057bp}
    \pgfpathclose
    \pgfusepathqfill
  \end{pgfscope}
  \begin{pgfscope}
    \definecolor{fc}{rgb}{0.0000,0.0000,0.0000}
    \pgfsetfillcolor{fc}
    \pgftransformshift{\pgfqpoint{139.3934bp}{42.3934bp}}
    \pgftransformscale{1.0000}
    \pgftext[]{$0/6$}
  \end{pgfscope}
  \begin{pgfscope}
    \pgfsetlinewidth{2.5573bp}
    \definecolor{sc}{rgb}{0.0000,0.0000,1.0000}
    \pgfsetstrokecolor{sc}
    \pgfsetmiterjoin
    \pgfsetbuttcap
    \pgfpathqmoveto{217.5000bp}{103.0000bp}
    \pgfpathqlineto{282.5000bp}{103.0000bp}
    \pgfusepathqstroke
  \end{pgfscope}
  \begin{pgfscope}
    \definecolor{fc}{rgb}{0.0000,0.0000,1.0000}
    \pgfsetfillcolor{fc}
    \pgfpathqmoveto{279.6132bp}{100.0000bp}
    \pgfpathqlineto{288.2735bp}{103.0000bp}
    \pgfpathqlineto{279.6132bp}{106.0000bp}
    \pgfpathqlineto{279.6132bp}{100.0000bp}
    \pgfpathclose
    \pgfusepathqfill
  \end{pgfscope}
  \begin{pgfscope}
    \definecolor{fc}{rgb}{0.0000,0.0000,0.0000}
    \pgfsetfillcolor{fc}
    \pgftransformshift{\pgfqpoint{250.0000bp}{118.0000bp}}
    \pgftransformscale{1.0000}
    \pgftext[]{$1/10$}
  \end{pgfscope}
  \begin{pgfscope}
    \pgfsetlinewidth{1.0229bp}
    \definecolor{sc}{rgb}{0.0000,0.0000,0.0000}
    \pgfsetstrokecolor{sc}
    \pgfsetmiterjoin
    \pgfsetbuttcap
    \pgfpathqmoveto{200.0000bp}{85.5000bp}
    \pgfpathqlineto{200.0000bp}{20.5000bp}
    \pgfusepathqstroke
  \end{pgfscope}
  \begin{pgfscope}
    \definecolor{fc}{rgb}{0.0000,0.0000,0.0000}
    \pgfsetfillcolor{fc}
    \pgfpathqmoveto{197.0000bp}{23.3868bp}
    \pgfpathqlineto{200.0000bp}{14.7265bp}
    \pgfpathqlineto{203.0000bp}{23.3868bp}
    \pgfpathqlineto{197.0000bp}{23.3868bp}
    \pgfpathclose
    \pgfusepathqfill
  \end{pgfscope}
  \begin{pgfscope}
    \definecolor{fc}{rgb}{0.0000,0.0000,0.0000}
    \pgfsetfillcolor{fc}
    \pgftransformshift{\pgfqpoint{215.0000bp}{53.0000bp}}
    \pgftransformscale{1.0000}
    \pgftext[]{$0/15$}
  \end{pgfscope}
  \begin{pgfscope}
    \pgfsetlinewidth{2.5573bp}
    \definecolor{sc}{rgb}{0.0000,0.0000,1.0000}
    \pgfsetstrokecolor{sc}
    \pgfsetmiterjoin
    \pgfsetbuttcap
    \pgfpathqmoveto{212.3744bp}{190.6256bp}
    \pgfpathqlineto{287.6256bp}{115.3744bp}
    \pgfusepathqstroke
  \end{pgfscope}
  \begin{pgfscope}
    \definecolor{fc}{rgb}{0.0000,0.0000,1.0000}
    \pgfsetfillcolor{fc}
    \pgfpathqmoveto{283.4631bp}{115.2943bp}
    \pgfpathqlineto{291.7081bp}{111.2919bp}
    \pgfpathqlineto{287.7057bp}{119.5369bp}
    \pgfpathqlineto{283.4631bp}{115.2943bp}
    \pgfpathclose
    \pgfusepathqfill
  \end{pgfscope}
  \begin{pgfscope}
    \definecolor{fc}{rgb}{0.0000,0.0000,0.0000}
    \pgfsetfillcolor{fc}
    \pgftransformshift{\pgfqpoint{260.6066bp}{163.6066bp}}
    \pgftransformscale{1.0000}
    \pgftext[]{$9/10$}
  \end{pgfscope}
  \begin{pgfscope}
    \pgfsetlinewidth{1.0229bp}
    \definecolor{sc}{rgb}{0.0000,0.0000,0.0000}
    \pgfsetstrokecolor{sc}
    \pgfsetmiterjoin
    \pgfsetbuttcap
    \pgfpathqmoveto{200.0000bp}{185.5000bp}
    \pgfpathqlineto{200.0000bp}{120.5000bp}
    \pgfusepathqstroke
  \end{pgfscope}
  \begin{pgfscope}
    \definecolor{fc}{rgb}{0.0000,0.0000,0.0000}
    \pgfsetfillcolor{fc}
    \pgfpathqmoveto{197.0000bp}{123.3868bp}
    \pgfpathqlineto{200.0000bp}{114.7265bp}
    \pgfpathqlineto{203.0000bp}{123.3868bp}
    \pgfpathqlineto{197.0000bp}{123.3868bp}
    \pgfpathclose
    \pgfusepathqfill
  \end{pgfscope}
  \begin{pgfscope}
    \definecolor{fc}{rgb}{0.0000,0.0000,0.0000}
    \pgfsetfillcolor{fc}
    \pgftransformshift{\pgfqpoint{215.0000bp}{153.0000bp}}
    \pgftransformscale{1.0000}
    \pgftext[]{$0/15$}
  \end{pgfscope}
  \begin{pgfscope}
    \pgfsetlinewidth{1.0229bp}
    \definecolor{sc}{rgb}{0.0000,0.0000,0.0000}
    \pgfsetstrokecolor{sc}
    \pgfsetmiterjoin
    \pgfsetbuttcap
    \pgfpathqmoveto{117.5000bp}{3.0000bp}
    \pgfpathqlineto{182.5000bp}{3.0000bp}
    \pgfusepathqstroke
  \end{pgfscope}
  \begin{pgfscope}
    \definecolor{fc}{rgb}{0.0000,0.0000,0.0000}
    \pgfsetfillcolor{fc}
    \pgfpathqmoveto{179.6132bp}{-0.0000bp}
    \pgfpathqlineto{188.2735bp}{3.0000bp}
    \pgfpathqlineto{179.6132bp}{6.0000bp}
    \pgfpathqlineto{179.6132bp}{-0.0000bp}
    \pgfpathclose
    \pgfusepathqfill
  \end{pgfscope}
  \begin{pgfscope}
    \definecolor{fc}{rgb}{0.0000,0.0000,0.0000}
    \pgfsetfillcolor{fc}
    \pgftransformshift{\pgfqpoint{150.0000bp}{18.0000bp}}
    \pgftransformscale{1.0000}
    \pgftext[]{$0/30$}
  \end{pgfscope}
  \begin{pgfscope}
    \pgfsetlinewidth{1.0229bp}
    \definecolor{sc}{rgb}{0.0000,0.0000,0.0000}
    \pgfsetstrokecolor{sc}
    \pgfsetmiterjoin
    \pgfsetbuttcap
    \pgfpathqmoveto{117.5000bp}{103.0000bp}
    \pgfpathqlineto{182.5000bp}{103.0000bp}
    \pgfusepathqstroke
  \end{pgfscope}
  \begin{pgfscope}
    \definecolor{fc}{rgb}{0.0000,0.0000,0.0000}
    \pgfsetfillcolor{fc}
    \pgfpathqmoveto{179.6132bp}{100.0000bp}
    \pgfpathqlineto{188.2735bp}{103.0000bp}
    \pgfpathqlineto{179.6132bp}{106.0000bp}
    \pgfpathqlineto{179.6132bp}{100.0000bp}
    \pgfpathclose
    \pgfusepathqfill
  \end{pgfscope}
  \begin{pgfscope}
    \definecolor{fc}{rgb}{0.0000,0.0000,0.0000}
    \pgfsetfillcolor{fc}
    \pgftransformshift{\pgfqpoint{150.0000bp}{118.0000bp}}
    \pgftransformscale{1.0000}
    \pgftext[]{$0/8$}
  \end{pgfscope}
  \begin{pgfscope}
    \pgfsetlinewidth{1.0229bp}
    \definecolor{sc}{rgb}{0.0000,0.0000,0.0000}
    \pgfsetstrokecolor{sc}
    \pgfsetmiterjoin
    \pgfsetbuttcap
    \pgfpathqmoveto{100.0000bp}{85.5000bp}
    \pgfpathqlineto{100.0000bp}{20.5000bp}
    \pgfusepathqstroke
  \end{pgfscope}
  \begin{pgfscope}
    \definecolor{fc}{rgb}{0.0000,0.0000,0.0000}
    \pgfsetfillcolor{fc}
    \pgfpathqmoveto{97.0000bp}{23.3868bp}
    \pgfpathqlineto{100.0000bp}{14.7265bp}
    \pgfpathqlineto{103.0000bp}{23.3868bp}
    \pgfpathqlineto{97.0000bp}{23.3868bp}
    \pgfpathclose
    \pgfusepathqfill
  \end{pgfscope}
  \begin{pgfscope}
    \definecolor{fc}{rgb}{0.0000,0.0000,0.0000}
    \pgfsetfillcolor{fc}
    \pgftransformshift{\pgfqpoint{115.0000bp}{53.0000bp}}
    \pgftransformscale{1.0000}
    \pgftext[]{$0/4$}
  \end{pgfscope}
  \begin{pgfscope}
    \pgfsetlinewidth{2.5573bp}
    \definecolor{sc}{rgb}{0.0000,0.0000,1.0000}
    \pgfsetstrokecolor{sc}
    \pgfsetmiterjoin
    \pgfsetbuttcap
    \pgfpathqmoveto{112.3744bp}{190.6256bp}
    \pgfpathqlineto{187.6256bp}{115.3744bp}
    \pgfusepathqstroke
  \end{pgfscope}
  \begin{pgfscope}
    \definecolor{fc}{rgb}{0.0000,0.0000,1.0000}
    \pgfsetfillcolor{fc}
    \pgfpathqmoveto{183.4631bp}{115.2943bp}
    \pgfpathqlineto{191.7081bp}{111.2919bp}
    \pgfpathqlineto{187.7057bp}{119.5369bp}
    \pgfpathqlineto{183.4631bp}{115.2943bp}
    \pgfpathclose
    \pgfusepathqfill
  \end{pgfscope}
  \begin{pgfscope}
    \definecolor{fc}{rgb}{0.0000,0.0000,0.0000}
    \pgfsetfillcolor{fc}
    \pgftransformshift{\pgfqpoint{160.6066bp}{163.6066bp}}
    \pgftransformscale{1.0000}
    \pgftext[]{$1/15$}
  \end{pgfscope}
  \begin{pgfscope}
    \pgfsetlinewidth{2.5573bp}
    \definecolor{sc}{rgb}{0.0000,0.5020,0.0000}
    \pgfsetstrokecolor{sc}
    \pgfsetmiterjoin
    \pgfsetbuttcap
    \pgfpathqmoveto{117.5000bp}{203.0000bp}
    \pgfpathqlineto{182.5000bp}{203.0000bp}
    \pgfusepathqstroke
  \end{pgfscope}
  \begin{pgfscope}
    \definecolor{fc}{rgb}{0.0000,0.5020,0.0000}
    \pgfsetfillcolor{fc}
    \pgfpathqmoveto{179.6132bp}{200.0000bp}
    \pgfpathqlineto{188.2735bp}{203.0000bp}
    \pgfpathqlineto{179.6132bp}{206.0000bp}
    \pgfpathqlineto{179.6132bp}{200.0000bp}
    \pgfpathclose
    \pgfusepathqfill
  \end{pgfscope}
  \begin{pgfscope}
    \definecolor{fc}{rgb}{0.0000,0.0000,0.0000}
    \pgfsetfillcolor{fc}
    \pgftransformshift{\pgfqpoint{150.0000bp}{218.0000bp}}
    \pgftransformscale{1.0000}
    \pgftext[]{$9/9$}
  \end{pgfscope}
  \begin{pgfscope}
    \pgfsetlinewidth{1.0229bp}
    \definecolor{sc}{rgb}{0.0000,0.0000,0.0000}
    \pgfsetstrokecolor{sc}
    \pgfsetmiterjoin
    \pgfsetbuttcap
    \pgfpathqmoveto{100.0000bp}{185.5000bp}
    \pgfpathqlineto{100.0000bp}{120.5000bp}
    \pgfusepathqstroke
  \end{pgfscope}
  \begin{pgfscope}
    \definecolor{fc}{rgb}{0.0000,0.0000,0.0000}
    \pgfsetfillcolor{fc}
    \pgfpathqmoveto{97.0000bp}{123.3868bp}
    \pgfpathqlineto{100.0000bp}{114.7265bp}
    \pgfpathqlineto{103.0000bp}{123.3868bp}
    \pgfpathqlineto{97.0000bp}{123.3868bp}
    \pgfpathclose
    \pgfusepathqfill
  \end{pgfscope}
  \begin{pgfscope}
    \definecolor{fc}{rgb}{0.0000,0.0000,0.0000}
    \pgfsetfillcolor{fc}
    \pgftransformshift{\pgfqpoint{115.0000bp}{153.0000bp}}
    \pgftransformscale{1.0000}
    \pgftext[]{$0/4$}
  \end{pgfscope}
  \begin{pgfscope}
    \definecolor{fc}{rgb}{0.0000,0.0000,0.0000}
    \pgfsetfillcolor{fc}
    \pgftransformshift{\pgfqpoint{300.0000bp}{103.0000bp}}
    \pgftransformscale{2.5000}
    \pgftext[]{$t$}
  \end{pgfscope}
  \begin{pgfscope}
    \definecolor{fc}{rgb}{0.0000,0.0000,0.0000}
    \pgfsetfillcolor{fc}
    \pgftransformshift{\pgfqpoint{0.0000bp}{103.0000bp}}
    \pgftransformscale{2.5000}
    \pgftext[]{$s$}
  \end{pgfscope}
  \begin{pgfscope}
    \definecolor{fc}{rgb}{0.0000,0.0000,0.0000}
    \pgfsetfillcolor{fc}
    \pgftransformshift{\pgfqpoint{200.0000bp}{3.0000bp}}
    \pgftransformscale{2.5000}
    \pgftext[]{$f$}
  \end{pgfscope}
  \begin{pgfscope}
    \definecolor{fc}{rgb}{0.0000,0.0000,0.0000}
    \pgfsetfillcolor{fc}
    \pgftransformshift{\pgfqpoint{200.0000bp}{103.0000bp}}
    \pgftransformscale{2.5000}
    \pgftext[]{$e$}
  \end{pgfscope}
  \begin{pgfscope}
    \definecolor{fc}{rgb}{0.0000,0.0000,0.0000}
    \pgfsetfillcolor{fc}
    \pgftransformshift{\pgfqpoint{200.0000bp}{203.0000bp}}
    \pgftransformscale{2.5000}
    \pgftext[]{$d$}
  \end{pgfscope}
  \begin{pgfscope}
    \definecolor{fc}{rgb}{0.0000,0.0000,0.0000}
    \pgfsetfillcolor{fc}
    \pgftransformshift{\pgfqpoint{100.0000bp}{3.0000bp}}
    \pgftransformscale{2.5000}
    \pgftext[]{$c$}
  \end{pgfscope}
  \begin{pgfscope}
    \definecolor{fc}{rgb}{0.0000,0.0000,0.0000}
    \pgfsetfillcolor{fc}
    \pgftransformshift{\pgfqpoint{100.0000bp}{103.0000bp}}
    \pgftransformscale{2.5000}
    \pgftext[]{$b$}
  \end{pgfscope}
  \begin{pgfscope}
    \definecolor{fc}{rgb}{0.0000,0.0000,0.0000}
    \pgfsetfillcolor{fc}
    \pgftransformshift{\pgfqpoint{100.0000bp}{203.0000bp}}
    \pgftransformscale{2.5000}
    \pgftext[]{$a$}
  \end{pgfscope}
\end{pgfpicture}

  \end{center}
  \bigskip \bigskip

  \hrule \bigskip

  \begin{center}
    {\huge D}
    \begin{pgfpicture}
  \pgfpathrectangle{\pgfpointorigin}{\pgfqpoint{300.0000bp}{217.0000bp}}
  \pgfusepath{use as bounding box}
  \begin{pgfscope}
    \pgfsetlinewidth{1.0229bp}
    \definecolor{sc}{rgb}{0.0000,0.0000,0.0000}
    \pgfsetstrokecolor{sc}
    \pgfsetmiterjoin
    \pgfsetbuttcap
    \pgfpathqmoveto{12.3744bp}{90.6256bp}
    \pgfpathqlineto{87.6256bp}{15.3744bp}
    \pgfusepathqstroke
  \end{pgfscope}
  \begin{pgfscope}
    \definecolor{fc}{rgb}{0.0000,0.0000,0.0000}
    \pgfsetfillcolor{fc}
    \pgfpathqmoveto{83.4631bp}{15.2943bp}
    \pgfpathqlineto{91.7081bp}{11.2919bp}
    \pgfpathqlineto{87.7057bp}{19.5369bp}
    \pgfpathqlineto{83.4631bp}{15.2943bp}
    \pgfpathclose
    \pgfusepathqfill
  \end{pgfscope}
  \begin{pgfscope}
    \definecolor{fc}{rgb}{0.0000,0.0000,0.0000}
    \pgfsetfillcolor{fc}
    \pgftransformshift{\pgfqpoint{60.6066bp}{63.6066bp}}
    \pgftransformscale{1.0000}
    \pgftext[]{$0/15$}
  \end{pgfscope}
  \begin{pgfscope}
    \pgfsetlinewidth{2.5573bp}
    \definecolor{sc}{rgb}{0.0000,0.5020,0.0000}
    \pgfsetstrokecolor{sc}
    \pgfsetmiterjoin
    \pgfsetbuttcap
    \pgfpathqmoveto{17.5000bp}{103.0000bp}
    \pgfpathqlineto{82.5000bp}{103.0000bp}
    \pgfusepathqstroke
  \end{pgfscope}
  \begin{pgfscope}
    \definecolor{fc}{rgb}{0.0000,0.5020,0.0000}
    \pgfsetfillcolor{fc}
    \pgfpathqmoveto{79.6132bp}{100.0000bp}
    \pgfpathqlineto{88.2735bp}{103.0000bp}
    \pgfpathqlineto{79.6132bp}{106.0000bp}
    \pgfpathqlineto{79.6132bp}{100.0000bp}
    \pgfpathclose
    \pgfusepathqfill
  \end{pgfscope}
  \begin{pgfscope}
    \definecolor{fc}{rgb}{0.0000,0.0000,0.0000}
    \pgfsetfillcolor{fc}
    \pgftransformshift{\pgfqpoint{50.0000bp}{118.0000bp}}
    \pgftransformscale{1.0000}
    \pgftext[]{$5/5$}
  \end{pgfscope}
  \begin{pgfscope}
    \pgfsetlinewidth{2.5573bp}
    \definecolor{sc}{rgb}{0.0000,0.5020,0.0000}
    \pgfsetstrokecolor{sc}
    \pgfsetmiterjoin
    \pgfsetbuttcap
    \pgfpathqmoveto{12.3744bp}{115.3744bp}
    \pgfpathqlineto{87.6256bp}{190.6256bp}
    \pgfusepathqstroke
  \end{pgfscope}
  \begin{pgfscope}
    \definecolor{fc}{rgb}{0.0000,0.5020,0.0000}
    \pgfsetfillcolor{fc}
    \pgfpathqmoveto{87.7057bp}{186.4631bp}
    \pgfpathqlineto{91.7081bp}{194.7081bp}
    \pgfpathqlineto{83.4631bp}{190.7057bp}
    \pgfpathqlineto{87.7057bp}{186.4631bp}
    \pgfpathclose
    \pgfusepathqfill
  \end{pgfscope}
  \begin{pgfscope}
    \definecolor{fc}{rgb}{0.0000,0.0000,0.0000}
    \pgfsetfillcolor{fc}
    \pgftransformshift{\pgfqpoint{39.3934bp}{163.6066bp}}
    \pgftransformscale{1.0000}
    \pgftext[]{$10/10$}
  \end{pgfscope}
  \begin{pgfscope}
    \pgfsetlinewidth{1.0229bp}
    \definecolor{sc}{rgb}{0.0000,0.0000,0.0000}
    \pgfsetstrokecolor{sc}
    \pgfsetmiterjoin
    \pgfsetbuttcap
    \pgfpathqmoveto{212.3744bp}{15.3744bp}
    \pgfpathqlineto{287.6256bp}{90.6256bp}
    \pgfusepathqstroke
  \end{pgfscope}
  \begin{pgfscope}
    \definecolor{fc}{rgb}{0.0000,0.0000,0.0000}
    \pgfsetfillcolor{fc}
    \pgfpathqmoveto{287.7057bp}{86.4631bp}
    \pgfpathqlineto{291.7081bp}{94.7081bp}
    \pgfpathqlineto{283.4631bp}{90.7057bp}
    \pgfpathqlineto{287.7057bp}{86.4631bp}
    \pgfpathclose
    \pgfusepathqfill
  \end{pgfscope}
  \begin{pgfscope}
    \definecolor{fc}{rgb}{0.0000,0.0000,0.0000}
    \pgfsetfillcolor{fc}
    \pgftransformshift{\pgfqpoint{239.3934bp}{63.6066bp}}
    \pgftransformscale{1.0000}
    \pgftext[]{$0/10$}
  \end{pgfscope}
  \begin{pgfscope}
    \pgfsetlinewidth{1.0229bp}
    \definecolor{sc}{rgb}{0.0000,0.0000,0.0000}
    \pgfsetstrokecolor{sc}
    \pgfsetmiterjoin
    \pgfsetbuttcap
    \pgfpathqmoveto{187.6256bp}{15.3744bp}
    \pgfpathqlineto{112.3744bp}{90.6256bp}
    \pgfusepathqstroke
  \end{pgfscope}
  \begin{pgfscope}
    \definecolor{fc}{rgb}{0.0000,0.0000,0.0000}
    \pgfsetfillcolor{fc}
    \pgfpathqmoveto{116.5369bp}{90.7057bp}
    \pgfpathqlineto{108.2919bp}{94.7081bp}
    \pgfpathqlineto{112.2943bp}{86.4631bp}
    \pgfpathqlineto{116.5369bp}{90.7057bp}
    \pgfpathclose
    \pgfusepathqfill
  \end{pgfscope}
  \begin{pgfscope}
    \definecolor{fc}{rgb}{0.0000,0.0000,0.0000}
    \pgfsetfillcolor{fc}
    \pgftransformshift{\pgfqpoint{139.3934bp}{42.3934bp}}
    \pgftransformscale{1.0000}
    \pgftext[]{$0/6$}
  \end{pgfscope}
  \begin{pgfscope}
    \pgfsetlinewidth{2.5573bp}
    \definecolor{sc}{rgb}{0.0000,0.0000,1.0000}
    \pgfsetstrokecolor{sc}
    \pgfsetmiterjoin
    \pgfsetbuttcap
    \pgfpathqmoveto{217.5000bp}{103.0000bp}
    \pgfpathqlineto{282.5000bp}{103.0000bp}
    \pgfusepathqstroke
  \end{pgfscope}
  \begin{pgfscope}
    \definecolor{fc}{rgb}{0.0000,0.0000,1.0000}
    \pgfsetfillcolor{fc}
    \pgfpathqmoveto{279.6132bp}{100.0000bp}
    \pgfpathqlineto{288.2735bp}{103.0000bp}
    \pgfpathqlineto{279.6132bp}{106.0000bp}
    \pgfpathqlineto{279.6132bp}{100.0000bp}
    \pgfpathclose
    \pgfusepathqfill
  \end{pgfscope}
  \begin{pgfscope}
    \definecolor{fc}{rgb}{0.0000,0.0000,0.0000}
    \pgfsetfillcolor{fc}
    \pgftransformshift{\pgfqpoint{250.0000bp}{118.0000bp}}
    \pgftransformscale{1.0000}
    \pgftext[]{$8/10$}
  \end{pgfscope}
  \begin{pgfscope}
    \pgfsetlinewidth{1.0229bp}
    \definecolor{sc}{rgb}{0.0000,0.0000,0.0000}
    \pgfsetstrokecolor{sc}
    \pgfsetmiterjoin
    \pgfsetbuttcap
    \pgfpathqmoveto{200.0000bp}{85.5000bp}
    \pgfpathqlineto{200.0000bp}{20.5000bp}
    \pgfusepathqstroke
  \end{pgfscope}
  \begin{pgfscope}
    \definecolor{fc}{rgb}{0.0000,0.0000,0.0000}
    \pgfsetfillcolor{fc}
    \pgfpathqmoveto{197.0000bp}{23.3868bp}
    \pgfpathqlineto{200.0000bp}{14.7265bp}
    \pgfpathqlineto{203.0000bp}{23.3868bp}
    \pgfpathqlineto{197.0000bp}{23.3868bp}
    \pgfpathclose
    \pgfusepathqfill
  \end{pgfscope}
  \begin{pgfscope}
    \definecolor{fc}{rgb}{0.0000,0.0000,0.0000}
    \pgfsetfillcolor{fc}
    \pgftransformshift{\pgfqpoint{215.0000bp}{53.0000bp}}
    \pgftransformscale{1.0000}
    \pgftext[]{$0/15$}
  \end{pgfscope}
  \begin{pgfscope}
    \pgfsetlinewidth{2.5573bp}
    \definecolor{sc}{rgb}{0.0000,0.0000,1.0000}
    \pgfsetstrokecolor{sc}
    \pgfsetmiterjoin
    \pgfsetbuttcap
    \pgfpathqmoveto{212.3744bp}{190.6256bp}
    \pgfpathqlineto{287.6256bp}{115.3744bp}
    \pgfusepathqstroke
  \end{pgfscope}
  \begin{pgfscope}
    \definecolor{fc}{rgb}{0.0000,0.0000,1.0000}
    \pgfsetfillcolor{fc}
    \pgfpathqmoveto{283.4631bp}{115.2943bp}
    \pgfpathqlineto{291.7081bp}{111.2919bp}
    \pgfpathqlineto{287.7057bp}{119.5369bp}
    \pgfpathqlineto{283.4631bp}{115.2943bp}
    \pgfpathclose
    \pgfusepathqfill
  \end{pgfscope}
  \begin{pgfscope}
    \definecolor{fc}{rgb}{0.0000,0.0000,0.0000}
    \pgfsetfillcolor{fc}
    \pgftransformshift{\pgfqpoint{260.6066bp}{163.6066bp}}
    \pgftransformscale{1.0000}
    \pgftext[]{$9/10$}
  \end{pgfscope}
  \begin{pgfscope}
    \pgfsetlinewidth{2.5573bp}
    \definecolor{sc}{rgb}{0.0000,0.0000,1.0000}
    \pgfsetstrokecolor{sc}
    \pgfsetmiterjoin
    \pgfsetbuttcap
    \pgfpathqmoveto{200.0000bp}{185.5000bp}
    \pgfpathqlineto{200.0000bp}{120.5000bp}
    \pgfusepathqstroke
  \end{pgfscope}
  \begin{pgfscope}
    \definecolor{fc}{rgb}{0.0000,0.0000,1.0000}
    \pgfsetfillcolor{fc}
    \pgfpathqmoveto{197.0000bp}{123.3868bp}
    \pgfpathqlineto{200.0000bp}{114.7265bp}
    \pgfpathqlineto{203.0000bp}{123.3868bp}
    \pgfpathqlineto{197.0000bp}{123.3868bp}
    \pgfpathclose
    \pgfusepathqfill
  \end{pgfscope}
  \begin{pgfscope}
    \definecolor{fc}{rgb}{0.0000,0.0000,0.0000}
    \pgfsetfillcolor{fc}
    \pgftransformshift{\pgfqpoint{215.0000bp}{153.0000bp}}
    \pgftransformscale{1.0000}
    \pgftext[]{$1/15$}
  \end{pgfscope}
  \begin{pgfscope}
    \pgfsetlinewidth{1.0229bp}
    \definecolor{sc}{rgb}{0.0000,0.0000,0.0000}
    \pgfsetstrokecolor{sc}
    \pgfsetmiterjoin
    \pgfsetbuttcap
    \pgfpathqmoveto{117.5000bp}{3.0000bp}
    \pgfpathqlineto{182.5000bp}{3.0000bp}
    \pgfusepathqstroke
  \end{pgfscope}
  \begin{pgfscope}
    \definecolor{fc}{rgb}{0.0000,0.0000,0.0000}
    \pgfsetfillcolor{fc}
    \pgfpathqmoveto{179.6132bp}{-0.0000bp}
    \pgfpathqlineto{188.2735bp}{3.0000bp}
    \pgfpathqlineto{179.6132bp}{6.0000bp}
    \pgfpathqlineto{179.6132bp}{-0.0000bp}
    \pgfpathclose
    \pgfusepathqfill
  \end{pgfscope}
  \begin{pgfscope}
    \definecolor{fc}{rgb}{0.0000,0.0000,0.0000}
    \pgfsetfillcolor{fc}
    \pgftransformshift{\pgfqpoint{150.0000bp}{18.0000bp}}
    \pgftransformscale{1.0000}
    \pgftext[]{$0/30$}
  \end{pgfscope}
  \begin{pgfscope}
    \pgfsetlinewidth{2.5573bp}
    \definecolor{sc}{rgb}{0.0000,0.0000,1.0000}
    \pgfsetstrokecolor{sc}
    \pgfsetmiterjoin
    \pgfsetbuttcap
    \pgfpathqmoveto{117.5000bp}{103.0000bp}
    \pgfpathqlineto{182.5000bp}{103.0000bp}
    \pgfusepathqstroke
  \end{pgfscope}
  \begin{pgfscope}
    \definecolor{fc}{rgb}{0.0000,0.0000,1.0000}
    \pgfsetfillcolor{fc}
    \pgfpathqmoveto{179.6132bp}{100.0000bp}
    \pgfpathqlineto{188.2735bp}{103.0000bp}
    \pgfpathqlineto{179.6132bp}{106.0000bp}
    \pgfpathqlineto{179.6132bp}{100.0000bp}
    \pgfpathclose
    \pgfusepathqfill
  \end{pgfscope}
  \begin{pgfscope}
    \definecolor{fc}{rgb}{0.0000,0.0000,0.0000}
    \pgfsetfillcolor{fc}
    \pgftransformshift{\pgfqpoint{150.0000bp}{118.0000bp}}
    \pgftransformscale{1.0000}
    \pgftext[]{$7/8$}
  \end{pgfscope}
  \begin{pgfscope}
    \pgfsetlinewidth{1.0229bp}
    \definecolor{sc}{rgb}{0.0000,0.0000,0.0000}
    \pgfsetstrokecolor{sc}
    \pgfsetmiterjoin
    \pgfsetbuttcap
    \pgfpathqmoveto{100.0000bp}{85.5000bp}
    \pgfpathqlineto{100.0000bp}{20.5000bp}
    \pgfusepathqstroke
  \end{pgfscope}
  \begin{pgfscope}
    \definecolor{fc}{rgb}{0.0000,0.0000,0.0000}
    \pgfsetfillcolor{fc}
    \pgfpathqmoveto{97.0000bp}{23.3868bp}
    \pgfpathqlineto{100.0000bp}{14.7265bp}
    \pgfpathqlineto{103.0000bp}{23.3868bp}
    \pgfpathqlineto{97.0000bp}{23.3868bp}
    \pgfpathclose
    \pgfusepathqfill
  \end{pgfscope}
  \begin{pgfscope}
    \definecolor{fc}{rgb}{0.0000,0.0000,0.0000}
    \pgfsetfillcolor{fc}
    \pgftransformshift{\pgfqpoint{115.0000bp}{53.0000bp}}
    \pgftransformscale{1.0000}
    \pgftext[]{$0/4$}
  \end{pgfscope}
  \begin{pgfscope}
    \pgfsetlinewidth{1.0229bp}
    \definecolor{sc}{rgb}{0.0000,0.0000,0.0000}
    \pgfsetstrokecolor{sc}
    \pgfsetmiterjoin
    \pgfsetbuttcap
    \pgfpathqmoveto{112.3744bp}{190.6256bp}
    \pgfpathqlineto{187.6256bp}{115.3744bp}
    \pgfusepathqstroke
  \end{pgfscope}
  \begin{pgfscope}
    \definecolor{fc}{rgb}{0.0000,0.0000,0.0000}
    \pgfsetfillcolor{fc}
    \pgfpathqmoveto{183.4631bp}{115.2943bp}
    \pgfpathqlineto{191.7081bp}{111.2919bp}
    \pgfpathqlineto{187.7057bp}{119.5369bp}
    \pgfpathqlineto{183.4631bp}{115.2943bp}
    \pgfpathclose
    \pgfusepathqfill
  \end{pgfscope}
  \begin{pgfscope}
    \definecolor{fc}{rgb}{0.0000,0.0000,0.0000}
    \pgfsetfillcolor{fc}
    \pgftransformshift{\pgfqpoint{160.6066bp}{163.6066bp}}
    \pgftransformscale{1.0000}
    \pgftext[]{$0/15$}
  \end{pgfscope}
  \begin{pgfscope}
    \pgfsetlinewidth{2.5573bp}
    \definecolor{sc}{rgb}{0.0000,0.5020,0.0000}
    \pgfsetstrokecolor{sc}
    \pgfsetmiterjoin
    \pgfsetbuttcap
    \pgfpathqmoveto{117.5000bp}{203.0000bp}
    \pgfpathqlineto{182.5000bp}{203.0000bp}
    \pgfusepathqstroke
  \end{pgfscope}
  \begin{pgfscope}
    \definecolor{fc}{rgb}{0.0000,0.5020,0.0000}
    \pgfsetfillcolor{fc}
    \pgfpathqmoveto{179.6132bp}{200.0000bp}
    \pgfpathqlineto{188.2735bp}{203.0000bp}
    \pgfpathqlineto{179.6132bp}{206.0000bp}
    \pgfpathqlineto{179.6132bp}{200.0000bp}
    \pgfpathclose
    \pgfusepathqfill
  \end{pgfscope}
  \begin{pgfscope}
    \definecolor{fc}{rgb}{0.0000,0.0000,0.0000}
    \pgfsetfillcolor{fc}
    \pgftransformshift{\pgfqpoint{150.0000bp}{218.0000bp}}
    \pgftransformscale{1.0000}
    \pgftext[]{$10/9$}
  \end{pgfscope}
  \begin{pgfscope}
    \pgfsetlinewidth{1.0229bp}
    \definecolor{sc}{rgb}{0.0000,0.0000,0.0000}
    \pgfsetstrokecolor{sc}
    \pgfsetmiterjoin
    \pgfsetbuttcap
    \pgfpathqmoveto{100.0000bp}{185.5000bp}
    \pgfpathqlineto{100.0000bp}{120.5000bp}
    \pgfusepathqstroke
  \end{pgfscope}
  \begin{pgfscope}
    \definecolor{fc}{rgb}{0.0000,0.0000,0.0000}
    \pgfsetfillcolor{fc}
    \pgfpathqmoveto{97.0000bp}{123.3868bp}
    \pgfpathqlineto{100.0000bp}{114.7265bp}
    \pgfpathqlineto{103.0000bp}{123.3868bp}
    \pgfpathqlineto{97.0000bp}{123.3868bp}
    \pgfpathclose
    \pgfusepathqfill
  \end{pgfscope}
  \begin{pgfscope}
    \definecolor{fc}{rgb}{0.0000,0.0000,0.0000}
    \pgfsetfillcolor{fc}
    \pgftransformshift{\pgfqpoint{115.0000bp}{153.0000bp}}
    \pgftransformscale{1.0000}
    \pgftext[]{$0/4$}
  \end{pgfscope}
  \begin{pgfscope}
    \definecolor{fc}{rgb}{0.0000,0.0000,0.0000}
    \pgfsetfillcolor{fc}
    \pgftransformshift{\pgfqpoint{300.0000bp}{103.0000bp}}
    \pgftransformscale{2.5000}
    \pgftext[]{$t$}
  \end{pgfscope}
  \begin{pgfscope}
    \definecolor{fc}{rgb}{0.0000,0.0000,0.0000}
    \pgfsetfillcolor{fc}
    \pgftransformshift{\pgfqpoint{0.0000bp}{103.0000bp}}
    \pgftransformscale{2.5000}
    \pgftext[]{$s$}
  \end{pgfscope}
  \begin{pgfscope}
    \definecolor{fc}{rgb}{0.0000,0.0000,0.0000}
    \pgfsetfillcolor{fc}
    \pgftransformshift{\pgfqpoint{200.0000bp}{3.0000bp}}
    \pgftransformscale{2.5000}
    \pgftext[]{$f$}
  \end{pgfscope}
  \begin{pgfscope}
    \definecolor{fc}{rgb}{0.0000,0.0000,0.0000}
    \pgfsetfillcolor{fc}
    \pgftransformshift{\pgfqpoint{200.0000bp}{103.0000bp}}
    \pgftransformscale{2.5000}
    \pgftext[]{$e$}
  \end{pgfscope}
  \begin{pgfscope}
    \definecolor{fc}{rgb}{0.0000,0.0000,0.0000}
    \pgfsetfillcolor{fc}
    \pgftransformshift{\pgfqpoint{200.0000bp}{203.0000bp}}
    \pgftransformscale{2.5000}
    \pgftext[]{$d$}
  \end{pgfscope}
  \begin{pgfscope}
    \definecolor{fc}{rgb}{0.0000,0.0000,0.0000}
    \pgfsetfillcolor{fc}
    \pgftransformshift{\pgfqpoint{100.0000bp}{3.0000bp}}
    \pgftransformscale{2.5000}
    \pgftext[]{$c$}
  \end{pgfscope}
  \begin{pgfscope}
    \definecolor{fc}{rgb}{0.0000,0.0000,0.0000}
    \pgfsetfillcolor{fc}
    \pgftransformshift{\pgfqpoint{100.0000bp}{103.0000bp}}
    \pgftransformscale{2.5000}
    \pgftext[]{$b$}
  \end{pgfscope}
  \begin{pgfscope}
    \definecolor{fc}{rgb}{0.0000,0.0000,0.0000}
    \pgfsetfillcolor{fc}
    \pgftransformshift{\pgfqpoint{100.0000bp}{203.0000bp}}
    \pgftransformscale{2.5000}
    \pgftext[]{$a$}
  \end{pgfscope}
\end{pgfpicture}

  \end{center}
\end{model}

Consider graph $A$. Once again we have a directed graph with weighted
edges.  However, instead of thinking of the weights as some sort of
length, we will now think of them as a \emph{capacity}: the ``maximum
amount of stuff'' that the edge can carry.  For example, the capacity
might be used to model things like:
\begin{itemize}
\item maximum gallons of water per minute that can flow through a pipe;
\item maximum number of trucks per hour that can drive along a road; or
\item maximum number of times a certain resource can be used before it
  is all used up.
\end{itemize}

\begin{questions}
\item Consider graph $B$.  How is it related to graph $A$?
\item What do the blue edges in graph $B$ all have in common?
\item What do you think the labels on the edges of graph $B$
  represent?
\item Now consider graph $C$.  Why do you think some of the edges are
  green?
\item Graph $D$ is invalid!  In fact, there are two things wrong with
  it.  What are they?
\end{questions}

\newcommand{\R}{\mathbb{R}}

\pause
\begin{defn}
  A \term{flow network} is a directed graph $G = (V,E)$ with
  \begin{itemize}
  \item a distinguished \term{source} vertex $s \in V$, with indegree 0;
  \item a distinguished \term{sink} or \term{target} vertex $t \in V$,
    with outdegree 0;
  \item a \term{capacity function} $c : E \to \R^+$ assigning a
    non-negative real number capacity $c(e)$ to each edge $e \in E$.
  \end{itemize}
\end{defn}

\begin{questions}
  \item Is graph $A$ a flow network?  Why or why not?
\end{questions}

Now let's define a \term{flow}.  Both graphs $B$ and $C$ depict valid
flows on $A$; graph $D$ does not.

\begin{defn}
  A \term{flow} on a flow network $G$ is a function $f : E \to \R^+$
  assigning a non-negative flow $f(e)$ to each edge, such that
  \begin{enumerate}
  \item $\underline{\phantom{XXXXXXXX}} \leq f(e) \leq
    \underline{\phantom{XXXXXXXX}}$ for every $e \in E$
  \item At each vertex $v \in V$ other than $s$ and $t$, \blank
    \newline \blank.
  \end{enumerate}

  Make sure graphs $B$ and $C$ are valid flows according to your
  definition, and that there are two different reasons why $D$ is
  invalid according to your definition.
\end{defn}

\begin{defn}
  The \term{value} of a flow, $v(f)$, is the sum of the flow on all
  edges leaving $s$.
\end{defn}

\begin{questions}
  \item What is the value of the flow on graph $B$?
  \item What is the value of the flow on graph $C$?
  \item Make a conjecture about the relationship between the value of
    a flow and the amount of flow entering $t$.
  \item For each amount, say whether you can construct a flow on graph
    $A$ with the given value.
    \begin{enumerate}[label=(\alph*)]
    \item $15$ \vspace{1in}
    \item $40$ \vspace{1in}
    \item $30$ \vspace{1in}
    \end{enumerate}
  \item What is the value of the biggest flow you can construct on
    graph $A$?
  \item (Bonus question) Brainstorm how you might create an algorithm
    to find the biggest possible flow for a given flow network.
\end{questions}

\end{document}
