% -*- compile-command: "pdflatex 05-LSS.tex" -*-
\documentclass{tufte-handout}

\usepackage{../algo-activity}

\title{\thecourse\ Activity nn: Largest Sum Subsequence}
\date{}

\begin{document}

\maketitle

\section{Model 1}

Consider these sequences of integers:
\begin{align*}
  A &= [10, 20, -50, 40] \\
  B &= [-5, 3, -2, 4, -1, 8, -20] \\
  C &= [1, 2, 3, 4] \\
  D &= [-1, -2, -3, -4]
\end{align*}

\begin{questions}
  \item Find the sum of each of the XXX
  \item What do these have in common: (some subsequences, some
    not subsequences.
\end{questions}

Let $A = [a_1, a_2, \dots, a_n]$ be a sequence of $n$ integers.

XXX

\begin{defn}
  A \term{subsequence} of $A$ is defined to be a sequence $[a_i,
  a_{i+1}, \dots, a_j]$ of zero or more \emph{consecutive} elements
  from $A$.
\end{defn}

\begin{questions}
  \item How many different subsequences does $A$ have?
  \item How many different subsequences does $B$ have?
  \item In general, how many different subsequences does a sequence of
    length $n$ have? \marginnote{Don't worry about the fact that some
      of the subsequences might not be distinct---we'll suppose that
      all of the elements of the sequence are distinct.}
\end{questions}

\begin{defn}
  The \term{largest sum subsequence} (LSS) is defined to be the
  subsequence with the largest possible sum.
\end{defn}

\begin{questions}
  \item Find the LSS for each of $A$, $B$, $C$, and $D$.
  \item Explain why finding the LSS becomes boring if we have
    sequences of \emph{natural numbers} instead of \emph{integers}.
\end{questions}

\begin{defn}
  The \term{LSS problem}, given a sequence of integers $A$, is to find
  the indices $i, j$ such that $a_i, \dots, a_j$ is the LSS of $A$.
\end{defn}

\begin{questions}
\item Explain how this problem fits into our framework for algorithmic
  problems. (XXX check this)
\item Describe a brute-force solution in English.
\item Now write some pseudocode for your brute-force algorithm. \vspace{2in}
\item Using big-$O$ notation, derive an upper bound for the running
  time of your algorithm.
\item XXX Wolfram Alpha --- derive big-Theta?
\end{questions}

\end{document}
