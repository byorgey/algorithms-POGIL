% -*- compile-command: "pdflatex POGIL-intro-plan.tex" -*-
\documentclass{tufte-handout}

\usepackage{../algo-activity}

\title{\thecourse: Introduction (Facilitation Plan)}
\date{}

\begin{document}

\maketitle

\section{Learning Objectives}

\subsection{Content objectives}

\begin{itemize}
\item Students will be able to list and explain the roles in a POGIL
  classroom.
\item Students will be able to explain course policies for weekly problem
  sets, late days, exams, and academic integrity.
\item Students will analyze problems in terms of inputs and outputs.
\item Students will write brute-force algorithms to solve search
  problems.
\end{itemize}

\subsection{Process objectives}

\section{Intro \& getting started (5m)}

\section{Introduction to POGIL (10m)}

\section{CSCI 382 Syllabus (10m + 5m discussion)}

Make chart of last Q on board (ac integrity), have reporters fill it
in, discuss.  This will probably take a while.

\section{Brute-Force Algorithms (15m)}

Report out on definitions of brute force.

\section{Closure (5m)}

A lot of this semester is going to focus on approaches and techniques
for doing better than brute-force.  But we will also explore limits:
for example, for scenario (c) we strongly suspect that it's not
possible to do fundamentally better than brute force!

\end{document}
