% -*- compile-command: "pdflatex POGIL-intro.tex" -*-

% History:
%   - 16c182a: used W, 23 Aug of F17.
%   - 978f0fb: used W, 22 Aug of F18.

\documentclass{tufte-handout}

\usepackage{../algo-activity}

\title{\thecourse: Introduction}
\date{}

\begin{document}

\maketitle

\section{Introduction to POGIL and CSCI 382}

Welcome to CSCI 382, Algorithms!  This semester we will often use a
methodology called POGIL (Process-Oriented Guided Inquiry
Learning).\marginnote{\url{http://www.pogil.org}} Before we jump into
learning about algorithms, we'll take a bit of time to learn about how
POGIL works. (We will leave the question of \emph{why} we are using
POGIL until next time.)  You'll also explore the syllabus of CSCI 382
so you know what to expect this semester.

\begin{objective}
  Students will be able to list and explain the roles in a POGIL
  classroom.
\end{objective}

% Make sure you are sitting in a team of three or four people.  If not,
% talk to your instructor.

You should be in a group of three or four people.  If not, either you
are trying to do this activity on your own (don't even think about
it!) or something has gone terribly wrong.

You should also have some paper and something to write with handy.

\begin{questions}
\item If not everyone on your team knows each other already, begin by
  introducing yourselves.
\end{questions}

Each member of a POGIL learning team has a specific \textbf{role},
although the roles will change from day to day.  Figure out which
member of your team has most recently had a birthday.  That person
will be the \textbf{manager} for today. The other members of your
team, in alphabetical order by first name, will be the
\textbf{recorder}, the \textbf{reporter}, and the \textbf{reflector},
respectively.  If your team only has three members, one person should
take the roles of both recorder and reporter.

\begin{questions}
\item Figure out which team member has each role for today, and write
  down the roles.
\item
  In the ``Files'' tab of your group channel, you should find some
  \textbf{role cards} which explain each of the roles.
  % The manager should get the provided \textbf{role cards} and
  % distribute them appropriately.
  Take two minutes to read over the role card(s) corresponding to your
  role(s).  What are the important functions of your role(s)?  Write
  down a summary.
\item Each team member should take \textbf{one minute} to explain the
  important aspects of their role(s) to the rest of the team.
\end{questions}

\newpage
\begin{model*}{CSCI 382 Syllabus}{syllabus}
  Visit the CSCI 382 course web page here: \bigskip

  \url{http://Hendrix-CS.github.io/csci382/}
\end{model*}

\begin{objective}
  Students will be able to explain course policies for weekly problem
  sets, late days, exams, and academic integrity.
\end{objective}

Say whether each statement below is true or false.  If false, explain
what a correct version of the statement would be.

Remember that you should \textbf{work together} to come up with
responses to the questions.  Make sure all team members agree before
moving on.

\begin{questions}

  \item Problem sets are due every Friday at 4pm, and should be turned
    in on Moodle.

  \item Problem sets can be submitted as either \texttt{.pdf} or
    \texttt{.doc} files.

  \item One week, Amanda has a virtual play performance, three midterm
    exams, a 100-page paper due, and a family trip.  As a result, she
    is not able to finish her Algorithms problem set before the
    deadline.  She must either turn in whatever she has by the
    deadline, or else take a zero on the assignment.
  \item There are three exams in the course, which together account
    for about 50\% of your final grade.
  \item In this course, which of the following are considered academic
    integrity violations? \marginnote{Hint: be sure to also look at
      \url{http://ozark.hendrix.edu/~yorgey/ac-integrity-policy.html}
      (which is also linked from the syllabus).}
    \begin{compactenum}[(a)]
      \item collaborating with another student on a weekly problem set
      \item collaborating with another student to write up solutions
        to a weekly problem set
      \item looking at another student's code
      \item referring to an online resource
      \item referring to an online resource without citing it
      \item collaborating with another student while preparing for the
        final exam
      \item obtaining a copy of exam questions before an exam
      \end{compactenum} \bigskip

    \item Finally, as a group, come up with one question you have for
      me about the syllabus or about the class in general. Find the
      link to Piazza (either on the course web page or in a tab in the
      CSCI 382 Algorithms Team) and post your question there.
\end{questions}

\end{document}
