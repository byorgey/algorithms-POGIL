% -*- compile-command: "rubber -d --unsafe POGIL-intro.tex" -*-

% History:
%   - 16c182a: used W, 23 Aug of F17.
%   - 978f0fb: used W, 22 Aug of F18.

\documentclass{tufte-handout}

\usepackage{algo-activity}

\title{\thecourse: Introduction}
\date{}

\begin{document}

\maketitle

\section{Introduction to POGIL and CSCI 382}

Welcome to CSCI 382, Algorithms!  This semester we will often use a
methodology called POGIL (Process-Oriented Guided Inquiry
Learning).\marginnote{\url{http://www.pogil.org}} Before we jump into
learning about algorithms, we'll take a bit of time to learn about how
POGIL works. (We will leave the question of \emph{why} we are using
POGIL until next time.)  You'll also explore the syllabus of CSCI 382
so you know what to expect this semester.

\newcommand{\lobjectiveA}{I can list and explain the roles in a POGIL
  classroom.}
\begin{objective}
  \lobjectiveA
\end{objective}

\newcommand{\lobjectiveB}{I can explain course
  policies for weekly problem sets, grading, exams, and academic
  integrity.}

\begin{objective}
  \lobjectiveB
\end{objective}

You should be in a group of three or four people.  If not, either you
are trying to do this activity on your own (don't even think about
it!) or something has gone terribly wrong.  You should also have
something to write with handy.

\begin{questions}
\item If not everyone on your team knows each other already, begin by
  introducing yourselves.
\end{questions}

Each member of a POGIL learning team has a specific \textbf{role},
although the roles will change from day to day.  Figure out which
member of your team has most recently had a birthday.  That person
will be the \textbf{manager} for today. The other members of your
team, in alphabetical order by first name, will be the
\textbf{recorder}, the \textbf{reporter}, and the \textbf{reflector},
respectively.  If your team only has three members, one person should
take the roles of both reporter and reflector.

\begin{questions}
\item Figure out which team member has each role for today, and write
  down the roles.
\item The \textbf{manager} should go ask for \textbf{role cards}
  from your instructor and distribute them appropriately.  Take two
  minutes to read over the role card(s) corresponding to your role(s).
\item Each team member should take \textbf{one minute} to explain the
  important aspects of their role(s) to the rest of the team.
\end{questions}

\newpage
\begin{model*}{CSCI 382 Syllabus}{syllabus}
  Using a phone, laptop, or other suitable device, visit the CSCI 382
  course web page at the following URL (you will most likely want to
  bookmark it!): \bigskip

  \url{http://hendrix-cs.github.io/csci382/}
\end{model*}

Say whether each statement below is true or false.  If false, explain
what a correct version of the statement would be.

Remember that you should \textbf{work together} to come up with
responses to the questions.  Make sure all team members agree before
moving on.

\begin{questions}

  \item Problem sets are typically due Friday at 4pm, and should be
    submitted electronically.

  \item Problem sets can be submitted as either \texttt{.pdf} or
    \texttt{.doc} files.

  \item One week, Amanda has a play performance, three midterm
    exams, a 100-page paper due, and a family trip.  As a result, she
    is not able to finish her Algorithms problem set before the
    deadline.  She must either turn in whatever she has by the
    deadline, or else take a zero on the assignment.

  \item Problem sets are graded out of 100 points.

  \item If you decide that you want to get a B in CSCI 382, Dr.\
    Yorgey will be disappointed.

  \newpage
  \item In this course, which of the following could be considered academic
    integrity violations? \marginnote{Hint: in addition to the
      syllabus, be sure to also look at
      \url{http://ozark.hendrix.edu/~yorgey/ac-integrity-policy.html}
      (which is also linked from the syllabus).}
    \begin{compactenum}[(a)]
      \item collaborating with another student on a weekly problem set
      \item collaborating with another student to write up solutions
        to a weekly problem set
      \item using generative AI to write solutions to a weekly problem set
      \item looking at another student's code
      \item referring to an online resource
      \item referring to an online resource without citing it
      \item collaborating with another student while preparing for the
        final exam
      \item obtaining a copy of exam questions before an exam
      \end{compactenum} \bigskip
\end{questions}

\newpage

\section{Facilitation plan}
\label{sec:facilitation}

\section{Learning Objectives}

\subsection{Content objectives}

\begin{itemize}
\item \lobjectiveA
\item \lobjectiveB
\end{itemize}

\subsection{Process objectives}

\section{Intro \& getting started (5m)}

\section{Introduction to POGIL (10m)}

\section{CSCI 382 Syllabus (10m + 5m discussion)}

Make chart of last Q on board (ac integrity), have reporters fill it
in, discuss.  This will probably take a while.

\end{document}


\end{document}
