% -*- compile-command: "./build.sh" -*-
\documentclass{tufte-handout}

\usepackage{algo-activity}

\title{\thecourse: Minimum Spanning Trees (answer key)}
\date{}

\begin{document}

\maketitle

% \section{Roles (2 mins)}

% \begin{itemize}
% \item Whoever has used social media least recently gets to choose whether
%   they want to be the manager, or whether they want to make the person
%   who has used social media most recently be the manager.  The manager
%   should make sure to pay attention to the suggested time limits for
%   each section.
% \item Clockwise from the manager are the recorder and reflector.  If
%   your group has four members, the remaining person is the reporter.
% \end{itemize}

\begin{model*}{Minimum-weight Spanning Subgraphs (13 mins)}{mwss}
  \begin{defn}
    Given an undirected graph $G = (V,E)$, a \term{spanning
      subgraph} of $G$ is a connected subgraph $G' = (V,E')$ of $G$;
    that is, a connected graph with the same vertices as $G$ and a
    subset of its edges.
  \end{defn}

  \begin{defn}
    Given a \emph{weighted}, undirected graph $G$, the
    \term{minimum-weight spanning subgraph} (MWSS) of $G$ is the
    spanning subgraph of $G$ whose total weight (\ie\ the sum of the
    weights of all its edges) is as small as possible.
  \end{defn}

  \begin{center}
    \begin{diagram}[width=300]
      {-# LANGUAGE ViewPatterns #-}

      import GraphDiagrams
      import Data.Bifunctor
      import System.Random

      perturb :: P2 Double -> IO (P2 Double)
      perturb (coords -> (x :& y)) = do
        dx <- randomRIO (-0.3, 0.3)
        dy <- randomRIO (-0.3, 0.3)
        return ((x + dx) ^& (y + dy))

      mvertices :: [(Char, P2 Double)]
      mvertices =
        [ ('a', 1 ^& 2)
        , ('b', 0 ^& 1)
        , ('c', 1 ^& 0)
        , ('d', 2 ^& 1)
        , ('i', 3 ^& 0)
        , ('f', 3 ^& 2)
        , ('e', 4 ^& 1)
        , ('j', 4 ^& (-1))
        , ('g', 5 ^& 2)
        , ('h', 5 ^& 0)
        ]

      mvertices' :: IO [(Char, P2 Double)]
      mvertices' = (traverse . traverse) perturb mvertices

      medges =
        [ (('a','b'), 3)
        , (('b','c'), 17)
        , (('b','d'), 16)
        , (('a','f'), 2)
        , (('c','d'), 8)
        , (('c','i'), 18)
        , (('d','i'), 4)
        , (('d','e'), 11)
        , (('i','e'), 10)
        , (('f','e'), 1)
        , (('f','g'), 7)
        , (('e','g'), 6)
        , (('e','h'), 5)
        , (('i','h'), 12)
        , (('i','j'), 9)
        , (('j','h'), 13)
        , (('g','h'), 15)
        ]

      g :: IO (Graph Char Int)
      g = do
        vs <- mvertices'
        return $ wgraph vs medges

      dia :: IO (Diagram B)
      dia = drawGraph (drawVTeX . first (:[])) (drawUWE (tex . show)) <$> g
  \end{diagram}

  \end{center}
\end{model*}

\begin{questions}
\item \label{q:draw-ss} Draw any spanning subgraph of the example
  graph from~\pref{model:mwss}.

  \begin{answer}
    Many are possible.
  \end{answer}
\item Can a graph $G$ have more than one spanning subgraph?

  \begin{answer}
    Yes, in general a graph can have lots of spanning subgraphs.
  \end{answer}
\item Is it possible for a graph $G$ to be a spanning subgraph of
  itself?  Why or why not?

  \begin{answer}
    Yes, if $G$ is a tree.
  \end{answer}
\item Is it possible for a disconnected graph to have a spanning
  subgraph?  Why or why not?

  \begin{answer}
    No, by definition a spanning subgraph is connected.
  \end{answer}
\item Can a graph $G$ have more than one minimum-weight spanning
  subgraph?  Give an example, or explain why it is not possible.

  \begin{answer}
    It is possible.  As a simple example, consider a graph with three
    vertices connected by three edges having equal weights.  There are
    three possible spanning subgraphs of this graph and all are
    minimum-weight.
  \end{answer}
\item Find a MWSS for the graph in~\pref{model:mwss}. Does it have
  more than one?\marginnote{Don't spend too much time on this
    question; just find a spanning subgraph which you reasonably think
    is the minimum, and move on.}

  \begin{answer}
    Answers may vary.
  \end{answer}
\item \label{q:scenarios} Which of the following scenarios could be
  modeled by finding a MWSS?
  \begin{subquestions}
  \item A railroad company wants to connect a given set of cities by
    train routes as cheaply as possible. Connecting two cities by a
    route costs an amount of money proportional to the distance
    between them.

    \begin{answer}
      Yes, in this case we can model the situation with a graph in
      which the cities by vertices, and an edge between every pair of
      cities weighted by their distance.  Then the company wants to
      construct a MWSS, which will connect all the cities as cheaply as possible.
    \end{answer}
  \item A company wants to network a set of data centers with
    high-speed fiber optic connections, and they want to mimimize the
    total amount of fiber optic cable used. There must be at least two
    routes between any pair of data centers, so that any one fiber
    optic link going down will not disconnect the network.

    \begin{answer}
      The requirement to have two links between any two data centers
      means we cannot model this by finding a MWSS; getting rid of a
      ``redundant'' link would keep the graph connected and reduce the
      total weight, so the MWSS will always have only one route
      between any two data centers.
    \end{answer}
  \item Given a network of train routes between a collection of
    cities and the cost of each route, find the cheapest route between
    two given cities.

    \begin{answer}
      No, the cheapest route between two cities may not be included in the
      MWSS!  This is an important distinction.  Although the MWSS is
      the minimum weight subgraph that connects all vertices, for a
      specific pair of vertices there may be a cheaper path than the
      one provided by the MWSS.
    \end{answer}
  \item You have the latest Megazorx puzzle toy, which can be in any
    one of a number of different states.  There are various moves
    which can transform the Megazorx from one state to another.  In
    order to impress your friends, you want to be able to transform it
    from any starting state into any other requested state (which may
    in general require a whole sequence of moves). You want to be able
    to do this
    \begin{enumerate}[label=(\roman*)]
    \item \dots using the fewest number of moves possible for each
      given pair of start and end states.

      \begin{answer}
        This cannot be modelled by finding a MWSS, for the same reason
        as the problem about finding cheapest train routes.
      \end{answer}
    \item \dots without having to memorize any more moves than
      absolutely necessary.

      \begin{answer}
        Consider the graph whose vertices are puzzle states, and edges
        are potential moves from one state to another, with weight 1.
        Then the MWSS is the minimal set of moves you should memorize
        to be able to go from any state to any other state.
      \end{answer}
    \end{enumerate}
  \end{subquestions}

\item (Review) What is the definition of a \term{tree} graph?

  \begin{answer}
    A tree is a connected graph with no cycles.
  \end{answer}
\item \label{q:mwss-tree} Prove: in a weighted, undirected graph with
  positive weights, a MWSS must always be a tree. \marginnote{Hint:
    use a proof by contradiction.}

  \begin{answer}
    Suppose on the contrary that some subgraph $M$ of $G$ is a MWSS
    but not a tree.  Since $M$ is a MWSS by definition it must be
    connected.  Thus, by definition of a tree, it must contain a
    cycle.  Pick any edge in the cycle and remove it.  This results in
    a new graph $M'$ which is still a connected, spanning subgraph of
    $G$ but has a smaller weight than $M$.  But this is a
    contradiction since we assumed $M$ was a \emph{minimum weight}
    spanning subgraph.  Thus, every MWSS must in fact be a tree.
  \end{answer}
\end{questions}

% \begin{figure*}
% \begin{center}
%   \textbf{Stop and check your answers to
%     Questions~\ref{q:draw-ss}--\ref{q:scenarios} with the other groups
%     (7 mins).  Can you all agree on correct answers?}
% \end{center}
% \end{figure*}
\pause

Because of the result from \pref{q:mwss-tree}, we typically refer to
a minimum-weight spanning subgraph as a \term{minimum spanning tree}
(MST).  Now that you understand the definition of a MST and some
scenarios that MSTs can be used to model, let's explore some
algorithms for finding them.

\begin{model*}{Four algorithms (15 mins)}{four-algorithms}
  Given a weighted, undirected graph $G$, the goal of each algorithm
  is to pick a subset of the edges of $G$ which constitute a MST.
  \begin{itemize}
  \item (\textbf{Kruskal's}) Consider all the edges in order from
    smallest to biggest weight; for each edge, pick it if and only if
    it does not complete a cycle with previously chosen edges.
  \item (\textbf{Prim's}) Choose an arbitrary vertex to start and mark
    it as visited.  At each step, pick the smallest edge which
    connects any visited vertex to any unvisited vertex and which would
    not complete a cycle with previously chosen edges; mark the other
    end of the edge visited.
  \item (\textbf{Johnson's}) Choose any vertex to start.  At each step,
    pick the smallest edge connected to the current vertex which has
    not already been chosen and would not complete a cycle with
    previously chosen edges.  Repeat until running out of options;
    then pick a new starting vertex and repeat the entire process.  Do
    this until all vertices are connected.
  \item (\textbf{Reverse delete}) Start with all edges initially
    ``picked'', and consider them in order from biggest to smallest
    weight.  For each edge, throw it out (\ie ``unpick'' it) if and
    only if doing so would not disconnect the remaining edges.
  \end{itemize}
\end{model*}

\begin{questions}
\item On the next page are four copies of the same graph.  Use these
  to trace the execution of each algorithm in the model.
  \begin{figure}
  \begin{center}
    \begin{diagram}[width=400]
      import GraphDiagrams
      import Data.Bifunctor

      g :: Graph Char Int
      g = wgraph
        [ ('a', 1 ^& 2)
        , ('b', 0 ^& 1)
        , ('c', 2 ^& 1)
        , ('d', 1 ^& 0)
        , ('e', 2 ^& 2)
        ]

        [ (('a','b'), 5)
        , (('a','d'), 9)
        , (('b','d'), 20)
        , (('a','c'), 3)
        , (('c','d'), 1)
        , (('a','e'), 2)
        , (('e','c'), 4)
        ]

      gDia = drawGraph (drawVTeX . first (:[])) (drawUWE (tex . show)) g

      dia = vsep 1 . map (hsep 1) $  -- $
        [[gDia, gDia], [gDia, gDia]]
    \end{diagram}
  \end{center}
  \end{figure}
\item One of these algorithms is fake.  Which one?  Give an example
  showing why it does not work.

  \begin{answer}
    The fake one is Johnson's algorithm.  (There \emph{is} an algorithm
    called Johnson's algorithm, but this is not it, and it doesn't
    find MST's in any case.)
  \end{answer}
\end{questions}

\end{document}
