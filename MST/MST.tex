% -*- compile-command: "./build.sh" -*-
\documentclass{tufte-handout}

\usepackage{algo-activity}

\title{\thecourse: Minimum Spanning Trees}
\date{}

\begin{document}

\maketitle

% \section{Roles (2 mins)}

% \begin{itemize}
% \item Whoever has used social media least recently gets to choose whether
%   they want to be the manager, or whether they want to make the person
%   who has used social media most recently be the manager.  The manager
%   should make sure to pay attention to the suggested time limits for
%   each section.
% \item Clockwise from the manager are the recorder and reflector.  If
%   your group has four members, the remaining person is the reporter.
% \end{itemize}

\begin{model*}{Minimum-weight Spanning Subgraphs (13 mins)}{mwss}
  \begin{defn}
    Given an undirected graph $G = (V,E)$, a \term{spanning
      subgraph} of $G$ is a connected subgraph $G' = (V,E')$ of $G$;
    that is, a connected graph with the same vertices as $G$ and a
    subset of its edges.
  \end{defn}

  \begin{defn}
    Given a \emph{weighted}, undirected graph $G$, the
    \term{minimum-weight spanning subgraph} (MWSS) of $G$ is the
    spanning subgraph of $G$ whose total weight (\ie\ the sum of the
    weights of all its edges) is as small as possible.
  \end{defn}

  \begin{center}
    \begin{diagram}[width=300]
      {-# LANGUAGE ViewPatterns #-}

      import GraphDiagrams
      import Data.Bifunctor
      import System.Random

      perturb :: P2 Double -> IO (P2 Double)
      perturb (coords -> (x :& y)) = do
        dx <- randomRIO (-0.3, 0.3)
        dy <- randomRIO (-0.3, 0.3)
        return ((x + dx) ^& (y + dy))

      mvertices :: [(Char, P2 Double)]
      mvertices =
        [ ('a', 1 ^& 2)
        , ('b', 0 ^& 1)
        , ('c', 1 ^& 0)
        , ('d', 2 ^& 1)
        , ('i', 3 ^& 0)
        , ('f', 3 ^& 2)
        , ('e', 4 ^& 1)
        , ('j', 4 ^& (-1))
        , ('g', 5 ^& 2)
        , ('h', 5 ^& 0)
        ]

      mvertices' :: IO [(Char, P2 Double)]
      mvertices' = (traverse . traverse) perturb mvertices

      medges =
        [ (('a','b'), 3)
        , (('b','c'), 17)
        , (('b','d'), 16)
        , (('a','f'), 2)
        , (('c','d'), 8)
        , (('c','i'), 18)
        , (('d','i'), 4)
        , (('d','e'), 11)
        , (('i','e'), 10)
        , (('f','e'), 1)
        , (('f','g'), 7)
        , (('e','g'), 6)
        , (('e','h'), 5)
        , (('i','h'), 12)
        , (('i','j'), 9)
        , (('j','h'), 13)
        , (('g','h'), 15)
        ]

      g :: IO (Graph Char Int)
      g = do
        vs <- mvertices'
        return $ wgraph vs medges

      dia :: IO (Diagram B)
      dia = drawGraph (drawVTeX . first (:[])) (drawUWE (tex . show)) <$> g
  \end{diagram}

  \end{center}
\end{model*}

\begin{questions}
\item \label{q:draw-ss} Draw any spanning subgraph of the example
  graph from~\pref{model:mwss}.
\item Can a graph $G$ have more than one spanning subgraph?
\item Is it possible for a graph $G$ to be a spanning subgraph of
  itself?  Why or why not?
\item Is it possible for a disconnected graph to have a spanning
  subgraph?  Why or why not?
\item Can a graph $G$ have more than one minimum-weight spanning
  subgraph?  Give an example, or explain why it is not possible.
\item Find a MWSS for the graph in~\pref{model:mwss}. Does it have
  more than one?\marginnote{Don't spend too much time on this
    question; just find a spanning subgraph which you reasonably think
    is the minimum, and move on.}
\item \label{q:scenarios} Which of the following scenarios could be
  modeled by finding a MWSS?
  \begin{subquestions}
  \item A railroad company wants to connect a given set of cities by
    train routes as cheaply as possible. Connecting two cities by a
    route costs an amount of money proportional to the distance
    between them.
  \item A company wants to network a set of data centers with
    high-speed fiber optic connections, and they want to mimimize the
    total amount of fiber optic cable used. There must be at least two
    routes between any pair of data centers, so that any one fiber
    optic link going down will not disconnect the network.
  \item Given a network of train routes between a collection of
    cities and the cost of each route, find the cheapest route between
    two given cities.
  \item You have the latest Megazorx puzzle toy, which can be in any
    one of a number of different states.  There are various moves
    which can transform the Megazorx from one state to another.  In
    order to impress your friends, you want to be able to transform it
    from any starting state into any other requested state (which may
    in general require a whole sequence of moves). You want to be able
    to do this
    \begin{enumerate}[label=(\roman*)]
    \item \dots using the fewest number of moves possible for each
      given pair of start and end states.
    \item \dots without having to memorize any more moves than
      absolutely necessary.
    \end{enumerate}
  \end{subquestions}

\item (Review) What is the definition of a \term{tree} graph?
\item \label{q:mwss-tree} Prove: a MWSS must always be a tree. \marginnote{Hint: use a
    proof by contradiction.}
\end{questions}

% \begin{figure*}
% \begin{center}
%   \textbf{Stop and check your answers to
%     Questions~\ref{q:draw-ss}--\ref{q:scenarios} with the other groups
%     (7 mins).  Can you all agree on correct answers?}
% \end{center}
% \end{figure*}
\pause

Because of the result from \pref{q:mwss-tree}, we typically refer to
a minimum-weight spanning subgraph as a \term{minimum spanning tree}
(MST).  Now that you understand the definition of a MST and some
scenarios that MSTs can be used to model, let's explore some
algorithms for finding them.

\begin{model*}{Four algorithms (15 mins)}{four-algorithms}
  Given a weighted, undirected graph $G$, the goal of each algorithm
  is to pick a subset of the edges of $G$ which constitute a MST.
  \begin{itemize}
  \item (\textbf{Kruskal's}) Consider all the edges in order from
    smallest to biggest weight; for each edge, pick it if and only if
    it does not complete a cycle with previously chosen edges.
  \item (\textbf{Prim's}) Choose an arbitrary vertex to start and mark
    it as visited.  At each step, pick the smallest edge which
    connects any visited vertex to any unvisited vertex and which would
    not complete a cycle with previously chosen edges; mark the other
    end of the edge visited.
  \item (\textbf{Johnson's}) Choose any vertex to start.  At each step,
    pick the smallest edge connected to the current vertex which has
    not already been chosen and would not complete a cycle with
    previously chosen edges.  Repeat until running out of options;
    then pick a new starting vertex and repeat the entire process.  Do
    this until all vertices are connected.
  \item (\textbf{Reverse delete}) Start with all edges initially
    ``picked'', and consider them in order from biggest to smallest
    weight.  For each edge, throw it out (\ie ``unpick'' it) if and
    only if doing so would not disconnect the remaining edges.
  \end{itemize}
\end{model*}

\begin{questions}
\item On the next page are four copies of the same graph.  Use these
  to trace the execution of each algorithm in the model.
  \begin{figure}
  \begin{center}
    \begin{diagram}[width=400]
      import GraphDiagrams
      import Data.Bifunctor

      g :: Graph Char Int
      g = wgraph
        [ ('a', 1 ^& 2)
        , ('b', 0 ^& 1)
        , ('c', 2 ^& 1)
        , ('d', 1 ^& 0)
        , ('e', 2 ^& 2)
        ]

        [ (('a','b'), 5)
        , (('a','d'), 9)
        , (('b','d'), 20)
        , (('a','c'), 3)
        , (('c','d'), 1)
        , (('a','e'), 2)
        , (('e','c'), 4)
        ]

      gDia = drawGraph (drawVTeX . first (:[])) (drawUWE (tex . show)) g

      dia = vsep 1 . map (hsep 1) $  -- $
        [[gDia, gDia], [gDia, gDia]]
    \end{diagram}
  \end{center}
  \end{figure}
\item One of these algorithms is fake.  Which one?  Give an example
  showing why it does not work.
\end{questions}

\newpage
\begin{model*}{The Cut Property (15 mins)}{cut-property}
  \begin{defn}
    A \term{cut} in a graph $G = (V,E)$ is a partition of the vertices
    $V$ into two sets $S$ and $T$, that is, every vertex is in either
    $S$ or $T$ but not both.  We say that an edge $e$ \term{crosses}
    the cut $(S,T)$ if one vertex of $e$ is in $S$ and the other is in
    $T$.
  \end{defn}

  \begin{thm}[Cut Property]
    Given a weighted, undirected graph $G = (V,E)$, let $S$ and $T$ be
    any partition of $V$, and suppose $e$ is the smallest-weight edge
    crossing the $(S,T)$ cut.  Then every minimum spanning tree of $G$
    must include $e$.
  \end{thm}
\end{model*}

\begin{questions}
\item Give three examples of cuts in the graph from \pref{model:mwss}
  and identify the smallest edge crossing each cut.
\end{questions}

Let's prove the cut property.

\begin{proof}
  Let $G$ be a weighted, undirected graph $G = (V,E)$, let $S$ and $T$
  be an arbitrary partition of $V$ into two sets, and suppose
  $e = (x,y)$ is the smallest-weight edge with one endpoint in $S$ and
  one in $T$.  We wish to show that \blank.

  Suppose $M$ is a spanning tree of $G$ which does \textbf{not}
  contain the edge $e$.  Since $M$ is a \blank it contains a unique
  \blank\linebreak between any two \blank. So consider the unique
  \blank\linebreak in $M$ between \blank.  \marginnote{\emph{Hint}:
    draw a picture!} It must cross
  the cut at least once since\linebreak \mbox{} \blank; suppose it
  crosses at $e' = (x',y')$,\linebreak with $x' \in X$ and $y' \in Y$.
  We know that $w_e < w_{e'}$ since \blank.\linebreak Now take $M$ and
  replace \blank with \blank; the resulting graph is\linebreak still
  \blank because \blank,\linebreak but it has a smaller total \blank
  because \blank.

  So, we have shown that any spanning tree $M$ which does not contain
  the edge $e$ can be made into a \blank,\linebreak which means that
  $M$ is not a \blank. This proves what\linebreak we wanted to show, since
  \blank.
\end{proof}

The cut property can be used to directly show the correctness of
several MST algorithms.  Let's prove the correctness of Kruskal's
Algorithm; the proofs for the other algorithms are similar.

\begin{thm}
  Kruskal's Algorithm is correct.
\end{thm}

\begin{proof}
  Suppose at some step the algorithm picks the edge $e = (x,y)$.  Let
  $X$ be the set of vertices connected to $x$ by edges which have been
  picked so far (not including $e$), and let $Y$ be all other
  vertices. $x \in X$ by definition.  We know that $y \notin X$ since
  if it was, $e$ would \blank\linebreak but then Kruskal's Algorithm
  wouldn't \blank.\linebreak $e$ therefore crosses the cut $(X,Y)$. No
  other edges which have been picked previously cross the cut, since
  \blank.\linebreak Therefore $e$ must be the \blank\linebreak because
  \blank.\linebreak Therefore by the Cut Property $e$ must
  be in any MST and Kruskal's Algorithm is correct to pick it.
\end{proof}

\end{document}
