% -*- compile-command: "./build.sh" -*-
\documentclass{tufte-handout}

\usepackage{algo-activity}
\usepackage{algorithm, algorithmicx}
\usepackage[noend]{algpseudocode}

\title{\thecourse: Kruskal's Algorithm}
\date{}

\begin{document}

\maketitle

In the previous activity you learned about minimum spanning trees and
experimented with several different algorithms for finding them.  In
today's activity we will focus on Kruskal's Algorithm and prove that
it works correctly.

\begin{model*}{Kruskal's Algorithm (12 mins)}{kruskal}
  \begin{center}
    \begin{diagram}[width=300]
      {-# LANGUAGE ViewPatterns #-}

      import GraphDiagrams
      import Data.Bifunctor
      import System.Random

      perturb :: P2 Double -> IO (P2 Double)
      perturb (coords -> (x :& y)) = do
        dx <- randomRIO (-0.3, 0.3)
        dy <- randomRIO (-0.3, 0.3)
        return ((x + dx) ^& (y + dy))

      mvertices :: [(Char, P2 Double)]
      mvertices =
        [ ('a', 1 ^& 2)
        , ('b', 0 ^& 1)
        , ('c', 1 ^& 0)
        , ('d', 2 ^& 1)
        , ('i', 3 ^& 0)
        , ('f', 3 ^& 2)
        , ('e', 4 ^& 1)
        , ('j', 4 ^& (-1))
        , ('g', 5 ^& 2)
        , ('h', 5 ^& 0)
        ]

      mvertices' :: IO [(Char, P2 Double)]
      mvertices' = (traverse . traverse) perturb mvertices

      medges =
        [ (('a','b'), 3)
        , (('b','c'), 17)
        , (('b','d'), 16)
        , (('a','f'), 2)
        , (('c','d'), 8)
        , (('c','i'), 18)
        , (('d','i'), 4)
        , (('d','e'), 11)
        , (('i','e'), 10)
        , (('f','e'), 1)
        , (('f','g'), 7)
        , (('e','g'), 6)
        , (('e','h'), 5)
        , (('i','h'), 12)
        , (('i','j'), 9)
        , (('j','h'), 13)
        , (('g','h'), 15)
        ]

      g :: IO (Graph Char Int)
      g = do
        vs <- mvertices'
        return $ wgraph vs medges

      dia :: IO (Diagram B)
      dia = drawGraph (drawVTeX . first (:[])) (drawUWE (tex . show)) <$> g
  \end{diagram}

  \end{center}

  \begin{algorithm}[H]
    \begin{algorithmic}[1]
      \Require Undirected, weighted graph $G = (V,E)$
      \State $T \gets \varnothing$   \Comment{$T$ holds the set of
        edges in the MST}
      \State Sort $E$ from smallest to biggest weight
      \For{each edge $e \in E$}
        \If{$e$ does not make a cycle with other edges in $T$}
          \State Add $e$ to $T$
        \EndIf
      \EndFor
    \end{algorithmic}
    \label{alg:kruskal}
  \end{algorithm}
\end{model*}

\begin{questions}
\item Simulate Kruskal's Algorithm on the graph in
  \pref{model:kruskal}.  What is the total weight of the resulting
  spanning tree?
\item The way the algorithm is written in \pref{model:kruskal}, one
  must iterate through every single edge in $E$.  However, this is not
  always necessary.  Can you think of a simple way to tell when we can
  stop the loop early?
\item Explain why even in the worst case,
  $\Theta(\lg V) = \Theta(\lg E)$ in any graph.
\item In the above algorithm, how long does line 2 take?  Simplify
  your answer using the observation from the previous question.
\item Can you think of a way to implement line 4?  How long would it take?
\end{questions}

\pause

\begin{model*}{The Cut Property (20 mins)}{cut-property}
  \begin{center}
    \begin{diagram}[width=300]
      {-# LANGUAGE ViewPatterns #-}

      import GraphDiagrams
      import Data.Bifunctor
      import System.Random

      perturb :: P2 Double -> IO (P2 Double)
      perturb (coords -> (x :& y)) = do
        dx <- randomRIO (-0.3, 0.3)
        dy <- randomRIO (-0.3, 0.3)
        return ((x + dx) ^& (y + dy))

      mvertices :: [(Char, P2 Double)]
      mvertices =
        [ ('a', 1 ^& 2)
        , ('b', 0 ^& 1)
        , ('c', 1 ^& 0)
        , ('d', 2 ^& 1)
        , ('i', 3 ^& 0)
        , ('f', 3 ^& 2)
        , ('e', 4 ^& 1)
        , ('j', 4 ^& (-1))
        , ('g', 5 ^& 2)
        , ('h', 5 ^& 0)
        ]

      mvertices' :: IO [(Char, P2 Double)]
      mvertices' = (traverse . traverse) perturb mvertices

      medges =
        [ (('a','b'), 3)
        , (('b','c'), 17)
        , (('b','d'), 16)
        , (('a','f'), 2)
        , (('c','d'), 8)
        , (('c','i'), 18)
        , (('d','i'), 4)
        , (('d','e'), 11)
        , (('i','e'), 10)
        , (('f','e'), 1)
        , (('f','g'), 7)
        , (('e','g'), 6)
        , (('e','h'), 5)
        , (('i','h'), 12)
        , (('i','j'), 9)
        , (('j','h'), 13)
        , (('g','h'), 15)
        ]

      g :: IO (Graph Char Int)
      g = do
        vs <- mvertices'
        return $ wgraph vs medges

      dia :: IO (Diagram B)
      dia = drawGraph (drawVTeX . first (:[])) (drawUWE (tex . show)) <$> g
  \end{diagram}

  \end{center}

  \begin{defn}
    A \term{cut} in a graph $G = (V,E)$ is a partition of the vertices
    $V$ into two sets $S$ and $T$, that is, every vertex is in either
    $S$ or $T$ but not both.  We say that an edge $e$ \term{crosses}
    the cut $(S,T)$ if one vertex of $e$ is in $S$ and the other is in
    $T$.
  \end{defn}

  \begin{thm}[Cut Property]
    Given a weighted, undirected graph $G = (V,E)$, let $S$ and $T$ be
    any partition of $V$, and suppose $e$ is some edge crossing the
    $(S,T)$ cut, such that the weight of $e$ is strictly smaller than
    the weight of any other edge crossing the $(S,T)$ cut.  Then every
    minimum spanning tree of $G$ must include $e$.
  \end{thm}
\end{model*}

\begin{questions}
\item Give three examples of cuts in the graph from
  \pref{model:cut-property} and identify the smallest edge crossing
  each cut.
\end{questions}

Let's prove the cut property.

\begin{proof}
  Let $G$ be a weighted, undirected graph $G = (V,E)$, let $S$ and $T$
  be an arbitrary partition of $V$ into two sets, and suppose
  $e = (x,y)$ is the smallest-weight edge with one endpoint in $S$ and
  one in $T$.  We wish to show that \blank.

  We will prove the contrapositive. Suppose $M$ is a spanning tree of
  $G$ which does \textbf{not} contain the edge $e$.  Since $M$ is a
  \blank it contains a unique \blank\linebreak between any two
  \blank. So consider the unique \blank\linebreak in $M$ between
  \blank.  \marginnote{\emph{Hint}: draw a picture!} It must cross the
  cut at least once since\linebreak \mbox{} \blank; suppose it crosses
  at $e' = (x',y')$,\linebreak with $x' \in X$ and $y' \in Y$.  We
  know that the weight of $e$ is smaller than the weight of $e'$,
  since \blank.\linebreak Now take $M$ and replace \blank with \blank;
  the result is\linebreak still \blank because
  \blank,\linebreak but it has a smaller total \blank because \blank.

  So, we have shown that any spanning tree $M$ which does not contain
  the edge $e$ can be made into a \blank,\linebreak which means that
  $M$ is not a \blank.
\end{proof}

The cut property can be used to directly show the correctness of
several MST algorithms.  Let's prove the correctness of Kruskal's
Algorithm; the proofs for the other algorithms are similar.

\begin{thm}
  Kruskal's Algorithm is correct.
\end{thm}

\begin{proof}
  Suppose at some step the algorithm picks the edge $e = (x,y)$.  Let
  $X$ be the set of vertices connected to $x$ by edges which have been
  picked so far (not including $e$), and let $Y$ be all other
  vertices. $x \in X$ by definition.  We know that $y \notin X$ since
  if it was, $e$ would\linebreak make a \blank but then Kruskal's
  Algorithm wouldn't \blank.\linebreak $e$ therefore crosses the cut
  $(X,Y)$. No other edges which have been picked previously cross the
  cut, since \blank.\linebreak Therefore $e$ must be the smallest
  \blank\linebreak because \blank.\linebreak Therefore by the Cut
  Property $e$ must be in any MST and Kruskal's Algorithm is correct
  to pick it.
\end{proof}

\end{document}