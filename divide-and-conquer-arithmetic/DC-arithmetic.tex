% -*- compile-command: "pdflatex --enable-write18 DC-arithmetic.tex" -*-
\documentclass{tufte-handout}

\usepackage{algo-activity}

\title{\thecourse: Divide and conquer arithmetic}
\date{}

\begin{document}

\maketitle

% \begin{objective}
%   Students will XXX.
% \end{objective}

\begin{model*}{Arithmetic}{arithmetic}
  \[
    \begin{array}{cccccccc}
         & 1 & 0 & 1 & 1 & 0 & 1 & 0 \\
       + & 1 & 1 & 1 & 0 & 0 & 1 & 1 \\
      \hline
       1 & 1 & 0 & 0 & 1 & 1 & 0 & 1
    \end{array}
  \] \vspace{0.3in}

  \[
    \begin{array}{cccccccccccccc}
  & & & & & & &1&0&1&1&0&1&0 \\
  & & & & & &\times&1&1&1&0&0&1&1 \\
  \hline
  & & & & & & &1&0&1&1&0&1&0 \\
  & & & & & &1&0&1&1&0&1&0&0 \\
  & & &1&0&1&1&0&1&0&0&0&0&0 \\
  & &1&0&1&1&0&1&0&0&0&0&0&0 \\
 +&1&0&1&1&0&1&0&0&0&0&0&0&0 \\
 \hline
 1&0&1&0&0&0&0&1&1&0&1&1&1&0
    \end{array}
  \] \vspace{0.3in}
\end{model*}

In many situations, when we use arithmetic operations on integers such
as addition or multiplication, we simply assume they take constant
time.  This is appropriate when the integers are bounded by some fixed
size, \eg if all the integers are $64$-bit integers.  However, if we
want to be able to deal with integers of arbitrary size, this is no
longer appropriate; we must take the size of the integers into account.

\begin{objective}
  Students will analyze arithmetic operations on $n$-bit integers.
\end{objective}

\begin{questions}
\item Approximately how many single bit operations (\emph{e.g.} adding
  or comparing two bits) are needed to compute the addition
  $1011010_2 + 1110011_2$ shown in the model?
\item In general, suppose we want to add two integers of $n$ bits
  each.  In terms of $\Theta$, how many bit operations are needed?
\item Explain why it is not possible to do any better than this.
\item If we have a list of $\Theta(n)$ integers, each with
  $\Theta(n)$ bits, how long will it take (in terms of single bit
  operations) to add all of them?
\end{questions}

Now consider the multiplication shown in the model.

\begin{questions}
\item Why are there five rows in between the two horizontal lines?
\item How many operations are needed to produce each such row?
  (\emph{Hint}: you may assume that multiplying by a power of two
  takes constant time.)
\item If we are multiplying two integers with $n$ bits each, how many
  intermediate rows could there be?
  case?
\item How long will it take to add them all?
\end{questions}

\pause

\begin{model*}{Arithmetic by pieces}{pieces}
  \begin{align*}
    X &= 01101001_2 & X_1 &= 0110_2 & X_2 &= 1001_2 \\
    Y &= 11100100_2 & Y_1 &= 1110_2 & Y_2 &= 0100_2
  \end{align*}
\end{model*}

Let's now consider whether it is possible to multiply two $n$-bit
integers any faster.

\begin{questions}
\item What is the relationship between $X$, $X_1$ and $X_2$ in the
  model?  What about $Y$, $Y_1$, and $Y_2$?
\item Suppose $Z = 1011100101_2$.  What would $Z_1$ and $Z_2$ be?
\item What is $2^4 \cdot X_1$ in binary?
\item In general, if $b$ is some number expressed in binary, what is
  $2^4 \cdot b$?
\item Write two equations expressing $X$ in terms of $X_1$ and $X_2$
  and $Y$ in terms of $Y_1$ and $Y_2$.
\item In general, suppose $A$ is an $n$-bit integer, and we split it
  into $A_1$ and $A_2$.  Generalize your previous answer to express
  $A$ in terms of $A_1$ and $A_2$.
\newpage
\item \label{q:expr} Suppose $A$ and $B$ are $n$-bit integers, and consider the
  product $AB$. Expand both $A$ and $B$ using your previous answer,
  and distribute the resulting product.  You should end up with an
  expression involving only $A_1$, $A_2$, $B_1$, and $B_2$.
\item How many multiplications are required to compute the expression
  from \pref{q:expr}? (Remember that multiplying by a power of two
  takes constant time and does not need to be counted.)
\item How big (how many bits) are the inputs to each multiplication in
  \pref{q:expr}?
\item Explain how we can use the equation from \pref{q:expr} as a recursive
  algorithm to compute $AB$.
\item Let $M(n)$ denote the time taken to multiply two $n$-bit
  integers, and write a recurrence relation for $M(n)$ corresponding
  to this recursive algorithm.
\end{questions}

\end{document}
