% -*- compile-command: "pdflatex --enable-write18 binomial-heap.tex" -*-
\documentclass{tufte-handout}

\usepackage{algo-activity}

\title{Algorithms: Binomial Heaps}
\date{}

\begin{document}

\maketitle

\begin{objective}
  Students will describe binomial heaps and analyze the amortized
  running time for heap operations such as INSERT, DELETE-MIN, and MERGE.
\end{objective}

\begin{model*}{Binomial Trees}{binomial-trees}
  \begin{defn}
    A \term{binomial tree} of order $n$ (for $n \geq 0$) consists of a root
    node with $n$ subtrees. The leftmost subtree is of order $0$,
    the next subtree is of order $1$, and so forth, with the
    rightmost subtree being of order $n-1$.
  \end{defn}

  The image below depicts a series of \emph{binomial trees} in
  increasing order. The leftmost tree is order 0, the next one is
  order 1, the third one is order 2, the fourth one is order 3, and
  the final tree is order 4.

  \begin{center}
  \begin{pgfpicture}
  \pgfpathrectangle{\pgfpointorigin}{\pgfqpoint{400.0000bp}{102.0000bp}}
  \pgfusepath{use as bounding box}
  \begin{pgfscope}
    \definecolor{fc}{rgb}{0.0000,0.0000,0.0000}
    \pgfsetfillcolor{fc}
    \pgfsetfillopacity{0.0000}
    \pgfsetlinewidth{0.8113bp}
    \definecolor{sc}{rgb}{0.0000,0.0000,0.0000}
    \pgfsetstrokecolor{sc}
    \pgfsetmiterjoin
    \pgfsetbuttcap
    \pgfpathqmoveto{400.0000bp}{5.7143bp}
    \pgfpathqcurveto{400.0000bp}{8.8702bp}{397.4416bp}{11.4286bp}{394.2857bp}{11.4286bp}
    \pgfpathqcurveto{391.1298bp}{11.4286bp}{388.5714bp}{8.8702bp}{388.5714bp}{5.7143bp}
    \pgfpathqcurveto{388.5714bp}{2.5584bp}{391.1298bp}{0.0000bp}{394.2857bp}{0.0000bp}
    \pgfpathqcurveto{397.4416bp}{0.0000bp}{400.0000bp}{2.5584bp}{400.0000bp}{5.7143bp}
    \pgfpathclose
    \pgfusepathqfillstroke
  \end{pgfscope}
  \begin{pgfscope}
    \definecolor{fc}{rgb}{0.0000,0.0000,0.0000}
    \pgfsetfillcolor{fc}
    \pgfsetfillopacity{0.0000}
    \pgfsetlinewidth{0.8113bp}
    \definecolor{sc}{rgb}{0.0000,0.0000,0.0000}
    \pgfsetstrokecolor{sc}
    \pgfsetmiterjoin
    \pgfsetbuttcap
    \pgfpathqmoveto{400.0000bp}{28.5714bp}
    \pgfpathqcurveto{400.0000bp}{31.7273bp}{397.4416bp}{34.2857bp}{394.2857bp}{34.2857bp}
    \pgfpathqcurveto{391.1298bp}{34.2857bp}{388.5714bp}{31.7273bp}{388.5714bp}{28.5714bp}
    \pgfpathqcurveto{388.5714bp}{25.4155bp}{391.1298bp}{22.8571bp}{394.2857bp}{22.8571bp}
    \pgfpathqcurveto{397.4416bp}{22.8571bp}{400.0000bp}{25.4155bp}{400.0000bp}{28.5714bp}
    \pgfpathclose
    \pgfusepathqfillstroke
  \end{pgfscope}
  \begin{pgfscope}
    \pgfsetlinewidth{0.8113bp}
    \definecolor{sc}{rgb}{0.0000,0.0000,0.0000}
    \pgfsetstrokecolor{sc}
    \pgfsetmiterjoin
    \pgfsetbuttcap
    \pgfpathqmoveto{394.2857bp}{22.8571bp}
    \pgfpathqlineto{394.2857bp}{11.4286bp}
    \pgfusepathqstroke
  \end{pgfscope}
  \begin{pgfscope}
    \definecolor{fc}{rgb}{0.0000,0.0000,0.0000}
    \pgfsetfillcolor{fc}
    \pgfusepathqfill
  \end{pgfscope}
  \begin{pgfscope}
    \definecolor{fc}{rgb}{0.0000,0.0000,0.0000}
    \pgfsetfillcolor{fc}
    \pgfusepathqfill
  \end{pgfscope}
  \begin{pgfscope}
    \definecolor{fc}{rgb}{0.0000,0.0000,0.0000}
    \pgfsetfillcolor{fc}
    \pgfusepathqfill
  \end{pgfscope}
  \begin{pgfscope}
    \definecolor{fc}{rgb}{0.0000,0.0000,0.0000}
    \pgfsetfillcolor{fc}
    \pgfusepathqfill
  \end{pgfscope}
  \begin{pgfscope}
    \definecolor{fc}{rgb}{0.0000,0.0000,0.0000}
    \pgfsetfillcolor{fc}
    \pgfsetfillopacity{0.0000}
    \pgfsetlinewidth{0.8113bp}
    \definecolor{sc}{rgb}{0.0000,0.0000,0.0000}
    \pgfsetstrokecolor{sc}
    \pgfsetmiterjoin
    \pgfsetbuttcap
    \pgfpathqmoveto{377.1429bp}{28.5714bp}
    \pgfpathqcurveto{377.1429bp}{31.7273bp}{374.5845bp}{34.2857bp}{371.4286bp}{34.2857bp}
    \pgfpathqcurveto{368.2727bp}{34.2857bp}{365.7143bp}{31.7273bp}{365.7143bp}{28.5714bp}
    \pgfpathqcurveto{365.7143bp}{25.4155bp}{368.2727bp}{22.8571bp}{371.4286bp}{22.8571bp}
    \pgfpathqcurveto{374.5845bp}{22.8571bp}{377.1429bp}{25.4155bp}{377.1429bp}{28.5714bp}
    \pgfpathclose
    \pgfusepathqfillstroke
  \end{pgfscope}
  \begin{pgfscope}
    \definecolor{fc}{rgb}{0.0000,0.0000,0.0000}
    \pgfsetfillcolor{fc}
    \pgfsetfillopacity{0.0000}
    \pgfsetlinewidth{0.8113bp}
    \definecolor{sc}{rgb}{0.0000,0.0000,0.0000}
    \pgfsetstrokecolor{sc}
    \pgfsetmiterjoin
    \pgfsetbuttcap
    \pgfpathqmoveto{377.1429bp}{51.4286bp}
    \pgfpathqcurveto{377.1429bp}{54.5845bp}{374.5845bp}{57.1429bp}{371.4286bp}{57.1429bp}
    \pgfpathqcurveto{368.2727bp}{57.1429bp}{365.7143bp}{54.5845bp}{365.7143bp}{51.4286bp}
    \pgfpathqcurveto{365.7143bp}{48.2727bp}{368.2727bp}{45.7143bp}{371.4286bp}{45.7143bp}
    \pgfpathqcurveto{374.5845bp}{45.7143bp}{377.1429bp}{48.2727bp}{377.1429bp}{51.4286bp}
    \pgfpathclose
    \pgfusepathqfillstroke
  \end{pgfscope}
  \begin{pgfscope}
    \pgfsetlinewidth{0.8113bp}
    \definecolor{sc}{rgb}{0.0000,0.0000,0.0000}
    \pgfsetstrokecolor{sc}
    \pgfsetmiterjoin
    \pgfsetbuttcap
    \pgfpathqmoveto{375.4692bp}{47.3880bp}
    \pgfpathqlineto{390.2451bp}{32.6120bp}
    \pgfusepathqstroke
  \end{pgfscope}
  \begin{pgfscope}
    \definecolor{fc}{rgb}{0.0000,0.0000,0.0000}
    \pgfsetfillcolor{fc}
    \pgfusepathqfill
  \end{pgfscope}
  \begin{pgfscope}
    \definecolor{fc}{rgb}{0.0000,0.0000,0.0000}
    \pgfsetfillcolor{fc}
    \pgfusepathqfill
  \end{pgfscope}
  \begin{pgfscope}
    \definecolor{fc}{rgb}{0.0000,0.0000,0.0000}
    \pgfsetfillcolor{fc}
    \pgfusepathqfill
  \end{pgfscope}
  \begin{pgfscope}
    \definecolor{fc}{rgb}{0.0000,0.0000,0.0000}
    \pgfsetfillcolor{fc}
    \pgfusepathqfill
  \end{pgfscope}
  \begin{pgfscope}
    \pgfsetlinewidth{0.8113bp}
    \definecolor{sc}{rgb}{0.0000,0.0000,0.0000}
    \pgfsetstrokecolor{sc}
    \pgfsetmiterjoin
    \pgfsetbuttcap
    \pgfpathqmoveto{371.4286bp}{45.7143bp}
    \pgfpathqlineto{371.4286bp}{34.2857bp}
    \pgfusepathqstroke
  \end{pgfscope}
  \begin{pgfscope}
    \definecolor{fc}{rgb}{0.0000,0.0000,0.0000}
    \pgfsetfillcolor{fc}
    \pgfusepathqfill
  \end{pgfscope}
  \begin{pgfscope}
    \definecolor{fc}{rgb}{0.0000,0.0000,0.0000}
    \pgfsetfillcolor{fc}
    \pgfusepathqfill
  \end{pgfscope}
  \begin{pgfscope}
    \definecolor{fc}{rgb}{0.0000,0.0000,0.0000}
    \pgfsetfillcolor{fc}
    \pgfusepathqfill
  \end{pgfscope}
  \begin{pgfscope}
    \definecolor{fc}{rgb}{0.0000,0.0000,0.0000}
    \pgfsetfillcolor{fc}
    \pgfusepathqfill
  \end{pgfscope}
  \begin{pgfscope}
    \definecolor{fc}{rgb}{0.0000,0.0000,0.0000}
    \pgfsetfillcolor{fc}
    \pgfsetfillopacity{0.0000}
    \pgfsetlinewidth{0.8113bp}
    \definecolor{sc}{rgb}{0.0000,0.0000,0.0000}
    \pgfsetstrokecolor{sc}
    \pgfsetmiterjoin
    \pgfsetbuttcap
    \pgfpathqmoveto{354.2857bp}{28.5714bp}
    \pgfpathqcurveto{354.2857bp}{31.7273bp}{351.7273bp}{34.2857bp}{348.5714bp}{34.2857bp}
    \pgfpathqcurveto{345.4155bp}{34.2857bp}{342.8571bp}{31.7273bp}{342.8571bp}{28.5714bp}
    \pgfpathqcurveto{342.8571bp}{25.4155bp}{345.4155bp}{22.8571bp}{348.5714bp}{22.8571bp}
    \pgfpathqcurveto{351.7273bp}{22.8571bp}{354.2857bp}{25.4155bp}{354.2857bp}{28.5714bp}
    \pgfpathclose
    \pgfusepathqfillstroke
  \end{pgfscope}
  \begin{pgfscope}
    \definecolor{fc}{rgb}{0.0000,0.0000,0.0000}
    \pgfsetfillcolor{fc}
    \pgfsetfillopacity{0.0000}
    \pgfsetlinewidth{0.8113bp}
    \definecolor{sc}{rgb}{0.0000,0.0000,0.0000}
    \pgfsetstrokecolor{sc}
    \pgfsetmiterjoin
    \pgfsetbuttcap
    \pgfpathqmoveto{354.2857bp}{51.4286bp}
    \pgfpathqcurveto{354.2857bp}{54.5845bp}{351.7273bp}{57.1429bp}{348.5714bp}{57.1429bp}
    \pgfpathqcurveto{345.4155bp}{57.1429bp}{342.8571bp}{54.5845bp}{342.8571bp}{51.4286bp}
    \pgfpathqcurveto{342.8571bp}{48.2727bp}{345.4155bp}{45.7143bp}{348.5714bp}{45.7143bp}
    \pgfpathqcurveto{351.7273bp}{45.7143bp}{354.2857bp}{48.2727bp}{354.2857bp}{51.4286bp}
    \pgfpathclose
    \pgfusepathqfillstroke
  \end{pgfscope}
  \begin{pgfscope}
    \pgfsetlinewidth{0.8113bp}
    \definecolor{sc}{rgb}{0.0000,0.0000,0.0000}
    \pgfsetstrokecolor{sc}
    \pgfsetmiterjoin
    \pgfsetbuttcap
    \pgfpathqmoveto{348.5714bp}{45.7143bp}
    \pgfpathqlineto{348.5714bp}{34.2857bp}
    \pgfusepathqstroke
  \end{pgfscope}
  \begin{pgfscope}
    \definecolor{fc}{rgb}{0.0000,0.0000,0.0000}
    \pgfsetfillcolor{fc}
    \pgfusepathqfill
  \end{pgfscope}
  \begin{pgfscope}
    \definecolor{fc}{rgb}{0.0000,0.0000,0.0000}
    \pgfsetfillcolor{fc}
    \pgfusepathqfill
  \end{pgfscope}
  \begin{pgfscope}
    \definecolor{fc}{rgb}{0.0000,0.0000,0.0000}
    \pgfsetfillcolor{fc}
    \pgfusepathqfill
  \end{pgfscope}
  \begin{pgfscope}
    \definecolor{fc}{rgb}{0.0000,0.0000,0.0000}
    \pgfsetfillcolor{fc}
    \pgfusepathqfill
  \end{pgfscope}
  \begin{pgfscope}
    \definecolor{fc}{rgb}{0.0000,0.0000,0.0000}
    \pgfsetfillcolor{fc}
    \pgfsetfillopacity{0.0000}
    \pgfsetlinewidth{0.8113bp}
    \definecolor{sc}{rgb}{0.0000,0.0000,0.0000}
    \pgfsetstrokecolor{sc}
    \pgfsetmiterjoin
    \pgfsetbuttcap
    \pgfpathqmoveto{331.4286bp}{51.4286bp}
    \pgfpathqcurveto{331.4286bp}{54.5845bp}{328.8702bp}{57.1429bp}{325.7143bp}{57.1429bp}
    \pgfpathqcurveto{322.5584bp}{57.1429bp}{320.0000bp}{54.5845bp}{320.0000bp}{51.4286bp}
    \pgfpathqcurveto{320.0000bp}{48.2727bp}{322.5584bp}{45.7143bp}{325.7143bp}{45.7143bp}
    \pgfpathqcurveto{328.8702bp}{45.7143bp}{331.4286bp}{48.2727bp}{331.4286bp}{51.4286bp}
    \pgfpathclose
    \pgfusepathqfillstroke
  \end{pgfscope}
  \begin{pgfscope}
    \definecolor{fc}{rgb}{0.0000,0.0000,0.0000}
    \pgfsetfillcolor{fc}
    \pgfsetfillopacity{0.0000}
    \pgfsetlinewidth{0.8113bp}
    \definecolor{sc}{rgb}{0.0000,0.0000,0.0000}
    \pgfsetstrokecolor{sc}
    \pgfsetmiterjoin
    \pgfsetbuttcap
    \pgfpathqmoveto{331.4286bp}{74.2857bp}
    \pgfpathqcurveto{331.4286bp}{77.4416bp}{328.8702bp}{80.0000bp}{325.7143bp}{80.0000bp}
    \pgfpathqcurveto{322.5584bp}{80.0000bp}{320.0000bp}{77.4416bp}{320.0000bp}{74.2857bp}
    \pgfpathqcurveto{320.0000bp}{71.1298bp}{322.5584bp}{68.5714bp}{325.7143bp}{68.5714bp}
    \pgfpathqcurveto{328.8702bp}{68.5714bp}{331.4286bp}{71.1298bp}{331.4286bp}{74.2857bp}
    \pgfpathclose
    \pgfusepathqfillstroke
  \end{pgfscope}
  \begin{pgfscope}
    \pgfsetlinewidth{0.8113bp}
    \definecolor{sc}{rgb}{0.0000,0.0000,0.0000}
    \pgfsetstrokecolor{sc}
    \pgfsetmiterjoin
    \pgfsetbuttcap
    \pgfpathqmoveto{330.8264bp}{71.7296bp}
    \pgfpathqlineto{366.3164bp}{53.9846bp}
    \pgfusepathqstroke
  \end{pgfscope}
  \begin{pgfscope}
    \definecolor{fc}{rgb}{0.0000,0.0000,0.0000}
    \pgfsetfillcolor{fc}
    \pgfusepathqfill
  \end{pgfscope}
  \begin{pgfscope}
    \definecolor{fc}{rgb}{0.0000,0.0000,0.0000}
    \pgfsetfillcolor{fc}
    \pgfusepathqfill
  \end{pgfscope}
  \begin{pgfscope}
    \definecolor{fc}{rgb}{0.0000,0.0000,0.0000}
    \pgfsetfillcolor{fc}
    \pgfusepathqfill
  \end{pgfscope}
  \begin{pgfscope}
    \definecolor{fc}{rgb}{0.0000,0.0000,0.0000}
    \pgfsetfillcolor{fc}
    \pgfusepathqfill
  \end{pgfscope}
  \begin{pgfscope}
    \pgfsetlinewidth{0.8113bp}
    \definecolor{sc}{rgb}{0.0000,0.0000,0.0000}
    \pgfsetstrokecolor{sc}
    \pgfsetmiterjoin
    \pgfsetbuttcap
    \pgfpathqmoveto{329.7549bp}{70.2451bp}
    \pgfpathqlineto{344.5308bp}{55.4692bp}
    \pgfusepathqstroke
  \end{pgfscope}
  \begin{pgfscope}
    \definecolor{fc}{rgb}{0.0000,0.0000,0.0000}
    \pgfsetfillcolor{fc}
    \pgfusepathqfill
  \end{pgfscope}
  \begin{pgfscope}
    \definecolor{fc}{rgb}{0.0000,0.0000,0.0000}
    \pgfsetfillcolor{fc}
    \pgfusepathqfill
  \end{pgfscope}
  \begin{pgfscope}
    \definecolor{fc}{rgb}{0.0000,0.0000,0.0000}
    \pgfsetfillcolor{fc}
    \pgfusepathqfill
  \end{pgfscope}
  \begin{pgfscope}
    \definecolor{fc}{rgb}{0.0000,0.0000,0.0000}
    \pgfsetfillcolor{fc}
    \pgfusepathqfill
  \end{pgfscope}
  \begin{pgfscope}
    \pgfsetlinewidth{0.8113bp}
    \definecolor{sc}{rgb}{0.0000,0.0000,0.0000}
    \pgfsetstrokecolor{sc}
    \pgfsetmiterjoin
    \pgfsetbuttcap
    \pgfpathqmoveto{325.7143bp}{68.5714bp}
    \pgfpathqlineto{325.7143bp}{57.1429bp}
    \pgfusepathqstroke
  \end{pgfscope}
  \begin{pgfscope}
    \definecolor{fc}{rgb}{0.0000,0.0000,0.0000}
    \pgfsetfillcolor{fc}
    \pgfusepathqfill
  \end{pgfscope}
  \begin{pgfscope}
    \definecolor{fc}{rgb}{0.0000,0.0000,0.0000}
    \pgfsetfillcolor{fc}
    \pgfusepathqfill
  \end{pgfscope}
  \begin{pgfscope}
    \definecolor{fc}{rgb}{0.0000,0.0000,0.0000}
    \pgfsetfillcolor{fc}
    \pgfusepathqfill
  \end{pgfscope}
  \begin{pgfscope}
    \definecolor{fc}{rgb}{0.0000,0.0000,0.0000}
    \pgfsetfillcolor{fc}
    \pgfusepathqfill
  \end{pgfscope}
  \begin{pgfscope}
    \definecolor{fc}{rgb}{0.0000,0.0000,0.0000}
    \pgfsetfillcolor{fc}
    \pgfsetfillopacity{0.0000}
    \pgfsetlinewidth{0.8113bp}
    \definecolor{sc}{rgb}{0.0000,0.0000,0.0000}
    \pgfsetstrokecolor{sc}
    \pgfsetmiterjoin
    \pgfsetbuttcap
    \pgfpathqmoveto{308.5714bp}{28.5714bp}
    \pgfpathqcurveto{308.5714bp}{31.7273bp}{306.0131bp}{34.2857bp}{302.8571bp}{34.2857bp}
    \pgfpathqcurveto{299.7012bp}{34.2857bp}{297.1429bp}{31.7273bp}{297.1429bp}{28.5714bp}
    \pgfpathqcurveto{297.1429bp}{25.4155bp}{299.7012bp}{22.8571bp}{302.8571bp}{22.8571bp}
    \pgfpathqcurveto{306.0131bp}{22.8571bp}{308.5714bp}{25.4155bp}{308.5714bp}{28.5714bp}
    \pgfpathclose
    \pgfusepathqfillstroke
  \end{pgfscope}
  \begin{pgfscope}
    \definecolor{fc}{rgb}{0.0000,0.0000,0.0000}
    \pgfsetfillcolor{fc}
    \pgfsetfillopacity{0.0000}
    \pgfsetlinewidth{0.8113bp}
    \definecolor{sc}{rgb}{0.0000,0.0000,0.0000}
    \pgfsetstrokecolor{sc}
    \pgfsetmiterjoin
    \pgfsetbuttcap
    \pgfpathqmoveto{308.5714bp}{51.4286bp}
    \pgfpathqcurveto{308.5714bp}{54.5845bp}{306.0131bp}{57.1429bp}{302.8571bp}{57.1429bp}
    \pgfpathqcurveto{299.7012bp}{57.1429bp}{297.1429bp}{54.5845bp}{297.1429bp}{51.4286bp}
    \pgfpathqcurveto{297.1429bp}{48.2727bp}{299.7012bp}{45.7143bp}{302.8571bp}{45.7143bp}
    \pgfpathqcurveto{306.0131bp}{45.7143bp}{308.5714bp}{48.2727bp}{308.5714bp}{51.4286bp}
    \pgfpathclose
    \pgfusepathqfillstroke
  \end{pgfscope}
  \begin{pgfscope}
    \pgfsetlinewidth{0.8113bp}
    \definecolor{sc}{rgb}{0.0000,0.0000,0.0000}
    \pgfsetstrokecolor{sc}
    \pgfsetmiterjoin
    \pgfsetbuttcap
    \pgfpathqmoveto{302.8571bp}{45.7143bp}
    \pgfpathqlineto{302.8571bp}{34.2857bp}
    \pgfusepathqstroke
  \end{pgfscope}
  \begin{pgfscope}
    \definecolor{fc}{rgb}{0.0000,0.0000,0.0000}
    \pgfsetfillcolor{fc}
    \pgfusepathqfill
  \end{pgfscope}
  \begin{pgfscope}
    \definecolor{fc}{rgb}{0.0000,0.0000,0.0000}
    \pgfsetfillcolor{fc}
    \pgfusepathqfill
  \end{pgfscope}
  \begin{pgfscope}
    \definecolor{fc}{rgb}{0.0000,0.0000,0.0000}
    \pgfsetfillcolor{fc}
    \pgfusepathqfill
  \end{pgfscope}
  \begin{pgfscope}
    \definecolor{fc}{rgb}{0.0000,0.0000,0.0000}
    \pgfsetfillcolor{fc}
    \pgfusepathqfill
  \end{pgfscope}
  \begin{pgfscope}
    \definecolor{fc}{rgb}{0.0000,0.0000,0.0000}
    \pgfsetfillcolor{fc}
    \pgfsetfillopacity{0.0000}
    \pgfsetlinewidth{0.8113bp}
    \definecolor{sc}{rgb}{0.0000,0.0000,0.0000}
    \pgfsetstrokecolor{sc}
    \pgfsetmiterjoin
    \pgfsetbuttcap
    \pgfpathqmoveto{285.7143bp}{51.4286bp}
    \pgfpathqcurveto{285.7143bp}{54.5845bp}{283.1559bp}{57.1429bp}{280.0000bp}{57.1429bp}
    \pgfpathqcurveto{276.8441bp}{57.1429bp}{274.2857bp}{54.5845bp}{274.2857bp}{51.4286bp}
    \pgfpathqcurveto{274.2857bp}{48.2727bp}{276.8441bp}{45.7143bp}{280.0000bp}{45.7143bp}
    \pgfpathqcurveto{283.1559bp}{45.7143bp}{285.7143bp}{48.2727bp}{285.7143bp}{51.4286bp}
    \pgfpathclose
    \pgfusepathqfillstroke
  \end{pgfscope}
  \begin{pgfscope}
    \definecolor{fc}{rgb}{0.0000,0.0000,0.0000}
    \pgfsetfillcolor{fc}
    \pgfsetfillopacity{0.0000}
    \pgfsetlinewidth{0.8113bp}
    \definecolor{sc}{rgb}{0.0000,0.0000,0.0000}
    \pgfsetstrokecolor{sc}
    \pgfsetmiterjoin
    \pgfsetbuttcap
    \pgfpathqmoveto{285.7143bp}{74.2857bp}
    \pgfpathqcurveto{285.7143bp}{77.4416bp}{283.1559bp}{80.0000bp}{280.0000bp}{80.0000bp}
    \pgfpathqcurveto{276.8441bp}{80.0000bp}{274.2857bp}{77.4416bp}{274.2857bp}{74.2857bp}
    \pgfpathqcurveto{274.2857bp}{71.1298bp}{276.8441bp}{68.5714bp}{280.0000bp}{68.5714bp}
    \pgfpathqcurveto{283.1559bp}{68.5714bp}{285.7143bp}{71.1298bp}{285.7143bp}{74.2857bp}
    \pgfpathclose
    \pgfusepathqfillstroke
  \end{pgfscope}
  \begin{pgfscope}
    \pgfsetlinewidth{0.8113bp}
    \definecolor{sc}{rgb}{0.0000,0.0000,0.0000}
    \pgfsetstrokecolor{sc}
    \pgfsetmiterjoin
    \pgfsetbuttcap
    \pgfpathqmoveto{284.0406bp}{70.2451bp}
    \pgfpathqlineto{298.8165bp}{55.4692bp}
    \pgfusepathqstroke
  \end{pgfscope}
  \begin{pgfscope}
    \definecolor{fc}{rgb}{0.0000,0.0000,0.0000}
    \pgfsetfillcolor{fc}
    \pgfusepathqfill
  \end{pgfscope}
  \begin{pgfscope}
    \definecolor{fc}{rgb}{0.0000,0.0000,0.0000}
    \pgfsetfillcolor{fc}
    \pgfusepathqfill
  \end{pgfscope}
  \begin{pgfscope}
    \definecolor{fc}{rgb}{0.0000,0.0000,0.0000}
    \pgfsetfillcolor{fc}
    \pgfusepathqfill
  \end{pgfscope}
  \begin{pgfscope}
    \definecolor{fc}{rgb}{0.0000,0.0000,0.0000}
    \pgfsetfillcolor{fc}
    \pgfusepathqfill
  \end{pgfscope}
  \begin{pgfscope}
    \pgfsetlinewidth{0.8113bp}
    \definecolor{sc}{rgb}{0.0000,0.0000,0.0000}
    \pgfsetstrokecolor{sc}
    \pgfsetmiterjoin
    \pgfsetbuttcap
    \pgfpathqmoveto{280.0000bp}{68.5714bp}
    \pgfpathqlineto{280.0000bp}{57.1429bp}
    \pgfusepathqstroke
  \end{pgfscope}
  \begin{pgfscope}
    \definecolor{fc}{rgb}{0.0000,0.0000,0.0000}
    \pgfsetfillcolor{fc}
    \pgfusepathqfill
  \end{pgfscope}
  \begin{pgfscope}
    \definecolor{fc}{rgb}{0.0000,0.0000,0.0000}
    \pgfsetfillcolor{fc}
    \pgfusepathqfill
  \end{pgfscope}
  \begin{pgfscope}
    \definecolor{fc}{rgb}{0.0000,0.0000,0.0000}
    \pgfsetfillcolor{fc}
    \pgfusepathqfill
  \end{pgfscope}
  \begin{pgfscope}
    \definecolor{fc}{rgb}{0.0000,0.0000,0.0000}
    \pgfsetfillcolor{fc}
    \pgfusepathqfill
  \end{pgfscope}
  \begin{pgfscope}
    \definecolor{fc}{rgb}{0.0000,0.0000,0.0000}
    \pgfsetfillcolor{fc}
    \pgfsetfillopacity{0.0000}
    \pgfsetlinewidth{0.8113bp}
    \definecolor{sc}{rgb}{0.0000,0.0000,0.0000}
    \pgfsetstrokecolor{sc}
    \pgfsetmiterjoin
    \pgfsetbuttcap
    \pgfpathqmoveto{262.8571bp}{51.4286bp}
    \pgfpathqcurveto{262.8571bp}{54.5845bp}{260.2988bp}{57.1429bp}{257.1429bp}{57.1429bp}
    \pgfpathqcurveto{253.9869bp}{57.1429bp}{251.4286bp}{54.5845bp}{251.4286bp}{51.4286bp}
    \pgfpathqcurveto{251.4286bp}{48.2727bp}{253.9869bp}{45.7143bp}{257.1429bp}{45.7143bp}
    \pgfpathqcurveto{260.2988bp}{45.7143bp}{262.8571bp}{48.2727bp}{262.8571bp}{51.4286bp}
    \pgfpathclose
    \pgfusepathqfillstroke
  \end{pgfscope}
  \begin{pgfscope}
    \definecolor{fc}{rgb}{0.0000,0.0000,0.0000}
    \pgfsetfillcolor{fc}
    \pgfsetfillopacity{0.0000}
    \pgfsetlinewidth{0.8113bp}
    \definecolor{sc}{rgb}{0.0000,0.0000,0.0000}
    \pgfsetstrokecolor{sc}
    \pgfsetmiterjoin
    \pgfsetbuttcap
    \pgfpathqmoveto{262.8571bp}{74.2857bp}
    \pgfpathqcurveto{262.8571bp}{77.4416bp}{260.2988bp}{80.0000bp}{257.1429bp}{80.0000bp}
    \pgfpathqcurveto{253.9869bp}{80.0000bp}{251.4286bp}{77.4416bp}{251.4286bp}{74.2857bp}
    \pgfpathqcurveto{251.4286bp}{71.1298bp}{253.9869bp}{68.5714bp}{257.1429bp}{68.5714bp}
    \pgfpathqcurveto{260.2988bp}{68.5714bp}{262.8571bp}{71.1298bp}{262.8571bp}{74.2857bp}
    \pgfpathclose
    \pgfusepathqfillstroke
  \end{pgfscope}
  \begin{pgfscope}
    \pgfsetlinewidth{0.8113bp}
    \definecolor{sc}{rgb}{0.0000,0.0000,0.0000}
    \pgfsetstrokecolor{sc}
    \pgfsetmiterjoin
    \pgfsetbuttcap
    \pgfpathqmoveto{257.1429bp}{68.5714bp}
    \pgfpathqlineto{257.1429bp}{57.1429bp}
    \pgfusepathqstroke
  \end{pgfscope}
  \begin{pgfscope}
    \definecolor{fc}{rgb}{0.0000,0.0000,0.0000}
    \pgfsetfillcolor{fc}
    \pgfusepathqfill
  \end{pgfscope}
  \begin{pgfscope}
    \definecolor{fc}{rgb}{0.0000,0.0000,0.0000}
    \pgfsetfillcolor{fc}
    \pgfusepathqfill
  \end{pgfscope}
  \begin{pgfscope}
    \definecolor{fc}{rgb}{0.0000,0.0000,0.0000}
    \pgfsetfillcolor{fc}
    \pgfusepathqfill
  \end{pgfscope}
  \begin{pgfscope}
    \definecolor{fc}{rgb}{0.0000,0.0000,0.0000}
    \pgfsetfillcolor{fc}
    \pgfusepathqfill
  \end{pgfscope}
  \begin{pgfscope}
    \definecolor{fc}{rgb}{0.0000,0.0000,0.0000}
    \pgfsetfillcolor{fc}
    \pgfsetfillopacity{0.0000}
    \pgfsetlinewidth{0.8113bp}
    \definecolor{sc}{rgb}{0.0000,0.0000,0.0000}
    \pgfsetstrokecolor{sc}
    \pgfsetmiterjoin
    \pgfsetbuttcap
    \pgfpathqmoveto{240.0000bp}{74.2857bp}
    \pgfpathqcurveto{240.0000bp}{77.4416bp}{237.4416bp}{80.0000bp}{234.2857bp}{80.0000bp}
    \pgfpathqcurveto{231.1298bp}{80.0000bp}{228.5714bp}{77.4416bp}{228.5714bp}{74.2857bp}
    \pgfpathqcurveto{228.5714bp}{71.1298bp}{231.1298bp}{68.5714bp}{234.2857bp}{68.5714bp}
    \pgfpathqcurveto{237.4416bp}{68.5714bp}{240.0000bp}{71.1298bp}{240.0000bp}{74.2857bp}
    \pgfpathclose
    \pgfusepathqfillstroke
  \end{pgfscope}
  \begin{pgfscope}
    \definecolor{fc}{rgb}{0.0000,0.0000,0.0000}
    \pgfsetfillcolor{fc}
    \pgfsetfillopacity{0.0000}
    \pgfsetlinewidth{0.8113bp}
    \definecolor{sc}{rgb}{0.0000,0.0000,0.0000}
    \pgfsetstrokecolor{sc}
    \pgfsetmiterjoin
    \pgfsetbuttcap
    \pgfpathqmoveto{240.0000bp}{97.1429bp}
    \pgfpathqcurveto{240.0000bp}{100.2988bp}{237.4416bp}{102.8571bp}{234.2857bp}{102.8571bp}
    \pgfpathqcurveto{231.1298bp}{102.8571bp}{228.5714bp}{100.2988bp}{228.5714bp}{97.1429bp}
    \pgfpathqcurveto{228.5714bp}{93.9869bp}{231.1298bp}{91.4286bp}{234.2857bp}{91.4286bp}
    \pgfpathqcurveto{237.4416bp}{91.4286bp}{240.0000bp}{93.9869bp}{240.0000bp}{97.1429bp}
    \pgfpathclose
    \pgfusepathqfillstroke
  \end{pgfscope}
  \begin{pgfscope}
    \pgfsetlinewidth{0.8113bp}
    \definecolor{sc}{rgb}{0.0000,0.0000,0.0000}
    \pgfsetstrokecolor{sc}
    \pgfsetmiterjoin
    \pgfsetbuttcap
    \pgfpathqmoveto{239.8307bp}{95.7566bp}
    \pgfpathqlineto{320.1693bp}{75.6720bp}
    \pgfusepathqstroke
  \end{pgfscope}
  \begin{pgfscope}
    \definecolor{fc}{rgb}{0.0000,0.0000,0.0000}
    \pgfsetfillcolor{fc}
    \pgfusepathqfill
  \end{pgfscope}
  \begin{pgfscope}
    \definecolor{fc}{rgb}{0.0000,0.0000,0.0000}
    \pgfsetfillcolor{fc}
    \pgfusepathqfill
  \end{pgfscope}
  \begin{pgfscope}
    \definecolor{fc}{rgb}{0.0000,0.0000,0.0000}
    \pgfsetfillcolor{fc}
    \pgfusepathqfill
  \end{pgfscope}
  \begin{pgfscope}
    \definecolor{fc}{rgb}{0.0000,0.0000,0.0000}
    \pgfsetfillcolor{fc}
    \pgfusepathqfill
  \end{pgfscope}
  \begin{pgfscope}
    \pgfsetlinewidth{0.8113bp}
    \definecolor{sc}{rgb}{0.0000,0.0000,0.0000}
    \pgfsetstrokecolor{sc}
    \pgfsetmiterjoin
    \pgfsetbuttcap
    \pgfpathqmoveto{239.3978bp}{94.5868bp}
    \pgfpathqlineto{274.8879bp}{76.8418bp}
    \pgfusepathqstroke
  \end{pgfscope}
  \begin{pgfscope}
    \definecolor{fc}{rgb}{0.0000,0.0000,0.0000}
    \pgfsetfillcolor{fc}
    \pgfusepathqfill
  \end{pgfscope}
  \begin{pgfscope}
    \definecolor{fc}{rgb}{0.0000,0.0000,0.0000}
    \pgfsetfillcolor{fc}
    \pgfusepathqfill
  \end{pgfscope}
  \begin{pgfscope}
    \definecolor{fc}{rgb}{0.0000,0.0000,0.0000}
    \pgfsetfillcolor{fc}
    \pgfusepathqfill
  \end{pgfscope}
  \begin{pgfscope}
    \definecolor{fc}{rgb}{0.0000,0.0000,0.0000}
    \pgfsetfillcolor{fc}
    \pgfusepathqfill
  \end{pgfscope}
  \begin{pgfscope}
    \pgfsetlinewidth{0.8113bp}
    \definecolor{sc}{rgb}{0.0000,0.0000,0.0000}
    \pgfsetstrokecolor{sc}
    \pgfsetmiterjoin
    \pgfsetbuttcap
    \pgfpathqmoveto{238.3263bp}{93.1022bp}
    \pgfpathqlineto{253.1022bp}{78.3263bp}
    \pgfusepathqstroke
  \end{pgfscope}
  \begin{pgfscope}
    \definecolor{fc}{rgb}{0.0000,0.0000,0.0000}
    \pgfsetfillcolor{fc}
    \pgfusepathqfill
  \end{pgfscope}
  \begin{pgfscope}
    \definecolor{fc}{rgb}{0.0000,0.0000,0.0000}
    \pgfsetfillcolor{fc}
    \pgfusepathqfill
  \end{pgfscope}
  \begin{pgfscope}
    \definecolor{fc}{rgb}{0.0000,0.0000,0.0000}
    \pgfsetfillcolor{fc}
    \pgfusepathqfill
  \end{pgfscope}
  \begin{pgfscope}
    \definecolor{fc}{rgb}{0.0000,0.0000,0.0000}
    \pgfsetfillcolor{fc}
    \pgfusepathqfill
  \end{pgfscope}
  \begin{pgfscope}
    \pgfsetlinewidth{0.8113bp}
    \definecolor{sc}{rgb}{0.0000,0.0000,0.0000}
    \pgfsetstrokecolor{sc}
    \pgfsetmiterjoin
    \pgfsetbuttcap
    \pgfpathqmoveto{234.2857bp}{91.4286bp}
    \pgfpathqlineto{234.2857bp}{80.0000bp}
    \pgfusepathqstroke
  \end{pgfscope}
  \begin{pgfscope}
    \definecolor{fc}{rgb}{0.0000,0.0000,0.0000}
    \pgfsetfillcolor{fc}
    \pgfusepathqfill
  \end{pgfscope}
  \begin{pgfscope}
    \definecolor{fc}{rgb}{0.0000,0.0000,0.0000}
    \pgfsetfillcolor{fc}
    \pgfusepathqfill
  \end{pgfscope}
  \begin{pgfscope}
    \definecolor{fc}{rgb}{0.0000,0.0000,0.0000}
    \pgfsetfillcolor{fc}
    \pgfusepathqfill
  \end{pgfscope}
  \begin{pgfscope}
    \definecolor{fc}{rgb}{0.0000,0.0000,0.0000}
    \pgfsetfillcolor{fc}
    \pgfusepathqfill
  \end{pgfscope}
  \begin{pgfscope}
    \definecolor{fc}{rgb}{0.0000,0.0000,0.0000}
    \pgfsetfillcolor{fc}
    \pgfsetfillopacity{0.0000}
    \pgfsetlinewidth{0.8113bp}
    \definecolor{sc}{rgb}{0.0000,0.0000,0.0000}
    \pgfsetstrokecolor{sc}
    \pgfsetmiterjoin
    \pgfsetbuttcap
    \pgfpathqmoveto{205.7143bp}{28.5714bp}
    \pgfpathqcurveto{205.7143bp}{31.7273bp}{203.1559bp}{34.2857bp}{200.0000bp}{34.2857bp}
    \pgfpathqcurveto{196.8441bp}{34.2857bp}{194.2857bp}{31.7273bp}{194.2857bp}{28.5714bp}
    \pgfpathqcurveto{194.2857bp}{25.4155bp}{196.8441bp}{22.8571bp}{200.0000bp}{22.8571bp}
    \pgfpathqcurveto{203.1559bp}{22.8571bp}{205.7143bp}{25.4155bp}{205.7143bp}{28.5714bp}
    \pgfpathclose
    \pgfusepathqfillstroke
  \end{pgfscope}
  \begin{pgfscope}
    \definecolor{fc}{rgb}{0.0000,0.0000,0.0000}
    \pgfsetfillcolor{fc}
    \pgfsetfillopacity{0.0000}
    \pgfsetlinewidth{0.8113bp}
    \definecolor{sc}{rgb}{0.0000,0.0000,0.0000}
    \pgfsetstrokecolor{sc}
    \pgfsetmiterjoin
    \pgfsetbuttcap
    \pgfpathqmoveto{205.7143bp}{51.4286bp}
    \pgfpathqcurveto{205.7143bp}{54.5845bp}{203.1559bp}{57.1429bp}{200.0000bp}{57.1429bp}
    \pgfpathqcurveto{196.8441bp}{57.1429bp}{194.2857bp}{54.5845bp}{194.2857bp}{51.4286bp}
    \pgfpathqcurveto{194.2857bp}{48.2727bp}{196.8441bp}{45.7143bp}{200.0000bp}{45.7143bp}
    \pgfpathqcurveto{203.1559bp}{45.7143bp}{205.7143bp}{48.2727bp}{205.7143bp}{51.4286bp}
    \pgfpathclose
    \pgfusepathqfillstroke
  \end{pgfscope}
  \begin{pgfscope}
    \pgfsetlinewidth{0.8113bp}
    \definecolor{sc}{rgb}{0.0000,0.0000,0.0000}
    \pgfsetstrokecolor{sc}
    \pgfsetmiterjoin
    \pgfsetbuttcap
    \pgfpathqmoveto{200.0000bp}{45.7143bp}
    \pgfpathqlineto{200.0000bp}{34.2857bp}
    \pgfusepathqstroke
  \end{pgfscope}
  \begin{pgfscope}
    \definecolor{fc}{rgb}{0.0000,0.0000,0.0000}
    \pgfsetfillcolor{fc}
    \pgfusepathqfill
  \end{pgfscope}
  \begin{pgfscope}
    \definecolor{fc}{rgb}{0.0000,0.0000,0.0000}
    \pgfsetfillcolor{fc}
    \pgfusepathqfill
  \end{pgfscope}
  \begin{pgfscope}
    \definecolor{fc}{rgb}{0.0000,0.0000,0.0000}
    \pgfsetfillcolor{fc}
    \pgfusepathqfill
  \end{pgfscope}
  \begin{pgfscope}
    \definecolor{fc}{rgb}{0.0000,0.0000,0.0000}
    \pgfsetfillcolor{fc}
    \pgfusepathqfill
  \end{pgfscope}
  \begin{pgfscope}
    \definecolor{fc}{rgb}{0.0000,0.0000,0.0000}
    \pgfsetfillcolor{fc}
    \pgfsetfillopacity{0.0000}
    \pgfsetlinewidth{0.8113bp}
    \definecolor{sc}{rgb}{0.0000,0.0000,0.0000}
    \pgfsetstrokecolor{sc}
    \pgfsetmiterjoin
    \pgfsetbuttcap
    \pgfpathqmoveto{182.8571bp}{51.4286bp}
    \pgfpathqcurveto{182.8571bp}{54.5845bp}{180.2988bp}{57.1429bp}{177.1429bp}{57.1429bp}
    \pgfpathqcurveto{173.9869bp}{57.1429bp}{171.4286bp}{54.5845bp}{171.4286bp}{51.4286bp}
    \pgfpathqcurveto{171.4286bp}{48.2727bp}{173.9869bp}{45.7143bp}{177.1429bp}{45.7143bp}
    \pgfpathqcurveto{180.2988bp}{45.7143bp}{182.8571bp}{48.2727bp}{182.8571bp}{51.4286bp}
    \pgfpathclose
    \pgfusepathqfillstroke
  \end{pgfscope}
  \begin{pgfscope}
    \definecolor{fc}{rgb}{0.0000,0.0000,0.0000}
    \pgfsetfillcolor{fc}
    \pgfsetfillopacity{0.0000}
    \pgfsetlinewidth{0.8113bp}
    \definecolor{sc}{rgb}{0.0000,0.0000,0.0000}
    \pgfsetstrokecolor{sc}
    \pgfsetmiterjoin
    \pgfsetbuttcap
    \pgfpathqmoveto{182.8571bp}{74.2857bp}
    \pgfpathqcurveto{182.8571bp}{77.4416bp}{180.2988bp}{80.0000bp}{177.1429bp}{80.0000bp}
    \pgfpathqcurveto{173.9869bp}{80.0000bp}{171.4286bp}{77.4416bp}{171.4286bp}{74.2857bp}
    \pgfpathqcurveto{171.4286bp}{71.1298bp}{173.9869bp}{68.5714bp}{177.1429bp}{68.5714bp}
    \pgfpathqcurveto{180.2988bp}{68.5714bp}{182.8571bp}{71.1298bp}{182.8571bp}{74.2857bp}
    \pgfpathclose
    \pgfusepathqfillstroke
  \end{pgfscope}
  \begin{pgfscope}
    \pgfsetlinewidth{0.8113bp}
    \definecolor{sc}{rgb}{0.0000,0.0000,0.0000}
    \pgfsetstrokecolor{sc}
    \pgfsetmiterjoin
    \pgfsetbuttcap
    \pgfpathqmoveto{181.1835bp}{70.2451bp}
    \pgfpathqlineto{195.9594bp}{55.4692bp}
    \pgfusepathqstroke
  \end{pgfscope}
  \begin{pgfscope}
    \definecolor{fc}{rgb}{0.0000,0.0000,0.0000}
    \pgfsetfillcolor{fc}
    \pgfusepathqfill
  \end{pgfscope}
  \begin{pgfscope}
    \definecolor{fc}{rgb}{0.0000,0.0000,0.0000}
    \pgfsetfillcolor{fc}
    \pgfusepathqfill
  \end{pgfscope}
  \begin{pgfscope}
    \definecolor{fc}{rgb}{0.0000,0.0000,0.0000}
    \pgfsetfillcolor{fc}
    \pgfusepathqfill
  \end{pgfscope}
  \begin{pgfscope}
    \definecolor{fc}{rgb}{0.0000,0.0000,0.0000}
    \pgfsetfillcolor{fc}
    \pgfusepathqfill
  \end{pgfscope}
  \begin{pgfscope}
    \pgfsetlinewidth{0.8113bp}
    \definecolor{sc}{rgb}{0.0000,0.0000,0.0000}
    \pgfsetstrokecolor{sc}
    \pgfsetmiterjoin
    \pgfsetbuttcap
    \pgfpathqmoveto{177.1429bp}{68.5714bp}
    \pgfpathqlineto{177.1429bp}{57.1429bp}
    \pgfusepathqstroke
  \end{pgfscope}
  \begin{pgfscope}
    \definecolor{fc}{rgb}{0.0000,0.0000,0.0000}
    \pgfsetfillcolor{fc}
    \pgfusepathqfill
  \end{pgfscope}
  \begin{pgfscope}
    \definecolor{fc}{rgb}{0.0000,0.0000,0.0000}
    \pgfsetfillcolor{fc}
    \pgfusepathqfill
  \end{pgfscope}
  \begin{pgfscope}
    \definecolor{fc}{rgb}{0.0000,0.0000,0.0000}
    \pgfsetfillcolor{fc}
    \pgfusepathqfill
  \end{pgfscope}
  \begin{pgfscope}
    \definecolor{fc}{rgb}{0.0000,0.0000,0.0000}
    \pgfsetfillcolor{fc}
    \pgfusepathqfill
  \end{pgfscope}
  \begin{pgfscope}
    \definecolor{fc}{rgb}{0.0000,0.0000,0.0000}
    \pgfsetfillcolor{fc}
    \pgfsetfillopacity{0.0000}
    \pgfsetlinewidth{0.8113bp}
    \definecolor{sc}{rgb}{0.0000,0.0000,0.0000}
    \pgfsetstrokecolor{sc}
    \pgfsetmiterjoin
    \pgfsetbuttcap
    \pgfpathqmoveto{160.0000bp}{51.4286bp}
    \pgfpathqcurveto{160.0000bp}{54.5845bp}{157.4416bp}{57.1429bp}{154.2857bp}{57.1429bp}
    \pgfpathqcurveto{151.1298bp}{57.1429bp}{148.5714bp}{54.5845bp}{148.5714bp}{51.4286bp}
    \pgfpathqcurveto{148.5714bp}{48.2727bp}{151.1298bp}{45.7143bp}{154.2857bp}{45.7143bp}
    \pgfpathqcurveto{157.4416bp}{45.7143bp}{160.0000bp}{48.2727bp}{160.0000bp}{51.4286bp}
    \pgfpathclose
    \pgfusepathqfillstroke
  \end{pgfscope}
  \begin{pgfscope}
    \definecolor{fc}{rgb}{0.0000,0.0000,0.0000}
    \pgfsetfillcolor{fc}
    \pgfsetfillopacity{0.0000}
    \pgfsetlinewidth{0.8113bp}
    \definecolor{sc}{rgb}{0.0000,0.0000,0.0000}
    \pgfsetstrokecolor{sc}
    \pgfsetmiterjoin
    \pgfsetbuttcap
    \pgfpathqmoveto{160.0000bp}{74.2857bp}
    \pgfpathqcurveto{160.0000bp}{77.4416bp}{157.4416bp}{80.0000bp}{154.2857bp}{80.0000bp}
    \pgfpathqcurveto{151.1298bp}{80.0000bp}{148.5714bp}{77.4416bp}{148.5714bp}{74.2857bp}
    \pgfpathqcurveto{148.5714bp}{71.1298bp}{151.1298bp}{68.5714bp}{154.2857bp}{68.5714bp}
    \pgfpathqcurveto{157.4416bp}{68.5714bp}{160.0000bp}{71.1298bp}{160.0000bp}{74.2857bp}
    \pgfpathclose
    \pgfusepathqfillstroke
  \end{pgfscope}
  \begin{pgfscope}
    \pgfsetlinewidth{0.8113bp}
    \definecolor{sc}{rgb}{0.0000,0.0000,0.0000}
    \pgfsetstrokecolor{sc}
    \pgfsetmiterjoin
    \pgfsetbuttcap
    \pgfpathqmoveto{154.2857bp}{68.5714bp}
    \pgfpathqlineto{154.2857bp}{57.1429bp}
    \pgfusepathqstroke
  \end{pgfscope}
  \begin{pgfscope}
    \definecolor{fc}{rgb}{0.0000,0.0000,0.0000}
    \pgfsetfillcolor{fc}
    \pgfusepathqfill
  \end{pgfscope}
  \begin{pgfscope}
    \definecolor{fc}{rgb}{0.0000,0.0000,0.0000}
    \pgfsetfillcolor{fc}
    \pgfusepathqfill
  \end{pgfscope}
  \begin{pgfscope}
    \definecolor{fc}{rgb}{0.0000,0.0000,0.0000}
    \pgfsetfillcolor{fc}
    \pgfusepathqfill
  \end{pgfscope}
  \begin{pgfscope}
    \definecolor{fc}{rgb}{0.0000,0.0000,0.0000}
    \pgfsetfillcolor{fc}
    \pgfusepathqfill
  \end{pgfscope}
  \begin{pgfscope}
    \definecolor{fc}{rgb}{0.0000,0.0000,0.0000}
    \pgfsetfillcolor{fc}
    \pgfsetfillopacity{0.0000}
    \pgfsetlinewidth{0.8113bp}
    \definecolor{sc}{rgb}{0.0000,0.0000,0.0000}
    \pgfsetstrokecolor{sc}
    \pgfsetmiterjoin
    \pgfsetbuttcap
    \pgfpathqmoveto{137.1429bp}{74.2857bp}
    \pgfpathqcurveto{137.1429bp}{77.4416bp}{134.5845bp}{80.0000bp}{131.4286bp}{80.0000bp}
    \pgfpathqcurveto{128.2727bp}{80.0000bp}{125.7143bp}{77.4416bp}{125.7143bp}{74.2857bp}
    \pgfpathqcurveto{125.7143bp}{71.1298bp}{128.2727bp}{68.5714bp}{131.4286bp}{68.5714bp}
    \pgfpathqcurveto{134.5845bp}{68.5714bp}{137.1429bp}{71.1298bp}{137.1429bp}{74.2857bp}
    \pgfpathclose
    \pgfusepathqfillstroke
  \end{pgfscope}
  \begin{pgfscope}
    \definecolor{fc}{rgb}{0.0000,0.0000,0.0000}
    \pgfsetfillcolor{fc}
    \pgfsetfillopacity{0.0000}
    \pgfsetlinewidth{0.8113bp}
    \definecolor{sc}{rgb}{0.0000,0.0000,0.0000}
    \pgfsetstrokecolor{sc}
    \pgfsetmiterjoin
    \pgfsetbuttcap
    \pgfpathqmoveto{137.1429bp}{97.1429bp}
    \pgfpathqcurveto{137.1429bp}{100.2988bp}{134.5845bp}{102.8571bp}{131.4286bp}{102.8571bp}
    \pgfpathqcurveto{128.2727bp}{102.8571bp}{125.7143bp}{100.2988bp}{125.7143bp}{97.1429bp}
    \pgfpathqcurveto{125.7143bp}{93.9869bp}{128.2727bp}{91.4286bp}{131.4286bp}{91.4286bp}
    \pgfpathqcurveto{134.5845bp}{91.4286bp}{137.1429bp}{93.9869bp}{137.1429bp}{97.1429bp}
    \pgfpathclose
    \pgfusepathqfillstroke
  \end{pgfscope}
  \begin{pgfscope}
    \pgfsetlinewidth{0.8113bp}
    \definecolor{sc}{rgb}{0.0000,0.0000,0.0000}
    \pgfsetstrokecolor{sc}
    \pgfsetmiterjoin
    \pgfsetbuttcap
    \pgfpathqmoveto{136.5407bp}{94.5868bp}
    \pgfpathqlineto{172.0307bp}{76.8418bp}
    \pgfusepathqstroke
  \end{pgfscope}
  \begin{pgfscope}
    \definecolor{fc}{rgb}{0.0000,0.0000,0.0000}
    \pgfsetfillcolor{fc}
    \pgfusepathqfill
  \end{pgfscope}
  \begin{pgfscope}
    \definecolor{fc}{rgb}{0.0000,0.0000,0.0000}
    \pgfsetfillcolor{fc}
    \pgfusepathqfill
  \end{pgfscope}
  \begin{pgfscope}
    \definecolor{fc}{rgb}{0.0000,0.0000,0.0000}
    \pgfsetfillcolor{fc}
    \pgfusepathqfill
  \end{pgfscope}
  \begin{pgfscope}
    \definecolor{fc}{rgb}{0.0000,0.0000,0.0000}
    \pgfsetfillcolor{fc}
    \pgfusepathqfill
  \end{pgfscope}
  \begin{pgfscope}
    \pgfsetlinewidth{0.8113bp}
    \definecolor{sc}{rgb}{0.0000,0.0000,0.0000}
    \pgfsetstrokecolor{sc}
    \pgfsetmiterjoin
    \pgfsetbuttcap
    \pgfpathqmoveto{135.4692bp}{93.1022bp}
    \pgfpathqlineto{150.2451bp}{78.3263bp}
    \pgfusepathqstroke
  \end{pgfscope}
  \begin{pgfscope}
    \definecolor{fc}{rgb}{0.0000,0.0000,0.0000}
    \pgfsetfillcolor{fc}
    \pgfusepathqfill
  \end{pgfscope}
  \begin{pgfscope}
    \definecolor{fc}{rgb}{0.0000,0.0000,0.0000}
    \pgfsetfillcolor{fc}
    \pgfusepathqfill
  \end{pgfscope}
  \begin{pgfscope}
    \definecolor{fc}{rgb}{0.0000,0.0000,0.0000}
    \pgfsetfillcolor{fc}
    \pgfusepathqfill
  \end{pgfscope}
  \begin{pgfscope}
    \definecolor{fc}{rgb}{0.0000,0.0000,0.0000}
    \pgfsetfillcolor{fc}
    \pgfusepathqfill
  \end{pgfscope}
  \begin{pgfscope}
    \pgfsetlinewidth{0.8113bp}
    \definecolor{sc}{rgb}{0.0000,0.0000,0.0000}
    \pgfsetstrokecolor{sc}
    \pgfsetmiterjoin
    \pgfsetbuttcap
    \pgfpathqmoveto{131.4286bp}{91.4286bp}
    \pgfpathqlineto{131.4286bp}{80.0000bp}
    \pgfusepathqstroke
  \end{pgfscope}
  \begin{pgfscope}
    \definecolor{fc}{rgb}{0.0000,0.0000,0.0000}
    \pgfsetfillcolor{fc}
    \pgfusepathqfill
  \end{pgfscope}
  \begin{pgfscope}
    \definecolor{fc}{rgb}{0.0000,0.0000,0.0000}
    \pgfsetfillcolor{fc}
    \pgfusepathqfill
  \end{pgfscope}
  \begin{pgfscope}
    \definecolor{fc}{rgb}{0.0000,0.0000,0.0000}
    \pgfsetfillcolor{fc}
    \pgfusepathqfill
  \end{pgfscope}
  \begin{pgfscope}
    \definecolor{fc}{rgb}{0.0000,0.0000,0.0000}
    \pgfsetfillcolor{fc}
    \pgfusepathqfill
  \end{pgfscope}
  \begin{pgfscope}
    \definecolor{fc}{rgb}{0.0000,0.0000,0.0000}
    \pgfsetfillcolor{fc}
    \pgfsetfillopacity{0.0000}
    \pgfsetlinewidth{0.8113bp}
    \definecolor{sc}{rgb}{0.0000,0.0000,0.0000}
    \pgfsetstrokecolor{sc}
    \pgfsetmiterjoin
    \pgfsetbuttcap
    \pgfpathqmoveto{102.8571bp}{51.4286bp}
    \pgfpathqcurveto{102.8571bp}{54.5845bp}{100.2988bp}{57.1429bp}{97.1429bp}{57.1429bp}
    \pgfpathqcurveto{93.9869bp}{57.1429bp}{91.4286bp}{54.5845bp}{91.4286bp}{51.4286bp}
    \pgfpathqcurveto{91.4286bp}{48.2727bp}{93.9869bp}{45.7143bp}{97.1429bp}{45.7143bp}
    \pgfpathqcurveto{100.2988bp}{45.7143bp}{102.8571bp}{48.2727bp}{102.8571bp}{51.4286bp}
    \pgfpathclose
    \pgfusepathqfillstroke
  \end{pgfscope}
  \begin{pgfscope}
    \definecolor{fc}{rgb}{0.0000,0.0000,0.0000}
    \pgfsetfillcolor{fc}
    \pgfsetfillopacity{0.0000}
    \pgfsetlinewidth{0.8113bp}
    \definecolor{sc}{rgb}{0.0000,0.0000,0.0000}
    \pgfsetstrokecolor{sc}
    \pgfsetmiterjoin
    \pgfsetbuttcap
    \pgfpathqmoveto{102.8571bp}{74.2857bp}
    \pgfpathqcurveto{102.8571bp}{77.4416bp}{100.2988bp}{80.0000bp}{97.1429bp}{80.0000bp}
    \pgfpathqcurveto{93.9869bp}{80.0000bp}{91.4286bp}{77.4416bp}{91.4286bp}{74.2857bp}
    \pgfpathqcurveto{91.4286bp}{71.1298bp}{93.9869bp}{68.5714bp}{97.1429bp}{68.5714bp}
    \pgfpathqcurveto{100.2988bp}{68.5714bp}{102.8571bp}{71.1298bp}{102.8571bp}{74.2857bp}
    \pgfpathclose
    \pgfusepathqfillstroke
  \end{pgfscope}
  \begin{pgfscope}
    \pgfsetlinewidth{0.8113bp}
    \definecolor{sc}{rgb}{0.0000,0.0000,0.0000}
    \pgfsetstrokecolor{sc}
    \pgfsetmiterjoin
    \pgfsetbuttcap
    \pgfpathqmoveto{97.1429bp}{68.5714bp}
    \pgfpathqlineto{97.1429bp}{57.1429bp}
    \pgfusepathqstroke
  \end{pgfscope}
  \begin{pgfscope}
    \definecolor{fc}{rgb}{0.0000,0.0000,0.0000}
    \pgfsetfillcolor{fc}
    \pgfusepathqfill
  \end{pgfscope}
  \begin{pgfscope}
    \definecolor{fc}{rgb}{0.0000,0.0000,0.0000}
    \pgfsetfillcolor{fc}
    \pgfusepathqfill
  \end{pgfscope}
  \begin{pgfscope}
    \definecolor{fc}{rgb}{0.0000,0.0000,0.0000}
    \pgfsetfillcolor{fc}
    \pgfusepathqfill
  \end{pgfscope}
  \begin{pgfscope}
    \definecolor{fc}{rgb}{0.0000,0.0000,0.0000}
    \pgfsetfillcolor{fc}
    \pgfusepathqfill
  \end{pgfscope}
  \begin{pgfscope}
    \definecolor{fc}{rgb}{0.0000,0.0000,0.0000}
    \pgfsetfillcolor{fc}
    \pgfsetfillopacity{0.0000}
    \pgfsetlinewidth{0.8113bp}
    \definecolor{sc}{rgb}{0.0000,0.0000,0.0000}
    \pgfsetstrokecolor{sc}
    \pgfsetmiterjoin
    \pgfsetbuttcap
    \pgfpathqmoveto{80.0000bp}{74.2857bp}
    \pgfpathqcurveto{80.0000bp}{77.4416bp}{77.4416bp}{80.0000bp}{74.2857bp}{80.0000bp}
    \pgfpathqcurveto{71.1298bp}{80.0000bp}{68.5714bp}{77.4416bp}{68.5714bp}{74.2857bp}
    \pgfpathqcurveto{68.5714bp}{71.1298bp}{71.1298bp}{68.5714bp}{74.2857bp}{68.5714bp}
    \pgfpathqcurveto{77.4416bp}{68.5714bp}{80.0000bp}{71.1298bp}{80.0000bp}{74.2857bp}
    \pgfpathclose
    \pgfusepathqfillstroke
  \end{pgfscope}
  \begin{pgfscope}
    \definecolor{fc}{rgb}{0.0000,0.0000,0.0000}
    \pgfsetfillcolor{fc}
    \pgfsetfillopacity{0.0000}
    \pgfsetlinewidth{0.8113bp}
    \definecolor{sc}{rgb}{0.0000,0.0000,0.0000}
    \pgfsetstrokecolor{sc}
    \pgfsetmiterjoin
    \pgfsetbuttcap
    \pgfpathqmoveto{80.0000bp}{97.1429bp}
    \pgfpathqcurveto{80.0000bp}{100.2988bp}{77.4416bp}{102.8571bp}{74.2857bp}{102.8571bp}
    \pgfpathqcurveto{71.1298bp}{102.8571bp}{68.5714bp}{100.2988bp}{68.5714bp}{97.1429bp}
    \pgfpathqcurveto{68.5714bp}{93.9869bp}{71.1298bp}{91.4286bp}{74.2857bp}{91.4286bp}
    \pgfpathqcurveto{77.4416bp}{91.4286bp}{80.0000bp}{93.9869bp}{80.0000bp}{97.1429bp}
    \pgfpathclose
    \pgfusepathqfillstroke
  \end{pgfscope}
  \begin{pgfscope}
    \pgfsetlinewidth{0.8113bp}
    \definecolor{sc}{rgb}{0.0000,0.0000,0.0000}
    \pgfsetstrokecolor{sc}
    \pgfsetmiterjoin
    \pgfsetbuttcap
    \pgfpathqmoveto{78.3263bp}{93.1022bp}
    \pgfpathqlineto{93.1022bp}{78.3263bp}
    \pgfusepathqstroke
  \end{pgfscope}
  \begin{pgfscope}
    \definecolor{fc}{rgb}{0.0000,0.0000,0.0000}
    \pgfsetfillcolor{fc}
    \pgfusepathqfill
  \end{pgfscope}
  \begin{pgfscope}
    \definecolor{fc}{rgb}{0.0000,0.0000,0.0000}
    \pgfsetfillcolor{fc}
    \pgfusepathqfill
  \end{pgfscope}
  \begin{pgfscope}
    \definecolor{fc}{rgb}{0.0000,0.0000,0.0000}
    \pgfsetfillcolor{fc}
    \pgfusepathqfill
  \end{pgfscope}
  \begin{pgfscope}
    \definecolor{fc}{rgb}{0.0000,0.0000,0.0000}
    \pgfsetfillcolor{fc}
    \pgfusepathqfill
  \end{pgfscope}
  \begin{pgfscope}
    \pgfsetlinewidth{0.8113bp}
    \definecolor{sc}{rgb}{0.0000,0.0000,0.0000}
    \pgfsetstrokecolor{sc}
    \pgfsetmiterjoin
    \pgfsetbuttcap
    \pgfpathqmoveto{74.2857bp}{91.4286bp}
    \pgfpathqlineto{74.2857bp}{80.0000bp}
    \pgfusepathqstroke
  \end{pgfscope}
  \begin{pgfscope}
    \definecolor{fc}{rgb}{0.0000,0.0000,0.0000}
    \pgfsetfillcolor{fc}
    \pgfusepathqfill
  \end{pgfscope}
  \begin{pgfscope}
    \definecolor{fc}{rgb}{0.0000,0.0000,0.0000}
    \pgfsetfillcolor{fc}
    \pgfusepathqfill
  \end{pgfscope}
  \begin{pgfscope}
    \definecolor{fc}{rgb}{0.0000,0.0000,0.0000}
    \pgfsetfillcolor{fc}
    \pgfusepathqfill
  \end{pgfscope}
  \begin{pgfscope}
    \definecolor{fc}{rgb}{0.0000,0.0000,0.0000}
    \pgfsetfillcolor{fc}
    \pgfusepathqfill
  \end{pgfscope}
  \begin{pgfscope}
    \definecolor{fc}{rgb}{0.0000,0.0000,0.0000}
    \pgfsetfillcolor{fc}
    \pgfsetfillopacity{0.0000}
    \pgfsetlinewidth{0.8113bp}
    \definecolor{sc}{rgb}{0.0000,0.0000,0.0000}
    \pgfsetstrokecolor{sc}
    \pgfsetmiterjoin
    \pgfsetbuttcap
    \pgfpathqmoveto{45.7143bp}{74.2857bp}
    \pgfpathqcurveto{45.7143bp}{77.4416bp}{43.1559bp}{80.0000bp}{40.0000bp}{80.0000bp}
    \pgfpathqcurveto{36.8441bp}{80.0000bp}{34.2857bp}{77.4416bp}{34.2857bp}{74.2857bp}
    \pgfpathqcurveto{34.2857bp}{71.1298bp}{36.8441bp}{68.5714bp}{40.0000bp}{68.5714bp}
    \pgfpathqcurveto{43.1559bp}{68.5714bp}{45.7143bp}{71.1298bp}{45.7143bp}{74.2857bp}
    \pgfpathclose
    \pgfusepathqfillstroke
  \end{pgfscope}
  \begin{pgfscope}
    \definecolor{fc}{rgb}{0.0000,0.0000,0.0000}
    \pgfsetfillcolor{fc}
    \pgfsetfillopacity{0.0000}
    \pgfsetlinewidth{0.8113bp}
    \definecolor{sc}{rgb}{0.0000,0.0000,0.0000}
    \pgfsetstrokecolor{sc}
    \pgfsetmiterjoin
    \pgfsetbuttcap
    \pgfpathqmoveto{45.7143bp}{97.1429bp}
    \pgfpathqcurveto{45.7143bp}{100.2988bp}{43.1559bp}{102.8571bp}{40.0000bp}{102.8571bp}
    \pgfpathqcurveto{36.8441bp}{102.8571bp}{34.2857bp}{100.2988bp}{34.2857bp}{97.1429bp}
    \pgfpathqcurveto{34.2857bp}{93.9869bp}{36.8441bp}{91.4286bp}{40.0000bp}{91.4286bp}
    \pgfpathqcurveto{43.1559bp}{91.4286bp}{45.7143bp}{93.9869bp}{45.7143bp}{97.1429bp}
    \pgfpathclose
    \pgfusepathqfillstroke
  \end{pgfscope}
  \begin{pgfscope}
    \pgfsetlinewidth{0.8113bp}
    \definecolor{sc}{rgb}{0.0000,0.0000,0.0000}
    \pgfsetstrokecolor{sc}
    \pgfsetmiterjoin
    \pgfsetbuttcap
    \pgfpathqmoveto{40.0000bp}{91.4286bp}
    \pgfpathqlineto{40.0000bp}{80.0000bp}
    \pgfusepathqstroke
  \end{pgfscope}
  \begin{pgfscope}
    \definecolor{fc}{rgb}{0.0000,0.0000,0.0000}
    \pgfsetfillcolor{fc}
    \pgfusepathqfill
  \end{pgfscope}
  \begin{pgfscope}
    \definecolor{fc}{rgb}{0.0000,0.0000,0.0000}
    \pgfsetfillcolor{fc}
    \pgfusepathqfill
  \end{pgfscope}
  \begin{pgfscope}
    \definecolor{fc}{rgb}{0.0000,0.0000,0.0000}
    \pgfsetfillcolor{fc}
    \pgfusepathqfill
  \end{pgfscope}
  \begin{pgfscope}
    \definecolor{fc}{rgb}{0.0000,0.0000,0.0000}
    \pgfsetfillcolor{fc}
    \pgfusepathqfill
  \end{pgfscope}
  \begin{pgfscope}
    \definecolor{fc}{rgb}{0.0000,0.0000,0.0000}
    \pgfsetfillcolor{fc}
    \pgfsetfillopacity{0.0000}
    \pgfsetlinewidth{0.8113bp}
    \definecolor{sc}{rgb}{0.0000,0.0000,0.0000}
    \pgfsetstrokecolor{sc}
    \pgfsetmiterjoin
    \pgfsetbuttcap
    \pgfpathqmoveto{11.4286bp}{97.1429bp}
    \pgfpathqcurveto{11.4286bp}{100.2988bp}{8.8702bp}{102.8571bp}{5.7143bp}{102.8571bp}
    \pgfpathqcurveto{2.5584bp}{102.8571bp}{0.0000bp}{100.2988bp}{0.0000bp}{97.1429bp}
    \pgfpathqcurveto{0.0000bp}{93.9869bp}{2.5584bp}{91.4286bp}{5.7143bp}{91.4286bp}
    \pgfpathqcurveto{8.8702bp}{91.4286bp}{11.4286bp}{93.9869bp}{11.4286bp}{97.1429bp}
    \pgfpathclose
    \pgfusepathqfillstroke
  \end{pgfscope}
\end{pgfpicture}

  \end{center}
  \label{binomial-trees}
\end{model*}

\begin{questions}
\item What patterns do you notice?  Write down at least three
  observations.

\item Send your reporter to some other groups to share your
  observations and collect as many additional observations as possible.

\item If it was not already among your observations from the previous
  question, explain how we can make a binomial tree of order $n$ from
  two binomial trees of order $n - 1$. \label{binomial-merge}

\item Assuming a binomial tree stores a root value and a list of
  children, how long does this operation (making two order-$(n-1)$ trees
  into an order-$n$ tree) take?

\item How many total nodes does a binomial tree of order $n$ have?
  Why?

\begin{model}{Binomial Heaps}{binomial-heap}
  \begin{defn}
    A \emph{binomial heap} is a list of binomial trees such that:
    \begin{itemize}
    \item Each binomial tree satisfies the heap property, i.e., each node's value is less than or equal to the values of all its children.
    \item There is at most one binomial tree of any given order.
    \end{itemize}
  \end{defn}
  \begin{center}
    \begin{pgfpicture}
  \pgfpathrectangle{\pgfpointorigin}{\pgfqpoint{200.0000bp}{266.0000bp}}
  \pgfusepath{use as bounding box}
  \begin{pgfscope}
    \definecolor{fc}{rgb}{0.0000,0.0000,0.0000}
    \pgfsetfillcolor{fc}
    \pgfsetfillopacity{0.0000}
    \pgfsetlinewidth{0.9238bp}
    \definecolor{sc}{rgb}{0.0000,0.0000,0.0000}
    \pgfsetstrokecolor{sc}
    \pgfsetmiterjoin
    \pgfsetbuttcap
    \pgfpathqmoveto{200.0000bp}{72.2222bp}
    \pgfpathqcurveto{200.0000bp}{75.2905bp}{197.5127bp}{77.7778bp}{194.4444bp}{77.7778bp}
    \pgfpathqcurveto{191.3762bp}{77.7778bp}{188.8889bp}{75.2905bp}{188.8889bp}{72.2222bp}
    \pgfpathqcurveto{188.8889bp}{69.1540bp}{191.3762bp}{66.6667bp}{194.4444bp}{66.6667bp}
    \pgfpathqcurveto{197.5127bp}{66.6667bp}{200.0000bp}{69.1540bp}{200.0000bp}{72.2222bp}
    \pgfpathclose
    \pgfusepathqfillstroke
  \end{pgfscope}
  \begin{pgfscope}
    \definecolor{fc}{rgb}{0.0000,0.0000,0.0000}
    \pgfsetfillcolor{fc}
    \pgftransformcm{1.0000}{0.0000}{0.0000}{1.0000}{\pgfqpoint{194.4444bp}{72.2222bp}}
    \pgftransformscale{0.4861}
    \pgftext[]{$8$}
  \end{pgfscope}
  \begin{pgfscope}
    \definecolor{fc}{rgb}{0.0000,0.0000,0.0000}
    \pgfsetfillcolor{fc}
    \pgfsetfillopacity{0.0000}
    \pgfsetlinewidth{0.9238bp}
    \definecolor{sc}{rgb}{0.0000,0.0000,0.0000}
    \pgfsetstrokecolor{sc}
    \pgfsetmiterjoin
    \pgfsetbuttcap
    \pgfpathqmoveto{166.6667bp}{27.7778bp}
    \pgfpathqcurveto{166.6667bp}{30.8460bp}{164.1794bp}{33.3333bp}{161.1111bp}{33.3333bp}
    \pgfpathqcurveto{158.0429bp}{33.3333bp}{155.5556bp}{30.8460bp}{155.5556bp}{27.7778bp}
    \pgfpathqcurveto{155.5556bp}{24.7095bp}{158.0429bp}{22.2222bp}{161.1111bp}{22.2222bp}
    \pgfpathqcurveto{164.1794bp}{22.2222bp}{166.6667bp}{24.7095bp}{166.6667bp}{27.7778bp}
    \pgfpathclose
    \pgfusepathqfillstroke
  \end{pgfscope}
  \begin{pgfscope}
    \definecolor{fc}{rgb}{0.0000,0.0000,0.0000}
    \pgfsetfillcolor{fc}
    \pgftransformcm{1.0000}{0.0000}{0.0000}{1.0000}{\pgfqpoint{161.1111bp}{27.7778bp}}
    \pgftransformscale{0.4861}
    \pgftext[]{$99$}
  \end{pgfscope}
  \begin{pgfscope}
    \definecolor{fc}{rgb}{0.0000,0.0000,0.0000}
    \pgfsetfillcolor{fc}
    \pgfsetfillopacity{0.0000}
    \pgfsetlinewidth{0.9238bp}
    \definecolor{sc}{rgb}{0.0000,0.0000,0.0000}
    \pgfsetstrokecolor{sc}
    \pgfsetmiterjoin
    \pgfsetbuttcap
    \pgfpathqmoveto{166.6667bp}{50.0000bp}
    \pgfpathqcurveto{166.6667bp}{53.0682bp}{164.1794bp}{55.5556bp}{161.1111bp}{55.5556bp}
    \pgfpathqcurveto{158.0429bp}{55.5556bp}{155.5556bp}{53.0682bp}{155.5556bp}{50.0000bp}
    \pgfpathqcurveto{155.5556bp}{46.9318bp}{158.0429bp}{44.4444bp}{161.1111bp}{44.4444bp}
    \pgfpathqcurveto{164.1794bp}{44.4444bp}{166.6667bp}{46.9318bp}{166.6667bp}{50.0000bp}
    \pgfpathclose
    \pgfusepathqfillstroke
  \end{pgfscope}
  \begin{pgfscope}
    \definecolor{fc}{rgb}{0.0000,0.0000,0.0000}
    \pgfsetfillcolor{fc}
    \pgftransformcm{1.0000}{0.0000}{0.0000}{1.0000}{\pgfqpoint{161.1111bp}{50.0000bp}}
    \pgftransformscale{0.4861}
    \pgftext[]{$17$}
  \end{pgfscope}
  \begin{pgfscope}
    \pgfsetlinewidth{0.9238bp}
    \definecolor{sc}{rgb}{0.0000,0.0000,0.0000}
    \pgfsetstrokecolor{sc}
    \pgfsetmiterjoin
    \pgfsetbuttcap
    \pgfpathqmoveto{161.1111bp}{44.4444bp}
    \pgfpathqlineto{161.1111bp}{33.3333bp}
    \pgfusepathqstroke
  \end{pgfscope}
  \begin{pgfscope}
    \definecolor{fc}{rgb}{0.0000,0.0000,0.0000}
    \pgfsetfillcolor{fc}
    \pgfusepathqfill
  \end{pgfscope}
  \begin{pgfscope}
    \definecolor{fc}{rgb}{0.0000,0.0000,0.0000}
    \pgfsetfillcolor{fc}
    \pgfusepathqfill
  \end{pgfscope}
  \begin{pgfscope}
    \definecolor{fc}{rgb}{0.0000,0.0000,0.0000}
    \pgfsetfillcolor{fc}
    \pgfusepathqfill
  \end{pgfscope}
  \begin{pgfscope}
    \definecolor{fc}{rgb}{0.0000,0.0000,0.0000}
    \pgfsetfillcolor{fc}
    \pgfusepathqfill
  \end{pgfscope}
  \begin{pgfscope}
    \definecolor{fc}{rgb}{0.0000,0.0000,0.0000}
    \pgfsetfillcolor{fc}
    \pgfsetfillopacity{0.0000}
    \pgfsetlinewidth{0.9238bp}
    \definecolor{sc}{rgb}{0.0000,0.0000,0.0000}
    \pgfsetstrokecolor{sc}
    \pgfsetmiterjoin
    \pgfsetbuttcap
    \pgfpathqmoveto{144.4444bp}{50.0000bp}
    \pgfpathqcurveto{144.4444bp}{53.0682bp}{141.9571bp}{55.5556bp}{138.8889bp}{55.5556bp}
    \pgfpathqcurveto{135.8206bp}{55.5556bp}{133.3333bp}{53.0682bp}{133.3333bp}{50.0000bp}
    \pgfpathqcurveto{133.3333bp}{46.9318bp}{135.8206bp}{44.4444bp}{138.8889bp}{44.4444bp}
    \pgfpathqcurveto{141.9571bp}{44.4444bp}{144.4444bp}{46.9318bp}{144.4444bp}{50.0000bp}
    \pgfpathclose
    \pgfusepathqfillstroke
  \end{pgfscope}
  \begin{pgfscope}
    \definecolor{fc}{rgb}{0.0000,0.0000,0.0000}
    \pgfsetfillcolor{fc}
    \pgftransformcm{1.0000}{0.0000}{0.0000}{1.0000}{\pgfqpoint{138.8889bp}{50.0000bp}}
    \pgftransformscale{0.4861}
    \pgftext[]{$21$}
  \end{pgfscope}
  \begin{pgfscope}
    \definecolor{fc}{rgb}{0.0000,0.0000,0.0000}
    \pgfsetfillcolor{fc}
    \pgfsetfillopacity{0.0000}
    \pgfsetlinewidth{0.9238bp}
    \definecolor{sc}{rgb}{0.0000,0.0000,0.0000}
    \pgfsetstrokecolor{sc}
    \pgfsetmiterjoin
    \pgfsetbuttcap
    \pgfpathqmoveto{144.4444bp}{72.2222bp}
    \pgfpathqcurveto{144.4444bp}{75.2905bp}{141.9571bp}{77.7778bp}{138.8889bp}{77.7778bp}
    \pgfpathqcurveto{135.8206bp}{77.7778bp}{133.3333bp}{75.2905bp}{133.3333bp}{72.2222bp}
    \pgfpathqcurveto{133.3333bp}{69.1540bp}{135.8206bp}{66.6667bp}{138.8889bp}{66.6667bp}
    \pgfpathqcurveto{141.9571bp}{66.6667bp}{144.4444bp}{69.1540bp}{144.4444bp}{72.2222bp}
    \pgfpathclose
    \pgfusepathqfillstroke
  \end{pgfscope}
  \begin{pgfscope}
    \definecolor{fc}{rgb}{0.0000,0.0000,0.0000}
    \pgfsetfillcolor{fc}
    \pgftransformcm{1.0000}{0.0000}{0.0000}{1.0000}{\pgfqpoint{138.8889bp}{72.2222bp}}
    \pgftransformscale{0.4861}
    \pgftext[]{$5$}
  \end{pgfscope}
  \begin{pgfscope}
    \pgfsetlinewidth{0.9238bp}
    \definecolor{sc}{rgb}{0.0000,0.0000,0.0000}
    \pgfsetstrokecolor{sc}
    \pgfsetmiterjoin
    \pgfsetbuttcap
    \pgfpathqmoveto{142.8173bp}{68.2939bp}
    \pgfpathqlineto{157.1827bp}{53.9284bp}
    \pgfusepathqstroke
  \end{pgfscope}
  \begin{pgfscope}
    \definecolor{fc}{rgb}{0.0000,0.0000,0.0000}
    \pgfsetfillcolor{fc}
    \pgfusepathqfill
  \end{pgfscope}
  \begin{pgfscope}
    \definecolor{fc}{rgb}{0.0000,0.0000,0.0000}
    \pgfsetfillcolor{fc}
    \pgfusepathqfill
  \end{pgfscope}
  \begin{pgfscope}
    \definecolor{fc}{rgb}{0.0000,0.0000,0.0000}
    \pgfsetfillcolor{fc}
    \pgfusepathqfill
  \end{pgfscope}
  \begin{pgfscope}
    \definecolor{fc}{rgb}{0.0000,0.0000,0.0000}
    \pgfsetfillcolor{fc}
    \pgfusepathqfill
  \end{pgfscope}
  \begin{pgfscope}
    \pgfsetlinewidth{0.9238bp}
    \definecolor{sc}{rgb}{0.0000,0.0000,0.0000}
    \pgfsetstrokecolor{sc}
    \pgfsetmiterjoin
    \pgfsetbuttcap
    \pgfpathqmoveto{138.8889bp}{66.6667bp}
    \pgfpathqlineto{138.8889bp}{55.5556bp}
    \pgfusepathqstroke
  \end{pgfscope}
  \begin{pgfscope}
    \definecolor{fc}{rgb}{0.0000,0.0000,0.0000}
    \pgfsetfillcolor{fc}
    \pgfusepathqfill
  \end{pgfscope}
  \begin{pgfscope}
    \definecolor{fc}{rgb}{0.0000,0.0000,0.0000}
    \pgfsetfillcolor{fc}
    \pgfusepathqfill
  \end{pgfscope}
  \begin{pgfscope}
    \definecolor{fc}{rgb}{0.0000,0.0000,0.0000}
    \pgfsetfillcolor{fc}
    \pgfusepathqfill
  \end{pgfscope}
  \begin{pgfscope}
    \definecolor{fc}{rgb}{0.0000,0.0000,0.0000}
    \pgfsetfillcolor{fc}
    \pgfusepathqfill
  \end{pgfscope}
  \begin{pgfscope}
    \definecolor{fc}{rgb}{0.0000,0.0000,0.0000}
    \pgfsetfillcolor{fc}
    \pgfsetfillopacity{0.0000}
    \pgfsetlinewidth{0.9238bp}
    \definecolor{sc}{rgb}{0.0000,0.0000,0.0000}
    \pgfsetstrokecolor{sc}
    \pgfsetmiterjoin
    \pgfsetbuttcap
    \pgfpathqmoveto{111.1111bp}{72.2222bp}
    \pgfpathqcurveto{111.1111bp}{75.2905bp}{108.6238bp}{77.7778bp}{105.5556bp}{77.7778bp}
    \pgfpathqcurveto{102.4873bp}{77.7778bp}{100.0000bp}{75.2905bp}{100.0000bp}{72.2222bp}
    \pgfpathqcurveto{100.0000bp}{69.1540bp}{102.4873bp}{66.6667bp}{105.5556bp}{66.6667bp}
    \pgfpathqcurveto{108.6238bp}{66.6667bp}{111.1111bp}{69.1540bp}{111.1111bp}{72.2222bp}
    \pgfpathclose
    \pgfusepathqfillstroke
  \end{pgfscope}
  \begin{pgfscope}
    \definecolor{fc}{rgb}{0.0000,0.0000,0.0000}
    \pgfsetfillcolor{fc}
    \pgftransformcm{1.0000}{0.0000}{0.0000}{1.0000}{\pgfqpoint{105.5556bp}{72.2222bp}}
    \pgftransformscale{0.4861}
    \pgftext[]{$4$}
  \end{pgfscope}
  \begin{pgfscope}
    \definecolor{fc}{rgb}{0.0000,0.0000,0.0000}
    \pgfsetfillcolor{fc}
    \pgfsetfillopacity{0.0000}
    \pgfsetlinewidth{0.9238bp}
    \definecolor{sc}{rgb}{0.0000,0.0000,0.0000}
    \pgfsetstrokecolor{sc}
    \pgfsetmiterjoin
    \pgfsetbuttcap
    \pgfpathqmoveto{77.7778bp}{5.5556bp}
    \pgfpathqcurveto{77.7778bp}{8.6238bp}{75.2905bp}{11.1111bp}{72.2222bp}{11.1111bp}
    \pgfpathqcurveto{69.1540bp}{11.1111bp}{66.6667bp}{8.6238bp}{66.6667bp}{5.5556bp}
    \pgfpathqcurveto{66.6667bp}{2.4873bp}{69.1540bp}{-0.0000bp}{72.2222bp}{-0.0000bp}
    \pgfpathqcurveto{75.2905bp}{-0.0000bp}{77.7778bp}{2.4873bp}{77.7778bp}{5.5556bp}
    \pgfpathclose
    \pgfusepathqfillstroke
  \end{pgfscope}
  \begin{pgfscope}
    \definecolor{fc}{rgb}{0.0000,0.0000,0.0000}
    \pgfsetfillcolor{fc}
    \pgftransformcm{1.0000}{0.0000}{0.0000}{1.0000}{\pgfqpoint{72.2222bp}{5.5556bp}}
    \pgftransformscale{0.4861}
    \pgftext[]{$53$}
  \end{pgfscope}
  \begin{pgfscope}
    \definecolor{fc}{rgb}{0.0000,0.0000,0.0000}
    \pgfsetfillcolor{fc}
    \pgfsetfillopacity{0.0000}
    \pgfsetlinewidth{0.9238bp}
    \definecolor{sc}{rgb}{0.0000,0.0000,0.0000}
    \pgfsetstrokecolor{sc}
    \pgfsetmiterjoin
    \pgfsetbuttcap
    \pgfpathqmoveto{77.7778bp}{27.7778bp}
    \pgfpathqcurveto{77.7778bp}{30.8460bp}{75.2905bp}{33.3333bp}{72.2222bp}{33.3333bp}
    \pgfpathqcurveto{69.1540bp}{33.3333bp}{66.6667bp}{30.8460bp}{66.6667bp}{27.7778bp}
    \pgfpathqcurveto{66.6667bp}{24.7095bp}{69.1540bp}{22.2222bp}{72.2222bp}{22.2222bp}
    \pgfpathqcurveto{75.2905bp}{22.2222bp}{77.7778bp}{24.7095bp}{77.7778bp}{27.7778bp}
    \pgfpathclose
    \pgfusepathqfillstroke
  \end{pgfscope}
  \begin{pgfscope}
    \definecolor{fc}{rgb}{0.0000,0.0000,0.0000}
    \pgfsetfillcolor{fc}
    \pgftransformcm{1.0000}{0.0000}{0.0000}{1.0000}{\pgfqpoint{72.2222bp}{27.7778bp}}
    \pgftransformscale{0.4861}
    \pgftext[]{$33$}
  \end{pgfscope}
  \begin{pgfscope}
    \pgfsetlinewidth{0.9238bp}
    \definecolor{sc}{rgb}{0.0000,0.0000,0.0000}
    \pgfsetstrokecolor{sc}
    \pgfsetmiterjoin
    \pgfsetbuttcap
    \pgfpathqmoveto{72.2222bp}{22.2222bp}
    \pgfpathqlineto{72.2222bp}{11.1111bp}
    \pgfusepathqstroke
  \end{pgfscope}
  \begin{pgfscope}
    \definecolor{fc}{rgb}{0.0000,0.0000,0.0000}
    \pgfsetfillcolor{fc}
    \pgfusepathqfill
  \end{pgfscope}
  \begin{pgfscope}
    \definecolor{fc}{rgb}{0.0000,0.0000,0.0000}
    \pgfsetfillcolor{fc}
    \pgfusepathqfill
  \end{pgfscope}
  \begin{pgfscope}
    \definecolor{fc}{rgb}{0.0000,0.0000,0.0000}
    \pgfsetfillcolor{fc}
    \pgfusepathqfill
  \end{pgfscope}
  \begin{pgfscope}
    \definecolor{fc}{rgb}{0.0000,0.0000,0.0000}
    \pgfsetfillcolor{fc}
    \pgfusepathqfill
  \end{pgfscope}
  \begin{pgfscope}
    \definecolor{fc}{rgb}{0.0000,0.0000,0.0000}
    \pgfsetfillcolor{fc}
    \pgfsetfillopacity{0.0000}
    \pgfsetlinewidth{0.9238bp}
    \definecolor{sc}{rgb}{0.0000,0.0000,0.0000}
    \pgfsetstrokecolor{sc}
    \pgfsetmiterjoin
    \pgfsetbuttcap
    \pgfpathqmoveto{55.5556bp}{27.7778bp}
    \pgfpathqcurveto{55.5556bp}{30.8460bp}{53.0682bp}{33.3333bp}{50.0000bp}{33.3333bp}
    \pgfpathqcurveto{46.9318bp}{33.3333bp}{44.4444bp}{30.8460bp}{44.4444bp}{27.7778bp}
    \pgfpathqcurveto{44.4444bp}{24.7095bp}{46.9318bp}{22.2222bp}{50.0000bp}{22.2222bp}
    \pgfpathqcurveto{53.0682bp}{22.2222bp}{55.5556bp}{24.7095bp}{55.5556bp}{27.7778bp}
    \pgfpathclose
    \pgfusepathqfillstroke
  \end{pgfscope}
  \begin{pgfscope}
    \definecolor{fc}{rgb}{0.0000,0.0000,0.0000}
    \pgfsetfillcolor{fc}
    \pgftransformcm{1.0000}{0.0000}{0.0000}{1.0000}{\pgfqpoint{50.0000bp}{27.7778bp}}
    \pgftransformscale{0.4861}
    \pgftext[]{$24$}
  \end{pgfscope}
  \begin{pgfscope}
    \definecolor{fc}{rgb}{0.0000,0.0000,0.0000}
    \pgfsetfillcolor{fc}
    \pgfsetfillopacity{0.0000}
    \pgfsetlinewidth{0.9238bp}
    \definecolor{sc}{rgb}{0.0000,0.0000,0.0000}
    \pgfsetstrokecolor{sc}
    \pgfsetmiterjoin
    \pgfsetbuttcap
    \pgfpathqmoveto{55.5556bp}{50.0000bp}
    \pgfpathqcurveto{55.5556bp}{53.0682bp}{53.0682bp}{55.5556bp}{50.0000bp}{55.5556bp}
    \pgfpathqcurveto{46.9318bp}{55.5556bp}{44.4444bp}{53.0682bp}{44.4444bp}{50.0000bp}
    \pgfpathqcurveto{44.4444bp}{46.9318bp}{46.9318bp}{44.4444bp}{50.0000bp}{44.4444bp}
    \pgfpathqcurveto{53.0682bp}{44.4444bp}{55.5556bp}{46.9318bp}{55.5556bp}{50.0000bp}
    \pgfpathclose
    \pgfusepathqfillstroke
  \end{pgfscope}
  \begin{pgfscope}
    \definecolor{fc}{rgb}{0.0000,0.0000,0.0000}
    \pgfsetfillcolor{fc}
    \pgftransformcm{1.0000}{0.0000}{0.0000}{1.0000}{\pgfqpoint{50.0000bp}{50.0000bp}}
    \pgftransformscale{0.4861}
    \pgftext[]{$23$}
  \end{pgfscope}
  \begin{pgfscope}
    \pgfsetlinewidth{0.9238bp}
    \definecolor{sc}{rgb}{0.0000,0.0000,0.0000}
    \pgfsetstrokecolor{sc}
    \pgfsetmiterjoin
    \pgfsetbuttcap
    \pgfpathqmoveto{53.9284bp}{46.0716bp}
    \pgfpathqlineto{68.2939bp}{31.7061bp}
    \pgfusepathqstroke
  \end{pgfscope}
  \begin{pgfscope}
    \definecolor{fc}{rgb}{0.0000,0.0000,0.0000}
    \pgfsetfillcolor{fc}
    \pgfusepathqfill
  \end{pgfscope}
  \begin{pgfscope}
    \definecolor{fc}{rgb}{0.0000,0.0000,0.0000}
    \pgfsetfillcolor{fc}
    \pgfusepathqfill
  \end{pgfscope}
  \begin{pgfscope}
    \definecolor{fc}{rgb}{0.0000,0.0000,0.0000}
    \pgfsetfillcolor{fc}
    \pgfusepathqfill
  \end{pgfscope}
  \begin{pgfscope}
    \definecolor{fc}{rgb}{0.0000,0.0000,0.0000}
    \pgfsetfillcolor{fc}
    \pgfusepathqfill
  \end{pgfscope}
  \begin{pgfscope}
    \pgfsetlinewidth{0.9238bp}
    \definecolor{sc}{rgb}{0.0000,0.0000,0.0000}
    \pgfsetstrokecolor{sc}
    \pgfsetmiterjoin
    \pgfsetbuttcap
    \pgfpathqmoveto{50.0000bp}{44.4444bp}
    \pgfpathqlineto{50.0000bp}{33.3333bp}
    \pgfusepathqstroke
  \end{pgfscope}
  \begin{pgfscope}
    \definecolor{fc}{rgb}{0.0000,0.0000,0.0000}
    \pgfsetfillcolor{fc}
    \pgfusepathqfill
  \end{pgfscope}
  \begin{pgfscope}
    \definecolor{fc}{rgb}{0.0000,0.0000,0.0000}
    \pgfsetfillcolor{fc}
    \pgfusepathqfill
  \end{pgfscope}
  \begin{pgfscope}
    \definecolor{fc}{rgb}{0.0000,0.0000,0.0000}
    \pgfsetfillcolor{fc}
    \pgfusepathqfill
  \end{pgfscope}
  \begin{pgfscope}
    \definecolor{fc}{rgb}{0.0000,0.0000,0.0000}
    \pgfsetfillcolor{fc}
    \pgfusepathqfill
  \end{pgfscope}
  \begin{pgfscope}
    \definecolor{fc}{rgb}{0.0000,0.0000,0.0000}
    \pgfsetfillcolor{fc}
    \pgfsetfillopacity{0.0000}
    \pgfsetlinewidth{0.9238bp}
    \definecolor{sc}{rgb}{0.0000,0.0000,0.0000}
    \pgfsetstrokecolor{sc}
    \pgfsetmiterjoin
    \pgfsetbuttcap
    \pgfpathqmoveto{33.3333bp}{27.7778bp}
    \pgfpathqcurveto{33.3333bp}{30.8460bp}{30.8460bp}{33.3333bp}{27.7778bp}{33.3333bp}
    \pgfpathqcurveto{24.7095bp}{33.3333bp}{22.2222bp}{30.8460bp}{22.2222bp}{27.7778bp}
    \pgfpathqcurveto{22.2222bp}{24.7095bp}{24.7095bp}{22.2222bp}{27.7778bp}{22.2222bp}
    \pgfpathqcurveto{30.8460bp}{22.2222bp}{33.3333bp}{24.7095bp}{33.3333bp}{27.7778bp}
    \pgfpathclose
    \pgfusepathqfillstroke
  \end{pgfscope}
  \begin{pgfscope}
    \definecolor{fc}{rgb}{0.0000,0.0000,0.0000}
    \pgfsetfillcolor{fc}
    \pgftransformcm{1.0000}{0.0000}{0.0000}{1.0000}{\pgfqpoint{27.7778bp}{27.7778bp}}
    \pgftransformscale{0.4861}
    \pgftext[]{$28$}
  \end{pgfscope}
  \begin{pgfscope}
    \definecolor{fc}{rgb}{0.0000,0.0000,0.0000}
    \pgfsetfillcolor{fc}
    \pgfsetfillopacity{0.0000}
    \pgfsetlinewidth{0.9238bp}
    \definecolor{sc}{rgb}{0.0000,0.0000,0.0000}
    \pgfsetstrokecolor{sc}
    \pgfsetmiterjoin
    \pgfsetbuttcap
    \pgfpathqmoveto{33.3333bp}{50.0000bp}
    \pgfpathqcurveto{33.3333bp}{53.0682bp}{30.8460bp}{55.5556bp}{27.7778bp}{55.5556bp}
    \pgfpathqcurveto{24.7095bp}{55.5556bp}{22.2222bp}{53.0682bp}{22.2222bp}{50.0000bp}
    \pgfpathqcurveto{22.2222bp}{46.9318bp}{24.7095bp}{44.4444bp}{27.7778bp}{44.4444bp}
    \pgfpathqcurveto{30.8460bp}{44.4444bp}{33.3333bp}{46.9318bp}{33.3333bp}{50.0000bp}
    \pgfpathclose
    \pgfusepathqfillstroke
  \end{pgfscope}
  \begin{pgfscope}
    \definecolor{fc}{rgb}{0.0000,0.0000,0.0000}
    \pgfsetfillcolor{fc}
    \pgftransformcm{1.0000}{0.0000}{0.0000}{1.0000}{\pgfqpoint{27.7778bp}{50.0000bp}}
    \pgftransformscale{0.4861}
    \pgftext[]{$13$}
  \end{pgfscope}
  \begin{pgfscope}
    \pgfsetlinewidth{0.9238bp}
    \definecolor{sc}{rgb}{0.0000,0.0000,0.0000}
    \pgfsetstrokecolor{sc}
    \pgfsetmiterjoin
    \pgfsetbuttcap
    \pgfpathqmoveto{27.7778bp}{44.4444bp}
    \pgfpathqlineto{27.7778bp}{33.3333bp}
    \pgfusepathqstroke
  \end{pgfscope}
  \begin{pgfscope}
    \definecolor{fc}{rgb}{0.0000,0.0000,0.0000}
    \pgfsetfillcolor{fc}
    \pgfusepathqfill
  \end{pgfscope}
  \begin{pgfscope}
    \definecolor{fc}{rgb}{0.0000,0.0000,0.0000}
    \pgfsetfillcolor{fc}
    \pgfusepathqfill
  \end{pgfscope}
  \begin{pgfscope}
    \definecolor{fc}{rgb}{0.0000,0.0000,0.0000}
    \pgfsetfillcolor{fc}
    \pgfusepathqfill
  \end{pgfscope}
  \begin{pgfscope}
    \definecolor{fc}{rgb}{0.0000,0.0000,0.0000}
    \pgfsetfillcolor{fc}
    \pgfusepathqfill
  \end{pgfscope}
  \begin{pgfscope}
    \definecolor{fc}{rgb}{0.0000,0.0000,0.0000}
    \pgfsetfillcolor{fc}
    \pgfsetfillopacity{0.0000}
    \pgfsetlinewidth{0.9238bp}
    \definecolor{sc}{rgb}{0.0000,0.0000,0.0000}
    \pgfsetstrokecolor{sc}
    \pgfsetmiterjoin
    \pgfsetbuttcap
    \pgfpathqmoveto{11.1111bp}{50.0000bp}
    \pgfpathqcurveto{11.1111bp}{53.0682bp}{8.6238bp}{55.5556bp}{5.5556bp}{55.5556bp}
    \pgfpathqcurveto{2.4873bp}{55.5556bp}{0.0000bp}{53.0682bp}{0.0000bp}{50.0000bp}
    \pgfpathqcurveto{0.0000bp}{46.9318bp}{2.4873bp}{44.4444bp}{5.5556bp}{44.4444bp}
    \pgfpathqcurveto{8.6238bp}{44.4444bp}{11.1111bp}{46.9318bp}{11.1111bp}{50.0000bp}
    \pgfpathclose
    \pgfusepathqfillstroke
  \end{pgfscope}
  \begin{pgfscope}
    \definecolor{fc}{rgb}{0.0000,0.0000,0.0000}
    \pgfsetfillcolor{fc}
    \pgftransformcm{1.0000}{0.0000}{0.0000}{1.0000}{\pgfqpoint{5.5556bp}{50.0000bp}}
    \pgftransformscale{0.4861}
    \pgftext[]{$77$}
  \end{pgfscope}
  \begin{pgfscope}
    \definecolor{fc}{rgb}{0.0000,0.0000,0.0000}
    \pgfsetfillcolor{fc}
    \pgfsetfillopacity{0.0000}
    \pgfsetlinewidth{0.9238bp}
    \definecolor{sc}{rgb}{0.0000,0.0000,0.0000}
    \pgfsetstrokecolor{sc}
    \pgfsetmiterjoin
    \pgfsetbuttcap
    \pgfpathqmoveto{11.1111bp}{72.2222bp}
    \pgfpathqcurveto{11.1111bp}{75.2905bp}{8.6238bp}{77.7778bp}{5.5556bp}{77.7778bp}
    \pgfpathqcurveto{2.4873bp}{77.7778bp}{0.0000bp}{75.2905bp}{0.0000bp}{72.2222bp}
    \pgfpathqcurveto{0.0000bp}{69.1540bp}{2.4873bp}{66.6667bp}{5.5556bp}{66.6667bp}
    \pgfpathqcurveto{8.6238bp}{66.6667bp}{11.1111bp}{69.1540bp}{11.1111bp}{72.2222bp}
    \pgfpathclose
    \pgfusepathqfillstroke
  \end{pgfscope}
  \begin{pgfscope}
    \definecolor{fc}{rgb}{0.0000,0.0000,0.0000}
    \pgfsetfillcolor{fc}
    \pgftransformcm{1.0000}{0.0000}{0.0000}{1.0000}{\pgfqpoint{5.5556bp}{72.2222bp}}
    \pgftransformscale{0.4861}
    \pgftext[]{$12$}
  \end{pgfscope}
  \begin{pgfscope}
    \pgfsetlinewidth{0.9238bp}
    \definecolor{sc}{rgb}{0.0000,0.0000,0.0000}
    \pgfsetstrokecolor{sc}
    \pgfsetmiterjoin
    \pgfsetbuttcap
    \pgfpathqmoveto{10.5257bp}{69.7372bp}
    \pgfpathqlineto{45.0299bp}{52.4851bp}
    \pgfusepathqstroke
  \end{pgfscope}
  \begin{pgfscope}
    \definecolor{fc}{rgb}{0.0000,0.0000,0.0000}
    \pgfsetfillcolor{fc}
    \pgfusepathqfill
  \end{pgfscope}
  \begin{pgfscope}
    \definecolor{fc}{rgb}{0.0000,0.0000,0.0000}
    \pgfsetfillcolor{fc}
    \pgfusepathqfill
  \end{pgfscope}
  \begin{pgfscope}
    \definecolor{fc}{rgb}{0.0000,0.0000,0.0000}
    \pgfsetfillcolor{fc}
    \pgfusepathqfill
  \end{pgfscope}
  \begin{pgfscope}
    \definecolor{fc}{rgb}{0.0000,0.0000,0.0000}
    \pgfsetfillcolor{fc}
    \pgfusepathqfill
  \end{pgfscope}
  \begin{pgfscope}
    \pgfsetlinewidth{0.9238bp}
    \definecolor{sc}{rgb}{0.0000,0.0000,0.0000}
    \pgfsetstrokecolor{sc}
    \pgfsetmiterjoin
    \pgfsetbuttcap
    \pgfpathqmoveto{9.4839bp}{68.2939bp}
    \pgfpathqlineto{23.8494bp}{53.9284bp}
    \pgfusepathqstroke
  \end{pgfscope}
  \begin{pgfscope}
    \definecolor{fc}{rgb}{0.0000,0.0000,0.0000}
    \pgfsetfillcolor{fc}
    \pgfusepathqfill
  \end{pgfscope}
  \begin{pgfscope}
    \definecolor{fc}{rgb}{0.0000,0.0000,0.0000}
    \pgfsetfillcolor{fc}
    \pgfusepathqfill
  \end{pgfscope}
  \begin{pgfscope}
    \definecolor{fc}{rgb}{0.0000,0.0000,0.0000}
    \pgfsetfillcolor{fc}
    \pgfusepathqfill
  \end{pgfscope}
  \begin{pgfscope}
    \definecolor{fc}{rgb}{0.0000,0.0000,0.0000}
    \pgfsetfillcolor{fc}
    \pgfusepathqfill
  \end{pgfscope}
  \begin{pgfscope}
    \pgfsetlinewidth{0.9238bp}
    \definecolor{sc}{rgb}{0.0000,0.0000,0.0000}
    \pgfsetstrokecolor{sc}
    \pgfsetmiterjoin
    \pgfsetbuttcap
    \pgfpathqmoveto{5.5556bp}{66.6667bp}
    \pgfpathqlineto{5.5556bp}{55.5556bp}
    \pgfusepathqstroke
  \end{pgfscope}
  \begin{pgfscope}
    \definecolor{fc}{rgb}{0.0000,0.0000,0.0000}
    \pgfsetfillcolor{fc}
    \pgfusepathqfill
  \end{pgfscope}
  \begin{pgfscope}
    \definecolor{fc}{rgb}{0.0000,0.0000,0.0000}
    \pgfsetfillcolor{fc}
    \pgfusepathqfill
  \end{pgfscope}
  \begin{pgfscope}
    \definecolor{fc}{rgb}{0.0000,0.0000,0.0000}
    \pgfsetfillcolor{fc}
    \pgfusepathqfill
  \end{pgfscope}
  \begin{pgfscope}
    \definecolor{fc}{rgb}{0.0000,0.0000,0.0000}
    \pgfsetfillcolor{fc}
    \pgfusepathqfill
  \end{pgfscope}
  \begin{pgfscope}
    \definecolor{fc}{rgb}{0.0000,0.0000,0.0000}
    \pgfsetfillcolor{fc}
    \pgfsetfillopacity{0.0000}
    \pgfsetlinewidth{0.9238bp}
    \definecolor{sc}{rgb}{0.0000,0.0000,0.0000}
    \pgfsetstrokecolor{sc}
    \pgfsetmiterjoin
    \pgfsetbuttcap
    \pgfpathqmoveto{166.6667bp}{166.6667bp}
    \pgfpathqcurveto{166.6667bp}{169.7349bp}{164.1794bp}{172.2222bp}{161.1111bp}{172.2222bp}
    \pgfpathqcurveto{158.0429bp}{172.2222bp}{155.5556bp}{169.7349bp}{155.5556bp}{166.6667bp}
    \pgfpathqcurveto{155.5556bp}{163.5984bp}{158.0429bp}{161.1111bp}{161.1111bp}{161.1111bp}
    \pgfpathqcurveto{164.1794bp}{161.1111bp}{166.6667bp}{163.5984bp}{166.6667bp}{166.6667bp}
    \pgfpathclose
    \pgfusepathqfillstroke
  \end{pgfscope}
  \begin{pgfscope}
    \definecolor{fc}{rgb}{0.0000,0.0000,0.0000}
    \pgfsetfillcolor{fc}
    \pgftransformcm{1.0000}{0.0000}{0.0000}{1.0000}{\pgfqpoint{161.1111bp}{166.6667bp}}
    \pgftransformscale{0.4861}
    \pgftext[]{$8$}
  \end{pgfscope}
  \begin{pgfscope}
    \definecolor{fc}{rgb}{0.0000,0.0000,0.0000}
    \pgfsetfillcolor{fc}
    \pgfsetfillopacity{0.0000}
    \pgfsetlinewidth{0.9238bp}
    \definecolor{sc}{rgb}{0.0000,0.0000,0.0000}
    \pgfsetstrokecolor{sc}
    \pgfsetmiterjoin
    \pgfsetbuttcap
    \pgfpathqmoveto{133.3333bp}{122.2222bp}
    \pgfpathqcurveto{133.3333bp}{125.2905bp}{130.8460bp}{127.7778bp}{127.7778bp}{127.7778bp}
    \pgfpathqcurveto{124.7095bp}{127.7778bp}{122.2222bp}{125.2905bp}{122.2222bp}{122.2222bp}
    \pgfpathqcurveto{122.2222bp}{119.1540bp}{124.7095bp}{116.6667bp}{127.7778bp}{116.6667bp}
    \pgfpathqcurveto{130.8460bp}{116.6667bp}{133.3333bp}{119.1540bp}{133.3333bp}{122.2222bp}
    \pgfpathclose
    \pgfusepathqfillstroke
  \end{pgfscope}
  \begin{pgfscope}
    \definecolor{fc}{rgb}{0.0000,0.0000,0.0000}
    \pgfsetfillcolor{fc}
    \pgftransformcm{1.0000}{0.0000}{0.0000}{1.0000}{\pgfqpoint{127.7778bp}{122.2222bp}}
    \pgftransformscale{0.4861}
    \pgftext[]{$99$}
  \end{pgfscope}
  \begin{pgfscope}
    \definecolor{fc}{rgb}{0.0000,0.0000,0.0000}
    \pgfsetfillcolor{fc}
    \pgfsetfillopacity{0.0000}
    \pgfsetlinewidth{0.9238bp}
    \definecolor{sc}{rgb}{0.0000,0.0000,0.0000}
    \pgfsetstrokecolor{sc}
    \pgfsetmiterjoin
    \pgfsetbuttcap
    \pgfpathqmoveto{133.3333bp}{144.4444bp}
    \pgfpathqcurveto{133.3333bp}{147.5127bp}{130.8460bp}{150.0000bp}{127.7778bp}{150.0000bp}
    \pgfpathqcurveto{124.7095bp}{150.0000bp}{122.2222bp}{147.5127bp}{122.2222bp}{144.4444bp}
    \pgfpathqcurveto{122.2222bp}{141.3762bp}{124.7095bp}{138.8889bp}{127.7778bp}{138.8889bp}
    \pgfpathqcurveto{130.8460bp}{138.8889bp}{133.3333bp}{141.3762bp}{133.3333bp}{144.4444bp}
    \pgfpathclose
    \pgfusepathqfillstroke
  \end{pgfscope}
  \begin{pgfscope}
    \definecolor{fc}{rgb}{0.0000,0.0000,0.0000}
    \pgfsetfillcolor{fc}
    \pgftransformcm{1.0000}{0.0000}{0.0000}{1.0000}{\pgfqpoint{127.7778bp}{144.4444bp}}
    \pgftransformscale{0.4861}
    \pgftext[]{$17$}
  \end{pgfscope}
  \begin{pgfscope}
    \pgfsetlinewidth{0.9238bp}
    \definecolor{sc}{rgb}{0.0000,0.0000,0.0000}
    \pgfsetstrokecolor{sc}
    \pgfsetmiterjoin
    \pgfsetbuttcap
    \pgfpathqmoveto{127.7778bp}{138.8889bp}
    \pgfpathqlineto{127.7778bp}{127.7778bp}
    \pgfusepathqstroke
  \end{pgfscope}
  \begin{pgfscope}
    \definecolor{fc}{rgb}{0.0000,0.0000,0.0000}
    \pgfsetfillcolor{fc}
    \pgfusepathqfill
  \end{pgfscope}
  \begin{pgfscope}
    \definecolor{fc}{rgb}{0.0000,0.0000,0.0000}
    \pgfsetfillcolor{fc}
    \pgfusepathqfill
  \end{pgfscope}
  \begin{pgfscope}
    \definecolor{fc}{rgb}{0.0000,0.0000,0.0000}
    \pgfsetfillcolor{fc}
    \pgfusepathqfill
  \end{pgfscope}
  \begin{pgfscope}
    \definecolor{fc}{rgb}{0.0000,0.0000,0.0000}
    \pgfsetfillcolor{fc}
    \pgfusepathqfill
  \end{pgfscope}
  \begin{pgfscope}
    \definecolor{fc}{rgb}{0.0000,0.0000,0.0000}
    \pgfsetfillcolor{fc}
    \pgfsetfillopacity{0.0000}
    \pgfsetlinewidth{0.9238bp}
    \definecolor{sc}{rgb}{0.0000,0.0000,0.0000}
    \pgfsetstrokecolor{sc}
    \pgfsetmiterjoin
    \pgfsetbuttcap
    \pgfpathqmoveto{111.1111bp}{144.4444bp}
    \pgfpathqcurveto{111.1111bp}{147.5127bp}{108.6238bp}{150.0000bp}{105.5556bp}{150.0000bp}
    \pgfpathqcurveto{102.4873bp}{150.0000bp}{100.0000bp}{147.5127bp}{100.0000bp}{144.4444bp}
    \pgfpathqcurveto{100.0000bp}{141.3762bp}{102.4873bp}{138.8889bp}{105.5556bp}{138.8889bp}
    \pgfpathqcurveto{108.6238bp}{138.8889bp}{111.1111bp}{141.3762bp}{111.1111bp}{144.4444bp}
    \pgfpathclose
    \pgfusepathqfillstroke
  \end{pgfscope}
  \begin{pgfscope}
    \definecolor{fc}{rgb}{0.0000,0.0000,0.0000}
    \pgfsetfillcolor{fc}
    \pgftransformcm{1.0000}{0.0000}{0.0000}{1.0000}{\pgfqpoint{105.5556bp}{144.4444bp}}
    \pgftransformscale{0.4861}
    \pgftext[]{$21$}
  \end{pgfscope}
  \begin{pgfscope}
    \definecolor{fc}{rgb}{0.0000,0.0000,0.0000}
    \pgfsetfillcolor{fc}
    \pgfsetfillopacity{0.0000}
    \pgfsetlinewidth{0.9238bp}
    \definecolor{sc}{rgb}{0.0000,0.0000,0.0000}
    \pgfsetstrokecolor{sc}
    \pgfsetmiterjoin
    \pgfsetbuttcap
    \pgfpathqmoveto{111.1111bp}{166.6667bp}
    \pgfpathqcurveto{111.1111bp}{169.7349bp}{108.6238bp}{172.2222bp}{105.5556bp}{172.2222bp}
    \pgfpathqcurveto{102.4873bp}{172.2222bp}{100.0000bp}{169.7349bp}{100.0000bp}{166.6667bp}
    \pgfpathqcurveto{100.0000bp}{163.5984bp}{102.4873bp}{161.1111bp}{105.5556bp}{161.1111bp}
    \pgfpathqcurveto{108.6238bp}{161.1111bp}{111.1111bp}{163.5984bp}{111.1111bp}{166.6667bp}
    \pgfpathclose
    \pgfusepathqfillstroke
  \end{pgfscope}
  \begin{pgfscope}
    \definecolor{fc}{rgb}{0.0000,0.0000,0.0000}
    \pgfsetfillcolor{fc}
    \pgftransformcm{1.0000}{0.0000}{0.0000}{1.0000}{\pgfqpoint{105.5556bp}{166.6667bp}}
    \pgftransformscale{0.4861}
    \pgftext[]{$5$}
  \end{pgfscope}
  \begin{pgfscope}
    \pgfsetlinewidth{0.9238bp}
    \definecolor{sc}{rgb}{0.0000,0.0000,0.0000}
    \pgfsetstrokecolor{sc}
    \pgfsetmiterjoin
    \pgfsetbuttcap
    \pgfpathqmoveto{109.4839bp}{162.7383bp}
    \pgfpathqlineto{123.8494bp}{148.3728bp}
    \pgfusepathqstroke
  \end{pgfscope}
  \begin{pgfscope}
    \definecolor{fc}{rgb}{0.0000,0.0000,0.0000}
    \pgfsetfillcolor{fc}
    \pgfusepathqfill
  \end{pgfscope}
  \begin{pgfscope}
    \definecolor{fc}{rgb}{0.0000,0.0000,0.0000}
    \pgfsetfillcolor{fc}
    \pgfusepathqfill
  \end{pgfscope}
  \begin{pgfscope}
    \definecolor{fc}{rgb}{0.0000,0.0000,0.0000}
    \pgfsetfillcolor{fc}
    \pgfusepathqfill
  \end{pgfscope}
  \begin{pgfscope}
    \definecolor{fc}{rgb}{0.0000,0.0000,0.0000}
    \pgfsetfillcolor{fc}
    \pgfusepathqfill
  \end{pgfscope}
  \begin{pgfscope}
    \pgfsetlinewidth{0.9238bp}
    \definecolor{sc}{rgb}{0.0000,0.0000,0.0000}
    \pgfsetstrokecolor{sc}
    \pgfsetmiterjoin
    \pgfsetbuttcap
    \pgfpathqmoveto{105.5556bp}{161.1111bp}
    \pgfpathqlineto{105.5556bp}{150.0000bp}
    \pgfusepathqstroke
  \end{pgfscope}
  \begin{pgfscope}
    \definecolor{fc}{rgb}{0.0000,0.0000,0.0000}
    \pgfsetfillcolor{fc}
    \pgfusepathqfill
  \end{pgfscope}
  \begin{pgfscope}
    \definecolor{fc}{rgb}{0.0000,0.0000,0.0000}
    \pgfsetfillcolor{fc}
    \pgfusepathqfill
  \end{pgfscope}
  \begin{pgfscope}
    \definecolor{fc}{rgb}{0.0000,0.0000,0.0000}
    \pgfsetfillcolor{fc}
    \pgfusepathqfill
  \end{pgfscope}
  \begin{pgfscope}
    \definecolor{fc}{rgb}{0.0000,0.0000,0.0000}
    \pgfsetfillcolor{fc}
    \pgfusepathqfill
  \end{pgfscope}
  \begin{pgfscope}
    \definecolor{fc}{rgb}{0.0000,0.0000,0.0000}
    \pgfsetfillcolor{fc}
    \pgfsetfillopacity{0.0000}
    \pgfsetlinewidth{0.9238bp}
    \definecolor{sc}{rgb}{0.0000,0.0000,0.0000}
    \pgfsetstrokecolor{sc}
    \pgfsetmiterjoin
    \pgfsetbuttcap
    \pgfpathqmoveto{77.7778bp}{100.0000bp}
    \pgfpathqcurveto{77.7778bp}{103.0682bp}{75.2905bp}{105.5556bp}{72.2222bp}{105.5556bp}
    \pgfpathqcurveto{69.1540bp}{105.5556bp}{66.6667bp}{103.0682bp}{66.6667bp}{100.0000bp}
    \pgfpathqcurveto{66.6667bp}{96.9318bp}{69.1540bp}{94.4444bp}{72.2222bp}{94.4444bp}
    \pgfpathqcurveto{75.2905bp}{94.4444bp}{77.7778bp}{96.9318bp}{77.7778bp}{100.0000bp}
    \pgfpathclose
    \pgfusepathqfillstroke
  \end{pgfscope}
  \begin{pgfscope}
    \definecolor{fc}{rgb}{0.0000,0.0000,0.0000}
    \pgfsetfillcolor{fc}
    \pgftransformcm{1.0000}{0.0000}{0.0000}{1.0000}{\pgfqpoint{72.2222bp}{100.0000bp}}
    \pgftransformscale{0.4861}
    \pgftext[]{$53$}
  \end{pgfscope}
  \begin{pgfscope}
    \definecolor{fc}{rgb}{0.0000,0.0000,0.0000}
    \pgfsetfillcolor{fc}
    \pgfsetfillopacity{0.0000}
    \pgfsetlinewidth{0.9238bp}
    \definecolor{sc}{rgb}{0.0000,0.0000,0.0000}
    \pgfsetstrokecolor{sc}
    \pgfsetmiterjoin
    \pgfsetbuttcap
    \pgfpathqmoveto{77.7778bp}{122.2222bp}
    \pgfpathqcurveto{77.7778bp}{125.2905bp}{75.2905bp}{127.7778bp}{72.2222bp}{127.7778bp}
    \pgfpathqcurveto{69.1540bp}{127.7778bp}{66.6667bp}{125.2905bp}{66.6667bp}{122.2222bp}
    \pgfpathqcurveto{66.6667bp}{119.1540bp}{69.1540bp}{116.6667bp}{72.2222bp}{116.6667bp}
    \pgfpathqcurveto{75.2905bp}{116.6667bp}{77.7778bp}{119.1540bp}{77.7778bp}{122.2222bp}
    \pgfpathclose
    \pgfusepathqfillstroke
  \end{pgfscope}
  \begin{pgfscope}
    \definecolor{fc}{rgb}{0.0000,0.0000,0.0000}
    \pgfsetfillcolor{fc}
    \pgftransformcm{1.0000}{0.0000}{0.0000}{1.0000}{\pgfqpoint{72.2222bp}{122.2222bp}}
    \pgftransformscale{0.4861}
    \pgftext[]{$33$}
  \end{pgfscope}
  \begin{pgfscope}
    \pgfsetlinewidth{0.9238bp}
    \definecolor{sc}{rgb}{0.0000,0.0000,0.0000}
    \pgfsetstrokecolor{sc}
    \pgfsetmiterjoin
    \pgfsetbuttcap
    \pgfpathqmoveto{72.2222bp}{116.6667bp}
    \pgfpathqlineto{72.2222bp}{105.5556bp}
    \pgfusepathqstroke
  \end{pgfscope}
  \begin{pgfscope}
    \definecolor{fc}{rgb}{0.0000,0.0000,0.0000}
    \pgfsetfillcolor{fc}
    \pgfusepathqfill
  \end{pgfscope}
  \begin{pgfscope}
    \definecolor{fc}{rgb}{0.0000,0.0000,0.0000}
    \pgfsetfillcolor{fc}
    \pgfusepathqfill
  \end{pgfscope}
  \begin{pgfscope}
    \definecolor{fc}{rgb}{0.0000,0.0000,0.0000}
    \pgfsetfillcolor{fc}
    \pgfusepathqfill
  \end{pgfscope}
  \begin{pgfscope}
    \definecolor{fc}{rgb}{0.0000,0.0000,0.0000}
    \pgfsetfillcolor{fc}
    \pgfusepathqfill
  \end{pgfscope}
  \begin{pgfscope}
    \definecolor{fc}{rgb}{0.0000,0.0000,0.0000}
    \pgfsetfillcolor{fc}
    \pgfsetfillopacity{0.0000}
    \pgfsetlinewidth{0.9238bp}
    \definecolor{sc}{rgb}{0.0000,0.0000,0.0000}
    \pgfsetstrokecolor{sc}
    \pgfsetmiterjoin
    \pgfsetbuttcap
    \pgfpathqmoveto{55.5556bp}{122.2222bp}
    \pgfpathqcurveto{55.5556bp}{125.2905bp}{53.0682bp}{127.7778bp}{50.0000bp}{127.7778bp}
    \pgfpathqcurveto{46.9318bp}{127.7778bp}{44.4444bp}{125.2905bp}{44.4444bp}{122.2222bp}
    \pgfpathqcurveto{44.4444bp}{119.1540bp}{46.9318bp}{116.6667bp}{50.0000bp}{116.6667bp}
    \pgfpathqcurveto{53.0682bp}{116.6667bp}{55.5556bp}{119.1540bp}{55.5556bp}{122.2222bp}
    \pgfpathclose
    \pgfusepathqfillstroke
  \end{pgfscope}
  \begin{pgfscope}
    \definecolor{fc}{rgb}{0.0000,0.0000,0.0000}
    \pgfsetfillcolor{fc}
    \pgftransformcm{1.0000}{0.0000}{0.0000}{1.0000}{\pgfqpoint{50.0000bp}{122.2222bp}}
    \pgftransformscale{0.4861}
    \pgftext[]{$24$}
  \end{pgfscope}
  \begin{pgfscope}
    \definecolor{fc}{rgb}{0.0000,0.0000,0.0000}
    \pgfsetfillcolor{fc}
    \pgfsetfillopacity{0.0000}
    \pgfsetlinewidth{0.9238bp}
    \definecolor{sc}{rgb}{0.0000,0.0000,0.0000}
    \pgfsetstrokecolor{sc}
    \pgfsetmiterjoin
    \pgfsetbuttcap
    \pgfpathqmoveto{55.5556bp}{144.4444bp}
    \pgfpathqcurveto{55.5556bp}{147.5127bp}{53.0682bp}{150.0000bp}{50.0000bp}{150.0000bp}
    \pgfpathqcurveto{46.9318bp}{150.0000bp}{44.4444bp}{147.5127bp}{44.4444bp}{144.4444bp}
    \pgfpathqcurveto{44.4444bp}{141.3762bp}{46.9318bp}{138.8889bp}{50.0000bp}{138.8889bp}
    \pgfpathqcurveto{53.0682bp}{138.8889bp}{55.5556bp}{141.3762bp}{55.5556bp}{144.4444bp}
    \pgfpathclose
    \pgfusepathqfillstroke
  \end{pgfscope}
  \begin{pgfscope}
    \definecolor{fc}{rgb}{0.0000,0.0000,0.0000}
    \pgfsetfillcolor{fc}
    \pgftransformcm{1.0000}{0.0000}{0.0000}{1.0000}{\pgfqpoint{50.0000bp}{144.4444bp}}
    \pgftransformscale{0.4861}
    \pgftext[]{$23$}
  \end{pgfscope}
  \begin{pgfscope}
    \pgfsetlinewidth{0.9238bp}
    \definecolor{sc}{rgb}{0.0000,0.0000,0.0000}
    \pgfsetstrokecolor{sc}
    \pgfsetmiterjoin
    \pgfsetbuttcap
    \pgfpathqmoveto{53.9284bp}{140.5161bp}
    \pgfpathqlineto{68.2939bp}{126.1506bp}
    \pgfusepathqstroke
  \end{pgfscope}
  \begin{pgfscope}
    \definecolor{fc}{rgb}{0.0000,0.0000,0.0000}
    \pgfsetfillcolor{fc}
    \pgfusepathqfill
  \end{pgfscope}
  \begin{pgfscope}
    \definecolor{fc}{rgb}{0.0000,0.0000,0.0000}
    \pgfsetfillcolor{fc}
    \pgfusepathqfill
  \end{pgfscope}
  \begin{pgfscope}
    \definecolor{fc}{rgb}{0.0000,0.0000,0.0000}
    \pgfsetfillcolor{fc}
    \pgfusepathqfill
  \end{pgfscope}
  \begin{pgfscope}
    \definecolor{fc}{rgb}{0.0000,0.0000,0.0000}
    \pgfsetfillcolor{fc}
    \pgfusepathqfill
  \end{pgfscope}
  \begin{pgfscope}
    \pgfsetlinewidth{0.9238bp}
    \definecolor{sc}{rgb}{0.0000,0.0000,0.0000}
    \pgfsetstrokecolor{sc}
    \pgfsetmiterjoin
    \pgfsetbuttcap
    \pgfpathqmoveto{50.0000bp}{138.8889bp}
    \pgfpathqlineto{50.0000bp}{127.7778bp}
    \pgfusepathqstroke
  \end{pgfscope}
  \begin{pgfscope}
    \definecolor{fc}{rgb}{0.0000,0.0000,0.0000}
    \pgfsetfillcolor{fc}
    \pgfusepathqfill
  \end{pgfscope}
  \begin{pgfscope}
    \definecolor{fc}{rgb}{0.0000,0.0000,0.0000}
    \pgfsetfillcolor{fc}
    \pgfusepathqfill
  \end{pgfscope}
  \begin{pgfscope}
    \definecolor{fc}{rgb}{0.0000,0.0000,0.0000}
    \pgfsetfillcolor{fc}
    \pgfusepathqfill
  \end{pgfscope}
  \begin{pgfscope}
    \definecolor{fc}{rgb}{0.0000,0.0000,0.0000}
    \pgfsetfillcolor{fc}
    \pgfusepathqfill
  \end{pgfscope}
  \begin{pgfscope}
    \definecolor{fc}{rgb}{0.0000,0.0000,0.0000}
    \pgfsetfillcolor{fc}
    \pgfsetfillopacity{0.0000}
    \pgfsetlinewidth{0.9238bp}
    \definecolor{sc}{rgb}{0.0000,0.0000,0.0000}
    \pgfsetstrokecolor{sc}
    \pgfsetmiterjoin
    \pgfsetbuttcap
    \pgfpathqmoveto{33.3333bp}{122.2222bp}
    \pgfpathqcurveto{33.3333bp}{125.2905bp}{30.8460bp}{127.7778bp}{27.7778bp}{127.7778bp}
    \pgfpathqcurveto{24.7095bp}{127.7778bp}{22.2222bp}{125.2905bp}{22.2222bp}{122.2222bp}
    \pgfpathqcurveto{22.2222bp}{119.1540bp}{24.7095bp}{116.6667bp}{27.7778bp}{116.6667bp}
    \pgfpathqcurveto{30.8460bp}{116.6667bp}{33.3333bp}{119.1540bp}{33.3333bp}{122.2222bp}
    \pgfpathclose
    \pgfusepathqfillstroke
  \end{pgfscope}
  \begin{pgfscope}
    \definecolor{fc}{rgb}{0.0000,0.0000,0.0000}
    \pgfsetfillcolor{fc}
    \pgftransformcm{1.0000}{0.0000}{0.0000}{1.0000}{\pgfqpoint{27.7778bp}{122.2222bp}}
    \pgftransformscale{0.4861}
    \pgftext[]{$28$}
  \end{pgfscope}
  \begin{pgfscope}
    \definecolor{fc}{rgb}{0.0000,0.0000,0.0000}
    \pgfsetfillcolor{fc}
    \pgfsetfillopacity{0.0000}
    \pgfsetlinewidth{0.9238bp}
    \definecolor{sc}{rgb}{0.0000,0.0000,0.0000}
    \pgfsetstrokecolor{sc}
    \pgfsetmiterjoin
    \pgfsetbuttcap
    \pgfpathqmoveto{33.3333bp}{144.4444bp}
    \pgfpathqcurveto{33.3333bp}{147.5127bp}{30.8460bp}{150.0000bp}{27.7778bp}{150.0000bp}
    \pgfpathqcurveto{24.7095bp}{150.0000bp}{22.2222bp}{147.5127bp}{22.2222bp}{144.4444bp}
    \pgfpathqcurveto{22.2222bp}{141.3762bp}{24.7095bp}{138.8889bp}{27.7778bp}{138.8889bp}
    \pgfpathqcurveto{30.8460bp}{138.8889bp}{33.3333bp}{141.3762bp}{33.3333bp}{144.4444bp}
    \pgfpathclose
    \pgfusepathqfillstroke
  \end{pgfscope}
  \begin{pgfscope}
    \definecolor{fc}{rgb}{0.0000,0.0000,0.0000}
    \pgfsetfillcolor{fc}
    \pgftransformcm{1.0000}{0.0000}{0.0000}{1.0000}{\pgfqpoint{27.7778bp}{144.4444bp}}
    \pgftransformscale{0.4861}
    \pgftext[]{$13$}
  \end{pgfscope}
  \begin{pgfscope}
    \pgfsetlinewidth{0.9238bp}
    \definecolor{sc}{rgb}{0.0000,0.0000,0.0000}
    \pgfsetstrokecolor{sc}
    \pgfsetmiterjoin
    \pgfsetbuttcap
    \pgfpathqmoveto{27.7778bp}{138.8889bp}
    \pgfpathqlineto{27.7778bp}{127.7778bp}
    \pgfusepathqstroke
  \end{pgfscope}
  \begin{pgfscope}
    \definecolor{fc}{rgb}{0.0000,0.0000,0.0000}
    \pgfsetfillcolor{fc}
    \pgfusepathqfill
  \end{pgfscope}
  \begin{pgfscope}
    \definecolor{fc}{rgb}{0.0000,0.0000,0.0000}
    \pgfsetfillcolor{fc}
    \pgfusepathqfill
  \end{pgfscope}
  \begin{pgfscope}
    \definecolor{fc}{rgb}{0.0000,0.0000,0.0000}
    \pgfsetfillcolor{fc}
    \pgfusepathqfill
  \end{pgfscope}
  \begin{pgfscope}
    \definecolor{fc}{rgb}{0.0000,0.0000,0.0000}
    \pgfsetfillcolor{fc}
    \pgfusepathqfill
  \end{pgfscope}
  \begin{pgfscope}
    \definecolor{fc}{rgb}{0.0000,0.0000,0.0000}
    \pgfsetfillcolor{fc}
    \pgfsetfillopacity{0.0000}
    \pgfsetlinewidth{0.9238bp}
    \definecolor{sc}{rgb}{0.0000,0.0000,0.0000}
    \pgfsetstrokecolor{sc}
    \pgfsetmiterjoin
    \pgfsetbuttcap
    \pgfpathqmoveto{11.1111bp}{144.4444bp}
    \pgfpathqcurveto{11.1111bp}{147.5127bp}{8.6238bp}{150.0000bp}{5.5556bp}{150.0000bp}
    \pgfpathqcurveto{2.4873bp}{150.0000bp}{0.0000bp}{147.5127bp}{0.0000bp}{144.4444bp}
    \pgfpathqcurveto{0.0000bp}{141.3762bp}{2.4873bp}{138.8889bp}{5.5556bp}{138.8889bp}
    \pgfpathqcurveto{8.6238bp}{138.8889bp}{11.1111bp}{141.3762bp}{11.1111bp}{144.4444bp}
    \pgfpathclose
    \pgfusepathqfillstroke
  \end{pgfscope}
  \begin{pgfscope}
    \definecolor{fc}{rgb}{0.0000,0.0000,0.0000}
    \pgfsetfillcolor{fc}
    \pgftransformcm{1.0000}{0.0000}{0.0000}{1.0000}{\pgfqpoint{5.5556bp}{144.4444bp}}
    \pgftransformscale{0.4861}
    \pgftext[]{$77$}
  \end{pgfscope}
  \begin{pgfscope}
    \definecolor{fc}{rgb}{0.0000,0.0000,0.0000}
    \pgfsetfillcolor{fc}
    \pgfsetfillopacity{0.0000}
    \pgfsetlinewidth{0.9238bp}
    \definecolor{sc}{rgb}{0.0000,0.0000,0.0000}
    \pgfsetstrokecolor{sc}
    \pgfsetmiterjoin
    \pgfsetbuttcap
    \pgfpathqmoveto{11.1111bp}{166.6667bp}
    \pgfpathqcurveto{11.1111bp}{169.7349bp}{8.6238bp}{172.2222bp}{5.5556bp}{172.2222bp}
    \pgfpathqcurveto{2.4873bp}{172.2222bp}{0.0000bp}{169.7349bp}{0.0000bp}{166.6667bp}
    \pgfpathqcurveto{0.0000bp}{163.5984bp}{2.4873bp}{161.1111bp}{5.5556bp}{161.1111bp}
    \pgfpathqcurveto{8.6238bp}{161.1111bp}{11.1111bp}{163.5984bp}{11.1111bp}{166.6667bp}
    \pgfpathclose
    \pgfusepathqfillstroke
  \end{pgfscope}
  \begin{pgfscope}
    \definecolor{fc}{rgb}{0.0000,0.0000,0.0000}
    \pgfsetfillcolor{fc}
    \pgftransformcm{1.0000}{0.0000}{0.0000}{1.0000}{\pgfqpoint{5.5556bp}{166.6667bp}}
    \pgftransformscale{0.4861}
    \pgftext[]{$12$}
  \end{pgfscope}
  \begin{pgfscope}
    \pgfsetlinewidth{0.9238bp}
    \definecolor{sc}{rgb}{0.0000,0.0000,0.0000}
    \pgfsetstrokecolor{sc}
    \pgfsetmiterjoin
    \pgfsetbuttcap
    \pgfpathqmoveto{10.5257bp}{164.1816bp}
    \pgfpathqlineto{45.0299bp}{146.9295bp}
    \pgfusepathqstroke
  \end{pgfscope}
  \begin{pgfscope}
    \definecolor{fc}{rgb}{0.0000,0.0000,0.0000}
    \pgfsetfillcolor{fc}
    \pgfusepathqfill
  \end{pgfscope}
  \begin{pgfscope}
    \definecolor{fc}{rgb}{0.0000,0.0000,0.0000}
    \pgfsetfillcolor{fc}
    \pgfusepathqfill
  \end{pgfscope}
  \begin{pgfscope}
    \definecolor{fc}{rgb}{0.0000,0.0000,0.0000}
    \pgfsetfillcolor{fc}
    \pgfusepathqfill
  \end{pgfscope}
  \begin{pgfscope}
    \definecolor{fc}{rgb}{0.0000,0.0000,0.0000}
    \pgfsetfillcolor{fc}
    \pgfusepathqfill
  \end{pgfscope}
  \begin{pgfscope}
    \pgfsetlinewidth{0.9238bp}
    \definecolor{sc}{rgb}{0.0000,0.0000,0.0000}
    \pgfsetstrokecolor{sc}
    \pgfsetmiterjoin
    \pgfsetbuttcap
    \pgfpathqmoveto{9.4839bp}{162.7383bp}
    \pgfpathqlineto{23.8494bp}{148.3728bp}
    \pgfusepathqstroke
  \end{pgfscope}
  \begin{pgfscope}
    \definecolor{fc}{rgb}{0.0000,0.0000,0.0000}
    \pgfsetfillcolor{fc}
    \pgfusepathqfill
  \end{pgfscope}
  \begin{pgfscope}
    \definecolor{fc}{rgb}{0.0000,0.0000,0.0000}
    \pgfsetfillcolor{fc}
    \pgfusepathqfill
  \end{pgfscope}
  \begin{pgfscope}
    \definecolor{fc}{rgb}{0.0000,0.0000,0.0000}
    \pgfsetfillcolor{fc}
    \pgfusepathqfill
  \end{pgfscope}
  \begin{pgfscope}
    \definecolor{fc}{rgb}{0.0000,0.0000,0.0000}
    \pgfsetfillcolor{fc}
    \pgfusepathqfill
  \end{pgfscope}
  \begin{pgfscope}
    \pgfsetlinewidth{0.9238bp}
    \definecolor{sc}{rgb}{0.0000,0.0000,0.0000}
    \pgfsetstrokecolor{sc}
    \pgfsetmiterjoin
    \pgfsetbuttcap
    \pgfpathqmoveto{5.5556bp}{161.1111bp}
    \pgfpathqlineto{5.5556bp}{150.0000bp}
    \pgfusepathqstroke
  \end{pgfscope}
  \begin{pgfscope}
    \definecolor{fc}{rgb}{0.0000,0.0000,0.0000}
    \pgfsetfillcolor{fc}
    \pgfusepathqfill
  \end{pgfscope}
  \begin{pgfscope}
    \definecolor{fc}{rgb}{0.0000,0.0000,0.0000}
    \pgfsetfillcolor{fc}
    \pgfusepathqfill
  \end{pgfscope}
  \begin{pgfscope}
    \definecolor{fc}{rgb}{0.0000,0.0000,0.0000}
    \pgfsetfillcolor{fc}
    \pgfusepathqfill
  \end{pgfscope}
  \begin{pgfscope}
    \definecolor{fc}{rgb}{0.0000,0.0000,0.0000}
    \pgfsetfillcolor{fc}
    \pgfusepathqfill
  \end{pgfscope}
  \begin{pgfscope}
    \definecolor{fc}{rgb}{0.0000,0.0000,0.0000}
    \pgfsetfillcolor{fc}
    \pgfsetfillopacity{0.0000}
    \pgfsetlinewidth{0.9238bp}
    \definecolor{sc}{rgb}{0.0000,0.0000,0.0000}
    \pgfsetstrokecolor{sc}
    \pgfsetmiterjoin
    \pgfsetbuttcap
    \pgfpathqmoveto{166.6667bp}{261.1111bp}
    \pgfpathqcurveto{166.6667bp}{264.1794bp}{164.1794bp}{266.6667bp}{161.1111bp}{266.6667bp}
    \pgfpathqcurveto{158.0429bp}{266.6667bp}{155.5556bp}{264.1794bp}{155.5556bp}{261.1111bp}
    \pgfpathqcurveto{155.5556bp}{258.0429bp}{158.0429bp}{255.5556bp}{161.1111bp}{255.5556bp}
    \pgfpathqcurveto{164.1794bp}{255.5556bp}{166.6667bp}{258.0429bp}{166.6667bp}{261.1111bp}
    \pgfpathclose
    \pgfusepathqfillstroke
  \end{pgfscope}
  \begin{pgfscope}
    \definecolor{fc}{rgb}{0.0000,0.0000,0.0000}
    \pgfsetfillcolor{fc}
    \pgftransformcm{1.0000}{0.0000}{0.0000}{1.0000}{\pgfqpoint{161.1111bp}{261.1111bp}}
    \pgftransformscale{0.4861}
    \pgftext[]{$8$}
  \end{pgfscope}
  \begin{pgfscope}
    \definecolor{fc}{rgb}{0.0000,0.0000,0.0000}
    \pgfsetfillcolor{fc}
    \pgfsetfillopacity{0.0000}
    \pgfsetlinewidth{0.9238bp}
    \definecolor{sc}{rgb}{0.0000,0.0000,0.0000}
    \pgfsetstrokecolor{sc}
    \pgfsetmiterjoin
    \pgfsetbuttcap
    \pgfpathqmoveto{133.3333bp}{216.6667bp}
    \pgfpathqcurveto{133.3333bp}{219.7349bp}{130.8460bp}{222.2222bp}{127.7778bp}{222.2222bp}
    \pgfpathqcurveto{124.7095bp}{222.2222bp}{122.2222bp}{219.7349bp}{122.2222bp}{216.6667bp}
    \pgfpathqcurveto{122.2222bp}{213.5984bp}{124.7095bp}{211.1111bp}{127.7778bp}{211.1111bp}
    \pgfpathqcurveto{130.8460bp}{211.1111bp}{133.3333bp}{213.5984bp}{133.3333bp}{216.6667bp}
    \pgfpathclose
    \pgfusepathqfillstroke
  \end{pgfscope}
  \begin{pgfscope}
    \definecolor{fc}{rgb}{0.0000,0.0000,0.0000}
    \pgfsetfillcolor{fc}
    \pgftransformcm{1.0000}{0.0000}{0.0000}{1.0000}{\pgfqpoint{127.7778bp}{216.6667bp}}
    \pgftransformscale{0.4861}
    \pgftext[]{$99$}
  \end{pgfscope}
  \begin{pgfscope}
    \definecolor{fc}{rgb}{0.0000,0.0000,0.0000}
    \pgfsetfillcolor{fc}
    \pgfsetfillopacity{0.0000}
    \pgfsetlinewidth{0.9238bp}
    \definecolor{sc}{rgb}{0.0000,0.0000,0.0000}
    \pgfsetstrokecolor{sc}
    \pgfsetmiterjoin
    \pgfsetbuttcap
    \pgfpathqmoveto{133.3333bp}{238.8889bp}
    \pgfpathqcurveto{133.3333bp}{241.9571bp}{130.8460bp}{244.4444bp}{127.7778bp}{244.4444bp}
    \pgfpathqcurveto{124.7095bp}{244.4444bp}{122.2222bp}{241.9571bp}{122.2222bp}{238.8889bp}
    \pgfpathqcurveto{122.2222bp}{235.8206bp}{124.7095bp}{233.3333bp}{127.7778bp}{233.3333bp}
    \pgfpathqcurveto{130.8460bp}{233.3333bp}{133.3333bp}{235.8206bp}{133.3333bp}{238.8889bp}
    \pgfpathclose
    \pgfusepathqfillstroke
  \end{pgfscope}
  \begin{pgfscope}
    \definecolor{fc}{rgb}{0.0000,0.0000,0.0000}
    \pgfsetfillcolor{fc}
    \pgftransformcm{1.0000}{0.0000}{0.0000}{1.0000}{\pgfqpoint{127.7778bp}{238.8889bp}}
    \pgftransformscale{0.4861}
    \pgftext[]{$17$}
  \end{pgfscope}
  \begin{pgfscope}
    \pgfsetlinewidth{0.9238bp}
    \definecolor{sc}{rgb}{0.0000,0.0000,0.0000}
    \pgfsetstrokecolor{sc}
    \pgfsetmiterjoin
    \pgfsetbuttcap
    \pgfpathqmoveto{127.7778bp}{233.3333bp}
    \pgfpathqlineto{127.7778bp}{222.2222bp}
    \pgfusepathqstroke
  \end{pgfscope}
  \begin{pgfscope}
    \definecolor{fc}{rgb}{0.0000,0.0000,0.0000}
    \pgfsetfillcolor{fc}
    \pgfusepathqfill
  \end{pgfscope}
  \begin{pgfscope}
    \definecolor{fc}{rgb}{0.0000,0.0000,0.0000}
    \pgfsetfillcolor{fc}
    \pgfusepathqfill
  \end{pgfscope}
  \begin{pgfscope}
    \definecolor{fc}{rgb}{0.0000,0.0000,0.0000}
    \pgfsetfillcolor{fc}
    \pgfusepathqfill
  \end{pgfscope}
  \begin{pgfscope}
    \definecolor{fc}{rgb}{0.0000,0.0000,0.0000}
    \pgfsetfillcolor{fc}
    \pgfusepathqfill
  \end{pgfscope}
  \begin{pgfscope}
    \definecolor{fc}{rgb}{0.0000,0.0000,0.0000}
    \pgfsetfillcolor{fc}
    \pgfsetfillopacity{0.0000}
    \pgfsetlinewidth{0.9238bp}
    \definecolor{sc}{rgb}{0.0000,0.0000,0.0000}
    \pgfsetstrokecolor{sc}
    \pgfsetmiterjoin
    \pgfsetbuttcap
    \pgfpathqmoveto{111.1111bp}{238.8889bp}
    \pgfpathqcurveto{111.1111bp}{241.9571bp}{108.6238bp}{244.4444bp}{105.5556bp}{244.4444bp}
    \pgfpathqcurveto{102.4873bp}{244.4444bp}{100.0000bp}{241.9571bp}{100.0000bp}{238.8889bp}
    \pgfpathqcurveto{100.0000bp}{235.8206bp}{102.4873bp}{233.3333bp}{105.5556bp}{233.3333bp}
    \pgfpathqcurveto{108.6238bp}{233.3333bp}{111.1111bp}{235.8206bp}{111.1111bp}{238.8889bp}
    \pgfpathclose
    \pgfusepathqfillstroke
  \end{pgfscope}
  \begin{pgfscope}
    \definecolor{fc}{rgb}{0.0000,0.0000,0.0000}
    \pgfsetfillcolor{fc}
    \pgftransformcm{1.0000}{0.0000}{0.0000}{1.0000}{\pgfqpoint{105.5556bp}{238.8889bp}}
    \pgftransformscale{0.4861}
    \pgftext[]{$21$}
  \end{pgfscope}
  \begin{pgfscope}
    \definecolor{fc}{rgb}{0.0000,0.0000,0.0000}
    \pgfsetfillcolor{fc}
    \pgfsetfillopacity{0.0000}
    \pgfsetlinewidth{0.9238bp}
    \definecolor{sc}{rgb}{0.0000,0.0000,0.0000}
    \pgfsetstrokecolor{sc}
    \pgfsetmiterjoin
    \pgfsetbuttcap
    \pgfpathqmoveto{111.1111bp}{261.1111bp}
    \pgfpathqcurveto{111.1111bp}{264.1794bp}{108.6238bp}{266.6667bp}{105.5556bp}{266.6667bp}
    \pgfpathqcurveto{102.4873bp}{266.6667bp}{100.0000bp}{264.1794bp}{100.0000bp}{261.1111bp}
    \pgfpathqcurveto{100.0000bp}{258.0429bp}{102.4873bp}{255.5556bp}{105.5556bp}{255.5556bp}
    \pgfpathqcurveto{108.6238bp}{255.5556bp}{111.1111bp}{258.0429bp}{111.1111bp}{261.1111bp}
    \pgfpathclose
    \pgfusepathqfillstroke
  \end{pgfscope}
  \begin{pgfscope}
    \definecolor{fc}{rgb}{0.0000,0.0000,0.0000}
    \pgfsetfillcolor{fc}
    \pgftransformcm{1.0000}{0.0000}{0.0000}{1.0000}{\pgfqpoint{105.5556bp}{261.1111bp}}
    \pgftransformscale{0.4861}
    \pgftext[]{$5$}
  \end{pgfscope}
  \begin{pgfscope}
    \pgfsetlinewidth{0.9238bp}
    \definecolor{sc}{rgb}{0.0000,0.0000,0.0000}
    \pgfsetstrokecolor{sc}
    \pgfsetmiterjoin
    \pgfsetbuttcap
    \pgfpathqmoveto{109.4839bp}{257.1827bp}
    \pgfpathqlineto{123.8494bp}{242.8173bp}
    \pgfusepathqstroke
  \end{pgfscope}
  \begin{pgfscope}
    \definecolor{fc}{rgb}{0.0000,0.0000,0.0000}
    \pgfsetfillcolor{fc}
    \pgfusepathqfill
  \end{pgfscope}
  \begin{pgfscope}
    \definecolor{fc}{rgb}{0.0000,0.0000,0.0000}
    \pgfsetfillcolor{fc}
    \pgfusepathqfill
  \end{pgfscope}
  \begin{pgfscope}
    \definecolor{fc}{rgb}{0.0000,0.0000,0.0000}
    \pgfsetfillcolor{fc}
    \pgfusepathqfill
  \end{pgfscope}
  \begin{pgfscope}
    \definecolor{fc}{rgb}{0.0000,0.0000,0.0000}
    \pgfsetfillcolor{fc}
    \pgfusepathqfill
  \end{pgfscope}
  \begin{pgfscope}
    \pgfsetlinewidth{0.9238bp}
    \definecolor{sc}{rgb}{0.0000,0.0000,0.0000}
    \pgfsetstrokecolor{sc}
    \pgfsetmiterjoin
    \pgfsetbuttcap
    \pgfpathqmoveto{105.5556bp}{255.5556bp}
    \pgfpathqlineto{105.5556bp}{244.4444bp}
    \pgfusepathqstroke
  \end{pgfscope}
  \begin{pgfscope}
    \definecolor{fc}{rgb}{0.0000,0.0000,0.0000}
    \pgfsetfillcolor{fc}
    \pgfusepathqfill
  \end{pgfscope}
  \begin{pgfscope}
    \definecolor{fc}{rgb}{0.0000,0.0000,0.0000}
    \pgfsetfillcolor{fc}
    \pgfusepathqfill
  \end{pgfscope}
  \begin{pgfscope}
    \definecolor{fc}{rgb}{0.0000,0.0000,0.0000}
    \pgfsetfillcolor{fc}
    \pgfusepathqfill
  \end{pgfscope}
  \begin{pgfscope}
    \definecolor{fc}{rgb}{0.0000,0.0000,0.0000}
    \pgfsetfillcolor{fc}
    \pgfusepathqfill
  \end{pgfscope}
  \begin{pgfscope}
    \definecolor{fc}{rgb}{0.0000,0.0000,0.0000}
    \pgfsetfillcolor{fc}
    \pgfsetfillopacity{0.0000}
    \pgfsetlinewidth{0.9238bp}
    \definecolor{sc}{rgb}{0.0000,0.0000,0.0000}
    \pgfsetstrokecolor{sc}
    \pgfsetmiterjoin
    \pgfsetbuttcap
    \pgfpathqmoveto{77.7778bp}{194.4444bp}
    \pgfpathqcurveto{77.7778bp}{197.5127bp}{75.2905bp}{200.0000bp}{72.2222bp}{200.0000bp}
    \pgfpathqcurveto{69.1540bp}{200.0000bp}{66.6667bp}{197.5127bp}{66.6667bp}{194.4444bp}
    \pgfpathqcurveto{66.6667bp}{191.3762bp}{69.1540bp}{188.8889bp}{72.2222bp}{188.8889bp}
    \pgfpathqcurveto{75.2905bp}{188.8889bp}{77.7778bp}{191.3762bp}{77.7778bp}{194.4444bp}
    \pgfpathclose
    \pgfusepathqfillstroke
  \end{pgfscope}
  \begin{pgfscope}
    \definecolor{fc}{rgb}{0.0000,0.0000,0.0000}
    \pgfsetfillcolor{fc}
    \pgftransformcm{1.0000}{0.0000}{0.0000}{1.0000}{\pgfqpoint{72.2222bp}{194.4444bp}}
    \pgftransformscale{0.4861}
    \pgftext[]{$53$}
  \end{pgfscope}
  \begin{pgfscope}
    \definecolor{fc}{rgb}{0.0000,0.0000,0.0000}
    \pgfsetfillcolor{fc}
    \pgfsetfillopacity{0.0000}
    \pgfsetlinewidth{0.9238bp}
    \definecolor{sc}{rgb}{0.0000,0.0000,0.0000}
    \pgfsetstrokecolor{sc}
    \pgfsetmiterjoin
    \pgfsetbuttcap
    \pgfpathqmoveto{77.7778bp}{216.6667bp}
    \pgfpathqcurveto{77.7778bp}{219.7349bp}{75.2905bp}{222.2222bp}{72.2222bp}{222.2222bp}
    \pgfpathqcurveto{69.1540bp}{222.2222bp}{66.6667bp}{219.7349bp}{66.6667bp}{216.6667bp}
    \pgfpathqcurveto{66.6667bp}{213.5984bp}{69.1540bp}{211.1111bp}{72.2222bp}{211.1111bp}
    \pgfpathqcurveto{75.2905bp}{211.1111bp}{77.7778bp}{213.5984bp}{77.7778bp}{216.6667bp}
    \pgfpathclose
    \pgfusepathqfillstroke
  \end{pgfscope}
  \begin{pgfscope}
    \definecolor{fc}{rgb}{0.0000,0.0000,0.0000}
    \pgfsetfillcolor{fc}
    \pgftransformcm{1.0000}{0.0000}{0.0000}{1.0000}{\pgfqpoint{72.2222bp}{216.6667bp}}
    \pgftransformscale{0.4861}
    \pgftext[]{$33$}
  \end{pgfscope}
  \begin{pgfscope}
    \pgfsetlinewidth{0.9238bp}
    \definecolor{sc}{rgb}{0.0000,0.0000,0.0000}
    \pgfsetstrokecolor{sc}
    \pgfsetmiterjoin
    \pgfsetbuttcap
    \pgfpathqmoveto{72.2222bp}{211.1111bp}
    \pgfpathqlineto{72.2222bp}{200.0000bp}
    \pgfusepathqstroke
  \end{pgfscope}
  \begin{pgfscope}
    \definecolor{fc}{rgb}{0.0000,0.0000,0.0000}
    \pgfsetfillcolor{fc}
    \pgfusepathqfill
  \end{pgfscope}
  \begin{pgfscope}
    \definecolor{fc}{rgb}{0.0000,0.0000,0.0000}
    \pgfsetfillcolor{fc}
    \pgfusepathqfill
  \end{pgfscope}
  \begin{pgfscope}
    \definecolor{fc}{rgb}{0.0000,0.0000,0.0000}
    \pgfsetfillcolor{fc}
    \pgfusepathqfill
  \end{pgfscope}
  \begin{pgfscope}
    \definecolor{fc}{rgb}{0.0000,0.0000,0.0000}
    \pgfsetfillcolor{fc}
    \pgfusepathqfill
  \end{pgfscope}
  \begin{pgfscope}
    \definecolor{fc}{rgb}{0.0000,0.0000,0.0000}
    \pgfsetfillcolor{fc}
    \pgfsetfillopacity{0.0000}
    \pgfsetlinewidth{0.9238bp}
    \definecolor{sc}{rgb}{0.0000,0.0000,0.0000}
    \pgfsetstrokecolor{sc}
    \pgfsetmiterjoin
    \pgfsetbuttcap
    \pgfpathqmoveto{55.5556bp}{216.6667bp}
    \pgfpathqcurveto{55.5556bp}{219.7349bp}{53.0682bp}{222.2222bp}{50.0000bp}{222.2222bp}
    \pgfpathqcurveto{46.9318bp}{222.2222bp}{44.4444bp}{219.7349bp}{44.4444bp}{216.6667bp}
    \pgfpathqcurveto{44.4444bp}{213.5984bp}{46.9318bp}{211.1111bp}{50.0000bp}{211.1111bp}
    \pgfpathqcurveto{53.0682bp}{211.1111bp}{55.5556bp}{213.5984bp}{55.5556bp}{216.6667bp}
    \pgfpathclose
    \pgfusepathqfillstroke
  \end{pgfscope}
  \begin{pgfscope}
    \definecolor{fc}{rgb}{0.0000,0.0000,0.0000}
    \pgfsetfillcolor{fc}
    \pgftransformcm{1.0000}{0.0000}{0.0000}{1.0000}{\pgfqpoint{50.0000bp}{216.6667bp}}
    \pgftransformscale{0.4861}
    \pgftext[]{$24$}
  \end{pgfscope}
  \begin{pgfscope}
    \definecolor{fc}{rgb}{0.0000,0.0000,0.0000}
    \pgfsetfillcolor{fc}
    \pgfsetfillopacity{0.0000}
    \pgfsetlinewidth{0.9238bp}
    \definecolor{sc}{rgb}{0.0000,0.0000,0.0000}
    \pgfsetstrokecolor{sc}
    \pgfsetmiterjoin
    \pgfsetbuttcap
    \pgfpathqmoveto{55.5556bp}{238.8889bp}
    \pgfpathqcurveto{55.5556bp}{241.9571bp}{53.0682bp}{244.4444bp}{50.0000bp}{244.4444bp}
    \pgfpathqcurveto{46.9318bp}{244.4444bp}{44.4444bp}{241.9571bp}{44.4444bp}{238.8889bp}
    \pgfpathqcurveto{44.4444bp}{235.8206bp}{46.9318bp}{233.3333bp}{50.0000bp}{233.3333bp}
    \pgfpathqcurveto{53.0682bp}{233.3333bp}{55.5556bp}{235.8206bp}{55.5556bp}{238.8889bp}
    \pgfpathclose
    \pgfusepathqfillstroke
  \end{pgfscope}
  \begin{pgfscope}
    \definecolor{fc}{rgb}{0.0000,0.0000,0.0000}
    \pgfsetfillcolor{fc}
    \pgftransformcm{1.0000}{0.0000}{0.0000}{1.0000}{\pgfqpoint{50.0000bp}{238.8889bp}}
    \pgftransformscale{0.4861}
    \pgftext[]{$25$}
  \end{pgfscope}
  \begin{pgfscope}
    \pgfsetlinewidth{0.9238bp}
    \definecolor{sc}{rgb}{0.0000,0.0000,0.0000}
    \pgfsetstrokecolor{sc}
    \pgfsetmiterjoin
    \pgfsetbuttcap
    \pgfpathqmoveto{53.9284bp}{234.9605bp}
    \pgfpathqlineto{68.2939bp}{220.5950bp}
    \pgfusepathqstroke
  \end{pgfscope}
  \begin{pgfscope}
    \definecolor{fc}{rgb}{0.0000,0.0000,0.0000}
    \pgfsetfillcolor{fc}
    \pgfusepathqfill
  \end{pgfscope}
  \begin{pgfscope}
    \definecolor{fc}{rgb}{0.0000,0.0000,0.0000}
    \pgfsetfillcolor{fc}
    \pgfusepathqfill
  \end{pgfscope}
  \begin{pgfscope}
    \definecolor{fc}{rgb}{0.0000,0.0000,0.0000}
    \pgfsetfillcolor{fc}
    \pgfusepathqfill
  \end{pgfscope}
  \begin{pgfscope}
    \definecolor{fc}{rgb}{0.0000,0.0000,0.0000}
    \pgfsetfillcolor{fc}
    \pgfusepathqfill
  \end{pgfscope}
  \begin{pgfscope}
    \pgfsetlinewidth{0.9238bp}
    \definecolor{sc}{rgb}{0.0000,0.0000,0.0000}
    \pgfsetstrokecolor{sc}
    \pgfsetmiterjoin
    \pgfsetbuttcap
    \pgfpathqmoveto{50.0000bp}{233.3333bp}
    \pgfpathqlineto{50.0000bp}{222.2222bp}
    \pgfusepathqstroke
  \end{pgfscope}
  \begin{pgfscope}
    \definecolor{fc}{rgb}{0.0000,0.0000,0.0000}
    \pgfsetfillcolor{fc}
    \pgfusepathqfill
  \end{pgfscope}
  \begin{pgfscope}
    \definecolor{fc}{rgb}{0.0000,0.0000,0.0000}
    \pgfsetfillcolor{fc}
    \pgfusepathqfill
  \end{pgfscope}
  \begin{pgfscope}
    \definecolor{fc}{rgb}{0.0000,0.0000,0.0000}
    \pgfsetfillcolor{fc}
    \pgfusepathqfill
  \end{pgfscope}
  \begin{pgfscope}
    \definecolor{fc}{rgb}{0.0000,0.0000,0.0000}
    \pgfsetfillcolor{fc}
    \pgfusepathqfill
  \end{pgfscope}
  \begin{pgfscope}
    \definecolor{fc}{rgb}{0.0000,0.0000,0.0000}
    \pgfsetfillcolor{fc}
    \pgfsetfillopacity{0.0000}
    \pgfsetlinewidth{0.9238bp}
    \definecolor{sc}{rgb}{0.0000,0.0000,0.0000}
    \pgfsetstrokecolor{sc}
    \pgfsetmiterjoin
    \pgfsetbuttcap
    \pgfpathqmoveto{33.3333bp}{216.6667bp}
    \pgfpathqcurveto{33.3333bp}{219.7349bp}{30.8460bp}{222.2222bp}{27.7778bp}{222.2222bp}
    \pgfpathqcurveto{24.7095bp}{222.2222bp}{22.2222bp}{219.7349bp}{22.2222bp}{216.6667bp}
    \pgfpathqcurveto{22.2222bp}{213.5984bp}{24.7095bp}{211.1111bp}{27.7778bp}{211.1111bp}
    \pgfpathqcurveto{30.8460bp}{211.1111bp}{33.3333bp}{213.5984bp}{33.3333bp}{216.6667bp}
    \pgfpathclose
    \pgfusepathqfillstroke
  \end{pgfscope}
  \begin{pgfscope}
    \definecolor{fc}{rgb}{0.0000,0.0000,0.0000}
    \pgfsetfillcolor{fc}
    \pgftransformcm{1.0000}{0.0000}{0.0000}{1.0000}{\pgfqpoint{27.7778bp}{216.6667bp}}
    \pgftransformscale{0.4861}
    \pgftext[]{$28$}
  \end{pgfscope}
  \begin{pgfscope}
    \definecolor{fc}{rgb}{0.0000,0.0000,0.0000}
    \pgfsetfillcolor{fc}
    \pgfsetfillopacity{0.0000}
    \pgfsetlinewidth{0.9238bp}
    \definecolor{sc}{rgb}{0.0000,0.0000,0.0000}
    \pgfsetstrokecolor{sc}
    \pgfsetmiterjoin
    \pgfsetbuttcap
    \pgfpathqmoveto{33.3333bp}{238.8889bp}
    \pgfpathqcurveto{33.3333bp}{241.9571bp}{30.8460bp}{244.4444bp}{27.7778bp}{244.4444bp}
    \pgfpathqcurveto{24.7095bp}{244.4444bp}{22.2222bp}{241.9571bp}{22.2222bp}{238.8889bp}
    \pgfpathqcurveto{22.2222bp}{235.8206bp}{24.7095bp}{233.3333bp}{27.7778bp}{233.3333bp}
    \pgfpathqcurveto{30.8460bp}{233.3333bp}{33.3333bp}{235.8206bp}{33.3333bp}{238.8889bp}
    \pgfpathclose
    \pgfusepathqfillstroke
  \end{pgfscope}
  \begin{pgfscope}
    \definecolor{fc}{rgb}{0.0000,0.0000,0.0000}
    \pgfsetfillcolor{fc}
    \pgftransformcm{1.0000}{0.0000}{0.0000}{1.0000}{\pgfqpoint{27.7778bp}{238.8889bp}}
    \pgftransformscale{0.4861}
    \pgftext[]{$13$}
  \end{pgfscope}
  \begin{pgfscope}
    \pgfsetlinewidth{0.9238bp}
    \definecolor{sc}{rgb}{0.0000,0.0000,0.0000}
    \pgfsetstrokecolor{sc}
    \pgfsetmiterjoin
    \pgfsetbuttcap
    \pgfpathqmoveto{27.7778bp}{233.3333bp}
    \pgfpathqlineto{27.7778bp}{222.2222bp}
    \pgfusepathqstroke
  \end{pgfscope}
  \begin{pgfscope}
    \definecolor{fc}{rgb}{0.0000,0.0000,0.0000}
    \pgfsetfillcolor{fc}
    \pgfusepathqfill
  \end{pgfscope}
  \begin{pgfscope}
    \definecolor{fc}{rgb}{0.0000,0.0000,0.0000}
    \pgfsetfillcolor{fc}
    \pgfusepathqfill
  \end{pgfscope}
  \begin{pgfscope}
    \definecolor{fc}{rgb}{0.0000,0.0000,0.0000}
    \pgfsetfillcolor{fc}
    \pgfusepathqfill
  \end{pgfscope}
  \begin{pgfscope}
    \definecolor{fc}{rgb}{0.0000,0.0000,0.0000}
    \pgfsetfillcolor{fc}
    \pgfusepathqfill
  \end{pgfscope}
  \begin{pgfscope}
    \definecolor{fc}{rgb}{0.0000,0.0000,0.0000}
    \pgfsetfillcolor{fc}
    \pgfsetfillopacity{0.0000}
    \pgfsetlinewidth{0.9238bp}
    \definecolor{sc}{rgb}{0.0000,0.0000,0.0000}
    \pgfsetstrokecolor{sc}
    \pgfsetmiterjoin
    \pgfsetbuttcap
    \pgfpathqmoveto{11.1111bp}{238.8889bp}
    \pgfpathqcurveto{11.1111bp}{241.9571bp}{8.6238bp}{244.4444bp}{5.5556bp}{244.4444bp}
    \pgfpathqcurveto{2.4873bp}{244.4444bp}{0.0000bp}{241.9571bp}{0.0000bp}{238.8889bp}
    \pgfpathqcurveto{0.0000bp}{235.8206bp}{2.4873bp}{233.3333bp}{5.5556bp}{233.3333bp}
    \pgfpathqcurveto{8.6238bp}{233.3333bp}{11.1111bp}{235.8206bp}{11.1111bp}{238.8889bp}
    \pgfpathclose
    \pgfusepathqfillstroke
  \end{pgfscope}
  \begin{pgfscope}
    \definecolor{fc}{rgb}{0.0000,0.0000,0.0000}
    \pgfsetfillcolor{fc}
    \pgftransformcm{1.0000}{0.0000}{0.0000}{1.0000}{\pgfqpoint{5.5556bp}{238.8889bp}}
    \pgftransformscale{0.4861}
    \pgftext[]{$77$}
  \end{pgfscope}
  \begin{pgfscope}
    \definecolor{fc}{rgb}{0.0000,0.0000,0.0000}
    \pgfsetfillcolor{fc}
    \pgfsetfillopacity{0.0000}
    \pgfsetlinewidth{0.9238bp}
    \definecolor{sc}{rgb}{0.0000,0.0000,0.0000}
    \pgfsetstrokecolor{sc}
    \pgfsetmiterjoin
    \pgfsetbuttcap
    \pgfpathqmoveto{11.1111bp}{261.1111bp}
    \pgfpathqcurveto{11.1111bp}{264.1794bp}{8.6238bp}{266.6667bp}{5.5556bp}{266.6667bp}
    \pgfpathqcurveto{2.4873bp}{266.6667bp}{0.0000bp}{264.1794bp}{0.0000bp}{261.1111bp}
    \pgfpathqcurveto{0.0000bp}{258.0429bp}{2.4873bp}{255.5556bp}{5.5556bp}{255.5556bp}
    \pgfpathqcurveto{8.6238bp}{255.5556bp}{11.1111bp}{258.0429bp}{11.1111bp}{261.1111bp}
    \pgfpathclose
    \pgfusepathqfillstroke
  \end{pgfscope}
  \begin{pgfscope}
    \definecolor{fc}{rgb}{0.0000,0.0000,0.0000}
    \pgfsetfillcolor{fc}
    \pgftransformcm{1.0000}{0.0000}{0.0000}{1.0000}{\pgfqpoint{5.5556bp}{261.1111bp}}
    \pgftransformscale{0.4861}
    \pgftext[]{$12$}
  \end{pgfscope}
  \begin{pgfscope}
    \pgfsetlinewidth{0.9238bp}
    \definecolor{sc}{rgb}{0.0000,0.0000,0.0000}
    \pgfsetstrokecolor{sc}
    \pgfsetmiterjoin
    \pgfsetbuttcap
    \pgfpathqmoveto{10.5257bp}{258.6260bp}
    \pgfpathqlineto{45.0299bp}{241.3740bp}
    \pgfusepathqstroke
  \end{pgfscope}
  \begin{pgfscope}
    \definecolor{fc}{rgb}{0.0000,0.0000,0.0000}
    \pgfsetfillcolor{fc}
    \pgfusepathqfill
  \end{pgfscope}
  \begin{pgfscope}
    \definecolor{fc}{rgb}{0.0000,0.0000,0.0000}
    \pgfsetfillcolor{fc}
    \pgfusepathqfill
  \end{pgfscope}
  \begin{pgfscope}
    \definecolor{fc}{rgb}{0.0000,0.0000,0.0000}
    \pgfsetfillcolor{fc}
    \pgfusepathqfill
  \end{pgfscope}
  \begin{pgfscope}
    \definecolor{fc}{rgb}{0.0000,0.0000,0.0000}
    \pgfsetfillcolor{fc}
    \pgfusepathqfill
  \end{pgfscope}
  \begin{pgfscope}
    \pgfsetlinewidth{0.9238bp}
    \definecolor{sc}{rgb}{0.0000,0.0000,0.0000}
    \pgfsetstrokecolor{sc}
    \pgfsetmiterjoin
    \pgfsetbuttcap
    \pgfpathqmoveto{9.4839bp}{257.1827bp}
    \pgfpathqlineto{23.8494bp}{242.8173bp}
    \pgfusepathqstroke
  \end{pgfscope}
  \begin{pgfscope}
    \definecolor{fc}{rgb}{0.0000,0.0000,0.0000}
    \pgfsetfillcolor{fc}
    \pgfusepathqfill
  \end{pgfscope}
  \begin{pgfscope}
    \definecolor{fc}{rgb}{0.0000,0.0000,0.0000}
    \pgfsetfillcolor{fc}
    \pgfusepathqfill
  \end{pgfscope}
  \begin{pgfscope}
    \definecolor{fc}{rgb}{0.0000,0.0000,0.0000}
    \pgfsetfillcolor{fc}
    \pgfusepathqfill
  \end{pgfscope}
  \begin{pgfscope}
    \definecolor{fc}{rgb}{0.0000,0.0000,0.0000}
    \pgfsetfillcolor{fc}
    \pgfusepathqfill
  \end{pgfscope}
  \begin{pgfscope}
    \pgfsetlinewidth{0.9238bp}
    \definecolor{sc}{rgb}{0.0000,0.0000,0.0000}
    \pgfsetstrokecolor{sc}
    \pgfsetmiterjoin
    \pgfsetbuttcap
    \pgfpathqmoveto{5.5556bp}{255.5556bp}
    \pgfpathqlineto{5.5556bp}{244.4444bp}
    \pgfusepathqstroke
  \end{pgfscope}
  \begin{pgfscope}
    \definecolor{fc}{rgb}{0.0000,0.0000,0.0000}
    \pgfsetfillcolor{fc}
    \pgfusepathqfill
  \end{pgfscope}
  \begin{pgfscope}
    \definecolor{fc}{rgb}{0.0000,0.0000,0.0000}
    \pgfsetfillcolor{fc}
    \pgfusepathqfill
  \end{pgfscope}
  \begin{pgfscope}
    \definecolor{fc}{rgb}{0.0000,0.0000,0.0000}
    \pgfsetfillcolor{fc}
    \pgfusepathqfill
  \end{pgfscope}
  \begin{pgfscope}
    \definecolor{fc}{rgb}{0.0000,0.0000,0.0000}
    \pgfsetfillcolor{fc}
    \pgfusepathqfill
  \end{pgfscope}
\end{pgfpicture}

  \end{center}
  \label{binomial-heap}
\end{model}

\item Two of the binomial heaps shown above are invalid, and one is
  valid.  Which is which?  Cross out the invalid ones.

\item How many total nodes does the valid binomial heap in Model
  \ref{binomial-heap} contain?

\item Draw a valid binomial heap with\dots
  \begin{subquestions}
  \item 5 nodes
  \item 8 nodes
  \item 11 nodes
  \end{subquestions}

\item What is the relationship between the number of elements in a
  binomial heap and the orders of the binomial trees that it contains?

% \item What is the order of each binomial tree in Model \ref{binomial-heap}?

% \item The table in Model \ref{heaptrees} lists a total number of nodes and the number of trees of each order. Fill in the table.

% \begin{model*}{Trees in binomial heap of size $n$}{heaptrees}
%   \centering
%   % \tabcolsep=0.1cm
%   \begin{tabular}{c|cccccccccccccccc}
%     $n$ & $0$ & $1$ & $2$ & $3$ & $4$ & $5$ & $6$ & $7$ & $8$ & $9$ & $10$
%     & $11$ & $12$ & $13$ & $14$ & $15$  \\[8pt]
%     Order 0 trees & $0$ & $1$ & $0$ & & & & & & & & & & & & & \\[8pt]
%     Order 1 trees & $0$ & $0$ & $1$ & & & & & & & & & & & & & \\[8pt]
%     Order 2 trees & $0$ & $0$ & $0$ & & & & & & & & & & & & & \\[8pt]
%     Order 3 trees & $0$ & $0$ & $0$ & & & & & & & & & & & & & \\[8pt]
%   \end{tabular}
%   \label{heaptrees}
% \end{model*}

\item If a binomial heap has a total of $n$ elements, what is the
  maximum number of binomial trees that it contains?

% \item The \verb|merge| operation takes two binomial heaps as parameters and returns a new binomial heap containing all of the values from the first two heaps. If two binomial heaps each have one tree of order 0, explain how \verb|merge| will create a new binomial heap.

% \item Next, imagine that we want to \verb|merge| two binomial heaps, one of which has an order 0 binomial tree, the other of which has an order 1 binomial tree. Explain how \verb|merge| will create a new binomial heap.

% \item Now imagine that we want to \verb|merge| two binomial heaps, each of which has an order 0 binomial tree and also an order 1 binomial tree. Explain how \verb|merge| will create a new binomial heap.

% \item Based on the insights gained from the previous three questions, write pseudocode for \verb|merge|.

% \item What is the relationship between the \verb|merge| algorithm and the binary counter from the \emph{Introduction to Amortized Analysis} POGIL activity? \label{counter}

% \item What is the best-case execution time for one call to \verb|merge|?

% \item What is the worst-case execution time for one call to \verb|merge|?

% \item What is the worst-case amortized execution time for $n$ calls to \verb|merge|? Feel free to use your answer to Question \ref{counter} to support your answer to this question. \label{amortizedN}

% \item Based on your answer to Question \ref{amortizedN}, what is the worst-case amortized execution time for one call to \verb|merge|?

% \item Priority queues have three main operations:
% \begin{itemize}
%     \item INSERT: Insert a new value into the heap.
%     \item FIND-MIN: Find the smallest value in the heap.
%     \item DELETE-MIN: Remove the smallest value from the heap.
% \end{itemize}

% Devise an algorithm for INSERT that uses \verb|merge| as a subroutine. 

% \item What is the amortized worst-case execution time for INSERT?

% \item Devise an algorithm for DELETE that uses \verb|merge| as a subroutine. 

% \item What is its worst-case execution time for DELETE?

% \item Devise a constant-time algorithm for FIND-MIN. You may want to make a small modification to the binomial heap data structure to ensure that it runs in constant time.

% \item What advantages does a binomial heap have in comparison to the standard binary heap?

% \item What advantages does a standard binary heap have in comparison to a binomial heap?

\end{questions}

\end{document}