% -*- compile-command: "pdflatex --enable-write18 binomial-heap.tex" -*-
\documentclass{tufte-handout}

\usepackage{algo-activity}

\title{Algorithms: Binomial Heaps}
\date{}

\begin{document}

\maketitle

\begin{objective}
  Students will describe binomial heaps and analyze the amortized
  running time for heap operations such as INSERT, DELETE-MIN, and MERGE.
\end{objective}

\begin{model*}{Binomial Trees}{binomial-trees}
  \begin{defn}
    A \term{binomial tree} of order $n$ (for $n \geq 0$) consists of a root
    node with $n$ subtrees. The leftmost subtree is of order $0$,
    the next subtree is of order $1$, and so forth, with the
    rightmost subtree being of order $n-1$.
  \end{defn}

  The image below depicts a series of \emph{binomial trees} in
  increasing order. The leftmost tree is order 0, the next one is
  order 1, the third one is order 2, the fourth one is order 3, and
  the final tree is order 4.

  \begin{center}
  \begin{pgfpicture}
  \pgfpathrectangle{\pgfpointorigin}{\pgfqpoint{400.0000bp}{102.0000bp}}
  \pgfusepath{use as bounding box}
  \begin{pgfscope}
    \definecolor{fc}{rgb}{0.0000,0.0000,0.0000}
    \pgfsetfillcolor{fc}
    \pgfsetfillopacity{0.0000}
    \pgfsetlinewidth{0.8113bp}
    \definecolor{sc}{rgb}{0.0000,0.0000,0.0000}
    \pgfsetstrokecolor{sc}
    \pgfsetmiterjoin
    \pgfsetbuttcap
    \pgfpathqmoveto{400.0000bp}{5.7143bp}
    \pgfpathqcurveto{400.0000bp}{8.8702bp}{397.4416bp}{11.4286bp}{394.2857bp}{11.4286bp}
    \pgfpathqcurveto{391.1298bp}{11.4286bp}{388.5714bp}{8.8702bp}{388.5714bp}{5.7143bp}
    \pgfpathqcurveto{388.5714bp}{2.5584bp}{391.1298bp}{0.0000bp}{394.2857bp}{0.0000bp}
    \pgfpathqcurveto{397.4416bp}{0.0000bp}{400.0000bp}{2.5584bp}{400.0000bp}{5.7143bp}
    \pgfpathclose
    \pgfusepathqfillstroke
  \end{pgfscope}
  \begin{pgfscope}
    \definecolor{fc}{rgb}{0.0000,0.0000,0.0000}
    \pgfsetfillcolor{fc}
    \pgfsetfillopacity{0.0000}
    \pgfsetlinewidth{0.8113bp}
    \definecolor{sc}{rgb}{0.0000,0.0000,0.0000}
    \pgfsetstrokecolor{sc}
    \pgfsetmiterjoin
    \pgfsetbuttcap
    \pgfpathqmoveto{400.0000bp}{28.5714bp}
    \pgfpathqcurveto{400.0000bp}{31.7273bp}{397.4416bp}{34.2857bp}{394.2857bp}{34.2857bp}
    \pgfpathqcurveto{391.1298bp}{34.2857bp}{388.5714bp}{31.7273bp}{388.5714bp}{28.5714bp}
    \pgfpathqcurveto{388.5714bp}{25.4155bp}{391.1298bp}{22.8571bp}{394.2857bp}{22.8571bp}
    \pgfpathqcurveto{397.4416bp}{22.8571bp}{400.0000bp}{25.4155bp}{400.0000bp}{28.5714bp}
    \pgfpathclose
    \pgfusepathqfillstroke
  \end{pgfscope}
  \begin{pgfscope}
    \pgfsetlinewidth{0.8113bp}
    \definecolor{sc}{rgb}{0.0000,0.0000,0.0000}
    \pgfsetstrokecolor{sc}
    \pgfsetmiterjoin
    \pgfsetbuttcap
    \pgfpathqmoveto{394.2857bp}{22.8571bp}
    \pgfpathqlineto{394.2857bp}{11.4286bp}
    \pgfusepathqstroke
  \end{pgfscope}
  \begin{pgfscope}
    \definecolor{fc}{rgb}{0.0000,0.0000,0.0000}
    \pgfsetfillcolor{fc}
    \pgfusepathqfill
  \end{pgfscope}
  \begin{pgfscope}
    \definecolor{fc}{rgb}{0.0000,0.0000,0.0000}
    \pgfsetfillcolor{fc}
    \pgfusepathqfill
  \end{pgfscope}
  \begin{pgfscope}
    \definecolor{fc}{rgb}{0.0000,0.0000,0.0000}
    \pgfsetfillcolor{fc}
    \pgfusepathqfill
  \end{pgfscope}
  \begin{pgfscope}
    \definecolor{fc}{rgb}{0.0000,0.0000,0.0000}
    \pgfsetfillcolor{fc}
    \pgfusepathqfill
  \end{pgfscope}
  \begin{pgfscope}
    \definecolor{fc}{rgb}{0.0000,0.0000,0.0000}
    \pgfsetfillcolor{fc}
    \pgfsetfillopacity{0.0000}
    \pgfsetlinewidth{0.8113bp}
    \definecolor{sc}{rgb}{0.0000,0.0000,0.0000}
    \pgfsetstrokecolor{sc}
    \pgfsetmiterjoin
    \pgfsetbuttcap
    \pgfpathqmoveto{377.1429bp}{28.5714bp}
    \pgfpathqcurveto{377.1429bp}{31.7273bp}{374.5845bp}{34.2857bp}{371.4286bp}{34.2857bp}
    \pgfpathqcurveto{368.2727bp}{34.2857bp}{365.7143bp}{31.7273bp}{365.7143bp}{28.5714bp}
    \pgfpathqcurveto{365.7143bp}{25.4155bp}{368.2727bp}{22.8571bp}{371.4286bp}{22.8571bp}
    \pgfpathqcurveto{374.5845bp}{22.8571bp}{377.1429bp}{25.4155bp}{377.1429bp}{28.5714bp}
    \pgfpathclose
    \pgfusepathqfillstroke
  \end{pgfscope}
  \begin{pgfscope}
    \definecolor{fc}{rgb}{0.0000,0.0000,0.0000}
    \pgfsetfillcolor{fc}
    \pgfsetfillopacity{0.0000}
    \pgfsetlinewidth{0.8113bp}
    \definecolor{sc}{rgb}{0.0000,0.0000,0.0000}
    \pgfsetstrokecolor{sc}
    \pgfsetmiterjoin
    \pgfsetbuttcap
    \pgfpathqmoveto{377.1429bp}{51.4286bp}
    \pgfpathqcurveto{377.1429bp}{54.5845bp}{374.5845bp}{57.1429bp}{371.4286bp}{57.1429bp}
    \pgfpathqcurveto{368.2727bp}{57.1429bp}{365.7143bp}{54.5845bp}{365.7143bp}{51.4286bp}
    \pgfpathqcurveto{365.7143bp}{48.2727bp}{368.2727bp}{45.7143bp}{371.4286bp}{45.7143bp}
    \pgfpathqcurveto{374.5845bp}{45.7143bp}{377.1429bp}{48.2727bp}{377.1429bp}{51.4286bp}
    \pgfpathclose
    \pgfusepathqfillstroke
  \end{pgfscope}
  \begin{pgfscope}
    \pgfsetlinewidth{0.8113bp}
    \definecolor{sc}{rgb}{0.0000,0.0000,0.0000}
    \pgfsetstrokecolor{sc}
    \pgfsetmiterjoin
    \pgfsetbuttcap
    \pgfpathqmoveto{375.4692bp}{47.3880bp}
    \pgfpathqlineto{390.2451bp}{32.6120bp}
    \pgfusepathqstroke
  \end{pgfscope}
  \begin{pgfscope}
    \definecolor{fc}{rgb}{0.0000,0.0000,0.0000}
    \pgfsetfillcolor{fc}
    \pgfusepathqfill
  \end{pgfscope}
  \begin{pgfscope}
    \definecolor{fc}{rgb}{0.0000,0.0000,0.0000}
    \pgfsetfillcolor{fc}
    \pgfusepathqfill
  \end{pgfscope}
  \begin{pgfscope}
    \definecolor{fc}{rgb}{0.0000,0.0000,0.0000}
    \pgfsetfillcolor{fc}
    \pgfusepathqfill
  \end{pgfscope}
  \begin{pgfscope}
    \definecolor{fc}{rgb}{0.0000,0.0000,0.0000}
    \pgfsetfillcolor{fc}
    \pgfusepathqfill
  \end{pgfscope}
  \begin{pgfscope}
    \pgfsetlinewidth{0.8113bp}
    \definecolor{sc}{rgb}{0.0000,0.0000,0.0000}
    \pgfsetstrokecolor{sc}
    \pgfsetmiterjoin
    \pgfsetbuttcap
    \pgfpathqmoveto{371.4286bp}{45.7143bp}
    \pgfpathqlineto{371.4286bp}{34.2857bp}
    \pgfusepathqstroke
  \end{pgfscope}
  \begin{pgfscope}
    \definecolor{fc}{rgb}{0.0000,0.0000,0.0000}
    \pgfsetfillcolor{fc}
    \pgfusepathqfill
  \end{pgfscope}
  \begin{pgfscope}
    \definecolor{fc}{rgb}{0.0000,0.0000,0.0000}
    \pgfsetfillcolor{fc}
    \pgfusepathqfill
  \end{pgfscope}
  \begin{pgfscope}
    \definecolor{fc}{rgb}{0.0000,0.0000,0.0000}
    \pgfsetfillcolor{fc}
    \pgfusepathqfill
  \end{pgfscope}
  \begin{pgfscope}
    \definecolor{fc}{rgb}{0.0000,0.0000,0.0000}
    \pgfsetfillcolor{fc}
    \pgfusepathqfill
  \end{pgfscope}
  \begin{pgfscope}
    \definecolor{fc}{rgb}{0.0000,0.0000,0.0000}
    \pgfsetfillcolor{fc}
    \pgfsetfillopacity{0.0000}
    \pgfsetlinewidth{0.8113bp}
    \definecolor{sc}{rgb}{0.0000,0.0000,0.0000}
    \pgfsetstrokecolor{sc}
    \pgfsetmiterjoin
    \pgfsetbuttcap
    \pgfpathqmoveto{354.2857bp}{28.5714bp}
    \pgfpathqcurveto{354.2857bp}{31.7273bp}{351.7273bp}{34.2857bp}{348.5714bp}{34.2857bp}
    \pgfpathqcurveto{345.4155bp}{34.2857bp}{342.8571bp}{31.7273bp}{342.8571bp}{28.5714bp}
    \pgfpathqcurveto{342.8571bp}{25.4155bp}{345.4155bp}{22.8571bp}{348.5714bp}{22.8571bp}
    \pgfpathqcurveto{351.7273bp}{22.8571bp}{354.2857bp}{25.4155bp}{354.2857bp}{28.5714bp}
    \pgfpathclose
    \pgfusepathqfillstroke
  \end{pgfscope}
  \begin{pgfscope}
    \definecolor{fc}{rgb}{0.0000,0.0000,0.0000}
    \pgfsetfillcolor{fc}
    \pgfsetfillopacity{0.0000}
    \pgfsetlinewidth{0.8113bp}
    \definecolor{sc}{rgb}{0.0000,0.0000,0.0000}
    \pgfsetstrokecolor{sc}
    \pgfsetmiterjoin
    \pgfsetbuttcap
    \pgfpathqmoveto{354.2857bp}{51.4286bp}
    \pgfpathqcurveto{354.2857bp}{54.5845bp}{351.7273bp}{57.1429bp}{348.5714bp}{57.1429bp}
    \pgfpathqcurveto{345.4155bp}{57.1429bp}{342.8571bp}{54.5845bp}{342.8571bp}{51.4286bp}
    \pgfpathqcurveto{342.8571bp}{48.2727bp}{345.4155bp}{45.7143bp}{348.5714bp}{45.7143bp}
    \pgfpathqcurveto{351.7273bp}{45.7143bp}{354.2857bp}{48.2727bp}{354.2857bp}{51.4286bp}
    \pgfpathclose
    \pgfusepathqfillstroke
  \end{pgfscope}
  \begin{pgfscope}
    \pgfsetlinewidth{0.8113bp}
    \definecolor{sc}{rgb}{0.0000,0.0000,0.0000}
    \pgfsetstrokecolor{sc}
    \pgfsetmiterjoin
    \pgfsetbuttcap
    \pgfpathqmoveto{348.5714bp}{45.7143bp}
    \pgfpathqlineto{348.5714bp}{34.2857bp}
    \pgfusepathqstroke
  \end{pgfscope}
  \begin{pgfscope}
    \definecolor{fc}{rgb}{0.0000,0.0000,0.0000}
    \pgfsetfillcolor{fc}
    \pgfusepathqfill
  \end{pgfscope}
  \begin{pgfscope}
    \definecolor{fc}{rgb}{0.0000,0.0000,0.0000}
    \pgfsetfillcolor{fc}
    \pgfusepathqfill
  \end{pgfscope}
  \begin{pgfscope}
    \definecolor{fc}{rgb}{0.0000,0.0000,0.0000}
    \pgfsetfillcolor{fc}
    \pgfusepathqfill
  \end{pgfscope}
  \begin{pgfscope}
    \definecolor{fc}{rgb}{0.0000,0.0000,0.0000}
    \pgfsetfillcolor{fc}
    \pgfusepathqfill
  \end{pgfscope}
  \begin{pgfscope}
    \definecolor{fc}{rgb}{0.0000,0.0000,0.0000}
    \pgfsetfillcolor{fc}
    \pgfsetfillopacity{0.0000}
    \pgfsetlinewidth{0.8113bp}
    \definecolor{sc}{rgb}{0.0000,0.0000,0.0000}
    \pgfsetstrokecolor{sc}
    \pgfsetmiterjoin
    \pgfsetbuttcap
    \pgfpathqmoveto{331.4286bp}{51.4286bp}
    \pgfpathqcurveto{331.4286bp}{54.5845bp}{328.8702bp}{57.1429bp}{325.7143bp}{57.1429bp}
    \pgfpathqcurveto{322.5584bp}{57.1429bp}{320.0000bp}{54.5845bp}{320.0000bp}{51.4286bp}
    \pgfpathqcurveto{320.0000bp}{48.2727bp}{322.5584bp}{45.7143bp}{325.7143bp}{45.7143bp}
    \pgfpathqcurveto{328.8702bp}{45.7143bp}{331.4286bp}{48.2727bp}{331.4286bp}{51.4286bp}
    \pgfpathclose
    \pgfusepathqfillstroke
  \end{pgfscope}
  \begin{pgfscope}
    \definecolor{fc}{rgb}{0.0000,0.0000,0.0000}
    \pgfsetfillcolor{fc}
    \pgfsetfillopacity{0.0000}
    \pgfsetlinewidth{0.8113bp}
    \definecolor{sc}{rgb}{0.0000,0.0000,0.0000}
    \pgfsetstrokecolor{sc}
    \pgfsetmiterjoin
    \pgfsetbuttcap
    \pgfpathqmoveto{331.4286bp}{74.2857bp}
    \pgfpathqcurveto{331.4286bp}{77.4416bp}{328.8702bp}{80.0000bp}{325.7143bp}{80.0000bp}
    \pgfpathqcurveto{322.5584bp}{80.0000bp}{320.0000bp}{77.4416bp}{320.0000bp}{74.2857bp}
    \pgfpathqcurveto{320.0000bp}{71.1298bp}{322.5584bp}{68.5714bp}{325.7143bp}{68.5714bp}
    \pgfpathqcurveto{328.8702bp}{68.5714bp}{331.4286bp}{71.1298bp}{331.4286bp}{74.2857bp}
    \pgfpathclose
    \pgfusepathqfillstroke
  \end{pgfscope}
  \begin{pgfscope}
    \pgfsetlinewidth{0.8113bp}
    \definecolor{sc}{rgb}{0.0000,0.0000,0.0000}
    \pgfsetstrokecolor{sc}
    \pgfsetmiterjoin
    \pgfsetbuttcap
    \pgfpathqmoveto{330.8264bp}{71.7296bp}
    \pgfpathqlineto{366.3164bp}{53.9846bp}
    \pgfusepathqstroke
  \end{pgfscope}
  \begin{pgfscope}
    \definecolor{fc}{rgb}{0.0000,0.0000,0.0000}
    \pgfsetfillcolor{fc}
    \pgfusepathqfill
  \end{pgfscope}
  \begin{pgfscope}
    \definecolor{fc}{rgb}{0.0000,0.0000,0.0000}
    \pgfsetfillcolor{fc}
    \pgfusepathqfill
  \end{pgfscope}
  \begin{pgfscope}
    \definecolor{fc}{rgb}{0.0000,0.0000,0.0000}
    \pgfsetfillcolor{fc}
    \pgfusepathqfill
  \end{pgfscope}
  \begin{pgfscope}
    \definecolor{fc}{rgb}{0.0000,0.0000,0.0000}
    \pgfsetfillcolor{fc}
    \pgfusepathqfill
  \end{pgfscope}
  \begin{pgfscope}
    \pgfsetlinewidth{0.8113bp}
    \definecolor{sc}{rgb}{0.0000,0.0000,0.0000}
    \pgfsetstrokecolor{sc}
    \pgfsetmiterjoin
    \pgfsetbuttcap
    \pgfpathqmoveto{329.7549bp}{70.2451bp}
    \pgfpathqlineto{344.5308bp}{55.4692bp}
    \pgfusepathqstroke
  \end{pgfscope}
  \begin{pgfscope}
    \definecolor{fc}{rgb}{0.0000,0.0000,0.0000}
    \pgfsetfillcolor{fc}
    \pgfusepathqfill
  \end{pgfscope}
  \begin{pgfscope}
    \definecolor{fc}{rgb}{0.0000,0.0000,0.0000}
    \pgfsetfillcolor{fc}
    \pgfusepathqfill
  \end{pgfscope}
  \begin{pgfscope}
    \definecolor{fc}{rgb}{0.0000,0.0000,0.0000}
    \pgfsetfillcolor{fc}
    \pgfusepathqfill
  \end{pgfscope}
  \begin{pgfscope}
    \definecolor{fc}{rgb}{0.0000,0.0000,0.0000}
    \pgfsetfillcolor{fc}
    \pgfusepathqfill
  \end{pgfscope}
  \begin{pgfscope}
    \pgfsetlinewidth{0.8113bp}
    \definecolor{sc}{rgb}{0.0000,0.0000,0.0000}
    \pgfsetstrokecolor{sc}
    \pgfsetmiterjoin
    \pgfsetbuttcap
    \pgfpathqmoveto{325.7143bp}{68.5714bp}
    \pgfpathqlineto{325.7143bp}{57.1429bp}
    \pgfusepathqstroke
  \end{pgfscope}
  \begin{pgfscope}
    \definecolor{fc}{rgb}{0.0000,0.0000,0.0000}
    \pgfsetfillcolor{fc}
    \pgfusepathqfill
  \end{pgfscope}
  \begin{pgfscope}
    \definecolor{fc}{rgb}{0.0000,0.0000,0.0000}
    \pgfsetfillcolor{fc}
    \pgfusepathqfill
  \end{pgfscope}
  \begin{pgfscope}
    \definecolor{fc}{rgb}{0.0000,0.0000,0.0000}
    \pgfsetfillcolor{fc}
    \pgfusepathqfill
  \end{pgfscope}
  \begin{pgfscope}
    \definecolor{fc}{rgb}{0.0000,0.0000,0.0000}
    \pgfsetfillcolor{fc}
    \pgfusepathqfill
  \end{pgfscope}
  \begin{pgfscope}
    \definecolor{fc}{rgb}{0.0000,0.0000,0.0000}
    \pgfsetfillcolor{fc}
    \pgfsetfillopacity{0.0000}
    \pgfsetlinewidth{0.8113bp}
    \definecolor{sc}{rgb}{0.0000,0.0000,0.0000}
    \pgfsetstrokecolor{sc}
    \pgfsetmiterjoin
    \pgfsetbuttcap
    \pgfpathqmoveto{308.5714bp}{28.5714bp}
    \pgfpathqcurveto{308.5714bp}{31.7273bp}{306.0131bp}{34.2857bp}{302.8571bp}{34.2857bp}
    \pgfpathqcurveto{299.7012bp}{34.2857bp}{297.1429bp}{31.7273bp}{297.1429bp}{28.5714bp}
    \pgfpathqcurveto{297.1429bp}{25.4155bp}{299.7012bp}{22.8571bp}{302.8571bp}{22.8571bp}
    \pgfpathqcurveto{306.0131bp}{22.8571bp}{308.5714bp}{25.4155bp}{308.5714bp}{28.5714bp}
    \pgfpathclose
    \pgfusepathqfillstroke
  \end{pgfscope}
  \begin{pgfscope}
    \definecolor{fc}{rgb}{0.0000,0.0000,0.0000}
    \pgfsetfillcolor{fc}
    \pgfsetfillopacity{0.0000}
    \pgfsetlinewidth{0.8113bp}
    \definecolor{sc}{rgb}{0.0000,0.0000,0.0000}
    \pgfsetstrokecolor{sc}
    \pgfsetmiterjoin
    \pgfsetbuttcap
    \pgfpathqmoveto{308.5714bp}{51.4286bp}
    \pgfpathqcurveto{308.5714bp}{54.5845bp}{306.0131bp}{57.1429bp}{302.8571bp}{57.1429bp}
    \pgfpathqcurveto{299.7012bp}{57.1429bp}{297.1429bp}{54.5845bp}{297.1429bp}{51.4286bp}
    \pgfpathqcurveto{297.1429bp}{48.2727bp}{299.7012bp}{45.7143bp}{302.8571bp}{45.7143bp}
    \pgfpathqcurveto{306.0131bp}{45.7143bp}{308.5714bp}{48.2727bp}{308.5714bp}{51.4286bp}
    \pgfpathclose
    \pgfusepathqfillstroke
  \end{pgfscope}
  \begin{pgfscope}
    \pgfsetlinewidth{0.8113bp}
    \definecolor{sc}{rgb}{0.0000,0.0000,0.0000}
    \pgfsetstrokecolor{sc}
    \pgfsetmiterjoin
    \pgfsetbuttcap
    \pgfpathqmoveto{302.8571bp}{45.7143bp}
    \pgfpathqlineto{302.8571bp}{34.2857bp}
    \pgfusepathqstroke
  \end{pgfscope}
  \begin{pgfscope}
    \definecolor{fc}{rgb}{0.0000,0.0000,0.0000}
    \pgfsetfillcolor{fc}
    \pgfusepathqfill
  \end{pgfscope}
  \begin{pgfscope}
    \definecolor{fc}{rgb}{0.0000,0.0000,0.0000}
    \pgfsetfillcolor{fc}
    \pgfusepathqfill
  \end{pgfscope}
  \begin{pgfscope}
    \definecolor{fc}{rgb}{0.0000,0.0000,0.0000}
    \pgfsetfillcolor{fc}
    \pgfusepathqfill
  \end{pgfscope}
  \begin{pgfscope}
    \definecolor{fc}{rgb}{0.0000,0.0000,0.0000}
    \pgfsetfillcolor{fc}
    \pgfusepathqfill
  \end{pgfscope}
  \begin{pgfscope}
    \definecolor{fc}{rgb}{0.0000,0.0000,0.0000}
    \pgfsetfillcolor{fc}
    \pgfsetfillopacity{0.0000}
    \pgfsetlinewidth{0.8113bp}
    \definecolor{sc}{rgb}{0.0000,0.0000,0.0000}
    \pgfsetstrokecolor{sc}
    \pgfsetmiterjoin
    \pgfsetbuttcap
    \pgfpathqmoveto{285.7143bp}{51.4286bp}
    \pgfpathqcurveto{285.7143bp}{54.5845bp}{283.1559bp}{57.1429bp}{280.0000bp}{57.1429bp}
    \pgfpathqcurveto{276.8441bp}{57.1429bp}{274.2857bp}{54.5845bp}{274.2857bp}{51.4286bp}
    \pgfpathqcurveto{274.2857bp}{48.2727bp}{276.8441bp}{45.7143bp}{280.0000bp}{45.7143bp}
    \pgfpathqcurveto{283.1559bp}{45.7143bp}{285.7143bp}{48.2727bp}{285.7143bp}{51.4286bp}
    \pgfpathclose
    \pgfusepathqfillstroke
  \end{pgfscope}
  \begin{pgfscope}
    \definecolor{fc}{rgb}{0.0000,0.0000,0.0000}
    \pgfsetfillcolor{fc}
    \pgfsetfillopacity{0.0000}
    \pgfsetlinewidth{0.8113bp}
    \definecolor{sc}{rgb}{0.0000,0.0000,0.0000}
    \pgfsetstrokecolor{sc}
    \pgfsetmiterjoin
    \pgfsetbuttcap
    \pgfpathqmoveto{285.7143bp}{74.2857bp}
    \pgfpathqcurveto{285.7143bp}{77.4416bp}{283.1559bp}{80.0000bp}{280.0000bp}{80.0000bp}
    \pgfpathqcurveto{276.8441bp}{80.0000bp}{274.2857bp}{77.4416bp}{274.2857bp}{74.2857bp}
    \pgfpathqcurveto{274.2857bp}{71.1298bp}{276.8441bp}{68.5714bp}{280.0000bp}{68.5714bp}
    \pgfpathqcurveto{283.1559bp}{68.5714bp}{285.7143bp}{71.1298bp}{285.7143bp}{74.2857bp}
    \pgfpathclose
    \pgfusepathqfillstroke
  \end{pgfscope}
  \begin{pgfscope}
    \pgfsetlinewidth{0.8113bp}
    \definecolor{sc}{rgb}{0.0000,0.0000,0.0000}
    \pgfsetstrokecolor{sc}
    \pgfsetmiterjoin
    \pgfsetbuttcap
    \pgfpathqmoveto{284.0406bp}{70.2451bp}
    \pgfpathqlineto{298.8165bp}{55.4692bp}
    \pgfusepathqstroke
  \end{pgfscope}
  \begin{pgfscope}
    \definecolor{fc}{rgb}{0.0000,0.0000,0.0000}
    \pgfsetfillcolor{fc}
    \pgfusepathqfill
  \end{pgfscope}
  \begin{pgfscope}
    \definecolor{fc}{rgb}{0.0000,0.0000,0.0000}
    \pgfsetfillcolor{fc}
    \pgfusepathqfill
  \end{pgfscope}
  \begin{pgfscope}
    \definecolor{fc}{rgb}{0.0000,0.0000,0.0000}
    \pgfsetfillcolor{fc}
    \pgfusepathqfill
  \end{pgfscope}
  \begin{pgfscope}
    \definecolor{fc}{rgb}{0.0000,0.0000,0.0000}
    \pgfsetfillcolor{fc}
    \pgfusepathqfill
  \end{pgfscope}
  \begin{pgfscope}
    \pgfsetlinewidth{0.8113bp}
    \definecolor{sc}{rgb}{0.0000,0.0000,0.0000}
    \pgfsetstrokecolor{sc}
    \pgfsetmiterjoin
    \pgfsetbuttcap
    \pgfpathqmoveto{280.0000bp}{68.5714bp}
    \pgfpathqlineto{280.0000bp}{57.1429bp}
    \pgfusepathqstroke
  \end{pgfscope}
  \begin{pgfscope}
    \definecolor{fc}{rgb}{0.0000,0.0000,0.0000}
    \pgfsetfillcolor{fc}
    \pgfusepathqfill
  \end{pgfscope}
  \begin{pgfscope}
    \definecolor{fc}{rgb}{0.0000,0.0000,0.0000}
    \pgfsetfillcolor{fc}
    \pgfusepathqfill
  \end{pgfscope}
  \begin{pgfscope}
    \definecolor{fc}{rgb}{0.0000,0.0000,0.0000}
    \pgfsetfillcolor{fc}
    \pgfusepathqfill
  \end{pgfscope}
  \begin{pgfscope}
    \definecolor{fc}{rgb}{0.0000,0.0000,0.0000}
    \pgfsetfillcolor{fc}
    \pgfusepathqfill
  \end{pgfscope}
  \begin{pgfscope}
    \definecolor{fc}{rgb}{0.0000,0.0000,0.0000}
    \pgfsetfillcolor{fc}
    \pgfsetfillopacity{0.0000}
    \pgfsetlinewidth{0.8113bp}
    \definecolor{sc}{rgb}{0.0000,0.0000,0.0000}
    \pgfsetstrokecolor{sc}
    \pgfsetmiterjoin
    \pgfsetbuttcap
    \pgfpathqmoveto{262.8571bp}{51.4286bp}
    \pgfpathqcurveto{262.8571bp}{54.5845bp}{260.2988bp}{57.1429bp}{257.1429bp}{57.1429bp}
    \pgfpathqcurveto{253.9869bp}{57.1429bp}{251.4286bp}{54.5845bp}{251.4286bp}{51.4286bp}
    \pgfpathqcurveto{251.4286bp}{48.2727bp}{253.9869bp}{45.7143bp}{257.1429bp}{45.7143bp}
    \pgfpathqcurveto{260.2988bp}{45.7143bp}{262.8571bp}{48.2727bp}{262.8571bp}{51.4286bp}
    \pgfpathclose
    \pgfusepathqfillstroke
  \end{pgfscope}
  \begin{pgfscope}
    \definecolor{fc}{rgb}{0.0000,0.0000,0.0000}
    \pgfsetfillcolor{fc}
    \pgfsetfillopacity{0.0000}
    \pgfsetlinewidth{0.8113bp}
    \definecolor{sc}{rgb}{0.0000,0.0000,0.0000}
    \pgfsetstrokecolor{sc}
    \pgfsetmiterjoin
    \pgfsetbuttcap
    \pgfpathqmoveto{262.8571bp}{74.2857bp}
    \pgfpathqcurveto{262.8571bp}{77.4416bp}{260.2988bp}{80.0000bp}{257.1429bp}{80.0000bp}
    \pgfpathqcurveto{253.9869bp}{80.0000bp}{251.4286bp}{77.4416bp}{251.4286bp}{74.2857bp}
    \pgfpathqcurveto{251.4286bp}{71.1298bp}{253.9869bp}{68.5714bp}{257.1429bp}{68.5714bp}
    \pgfpathqcurveto{260.2988bp}{68.5714bp}{262.8571bp}{71.1298bp}{262.8571bp}{74.2857bp}
    \pgfpathclose
    \pgfusepathqfillstroke
  \end{pgfscope}
  \begin{pgfscope}
    \pgfsetlinewidth{0.8113bp}
    \definecolor{sc}{rgb}{0.0000,0.0000,0.0000}
    \pgfsetstrokecolor{sc}
    \pgfsetmiterjoin
    \pgfsetbuttcap
    \pgfpathqmoveto{257.1429bp}{68.5714bp}
    \pgfpathqlineto{257.1429bp}{57.1429bp}
    \pgfusepathqstroke
  \end{pgfscope}
  \begin{pgfscope}
    \definecolor{fc}{rgb}{0.0000,0.0000,0.0000}
    \pgfsetfillcolor{fc}
    \pgfusepathqfill
  \end{pgfscope}
  \begin{pgfscope}
    \definecolor{fc}{rgb}{0.0000,0.0000,0.0000}
    \pgfsetfillcolor{fc}
    \pgfusepathqfill
  \end{pgfscope}
  \begin{pgfscope}
    \definecolor{fc}{rgb}{0.0000,0.0000,0.0000}
    \pgfsetfillcolor{fc}
    \pgfusepathqfill
  \end{pgfscope}
  \begin{pgfscope}
    \definecolor{fc}{rgb}{0.0000,0.0000,0.0000}
    \pgfsetfillcolor{fc}
    \pgfusepathqfill
  \end{pgfscope}
  \begin{pgfscope}
    \definecolor{fc}{rgb}{0.0000,0.0000,0.0000}
    \pgfsetfillcolor{fc}
    \pgfsetfillopacity{0.0000}
    \pgfsetlinewidth{0.8113bp}
    \definecolor{sc}{rgb}{0.0000,0.0000,0.0000}
    \pgfsetstrokecolor{sc}
    \pgfsetmiterjoin
    \pgfsetbuttcap
    \pgfpathqmoveto{240.0000bp}{74.2857bp}
    \pgfpathqcurveto{240.0000bp}{77.4416bp}{237.4416bp}{80.0000bp}{234.2857bp}{80.0000bp}
    \pgfpathqcurveto{231.1298bp}{80.0000bp}{228.5714bp}{77.4416bp}{228.5714bp}{74.2857bp}
    \pgfpathqcurveto{228.5714bp}{71.1298bp}{231.1298bp}{68.5714bp}{234.2857bp}{68.5714bp}
    \pgfpathqcurveto{237.4416bp}{68.5714bp}{240.0000bp}{71.1298bp}{240.0000bp}{74.2857bp}
    \pgfpathclose
    \pgfusepathqfillstroke
  \end{pgfscope}
  \begin{pgfscope}
    \definecolor{fc}{rgb}{0.0000,0.0000,0.0000}
    \pgfsetfillcolor{fc}
    \pgfsetfillopacity{0.0000}
    \pgfsetlinewidth{0.8113bp}
    \definecolor{sc}{rgb}{0.0000,0.0000,0.0000}
    \pgfsetstrokecolor{sc}
    \pgfsetmiterjoin
    \pgfsetbuttcap
    \pgfpathqmoveto{240.0000bp}{97.1429bp}
    \pgfpathqcurveto{240.0000bp}{100.2988bp}{237.4416bp}{102.8571bp}{234.2857bp}{102.8571bp}
    \pgfpathqcurveto{231.1298bp}{102.8571bp}{228.5714bp}{100.2988bp}{228.5714bp}{97.1429bp}
    \pgfpathqcurveto{228.5714bp}{93.9869bp}{231.1298bp}{91.4286bp}{234.2857bp}{91.4286bp}
    \pgfpathqcurveto{237.4416bp}{91.4286bp}{240.0000bp}{93.9869bp}{240.0000bp}{97.1429bp}
    \pgfpathclose
    \pgfusepathqfillstroke
  \end{pgfscope}
  \begin{pgfscope}
    \pgfsetlinewidth{0.8113bp}
    \definecolor{sc}{rgb}{0.0000,0.0000,0.0000}
    \pgfsetstrokecolor{sc}
    \pgfsetmiterjoin
    \pgfsetbuttcap
    \pgfpathqmoveto{239.8307bp}{95.7566bp}
    \pgfpathqlineto{320.1693bp}{75.6720bp}
    \pgfusepathqstroke
  \end{pgfscope}
  \begin{pgfscope}
    \definecolor{fc}{rgb}{0.0000,0.0000,0.0000}
    \pgfsetfillcolor{fc}
    \pgfusepathqfill
  \end{pgfscope}
  \begin{pgfscope}
    \definecolor{fc}{rgb}{0.0000,0.0000,0.0000}
    \pgfsetfillcolor{fc}
    \pgfusepathqfill
  \end{pgfscope}
  \begin{pgfscope}
    \definecolor{fc}{rgb}{0.0000,0.0000,0.0000}
    \pgfsetfillcolor{fc}
    \pgfusepathqfill
  \end{pgfscope}
  \begin{pgfscope}
    \definecolor{fc}{rgb}{0.0000,0.0000,0.0000}
    \pgfsetfillcolor{fc}
    \pgfusepathqfill
  \end{pgfscope}
  \begin{pgfscope}
    \pgfsetlinewidth{0.8113bp}
    \definecolor{sc}{rgb}{0.0000,0.0000,0.0000}
    \pgfsetstrokecolor{sc}
    \pgfsetmiterjoin
    \pgfsetbuttcap
    \pgfpathqmoveto{239.3978bp}{94.5868bp}
    \pgfpathqlineto{274.8879bp}{76.8418bp}
    \pgfusepathqstroke
  \end{pgfscope}
  \begin{pgfscope}
    \definecolor{fc}{rgb}{0.0000,0.0000,0.0000}
    \pgfsetfillcolor{fc}
    \pgfusepathqfill
  \end{pgfscope}
  \begin{pgfscope}
    \definecolor{fc}{rgb}{0.0000,0.0000,0.0000}
    \pgfsetfillcolor{fc}
    \pgfusepathqfill
  \end{pgfscope}
  \begin{pgfscope}
    \definecolor{fc}{rgb}{0.0000,0.0000,0.0000}
    \pgfsetfillcolor{fc}
    \pgfusepathqfill
  \end{pgfscope}
  \begin{pgfscope}
    \definecolor{fc}{rgb}{0.0000,0.0000,0.0000}
    \pgfsetfillcolor{fc}
    \pgfusepathqfill
  \end{pgfscope}
  \begin{pgfscope}
    \pgfsetlinewidth{0.8113bp}
    \definecolor{sc}{rgb}{0.0000,0.0000,0.0000}
    \pgfsetstrokecolor{sc}
    \pgfsetmiterjoin
    \pgfsetbuttcap
    \pgfpathqmoveto{238.3263bp}{93.1022bp}
    \pgfpathqlineto{253.1022bp}{78.3263bp}
    \pgfusepathqstroke
  \end{pgfscope}
  \begin{pgfscope}
    \definecolor{fc}{rgb}{0.0000,0.0000,0.0000}
    \pgfsetfillcolor{fc}
    \pgfusepathqfill
  \end{pgfscope}
  \begin{pgfscope}
    \definecolor{fc}{rgb}{0.0000,0.0000,0.0000}
    \pgfsetfillcolor{fc}
    \pgfusepathqfill
  \end{pgfscope}
  \begin{pgfscope}
    \definecolor{fc}{rgb}{0.0000,0.0000,0.0000}
    \pgfsetfillcolor{fc}
    \pgfusepathqfill
  \end{pgfscope}
  \begin{pgfscope}
    \definecolor{fc}{rgb}{0.0000,0.0000,0.0000}
    \pgfsetfillcolor{fc}
    \pgfusepathqfill
  \end{pgfscope}
  \begin{pgfscope}
    \pgfsetlinewidth{0.8113bp}
    \definecolor{sc}{rgb}{0.0000,0.0000,0.0000}
    \pgfsetstrokecolor{sc}
    \pgfsetmiterjoin
    \pgfsetbuttcap
    \pgfpathqmoveto{234.2857bp}{91.4286bp}
    \pgfpathqlineto{234.2857bp}{80.0000bp}
    \pgfusepathqstroke
  \end{pgfscope}
  \begin{pgfscope}
    \definecolor{fc}{rgb}{0.0000,0.0000,0.0000}
    \pgfsetfillcolor{fc}
    \pgfusepathqfill
  \end{pgfscope}
  \begin{pgfscope}
    \definecolor{fc}{rgb}{0.0000,0.0000,0.0000}
    \pgfsetfillcolor{fc}
    \pgfusepathqfill
  \end{pgfscope}
  \begin{pgfscope}
    \definecolor{fc}{rgb}{0.0000,0.0000,0.0000}
    \pgfsetfillcolor{fc}
    \pgfusepathqfill
  \end{pgfscope}
  \begin{pgfscope}
    \definecolor{fc}{rgb}{0.0000,0.0000,0.0000}
    \pgfsetfillcolor{fc}
    \pgfusepathqfill
  \end{pgfscope}
  \begin{pgfscope}
    \definecolor{fc}{rgb}{0.0000,0.0000,0.0000}
    \pgfsetfillcolor{fc}
    \pgfsetfillopacity{0.0000}
    \pgfsetlinewidth{0.8113bp}
    \definecolor{sc}{rgb}{0.0000,0.0000,0.0000}
    \pgfsetstrokecolor{sc}
    \pgfsetmiterjoin
    \pgfsetbuttcap
    \pgfpathqmoveto{205.7143bp}{28.5714bp}
    \pgfpathqcurveto{205.7143bp}{31.7273bp}{203.1559bp}{34.2857bp}{200.0000bp}{34.2857bp}
    \pgfpathqcurveto{196.8441bp}{34.2857bp}{194.2857bp}{31.7273bp}{194.2857bp}{28.5714bp}
    \pgfpathqcurveto{194.2857bp}{25.4155bp}{196.8441bp}{22.8571bp}{200.0000bp}{22.8571bp}
    \pgfpathqcurveto{203.1559bp}{22.8571bp}{205.7143bp}{25.4155bp}{205.7143bp}{28.5714bp}
    \pgfpathclose
    \pgfusepathqfillstroke
  \end{pgfscope}
  \begin{pgfscope}
    \definecolor{fc}{rgb}{0.0000,0.0000,0.0000}
    \pgfsetfillcolor{fc}
    \pgfsetfillopacity{0.0000}
    \pgfsetlinewidth{0.8113bp}
    \definecolor{sc}{rgb}{0.0000,0.0000,0.0000}
    \pgfsetstrokecolor{sc}
    \pgfsetmiterjoin
    \pgfsetbuttcap
    \pgfpathqmoveto{205.7143bp}{51.4286bp}
    \pgfpathqcurveto{205.7143bp}{54.5845bp}{203.1559bp}{57.1429bp}{200.0000bp}{57.1429bp}
    \pgfpathqcurveto{196.8441bp}{57.1429bp}{194.2857bp}{54.5845bp}{194.2857bp}{51.4286bp}
    \pgfpathqcurveto{194.2857bp}{48.2727bp}{196.8441bp}{45.7143bp}{200.0000bp}{45.7143bp}
    \pgfpathqcurveto{203.1559bp}{45.7143bp}{205.7143bp}{48.2727bp}{205.7143bp}{51.4286bp}
    \pgfpathclose
    \pgfusepathqfillstroke
  \end{pgfscope}
  \begin{pgfscope}
    \pgfsetlinewidth{0.8113bp}
    \definecolor{sc}{rgb}{0.0000,0.0000,0.0000}
    \pgfsetstrokecolor{sc}
    \pgfsetmiterjoin
    \pgfsetbuttcap
    \pgfpathqmoveto{200.0000bp}{45.7143bp}
    \pgfpathqlineto{200.0000bp}{34.2857bp}
    \pgfusepathqstroke
  \end{pgfscope}
  \begin{pgfscope}
    \definecolor{fc}{rgb}{0.0000,0.0000,0.0000}
    \pgfsetfillcolor{fc}
    \pgfusepathqfill
  \end{pgfscope}
  \begin{pgfscope}
    \definecolor{fc}{rgb}{0.0000,0.0000,0.0000}
    \pgfsetfillcolor{fc}
    \pgfusepathqfill
  \end{pgfscope}
  \begin{pgfscope}
    \definecolor{fc}{rgb}{0.0000,0.0000,0.0000}
    \pgfsetfillcolor{fc}
    \pgfusepathqfill
  \end{pgfscope}
  \begin{pgfscope}
    \definecolor{fc}{rgb}{0.0000,0.0000,0.0000}
    \pgfsetfillcolor{fc}
    \pgfusepathqfill
  \end{pgfscope}
  \begin{pgfscope}
    \definecolor{fc}{rgb}{0.0000,0.0000,0.0000}
    \pgfsetfillcolor{fc}
    \pgfsetfillopacity{0.0000}
    \pgfsetlinewidth{0.8113bp}
    \definecolor{sc}{rgb}{0.0000,0.0000,0.0000}
    \pgfsetstrokecolor{sc}
    \pgfsetmiterjoin
    \pgfsetbuttcap
    \pgfpathqmoveto{182.8571bp}{51.4286bp}
    \pgfpathqcurveto{182.8571bp}{54.5845bp}{180.2988bp}{57.1429bp}{177.1429bp}{57.1429bp}
    \pgfpathqcurveto{173.9869bp}{57.1429bp}{171.4286bp}{54.5845bp}{171.4286bp}{51.4286bp}
    \pgfpathqcurveto{171.4286bp}{48.2727bp}{173.9869bp}{45.7143bp}{177.1429bp}{45.7143bp}
    \pgfpathqcurveto{180.2988bp}{45.7143bp}{182.8571bp}{48.2727bp}{182.8571bp}{51.4286bp}
    \pgfpathclose
    \pgfusepathqfillstroke
  \end{pgfscope}
  \begin{pgfscope}
    \definecolor{fc}{rgb}{0.0000,0.0000,0.0000}
    \pgfsetfillcolor{fc}
    \pgfsetfillopacity{0.0000}
    \pgfsetlinewidth{0.8113bp}
    \definecolor{sc}{rgb}{0.0000,0.0000,0.0000}
    \pgfsetstrokecolor{sc}
    \pgfsetmiterjoin
    \pgfsetbuttcap
    \pgfpathqmoveto{182.8571bp}{74.2857bp}
    \pgfpathqcurveto{182.8571bp}{77.4416bp}{180.2988bp}{80.0000bp}{177.1429bp}{80.0000bp}
    \pgfpathqcurveto{173.9869bp}{80.0000bp}{171.4286bp}{77.4416bp}{171.4286bp}{74.2857bp}
    \pgfpathqcurveto{171.4286bp}{71.1298bp}{173.9869bp}{68.5714bp}{177.1429bp}{68.5714bp}
    \pgfpathqcurveto{180.2988bp}{68.5714bp}{182.8571bp}{71.1298bp}{182.8571bp}{74.2857bp}
    \pgfpathclose
    \pgfusepathqfillstroke
  \end{pgfscope}
  \begin{pgfscope}
    \pgfsetlinewidth{0.8113bp}
    \definecolor{sc}{rgb}{0.0000,0.0000,0.0000}
    \pgfsetstrokecolor{sc}
    \pgfsetmiterjoin
    \pgfsetbuttcap
    \pgfpathqmoveto{181.1835bp}{70.2451bp}
    \pgfpathqlineto{195.9594bp}{55.4692bp}
    \pgfusepathqstroke
  \end{pgfscope}
  \begin{pgfscope}
    \definecolor{fc}{rgb}{0.0000,0.0000,0.0000}
    \pgfsetfillcolor{fc}
    \pgfusepathqfill
  \end{pgfscope}
  \begin{pgfscope}
    \definecolor{fc}{rgb}{0.0000,0.0000,0.0000}
    \pgfsetfillcolor{fc}
    \pgfusepathqfill
  \end{pgfscope}
  \begin{pgfscope}
    \definecolor{fc}{rgb}{0.0000,0.0000,0.0000}
    \pgfsetfillcolor{fc}
    \pgfusepathqfill
  \end{pgfscope}
  \begin{pgfscope}
    \definecolor{fc}{rgb}{0.0000,0.0000,0.0000}
    \pgfsetfillcolor{fc}
    \pgfusepathqfill
  \end{pgfscope}
  \begin{pgfscope}
    \pgfsetlinewidth{0.8113bp}
    \definecolor{sc}{rgb}{0.0000,0.0000,0.0000}
    \pgfsetstrokecolor{sc}
    \pgfsetmiterjoin
    \pgfsetbuttcap
    \pgfpathqmoveto{177.1429bp}{68.5714bp}
    \pgfpathqlineto{177.1429bp}{57.1429bp}
    \pgfusepathqstroke
  \end{pgfscope}
  \begin{pgfscope}
    \definecolor{fc}{rgb}{0.0000,0.0000,0.0000}
    \pgfsetfillcolor{fc}
    \pgfusepathqfill
  \end{pgfscope}
  \begin{pgfscope}
    \definecolor{fc}{rgb}{0.0000,0.0000,0.0000}
    \pgfsetfillcolor{fc}
    \pgfusepathqfill
  \end{pgfscope}
  \begin{pgfscope}
    \definecolor{fc}{rgb}{0.0000,0.0000,0.0000}
    \pgfsetfillcolor{fc}
    \pgfusepathqfill
  \end{pgfscope}
  \begin{pgfscope}
    \definecolor{fc}{rgb}{0.0000,0.0000,0.0000}
    \pgfsetfillcolor{fc}
    \pgfusepathqfill
  \end{pgfscope}
  \begin{pgfscope}
    \definecolor{fc}{rgb}{0.0000,0.0000,0.0000}
    \pgfsetfillcolor{fc}
    \pgfsetfillopacity{0.0000}
    \pgfsetlinewidth{0.8113bp}
    \definecolor{sc}{rgb}{0.0000,0.0000,0.0000}
    \pgfsetstrokecolor{sc}
    \pgfsetmiterjoin
    \pgfsetbuttcap
    \pgfpathqmoveto{160.0000bp}{51.4286bp}
    \pgfpathqcurveto{160.0000bp}{54.5845bp}{157.4416bp}{57.1429bp}{154.2857bp}{57.1429bp}
    \pgfpathqcurveto{151.1298bp}{57.1429bp}{148.5714bp}{54.5845bp}{148.5714bp}{51.4286bp}
    \pgfpathqcurveto{148.5714bp}{48.2727bp}{151.1298bp}{45.7143bp}{154.2857bp}{45.7143bp}
    \pgfpathqcurveto{157.4416bp}{45.7143bp}{160.0000bp}{48.2727bp}{160.0000bp}{51.4286bp}
    \pgfpathclose
    \pgfusepathqfillstroke
  \end{pgfscope}
  \begin{pgfscope}
    \definecolor{fc}{rgb}{0.0000,0.0000,0.0000}
    \pgfsetfillcolor{fc}
    \pgfsetfillopacity{0.0000}
    \pgfsetlinewidth{0.8113bp}
    \definecolor{sc}{rgb}{0.0000,0.0000,0.0000}
    \pgfsetstrokecolor{sc}
    \pgfsetmiterjoin
    \pgfsetbuttcap
    \pgfpathqmoveto{160.0000bp}{74.2857bp}
    \pgfpathqcurveto{160.0000bp}{77.4416bp}{157.4416bp}{80.0000bp}{154.2857bp}{80.0000bp}
    \pgfpathqcurveto{151.1298bp}{80.0000bp}{148.5714bp}{77.4416bp}{148.5714bp}{74.2857bp}
    \pgfpathqcurveto{148.5714bp}{71.1298bp}{151.1298bp}{68.5714bp}{154.2857bp}{68.5714bp}
    \pgfpathqcurveto{157.4416bp}{68.5714bp}{160.0000bp}{71.1298bp}{160.0000bp}{74.2857bp}
    \pgfpathclose
    \pgfusepathqfillstroke
  \end{pgfscope}
  \begin{pgfscope}
    \pgfsetlinewidth{0.8113bp}
    \definecolor{sc}{rgb}{0.0000,0.0000,0.0000}
    \pgfsetstrokecolor{sc}
    \pgfsetmiterjoin
    \pgfsetbuttcap
    \pgfpathqmoveto{154.2857bp}{68.5714bp}
    \pgfpathqlineto{154.2857bp}{57.1429bp}
    \pgfusepathqstroke
  \end{pgfscope}
  \begin{pgfscope}
    \definecolor{fc}{rgb}{0.0000,0.0000,0.0000}
    \pgfsetfillcolor{fc}
    \pgfusepathqfill
  \end{pgfscope}
  \begin{pgfscope}
    \definecolor{fc}{rgb}{0.0000,0.0000,0.0000}
    \pgfsetfillcolor{fc}
    \pgfusepathqfill
  \end{pgfscope}
  \begin{pgfscope}
    \definecolor{fc}{rgb}{0.0000,0.0000,0.0000}
    \pgfsetfillcolor{fc}
    \pgfusepathqfill
  \end{pgfscope}
  \begin{pgfscope}
    \definecolor{fc}{rgb}{0.0000,0.0000,0.0000}
    \pgfsetfillcolor{fc}
    \pgfusepathqfill
  \end{pgfscope}
  \begin{pgfscope}
    \definecolor{fc}{rgb}{0.0000,0.0000,0.0000}
    \pgfsetfillcolor{fc}
    \pgfsetfillopacity{0.0000}
    \pgfsetlinewidth{0.8113bp}
    \definecolor{sc}{rgb}{0.0000,0.0000,0.0000}
    \pgfsetstrokecolor{sc}
    \pgfsetmiterjoin
    \pgfsetbuttcap
    \pgfpathqmoveto{137.1429bp}{74.2857bp}
    \pgfpathqcurveto{137.1429bp}{77.4416bp}{134.5845bp}{80.0000bp}{131.4286bp}{80.0000bp}
    \pgfpathqcurveto{128.2727bp}{80.0000bp}{125.7143bp}{77.4416bp}{125.7143bp}{74.2857bp}
    \pgfpathqcurveto{125.7143bp}{71.1298bp}{128.2727bp}{68.5714bp}{131.4286bp}{68.5714bp}
    \pgfpathqcurveto{134.5845bp}{68.5714bp}{137.1429bp}{71.1298bp}{137.1429bp}{74.2857bp}
    \pgfpathclose
    \pgfusepathqfillstroke
  \end{pgfscope}
  \begin{pgfscope}
    \definecolor{fc}{rgb}{0.0000,0.0000,0.0000}
    \pgfsetfillcolor{fc}
    \pgfsetfillopacity{0.0000}
    \pgfsetlinewidth{0.8113bp}
    \definecolor{sc}{rgb}{0.0000,0.0000,0.0000}
    \pgfsetstrokecolor{sc}
    \pgfsetmiterjoin
    \pgfsetbuttcap
    \pgfpathqmoveto{137.1429bp}{97.1429bp}
    \pgfpathqcurveto{137.1429bp}{100.2988bp}{134.5845bp}{102.8571bp}{131.4286bp}{102.8571bp}
    \pgfpathqcurveto{128.2727bp}{102.8571bp}{125.7143bp}{100.2988bp}{125.7143bp}{97.1429bp}
    \pgfpathqcurveto{125.7143bp}{93.9869bp}{128.2727bp}{91.4286bp}{131.4286bp}{91.4286bp}
    \pgfpathqcurveto{134.5845bp}{91.4286bp}{137.1429bp}{93.9869bp}{137.1429bp}{97.1429bp}
    \pgfpathclose
    \pgfusepathqfillstroke
  \end{pgfscope}
  \begin{pgfscope}
    \pgfsetlinewidth{0.8113bp}
    \definecolor{sc}{rgb}{0.0000,0.0000,0.0000}
    \pgfsetstrokecolor{sc}
    \pgfsetmiterjoin
    \pgfsetbuttcap
    \pgfpathqmoveto{136.5407bp}{94.5868bp}
    \pgfpathqlineto{172.0307bp}{76.8418bp}
    \pgfusepathqstroke
  \end{pgfscope}
  \begin{pgfscope}
    \definecolor{fc}{rgb}{0.0000,0.0000,0.0000}
    \pgfsetfillcolor{fc}
    \pgfusepathqfill
  \end{pgfscope}
  \begin{pgfscope}
    \definecolor{fc}{rgb}{0.0000,0.0000,0.0000}
    \pgfsetfillcolor{fc}
    \pgfusepathqfill
  \end{pgfscope}
  \begin{pgfscope}
    \definecolor{fc}{rgb}{0.0000,0.0000,0.0000}
    \pgfsetfillcolor{fc}
    \pgfusepathqfill
  \end{pgfscope}
  \begin{pgfscope}
    \definecolor{fc}{rgb}{0.0000,0.0000,0.0000}
    \pgfsetfillcolor{fc}
    \pgfusepathqfill
  \end{pgfscope}
  \begin{pgfscope}
    \pgfsetlinewidth{0.8113bp}
    \definecolor{sc}{rgb}{0.0000,0.0000,0.0000}
    \pgfsetstrokecolor{sc}
    \pgfsetmiterjoin
    \pgfsetbuttcap
    \pgfpathqmoveto{135.4692bp}{93.1022bp}
    \pgfpathqlineto{150.2451bp}{78.3263bp}
    \pgfusepathqstroke
  \end{pgfscope}
  \begin{pgfscope}
    \definecolor{fc}{rgb}{0.0000,0.0000,0.0000}
    \pgfsetfillcolor{fc}
    \pgfusepathqfill
  \end{pgfscope}
  \begin{pgfscope}
    \definecolor{fc}{rgb}{0.0000,0.0000,0.0000}
    \pgfsetfillcolor{fc}
    \pgfusepathqfill
  \end{pgfscope}
  \begin{pgfscope}
    \definecolor{fc}{rgb}{0.0000,0.0000,0.0000}
    \pgfsetfillcolor{fc}
    \pgfusepathqfill
  \end{pgfscope}
  \begin{pgfscope}
    \definecolor{fc}{rgb}{0.0000,0.0000,0.0000}
    \pgfsetfillcolor{fc}
    \pgfusepathqfill
  \end{pgfscope}
  \begin{pgfscope}
    \pgfsetlinewidth{0.8113bp}
    \definecolor{sc}{rgb}{0.0000,0.0000,0.0000}
    \pgfsetstrokecolor{sc}
    \pgfsetmiterjoin
    \pgfsetbuttcap
    \pgfpathqmoveto{131.4286bp}{91.4286bp}
    \pgfpathqlineto{131.4286bp}{80.0000bp}
    \pgfusepathqstroke
  \end{pgfscope}
  \begin{pgfscope}
    \definecolor{fc}{rgb}{0.0000,0.0000,0.0000}
    \pgfsetfillcolor{fc}
    \pgfusepathqfill
  \end{pgfscope}
  \begin{pgfscope}
    \definecolor{fc}{rgb}{0.0000,0.0000,0.0000}
    \pgfsetfillcolor{fc}
    \pgfusepathqfill
  \end{pgfscope}
  \begin{pgfscope}
    \definecolor{fc}{rgb}{0.0000,0.0000,0.0000}
    \pgfsetfillcolor{fc}
    \pgfusepathqfill
  \end{pgfscope}
  \begin{pgfscope}
    \definecolor{fc}{rgb}{0.0000,0.0000,0.0000}
    \pgfsetfillcolor{fc}
    \pgfusepathqfill
  \end{pgfscope}
  \begin{pgfscope}
    \definecolor{fc}{rgb}{0.0000,0.0000,0.0000}
    \pgfsetfillcolor{fc}
    \pgfsetfillopacity{0.0000}
    \pgfsetlinewidth{0.8113bp}
    \definecolor{sc}{rgb}{0.0000,0.0000,0.0000}
    \pgfsetstrokecolor{sc}
    \pgfsetmiterjoin
    \pgfsetbuttcap
    \pgfpathqmoveto{102.8571bp}{51.4286bp}
    \pgfpathqcurveto{102.8571bp}{54.5845bp}{100.2988bp}{57.1429bp}{97.1429bp}{57.1429bp}
    \pgfpathqcurveto{93.9869bp}{57.1429bp}{91.4286bp}{54.5845bp}{91.4286bp}{51.4286bp}
    \pgfpathqcurveto{91.4286bp}{48.2727bp}{93.9869bp}{45.7143bp}{97.1429bp}{45.7143bp}
    \pgfpathqcurveto{100.2988bp}{45.7143bp}{102.8571bp}{48.2727bp}{102.8571bp}{51.4286bp}
    \pgfpathclose
    \pgfusepathqfillstroke
  \end{pgfscope}
  \begin{pgfscope}
    \definecolor{fc}{rgb}{0.0000,0.0000,0.0000}
    \pgfsetfillcolor{fc}
    \pgfsetfillopacity{0.0000}
    \pgfsetlinewidth{0.8113bp}
    \definecolor{sc}{rgb}{0.0000,0.0000,0.0000}
    \pgfsetstrokecolor{sc}
    \pgfsetmiterjoin
    \pgfsetbuttcap
    \pgfpathqmoveto{102.8571bp}{74.2857bp}
    \pgfpathqcurveto{102.8571bp}{77.4416bp}{100.2988bp}{80.0000bp}{97.1429bp}{80.0000bp}
    \pgfpathqcurveto{93.9869bp}{80.0000bp}{91.4286bp}{77.4416bp}{91.4286bp}{74.2857bp}
    \pgfpathqcurveto{91.4286bp}{71.1298bp}{93.9869bp}{68.5714bp}{97.1429bp}{68.5714bp}
    \pgfpathqcurveto{100.2988bp}{68.5714bp}{102.8571bp}{71.1298bp}{102.8571bp}{74.2857bp}
    \pgfpathclose
    \pgfusepathqfillstroke
  \end{pgfscope}
  \begin{pgfscope}
    \pgfsetlinewidth{0.8113bp}
    \definecolor{sc}{rgb}{0.0000,0.0000,0.0000}
    \pgfsetstrokecolor{sc}
    \pgfsetmiterjoin
    \pgfsetbuttcap
    \pgfpathqmoveto{97.1429bp}{68.5714bp}
    \pgfpathqlineto{97.1429bp}{57.1429bp}
    \pgfusepathqstroke
  \end{pgfscope}
  \begin{pgfscope}
    \definecolor{fc}{rgb}{0.0000,0.0000,0.0000}
    \pgfsetfillcolor{fc}
    \pgfusepathqfill
  \end{pgfscope}
  \begin{pgfscope}
    \definecolor{fc}{rgb}{0.0000,0.0000,0.0000}
    \pgfsetfillcolor{fc}
    \pgfusepathqfill
  \end{pgfscope}
  \begin{pgfscope}
    \definecolor{fc}{rgb}{0.0000,0.0000,0.0000}
    \pgfsetfillcolor{fc}
    \pgfusepathqfill
  \end{pgfscope}
  \begin{pgfscope}
    \definecolor{fc}{rgb}{0.0000,0.0000,0.0000}
    \pgfsetfillcolor{fc}
    \pgfusepathqfill
  \end{pgfscope}
  \begin{pgfscope}
    \definecolor{fc}{rgb}{0.0000,0.0000,0.0000}
    \pgfsetfillcolor{fc}
    \pgfsetfillopacity{0.0000}
    \pgfsetlinewidth{0.8113bp}
    \definecolor{sc}{rgb}{0.0000,0.0000,0.0000}
    \pgfsetstrokecolor{sc}
    \pgfsetmiterjoin
    \pgfsetbuttcap
    \pgfpathqmoveto{80.0000bp}{74.2857bp}
    \pgfpathqcurveto{80.0000bp}{77.4416bp}{77.4416bp}{80.0000bp}{74.2857bp}{80.0000bp}
    \pgfpathqcurveto{71.1298bp}{80.0000bp}{68.5714bp}{77.4416bp}{68.5714bp}{74.2857bp}
    \pgfpathqcurveto{68.5714bp}{71.1298bp}{71.1298bp}{68.5714bp}{74.2857bp}{68.5714bp}
    \pgfpathqcurveto{77.4416bp}{68.5714bp}{80.0000bp}{71.1298bp}{80.0000bp}{74.2857bp}
    \pgfpathclose
    \pgfusepathqfillstroke
  \end{pgfscope}
  \begin{pgfscope}
    \definecolor{fc}{rgb}{0.0000,0.0000,0.0000}
    \pgfsetfillcolor{fc}
    \pgfsetfillopacity{0.0000}
    \pgfsetlinewidth{0.8113bp}
    \definecolor{sc}{rgb}{0.0000,0.0000,0.0000}
    \pgfsetstrokecolor{sc}
    \pgfsetmiterjoin
    \pgfsetbuttcap
    \pgfpathqmoveto{80.0000bp}{97.1429bp}
    \pgfpathqcurveto{80.0000bp}{100.2988bp}{77.4416bp}{102.8571bp}{74.2857bp}{102.8571bp}
    \pgfpathqcurveto{71.1298bp}{102.8571bp}{68.5714bp}{100.2988bp}{68.5714bp}{97.1429bp}
    \pgfpathqcurveto{68.5714bp}{93.9869bp}{71.1298bp}{91.4286bp}{74.2857bp}{91.4286bp}
    \pgfpathqcurveto{77.4416bp}{91.4286bp}{80.0000bp}{93.9869bp}{80.0000bp}{97.1429bp}
    \pgfpathclose
    \pgfusepathqfillstroke
  \end{pgfscope}
  \begin{pgfscope}
    \pgfsetlinewidth{0.8113bp}
    \definecolor{sc}{rgb}{0.0000,0.0000,0.0000}
    \pgfsetstrokecolor{sc}
    \pgfsetmiterjoin
    \pgfsetbuttcap
    \pgfpathqmoveto{78.3263bp}{93.1022bp}
    \pgfpathqlineto{93.1022bp}{78.3263bp}
    \pgfusepathqstroke
  \end{pgfscope}
  \begin{pgfscope}
    \definecolor{fc}{rgb}{0.0000,0.0000,0.0000}
    \pgfsetfillcolor{fc}
    \pgfusepathqfill
  \end{pgfscope}
  \begin{pgfscope}
    \definecolor{fc}{rgb}{0.0000,0.0000,0.0000}
    \pgfsetfillcolor{fc}
    \pgfusepathqfill
  \end{pgfscope}
  \begin{pgfscope}
    \definecolor{fc}{rgb}{0.0000,0.0000,0.0000}
    \pgfsetfillcolor{fc}
    \pgfusepathqfill
  \end{pgfscope}
  \begin{pgfscope}
    \definecolor{fc}{rgb}{0.0000,0.0000,0.0000}
    \pgfsetfillcolor{fc}
    \pgfusepathqfill
  \end{pgfscope}
  \begin{pgfscope}
    \pgfsetlinewidth{0.8113bp}
    \definecolor{sc}{rgb}{0.0000,0.0000,0.0000}
    \pgfsetstrokecolor{sc}
    \pgfsetmiterjoin
    \pgfsetbuttcap
    \pgfpathqmoveto{74.2857bp}{91.4286bp}
    \pgfpathqlineto{74.2857bp}{80.0000bp}
    \pgfusepathqstroke
  \end{pgfscope}
  \begin{pgfscope}
    \definecolor{fc}{rgb}{0.0000,0.0000,0.0000}
    \pgfsetfillcolor{fc}
    \pgfusepathqfill
  \end{pgfscope}
  \begin{pgfscope}
    \definecolor{fc}{rgb}{0.0000,0.0000,0.0000}
    \pgfsetfillcolor{fc}
    \pgfusepathqfill
  \end{pgfscope}
  \begin{pgfscope}
    \definecolor{fc}{rgb}{0.0000,0.0000,0.0000}
    \pgfsetfillcolor{fc}
    \pgfusepathqfill
  \end{pgfscope}
  \begin{pgfscope}
    \definecolor{fc}{rgb}{0.0000,0.0000,0.0000}
    \pgfsetfillcolor{fc}
    \pgfusepathqfill
  \end{pgfscope}
  \begin{pgfscope}
    \definecolor{fc}{rgb}{0.0000,0.0000,0.0000}
    \pgfsetfillcolor{fc}
    \pgfsetfillopacity{0.0000}
    \pgfsetlinewidth{0.8113bp}
    \definecolor{sc}{rgb}{0.0000,0.0000,0.0000}
    \pgfsetstrokecolor{sc}
    \pgfsetmiterjoin
    \pgfsetbuttcap
    \pgfpathqmoveto{45.7143bp}{74.2857bp}
    \pgfpathqcurveto{45.7143bp}{77.4416bp}{43.1559bp}{80.0000bp}{40.0000bp}{80.0000bp}
    \pgfpathqcurveto{36.8441bp}{80.0000bp}{34.2857bp}{77.4416bp}{34.2857bp}{74.2857bp}
    \pgfpathqcurveto{34.2857bp}{71.1298bp}{36.8441bp}{68.5714bp}{40.0000bp}{68.5714bp}
    \pgfpathqcurveto{43.1559bp}{68.5714bp}{45.7143bp}{71.1298bp}{45.7143bp}{74.2857bp}
    \pgfpathclose
    \pgfusepathqfillstroke
  \end{pgfscope}
  \begin{pgfscope}
    \definecolor{fc}{rgb}{0.0000,0.0000,0.0000}
    \pgfsetfillcolor{fc}
    \pgfsetfillopacity{0.0000}
    \pgfsetlinewidth{0.8113bp}
    \definecolor{sc}{rgb}{0.0000,0.0000,0.0000}
    \pgfsetstrokecolor{sc}
    \pgfsetmiterjoin
    \pgfsetbuttcap
    \pgfpathqmoveto{45.7143bp}{97.1429bp}
    \pgfpathqcurveto{45.7143bp}{100.2988bp}{43.1559bp}{102.8571bp}{40.0000bp}{102.8571bp}
    \pgfpathqcurveto{36.8441bp}{102.8571bp}{34.2857bp}{100.2988bp}{34.2857bp}{97.1429bp}
    \pgfpathqcurveto{34.2857bp}{93.9869bp}{36.8441bp}{91.4286bp}{40.0000bp}{91.4286bp}
    \pgfpathqcurveto{43.1559bp}{91.4286bp}{45.7143bp}{93.9869bp}{45.7143bp}{97.1429bp}
    \pgfpathclose
    \pgfusepathqfillstroke
  \end{pgfscope}
  \begin{pgfscope}
    \pgfsetlinewidth{0.8113bp}
    \definecolor{sc}{rgb}{0.0000,0.0000,0.0000}
    \pgfsetstrokecolor{sc}
    \pgfsetmiterjoin
    \pgfsetbuttcap
    \pgfpathqmoveto{40.0000bp}{91.4286bp}
    \pgfpathqlineto{40.0000bp}{80.0000bp}
    \pgfusepathqstroke
  \end{pgfscope}
  \begin{pgfscope}
    \definecolor{fc}{rgb}{0.0000,0.0000,0.0000}
    \pgfsetfillcolor{fc}
    \pgfusepathqfill
  \end{pgfscope}
  \begin{pgfscope}
    \definecolor{fc}{rgb}{0.0000,0.0000,0.0000}
    \pgfsetfillcolor{fc}
    \pgfusepathqfill
  \end{pgfscope}
  \begin{pgfscope}
    \definecolor{fc}{rgb}{0.0000,0.0000,0.0000}
    \pgfsetfillcolor{fc}
    \pgfusepathqfill
  \end{pgfscope}
  \begin{pgfscope}
    \definecolor{fc}{rgb}{0.0000,0.0000,0.0000}
    \pgfsetfillcolor{fc}
    \pgfusepathqfill
  \end{pgfscope}
  \begin{pgfscope}
    \definecolor{fc}{rgb}{0.0000,0.0000,0.0000}
    \pgfsetfillcolor{fc}
    \pgfsetfillopacity{0.0000}
    \pgfsetlinewidth{0.8113bp}
    \definecolor{sc}{rgb}{0.0000,0.0000,0.0000}
    \pgfsetstrokecolor{sc}
    \pgfsetmiterjoin
    \pgfsetbuttcap
    \pgfpathqmoveto{11.4286bp}{97.1429bp}
    \pgfpathqcurveto{11.4286bp}{100.2988bp}{8.8702bp}{102.8571bp}{5.7143bp}{102.8571bp}
    \pgfpathqcurveto{2.5584bp}{102.8571bp}{0.0000bp}{100.2988bp}{0.0000bp}{97.1429bp}
    \pgfpathqcurveto{0.0000bp}{93.9869bp}{2.5584bp}{91.4286bp}{5.7143bp}{91.4286bp}
    \pgfpathqcurveto{8.8702bp}{91.4286bp}{11.4286bp}{93.9869bp}{11.4286bp}{97.1429bp}
    \pgfpathclose
    \pgfusepathqfillstroke
  \end{pgfscope}
\end{pgfpicture}

  \end{center}
  \label{binomial-trees}
\end{model*}

\begin{questions}
\item What patterns do you notice?  Write down at least three
  observations.

\item Send your reporter to some other groups to share your
  observations and collect as many additional observations as possible.

\item If it was not already among your observations from the previous
  question, explain how we can make a binomial tree of order $n$ from
  two binomial trees of order $n - 1$. \label{binomial-merge}

\item Assuming a binomial tree is stored as an object containing a
  root value and a list of children (which are themselves binomial
  tree objects), how long does this operation (making two
  order-$(n-1)$ trees into an order-$n$ tree) take?

\item How many total nodes does a binomial tree of order $n$ have?
  Why?
\end{questions}

\begin{model}{Binomial Heaps}{binomial-heap}
  \begin{defn}
    A \emph{binomial heap} is a list of binomial trees such that:
    \begin{itemize}
    \item Each binomial tree satisfies the heap property, i.e., each node's value is less than or equal to the values of all its children.
    \item There is at most one binomial tree of any given order.
    \end{itemize}
    Below are three potential binomial heaps (one per row).  However,
    only one of them is a valid binomial heap according to the
    definition above; the other two are invalid.
  \end{defn}
  \begin{center}
    \begin{pgfpicture}
  \pgfpathrectangle{\pgfpointorigin}{\pgfqpoint{300.0000bp}{300.0000bp}}
  \pgfusepath{use as bounding box}
  \begin{pgfscope}
    \definecolor{fc}{rgb}{0.0000,0.0000,0.0000}
    \pgfsetfillcolor{fc}
    \pgfsetfillopacity{0.0000}
    \pgfsetlinewidth{1.2000bp}
    \definecolor{sc}{rgb}{0.0000,0.0000,0.0000}
    \pgfsetstrokecolor{sc}
    \pgfsetmiterjoin
    \pgfsetbuttcap
    \pgfpathqmoveto{258.0000bp}{78.0000bp}
    \pgfpathqcurveto{258.0000bp}{81.3137bp}{255.3137bp}{84.0000bp}{252.0000bp}{84.0000bp}
    \pgfpathqcurveto{248.6863bp}{84.0000bp}{246.0000bp}{81.3137bp}{246.0000bp}{78.0000bp}
    \pgfpathqcurveto{246.0000bp}{74.6863bp}{248.6863bp}{72.0000bp}{252.0000bp}{72.0000bp}
    \pgfpathqcurveto{255.3137bp}{72.0000bp}{258.0000bp}{74.6863bp}{258.0000bp}{78.0000bp}
    \pgfpathclose
    \pgfusepathqfillstroke
  \end{pgfscope}
  \begin{pgfscope}
    \definecolor{fc}{rgb}{0.0000,0.0000,0.0000}
    \pgfsetfillcolor{fc}
    \pgftransformshift{\pgfqpoint{252.0000bp}{78.0000bp}}
    \pgftransformscale{0.5250}
    \pgftext[]{$8$}
  \end{pgfscope}
  \begin{pgfscope}
    \definecolor{fc}{rgb}{0.0000,0.0000,0.0000}
    \pgfsetfillcolor{fc}
    \pgfsetfillopacity{0.0000}
    \pgfsetlinewidth{1.2000bp}
    \definecolor{sc}{rgb}{0.0000,0.0000,0.0000}
    \pgfsetstrokecolor{sc}
    \pgfsetmiterjoin
    \pgfsetbuttcap
    \pgfpathqmoveto{222.0000bp}{30.0000bp}
    \pgfpathqcurveto{222.0000bp}{33.3137bp}{219.3137bp}{36.0000bp}{216.0000bp}{36.0000bp}
    \pgfpathqcurveto{212.6863bp}{36.0000bp}{210.0000bp}{33.3137bp}{210.0000bp}{30.0000bp}
    \pgfpathqcurveto{210.0000bp}{26.6863bp}{212.6863bp}{24.0000bp}{216.0000bp}{24.0000bp}
    \pgfpathqcurveto{219.3137bp}{24.0000bp}{222.0000bp}{26.6863bp}{222.0000bp}{30.0000bp}
    \pgfpathclose
    \pgfusepathqfillstroke
  \end{pgfscope}
  \begin{pgfscope}
    \definecolor{fc}{rgb}{0.0000,0.0000,0.0000}
    \pgfsetfillcolor{fc}
    \pgftransformshift{\pgfqpoint{216.0000bp}{30.0000bp}}
    \pgftransformscale{0.5250}
    \pgftext[]{$99$}
  \end{pgfscope}
  \begin{pgfscope}
    \definecolor{fc}{rgb}{0.0000,0.0000,0.0000}
    \pgfsetfillcolor{fc}
    \pgfsetfillopacity{0.0000}
    \pgfsetlinewidth{1.2000bp}
    \definecolor{sc}{rgb}{0.0000,0.0000,0.0000}
    \pgfsetstrokecolor{sc}
    \pgfsetmiterjoin
    \pgfsetbuttcap
    \pgfpathqmoveto{222.0000bp}{54.0000bp}
    \pgfpathqcurveto{222.0000bp}{57.3137bp}{219.3137bp}{60.0000bp}{216.0000bp}{60.0000bp}
    \pgfpathqcurveto{212.6863bp}{60.0000bp}{210.0000bp}{57.3137bp}{210.0000bp}{54.0000bp}
    \pgfpathqcurveto{210.0000bp}{50.6863bp}{212.6863bp}{48.0000bp}{216.0000bp}{48.0000bp}
    \pgfpathqcurveto{219.3137bp}{48.0000bp}{222.0000bp}{50.6863bp}{222.0000bp}{54.0000bp}
    \pgfpathclose
    \pgfusepathqfillstroke
  \end{pgfscope}
  \begin{pgfscope}
    \definecolor{fc}{rgb}{0.0000,0.0000,0.0000}
    \pgfsetfillcolor{fc}
    \pgftransformshift{\pgfqpoint{216.0000bp}{54.0000bp}}
    \pgftransformscale{0.5250}
    \pgftext[]{$17$}
  \end{pgfscope}
  \begin{pgfscope}
    \pgfsetlinewidth{1.2000bp}
    \definecolor{sc}{rgb}{0.0000,0.0000,0.0000}
    \pgfsetstrokecolor{sc}
    \pgfsetmiterjoin
    \pgfsetbuttcap
    \pgfpathqmoveto{216.0000bp}{48.0000bp}
    \pgfpathqlineto{216.0000bp}{36.0000bp}
    \pgfusepathqstroke
  \end{pgfscope}
  \begin{pgfscope}
    \definecolor{fc}{rgb}{0.0000,0.0000,0.0000}
    \pgfsetfillcolor{fc}
    \pgfusepathqfill
  \end{pgfscope}
  \begin{pgfscope}
    \definecolor{fc}{rgb}{0.0000,0.0000,0.0000}
    \pgfsetfillcolor{fc}
    \pgfusepathqfill
  \end{pgfscope}
  \begin{pgfscope}
    \definecolor{fc}{rgb}{0.0000,0.0000,0.0000}
    \pgfsetfillcolor{fc}
    \pgfusepathqfill
  \end{pgfscope}
  \begin{pgfscope}
    \definecolor{fc}{rgb}{0.0000,0.0000,0.0000}
    \pgfsetfillcolor{fc}
    \pgfusepathqfill
  \end{pgfscope}
  \begin{pgfscope}
    \definecolor{fc}{rgb}{0.0000,0.0000,0.0000}
    \pgfsetfillcolor{fc}
    \pgfsetfillopacity{0.0000}
    \pgfsetlinewidth{1.2000bp}
    \definecolor{sc}{rgb}{0.0000,0.0000,0.0000}
    \pgfsetstrokecolor{sc}
    \pgfsetmiterjoin
    \pgfsetbuttcap
    \pgfpathqmoveto{198.0000bp}{54.0000bp}
    \pgfpathqcurveto{198.0000bp}{57.3137bp}{195.3137bp}{60.0000bp}{192.0000bp}{60.0000bp}
    \pgfpathqcurveto{188.6863bp}{60.0000bp}{186.0000bp}{57.3137bp}{186.0000bp}{54.0000bp}
    \pgfpathqcurveto{186.0000bp}{50.6863bp}{188.6863bp}{48.0000bp}{192.0000bp}{48.0000bp}
    \pgfpathqcurveto{195.3137bp}{48.0000bp}{198.0000bp}{50.6863bp}{198.0000bp}{54.0000bp}
    \pgfpathclose
    \pgfusepathqfillstroke
  \end{pgfscope}
  \begin{pgfscope}
    \definecolor{fc}{rgb}{0.0000,0.0000,0.0000}
    \pgfsetfillcolor{fc}
    \pgftransformshift{\pgfqpoint{192.0000bp}{54.0000bp}}
    \pgftransformscale{0.5250}
    \pgftext[]{$21$}
  \end{pgfscope}
  \begin{pgfscope}
    \definecolor{fc}{rgb}{0.0000,0.0000,0.0000}
    \pgfsetfillcolor{fc}
    \pgfsetfillopacity{0.0000}
    \pgfsetlinewidth{1.2000bp}
    \definecolor{sc}{rgb}{0.0000,0.0000,0.0000}
    \pgfsetstrokecolor{sc}
    \pgfsetmiterjoin
    \pgfsetbuttcap
    \pgfpathqmoveto{198.0000bp}{78.0000bp}
    \pgfpathqcurveto{198.0000bp}{81.3137bp}{195.3137bp}{84.0000bp}{192.0000bp}{84.0000bp}
    \pgfpathqcurveto{188.6863bp}{84.0000bp}{186.0000bp}{81.3137bp}{186.0000bp}{78.0000bp}
    \pgfpathqcurveto{186.0000bp}{74.6863bp}{188.6863bp}{72.0000bp}{192.0000bp}{72.0000bp}
    \pgfpathqcurveto{195.3137bp}{72.0000bp}{198.0000bp}{74.6863bp}{198.0000bp}{78.0000bp}
    \pgfpathclose
    \pgfusepathqfillstroke
  \end{pgfscope}
  \begin{pgfscope}
    \definecolor{fc}{rgb}{0.0000,0.0000,0.0000}
    \pgfsetfillcolor{fc}
    \pgftransformshift{\pgfqpoint{192.0000bp}{78.0000bp}}
    \pgftransformscale{0.5250}
    \pgftext[]{$5$}
  \end{pgfscope}
  \begin{pgfscope}
    \pgfsetlinewidth{1.2000bp}
    \definecolor{sc}{rgb}{0.0000,0.0000,0.0000}
    \pgfsetstrokecolor{sc}
    \pgfsetmiterjoin
    \pgfsetbuttcap
    \pgfpathqmoveto{196.2426bp}{73.7574bp}
    \pgfpathqlineto{211.7574bp}{58.2426bp}
    \pgfusepathqstroke
  \end{pgfscope}
  \begin{pgfscope}
    \definecolor{fc}{rgb}{0.0000,0.0000,0.0000}
    \pgfsetfillcolor{fc}
    \pgfusepathqfill
  \end{pgfscope}
  \begin{pgfscope}
    \definecolor{fc}{rgb}{0.0000,0.0000,0.0000}
    \pgfsetfillcolor{fc}
    \pgfusepathqfill
  \end{pgfscope}
  \begin{pgfscope}
    \definecolor{fc}{rgb}{0.0000,0.0000,0.0000}
    \pgfsetfillcolor{fc}
    \pgfusepathqfill
  \end{pgfscope}
  \begin{pgfscope}
    \definecolor{fc}{rgb}{0.0000,0.0000,0.0000}
    \pgfsetfillcolor{fc}
    \pgfusepathqfill
  \end{pgfscope}
  \begin{pgfscope}
    \pgfsetlinewidth{1.2000bp}
    \definecolor{sc}{rgb}{0.0000,0.0000,0.0000}
    \pgfsetstrokecolor{sc}
    \pgfsetmiterjoin
    \pgfsetbuttcap
    \pgfpathqmoveto{192.0000bp}{72.0000bp}
    \pgfpathqlineto{192.0000bp}{60.0000bp}
    \pgfusepathqstroke
  \end{pgfscope}
  \begin{pgfscope}
    \definecolor{fc}{rgb}{0.0000,0.0000,0.0000}
    \pgfsetfillcolor{fc}
    \pgfusepathqfill
  \end{pgfscope}
  \begin{pgfscope}
    \definecolor{fc}{rgb}{0.0000,0.0000,0.0000}
    \pgfsetfillcolor{fc}
    \pgfusepathqfill
  \end{pgfscope}
  \begin{pgfscope}
    \definecolor{fc}{rgb}{0.0000,0.0000,0.0000}
    \pgfsetfillcolor{fc}
    \pgfusepathqfill
  \end{pgfscope}
  \begin{pgfscope}
    \definecolor{fc}{rgb}{0.0000,0.0000,0.0000}
    \pgfsetfillcolor{fc}
    \pgfusepathqfill
  \end{pgfscope}
  \begin{pgfscope}
    \definecolor{fc}{rgb}{0.0000,0.0000,0.0000}
    \pgfsetfillcolor{fc}
    \pgfsetfillopacity{0.0000}
    \pgfsetlinewidth{1.2000bp}
    \definecolor{sc}{rgb}{0.0000,0.0000,0.0000}
    \pgfsetstrokecolor{sc}
    \pgfsetmiterjoin
    \pgfsetbuttcap
    \pgfpathqmoveto{162.0000bp}{78.0000bp}
    \pgfpathqcurveto{162.0000bp}{81.3137bp}{159.3137bp}{84.0000bp}{156.0000bp}{84.0000bp}
    \pgfpathqcurveto{152.6863bp}{84.0000bp}{150.0000bp}{81.3137bp}{150.0000bp}{78.0000bp}
    \pgfpathqcurveto{150.0000bp}{74.6863bp}{152.6863bp}{72.0000bp}{156.0000bp}{72.0000bp}
    \pgfpathqcurveto{159.3137bp}{72.0000bp}{162.0000bp}{74.6863bp}{162.0000bp}{78.0000bp}
    \pgfpathclose
    \pgfusepathqfillstroke
  \end{pgfscope}
  \begin{pgfscope}
    \definecolor{fc}{rgb}{0.0000,0.0000,0.0000}
    \pgfsetfillcolor{fc}
    \pgftransformshift{\pgfqpoint{156.0000bp}{78.0000bp}}
    \pgftransformscale{0.5250}
    \pgftext[]{$4$}
  \end{pgfscope}
  \begin{pgfscope}
    \definecolor{fc}{rgb}{0.0000,0.0000,0.0000}
    \pgfsetfillcolor{fc}
    \pgfsetfillopacity{0.0000}
    \pgfsetlinewidth{1.2000bp}
    \definecolor{sc}{rgb}{0.0000,0.0000,0.0000}
    \pgfsetstrokecolor{sc}
    \pgfsetmiterjoin
    \pgfsetbuttcap
    \pgfpathqmoveto{126.0000bp}{6.0000bp}
    \pgfpathqcurveto{126.0000bp}{9.3137bp}{123.3137bp}{12.0000bp}{120.0000bp}{12.0000bp}
    \pgfpathqcurveto{116.6863bp}{12.0000bp}{114.0000bp}{9.3137bp}{114.0000bp}{6.0000bp}
    \pgfpathqcurveto{114.0000bp}{2.6863bp}{116.6863bp}{0.0000bp}{120.0000bp}{0.0000bp}
    \pgfpathqcurveto{123.3137bp}{0.0000bp}{126.0000bp}{2.6863bp}{126.0000bp}{6.0000bp}
    \pgfpathclose
    \pgfusepathqfillstroke
  \end{pgfscope}
  \begin{pgfscope}
    \definecolor{fc}{rgb}{0.0000,0.0000,0.0000}
    \pgfsetfillcolor{fc}
    \pgftransformshift{\pgfqpoint{120.0000bp}{6.0000bp}}
    \pgftransformscale{0.5250}
    \pgftext[]{$53$}
  \end{pgfscope}
  \begin{pgfscope}
    \definecolor{fc}{rgb}{0.0000,0.0000,0.0000}
    \pgfsetfillcolor{fc}
    \pgfsetfillopacity{0.0000}
    \pgfsetlinewidth{1.2000bp}
    \definecolor{sc}{rgb}{0.0000,0.0000,0.0000}
    \pgfsetstrokecolor{sc}
    \pgfsetmiterjoin
    \pgfsetbuttcap
    \pgfpathqmoveto{126.0000bp}{30.0000bp}
    \pgfpathqcurveto{126.0000bp}{33.3137bp}{123.3137bp}{36.0000bp}{120.0000bp}{36.0000bp}
    \pgfpathqcurveto{116.6863bp}{36.0000bp}{114.0000bp}{33.3137bp}{114.0000bp}{30.0000bp}
    \pgfpathqcurveto{114.0000bp}{26.6863bp}{116.6863bp}{24.0000bp}{120.0000bp}{24.0000bp}
    \pgfpathqcurveto{123.3137bp}{24.0000bp}{126.0000bp}{26.6863bp}{126.0000bp}{30.0000bp}
    \pgfpathclose
    \pgfusepathqfillstroke
  \end{pgfscope}
  \begin{pgfscope}
    \definecolor{fc}{rgb}{0.0000,0.0000,0.0000}
    \pgfsetfillcolor{fc}
    \pgftransformshift{\pgfqpoint{120.0000bp}{30.0000bp}}
    \pgftransformscale{0.5250}
    \pgftext[]{$33$}
  \end{pgfscope}
  \begin{pgfscope}
    \pgfsetlinewidth{1.2000bp}
    \definecolor{sc}{rgb}{0.0000,0.0000,0.0000}
    \pgfsetstrokecolor{sc}
    \pgfsetmiterjoin
    \pgfsetbuttcap
    \pgfpathqmoveto{120.0000bp}{24.0000bp}
    \pgfpathqlineto{120.0000bp}{12.0000bp}
    \pgfusepathqstroke
  \end{pgfscope}
  \begin{pgfscope}
    \definecolor{fc}{rgb}{0.0000,0.0000,0.0000}
    \pgfsetfillcolor{fc}
    \pgfusepathqfill
  \end{pgfscope}
  \begin{pgfscope}
    \definecolor{fc}{rgb}{0.0000,0.0000,0.0000}
    \pgfsetfillcolor{fc}
    \pgfusepathqfill
  \end{pgfscope}
  \begin{pgfscope}
    \definecolor{fc}{rgb}{0.0000,0.0000,0.0000}
    \pgfsetfillcolor{fc}
    \pgfusepathqfill
  \end{pgfscope}
  \begin{pgfscope}
    \definecolor{fc}{rgb}{0.0000,0.0000,0.0000}
    \pgfsetfillcolor{fc}
    \pgfusepathqfill
  \end{pgfscope}
  \begin{pgfscope}
    \definecolor{fc}{rgb}{0.0000,0.0000,0.0000}
    \pgfsetfillcolor{fc}
    \pgfsetfillopacity{0.0000}
    \pgfsetlinewidth{1.2000bp}
    \definecolor{sc}{rgb}{0.0000,0.0000,0.0000}
    \pgfsetstrokecolor{sc}
    \pgfsetmiterjoin
    \pgfsetbuttcap
    \pgfpathqmoveto{102.0000bp}{30.0000bp}
    \pgfpathqcurveto{102.0000bp}{33.3137bp}{99.3137bp}{36.0000bp}{96.0000bp}{36.0000bp}
    \pgfpathqcurveto{92.6863bp}{36.0000bp}{90.0000bp}{33.3137bp}{90.0000bp}{30.0000bp}
    \pgfpathqcurveto{90.0000bp}{26.6863bp}{92.6863bp}{24.0000bp}{96.0000bp}{24.0000bp}
    \pgfpathqcurveto{99.3137bp}{24.0000bp}{102.0000bp}{26.6863bp}{102.0000bp}{30.0000bp}
    \pgfpathclose
    \pgfusepathqfillstroke
  \end{pgfscope}
  \begin{pgfscope}
    \definecolor{fc}{rgb}{0.0000,0.0000,0.0000}
    \pgfsetfillcolor{fc}
    \pgftransformshift{\pgfqpoint{96.0000bp}{30.0000bp}}
    \pgftransformscale{0.5250}
    \pgftext[]{$24$}
  \end{pgfscope}
  \begin{pgfscope}
    \definecolor{fc}{rgb}{0.0000,0.0000,0.0000}
    \pgfsetfillcolor{fc}
    \pgfsetfillopacity{0.0000}
    \pgfsetlinewidth{1.2000bp}
    \definecolor{sc}{rgb}{0.0000,0.0000,0.0000}
    \pgfsetstrokecolor{sc}
    \pgfsetmiterjoin
    \pgfsetbuttcap
    \pgfpathqmoveto{102.0000bp}{54.0000bp}
    \pgfpathqcurveto{102.0000bp}{57.3137bp}{99.3137bp}{60.0000bp}{96.0000bp}{60.0000bp}
    \pgfpathqcurveto{92.6863bp}{60.0000bp}{90.0000bp}{57.3137bp}{90.0000bp}{54.0000bp}
    \pgfpathqcurveto{90.0000bp}{50.6863bp}{92.6863bp}{48.0000bp}{96.0000bp}{48.0000bp}
    \pgfpathqcurveto{99.3137bp}{48.0000bp}{102.0000bp}{50.6863bp}{102.0000bp}{54.0000bp}
    \pgfpathclose
    \pgfusepathqfillstroke
  \end{pgfscope}
  \begin{pgfscope}
    \definecolor{fc}{rgb}{0.0000,0.0000,0.0000}
    \pgfsetfillcolor{fc}
    \pgftransformshift{\pgfqpoint{96.0000bp}{54.0000bp}}
    \pgftransformscale{0.5250}
    \pgftext[]{$23$}
  \end{pgfscope}
  \begin{pgfscope}
    \pgfsetlinewidth{1.2000bp}
    \definecolor{sc}{rgb}{0.0000,0.0000,0.0000}
    \pgfsetstrokecolor{sc}
    \pgfsetmiterjoin
    \pgfsetbuttcap
    \pgfpathqmoveto{100.2426bp}{49.7574bp}
    \pgfpathqlineto{115.7574bp}{34.2426bp}
    \pgfusepathqstroke
  \end{pgfscope}
  \begin{pgfscope}
    \definecolor{fc}{rgb}{0.0000,0.0000,0.0000}
    \pgfsetfillcolor{fc}
    \pgfusepathqfill
  \end{pgfscope}
  \begin{pgfscope}
    \definecolor{fc}{rgb}{0.0000,0.0000,0.0000}
    \pgfsetfillcolor{fc}
    \pgfusepathqfill
  \end{pgfscope}
  \begin{pgfscope}
    \definecolor{fc}{rgb}{0.0000,0.0000,0.0000}
    \pgfsetfillcolor{fc}
    \pgfusepathqfill
  \end{pgfscope}
  \begin{pgfscope}
    \definecolor{fc}{rgb}{0.0000,0.0000,0.0000}
    \pgfsetfillcolor{fc}
    \pgfusepathqfill
  \end{pgfscope}
  \begin{pgfscope}
    \pgfsetlinewidth{1.2000bp}
    \definecolor{sc}{rgb}{0.0000,0.0000,0.0000}
    \pgfsetstrokecolor{sc}
    \pgfsetmiterjoin
    \pgfsetbuttcap
    \pgfpathqmoveto{96.0000bp}{48.0000bp}
    \pgfpathqlineto{96.0000bp}{36.0000bp}
    \pgfusepathqstroke
  \end{pgfscope}
  \begin{pgfscope}
    \definecolor{fc}{rgb}{0.0000,0.0000,0.0000}
    \pgfsetfillcolor{fc}
    \pgfusepathqfill
  \end{pgfscope}
  \begin{pgfscope}
    \definecolor{fc}{rgb}{0.0000,0.0000,0.0000}
    \pgfsetfillcolor{fc}
    \pgfusepathqfill
  \end{pgfscope}
  \begin{pgfscope}
    \definecolor{fc}{rgb}{0.0000,0.0000,0.0000}
    \pgfsetfillcolor{fc}
    \pgfusepathqfill
  \end{pgfscope}
  \begin{pgfscope}
    \definecolor{fc}{rgb}{0.0000,0.0000,0.0000}
    \pgfsetfillcolor{fc}
    \pgfusepathqfill
  \end{pgfscope}
  \begin{pgfscope}
    \definecolor{fc}{rgb}{0.0000,0.0000,0.0000}
    \pgfsetfillcolor{fc}
    \pgfsetfillopacity{0.0000}
    \pgfsetlinewidth{1.2000bp}
    \definecolor{sc}{rgb}{0.0000,0.0000,0.0000}
    \pgfsetstrokecolor{sc}
    \pgfsetmiterjoin
    \pgfsetbuttcap
    \pgfpathqmoveto{78.0000bp}{30.0000bp}
    \pgfpathqcurveto{78.0000bp}{33.3137bp}{75.3137bp}{36.0000bp}{72.0000bp}{36.0000bp}
    \pgfpathqcurveto{68.6863bp}{36.0000bp}{66.0000bp}{33.3137bp}{66.0000bp}{30.0000bp}
    \pgfpathqcurveto{66.0000bp}{26.6863bp}{68.6863bp}{24.0000bp}{72.0000bp}{24.0000bp}
    \pgfpathqcurveto{75.3137bp}{24.0000bp}{78.0000bp}{26.6863bp}{78.0000bp}{30.0000bp}
    \pgfpathclose
    \pgfusepathqfillstroke
  \end{pgfscope}
  \begin{pgfscope}
    \definecolor{fc}{rgb}{0.0000,0.0000,0.0000}
    \pgfsetfillcolor{fc}
    \pgftransformshift{\pgfqpoint{72.0000bp}{30.0000bp}}
    \pgftransformscale{0.5250}
    \pgftext[]{$28$}
  \end{pgfscope}
  \begin{pgfscope}
    \definecolor{fc}{rgb}{0.0000,0.0000,0.0000}
    \pgfsetfillcolor{fc}
    \pgfsetfillopacity{0.0000}
    \pgfsetlinewidth{1.2000bp}
    \definecolor{sc}{rgb}{0.0000,0.0000,0.0000}
    \pgfsetstrokecolor{sc}
    \pgfsetmiterjoin
    \pgfsetbuttcap
    \pgfpathqmoveto{78.0000bp}{54.0000bp}
    \pgfpathqcurveto{78.0000bp}{57.3137bp}{75.3137bp}{60.0000bp}{72.0000bp}{60.0000bp}
    \pgfpathqcurveto{68.6863bp}{60.0000bp}{66.0000bp}{57.3137bp}{66.0000bp}{54.0000bp}
    \pgfpathqcurveto{66.0000bp}{50.6863bp}{68.6863bp}{48.0000bp}{72.0000bp}{48.0000bp}
    \pgfpathqcurveto{75.3137bp}{48.0000bp}{78.0000bp}{50.6863bp}{78.0000bp}{54.0000bp}
    \pgfpathclose
    \pgfusepathqfillstroke
  \end{pgfscope}
  \begin{pgfscope}
    \definecolor{fc}{rgb}{0.0000,0.0000,0.0000}
    \pgfsetfillcolor{fc}
    \pgftransformshift{\pgfqpoint{72.0000bp}{54.0000bp}}
    \pgftransformscale{0.5250}
    \pgftext[]{$13$}
  \end{pgfscope}
  \begin{pgfscope}
    \pgfsetlinewidth{1.2000bp}
    \definecolor{sc}{rgb}{0.0000,0.0000,0.0000}
    \pgfsetstrokecolor{sc}
    \pgfsetmiterjoin
    \pgfsetbuttcap
    \pgfpathqmoveto{72.0000bp}{48.0000bp}
    \pgfpathqlineto{72.0000bp}{36.0000bp}
    \pgfusepathqstroke
  \end{pgfscope}
  \begin{pgfscope}
    \definecolor{fc}{rgb}{0.0000,0.0000,0.0000}
    \pgfsetfillcolor{fc}
    \pgfusepathqfill
  \end{pgfscope}
  \begin{pgfscope}
    \definecolor{fc}{rgb}{0.0000,0.0000,0.0000}
    \pgfsetfillcolor{fc}
    \pgfusepathqfill
  \end{pgfscope}
  \begin{pgfscope}
    \definecolor{fc}{rgb}{0.0000,0.0000,0.0000}
    \pgfsetfillcolor{fc}
    \pgfusepathqfill
  \end{pgfscope}
  \begin{pgfscope}
    \definecolor{fc}{rgb}{0.0000,0.0000,0.0000}
    \pgfsetfillcolor{fc}
    \pgfusepathqfill
  \end{pgfscope}
  \begin{pgfscope}
    \definecolor{fc}{rgb}{0.0000,0.0000,0.0000}
    \pgfsetfillcolor{fc}
    \pgfsetfillopacity{0.0000}
    \pgfsetlinewidth{1.2000bp}
    \definecolor{sc}{rgb}{0.0000,0.0000,0.0000}
    \pgfsetstrokecolor{sc}
    \pgfsetmiterjoin
    \pgfsetbuttcap
    \pgfpathqmoveto{54.0000bp}{54.0000bp}
    \pgfpathqcurveto{54.0000bp}{57.3137bp}{51.3137bp}{60.0000bp}{48.0000bp}{60.0000bp}
    \pgfpathqcurveto{44.6863bp}{60.0000bp}{42.0000bp}{57.3137bp}{42.0000bp}{54.0000bp}
    \pgfpathqcurveto{42.0000bp}{50.6863bp}{44.6863bp}{48.0000bp}{48.0000bp}{48.0000bp}
    \pgfpathqcurveto{51.3137bp}{48.0000bp}{54.0000bp}{50.6863bp}{54.0000bp}{54.0000bp}
    \pgfpathclose
    \pgfusepathqfillstroke
  \end{pgfscope}
  \begin{pgfscope}
    \definecolor{fc}{rgb}{0.0000,0.0000,0.0000}
    \pgfsetfillcolor{fc}
    \pgftransformshift{\pgfqpoint{48.0000bp}{54.0000bp}}
    \pgftransformscale{0.5250}
    \pgftext[]{$77$}
  \end{pgfscope}
  \begin{pgfscope}
    \definecolor{fc}{rgb}{0.0000,0.0000,0.0000}
    \pgfsetfillcolor{fc}
    \pgfsetfillopacity{0.0000}
    \pgfsetlinewidth{1.2000bp}
    \definecolor{sc}{rgb}{0.0000,0.0000,0.0000}
    \pgfsetstrokecolor{sc}
    \pgfsetmiterjoin
    \pgfsetbuttcap
    \pgfpathqmoveto{54.0000bp}{78.0000bp}
    \pgfpathqcurveto{54.0000bp}{81.3137bp}{51.3137bp}{84.0000bp}{48.0000bp}{84.0000bp}
    \pgfpathqcurveto{44.6863bp}{84.0000bp}{42.0000bp}{81.3137bp}{42.0000bp}{78.0000bp}
    \pgfpathqcurveto{42.0000bp}{74.6863bp}{44.6863bp}{72.0000bp}{48.0000bp}{72.0000bp}
    \pgfpathqcurveto{51.3137bp}{72.0000bp}{54.0000bp}{74.6863bp}{54.0000bp}{78.0000bp}
    \pgfpathclose
    \pgfusepathqfillstroke
  \end{pgfscope}
  \begin{pgfscope}
    \definecolor{fc}{rgb}{0.0000,0.0000,0.0000}
    \pgfsetfillcolor{fc}
    \pgftransformshift{\pgfqpoint{48.0000bp}{78.0000bp}}
    \pgftransformscale{0.5250}
    \pgftext[]{$12$}
  \end{pgfscope}
  \begin{pgfscope}
    \pgfsetlinewidth{1.2000bp}
    \definecolor{sc}{rgb}{0.0000,0.0000,0.0000}
    \pgfsetstrokecolor{sc}
    \pgfsetmiterjoin
    \pgfsetbuttcap
    \pgfpathqmoveto{53.3677bp}{75.3161bp}
    \pgfpathqlineto{90.6323bp}{56.6839bp}
    \pgfusepathqstroke
  \end{pgfscope}
  \begin{pgfscope}
    \definecolor{fc}{rgb}{0.0000,0.0000,0.0000}
    \pgfsetfillcolor{fc}
    \pgfusepathqfill
  \end{pgfscope}
  \begin{pgfscope}
    \definecolor{fc}{rgb}{0.0000,0.0000,0.0000}
    \pgfsetfillcolor{fc}
    \pgfusepathqfill
  \end{pgfscope}
  \begin{pgfscope}
    \definecolor{fc}{rgb}{0.0000,0.0000,0.0000}
    \pgfsetfillcolor{fc}
    \pgfusepathqfill
  \end{pgfscope}
  \begin{pgfscope}
    \definecolor{fc}{rgb}{0.0000,0.0000,0.0000}
    \pgfsetfillcolor{fc}
    \pgfusepathqfill
  \end{pgfscope}
  \begin{pgfscope}
    \pgfsetlinewidth{1.2000bp}
    \definecolor{sc}{rgb}{0.0000,0.0000,0.0000}
    \pgfsetstrokecolor{sc}
    \pgfsetmiterjoin
    \pgfsetbuttcap
    \pgfpathqmoveto{52.2426bp}{73.7574bp}
    \pgfpathqlineto{67.7574bp}{58.2426bp}
    \pgfusepathqstroke
  \end{pgfscope}
  \begin{pgfscope}
    \definecolor{fc}{rgb}{0.0000,0.0000,0.0000}
    \pgfsetfillcolor{fc}
    \pgfusepathqfill
  \end{pgfscope}
  \begin{pgfscope}
    \definecolor{fc}{rgb}{0.0000,0.0000,0.0000}
    \pgfsetfillcolor{fc}
    \pgfusepathqfill
  \end{pgfscope}
  \begin{pgfscope}
    \definecolor{fc}{rgb}{0.0000,0.0000,0.0000}
    \pgfsetfillcolor{fc}
    \pgfusepathqfill
  \end{pgfscope}
  \begin{pgfscope}
    \definecolor{fc}{rgb}{0.0000,0.0000,0.0000}
    \pgfsetfillcolor{fc}
    \pgfusepathqfill
  \end{pgfscope}
  \begin{pgfscope}
    \pgfsetlinewidth{1.2000bp}
    \definecolor{sc}{rgb}{0.0000,0.0000,0.0000}
    \pgfsetstrokecolor{sc}
    \pgfsetmiterjoin
    \pgfsetbuttcap
    \pgfpathqmoveto{48.0000bp}{72.0000bp}
    \pgfpathqlineto{48.0000bp}{60.0000bp}
    \pgfusepathqstroke
  \end{pgfscope}
  \begin{pgfscope}
    \definecolor{fc}{rgb}{0.0000,0.0000,0.0000}
    \pgfsetfillcolor{fc}
    \pgfusepathqfill
  \end{pgfscope}
  \begin{pgfscope}
    \definecolor{fc}{rgb}{0.0000,0.0000,0.0000}
    \pgfsetfillcolor{fc}
    \pgfusepathqfill
  \end{pgfscope}
  \begin{pgfscope}
    \definecolor{fc}{rgb}{0.0000,0.0000,0.0000}
    \pgfsetfillcolor{fc}
    \pgfusepathqfill
  \end{pgfscope}
  \begin{pgfscope}
    \definecolor{fc}{rgb}{0.0000,0.0000,0.0000}
    \pgfsetfillcolor{fc}
    \pgfusepathqfill
  \end{pgfscope}
  \begin{pgfscope}
    \pgfsetlinewidth{1.2000bp}
    \definecolor{sc}{rgb}{0.5020,0.5020,0.5020}
    \pgfsetstrokecolor{sc}
    \pgfsetmiterjoin
    \pgfsetbuttcap
    \pgfpathqmoveto{0.0000bp}{96.0000bp}
    \pgfpathqlineto{300.0000bp}{96.0000bp}
    \pgfusepathqstroke
  \end{pgfscope}
  \begin{pgfscope}
    \definecolor{fc}{rgb}{0.0000,0.0000,0.0000}
    \pgfsetfillcolor{fc}
    \pgfsetfillopacity{0.0000}
    \pgfsetlinewidth{1.2000bp}
    \definecolor{sc}{rgb}{0.0000,0.0000,0.0000}
    \pgfsetstrokecolor{sc}
    \pgfsetmiterjoin
    \pgfsetbuttcap
    \pgfpathqmoveto{240.0000bp}{186.0000bp}
    \pgfpathqcurveto{240.0000bp}{189.3137bp}{237.3137bp}{192.0000bp}{234.0000bp}{192.0000bp}
    \pgfpathqcurveto{230.6863bp}{192.0000bp}{228.0000bp}{189.3137bp}{228.0000bp}{186.0000bp}
    \pgfpathqcurveto{228.0000bp}{182.6863bp}{230.6863bp}{180.0000bp}{234.0000bp}{180.0000bp}
    \pgfpathqcurveto{237.3137bp}{180.0000bp}{240.0000bp}{182.6863bp}{240.0000bp}{186.0000bp}
    \pgfpathclose
    \pgfusepathqfillstroke
  \end{pgfscope}
  \begin{pgfscope}
    \definecolor{fc}{rgb}{0.0000,0.0000,0.0000}
    \pgfsetfillcolor{fc}
    \pgftransformshift{\pgfqpoint{234.0000bp}{186.0000bp}}
    \pgftransformscale{0.5250}
    \pgftext[]{$8$}
  \end{pgfscope}
  \begin{pgfscope}
    \definecolor{fc}{rgb}{0.0000,0.0000,0.0000}
    \pgfsetfillcolor{fc}
    \pgfsetfillopacity{0.0000}
    \pgfsetlinewidth{1.2000bp}
    \definecolor{sc}{rgb}{0.0000,0.0000,0.0000}
    \pgfsetstrokecolor{sc}
    \pgfsetmiterjoin
    \pgfsetbuttcap
    \pgfpathqmoveto{204.0000bp}{138.0000bp}
    \pgfpathqcurveto{204.0000bp}{141.3137bp}{201.3137bp}{144.0000bp}{198.0000bp}{144.0000bp}
    \pgfpathqcurveto{194.6863bp}{144.0000bp}{192.0000bp}{141.3137bp}{192.0000bp}{138.0000bp}
    \pgfpathqcurveto{192.0000bp}{134.6863bp}{194.6863bp}{132.0000bp}{198.0000bp}{132.0000bp}
    \pgfpathqcurveto{201.3137bp}{132.0000bp}{204.0000bp}{134.6863bp}{204.0000bp}{138.0000bp}
    \pgfpathclose
    \pgfusepathqfillstroke
  \end{pgfscope}
  \begin{pgfscope}
    \definecolor{fc}{rgb}{0.0000,0.0000,0.0000}
    \pgfsetfillcolor{fc}
    \pgftransformshift{\pgfqpoint{198.0000bp}{138.0000bp}}
    \pgftransformscale{0.5250}
    \pgftext[]{$99$}
  \end{pgfscope}
  \begin{pgfscope}
    \definecolor{fc}{rgb}{0.0000,0.0000,0.0000}
    \pgfsetfillcolor{fc}
    \pgfsetfillopacity{0.0000}
    \pgfsetlinewidth{1.2000bp}
    \definecolor{sc}{rgb}{0.0000,0.0000,0.0000}
    \pgfsetstrokecolor{sc}
    \pgfsetmiterjoin
    \pgfsetbuttcap
    \pgfpathqmoveto{204.0000bp}{162.0000bp}
    \pgfpathqcurveto{204.0000bp}{165.3137bp}{201.3137bp}{168.0000bp}{198.0000bp}{168.0000bp}
    \pgfpathqcurveto{194.6863bp}{168.0000bp}{192.0000bp}{165.3137bp}{192.0000bp}{162.0000bp}
    \pgfpathqcurveto{192.0000bp}{158.6863bp}{194.6863bp}{156.0000bp}{198.0000bp}{156.0000bp}
    \pgfpathqcurveto{201.3137bp}{156.0000bp}{204.0000bp}{158.6863bp}{204.0000bp}{162.0000bp}
    \pgfpathclose
    \pgfusepathqfillstroke
  \end{pgfscope}
  \begin{pgfscope}
    \definecolor{fc}{rgb}{0.0000,0.0000,0.0000}
    \pgfsetfillcolor{fc}
    \pgftransformshift{\pgfqpoint{198.0000bp}{162.0000bp}}
    \pgftransformscale{0.5250}
    \pgftext[]{$17$}
  \end{pgfscope}
  \begin{pgfscope}
    \pgfsetlinewidth{1.2000bp}
    \definecolor{sc}{rgb}{0.0000,0.0000,0.0000}
    \pgfsetstrokecolor{sc}
    \pgfsetmiterjoin
    \pgfsetbuttcap
    \pgfpathqmoveto{198.0000bp}{156.0000bp}
    \pgfpathqlineto{198.0000bp}{144.0000bp}
    \pgfusepathqstroke
  \end{pgfscope}
  \begin{pgfscope}
    \definecolor{fc}{rgb}{0.0000,0.0000,0.0000}
    \pgfsetfillcolor{fc}
    \pgfusepathqfill
  \end{pgfscope}
  \begin{pgfscope}
    \definecolor{fc}{rgb}{0.0000,0.0000,0.0000}
    \pgfsetfillcolor{fc}
    \pgfusepathqfill
  \end{pgfscope}
  \begin{pgfscope}
    \definecolor{fc}{rgb}{0.0000,0.0000,0.0000}
    \pgfsetfillcolor{fc}
    \pgfusepathqfill
  \end{pgfscope}
  \begin{pgfscope}
    \definecolor{fc}{rgb}{0.0000,0.0000,0.0000}
    \pgfsetfillcolor{fc}
    \pgfusepathqfill
  \end{pgfscope}
  \begin{pgfscope}
    \definecolor{fc}{rgb}{0.0000,0.0000,0.0000}
    \pgfsetfillcolor{fc}
    \pgfsetfillopacity{0.0000}
    \pgfsetlinewidth{1.2000bp}
    \definecolor{sc}{rgb}{0.0000,0.0000,0.0000}
    \pgfsetstrokecolor{sc}
    \pgfsetmiterjoin
    \pgfsetbuttcap
    \pgfpathqmoveto{180.0000bp}{162.0000bp}
    \pgfpathqcurveto{180.0000bp}{165.3137bp}{177.3137bp}{168.0000bp}{174.0000bp}{168.0000bp}
    \pgfpathqcurveto{170.6863bp}{168.0000bp}{168.0000bp}{165.3137bp}{168.0000bp}{162.0000bp}
    \pgfpathqcurveto{168.0000bp}{158.6863bp}{170.6863bp}{156.0000bp}{174.0000bp}{156.0000bp}
    \pgfpathqcurveto{177.3137bp}{156.0000bp}{180.0000bp}{158.6863bp}{180.0000bp}{162.0000bp}
    \pgfpathclose
    \pgfusepathqfillstroke
  \end{pgfscope}
  \begin{pgfscope}
    \definecolor{fc}{rgb}{0.0000,0.0000,0.0000}
    \pgfsetfillcolor{fc}
    \pgftransformshift{\pgfqpoint{174.0000bp}{162.0000bp}}
    \pgftransformscale{0.5250}
    \pgftext[]{$21$}
  \end{pgfscope}
  \begin{pgfscope}
    \definecolor{fc}{rgb}{0.0000,0.0000,0.0000}
    \pgfsetfillcolor{fc}
    \pgfsetfillopacity{0.0000}
    \pgfsetlinewidth{1.2000bp}
    \definecolor{sc}{rgb}{0.0000,0.0000,0.0000}
    \pgfsetstrokecolor{sc}
    \pgfsetmiterjoin
    \pgfsetbuttcap
    \pgfpathqmoveto{180.0000bp}{186.0000bp}
    \pgfpathqcurveto{180.0000bp}{189.3137bp}{177.3137bp}{192.0000bp}{174.0000bp}{192.0000bp}
    \pgfpathqcurveto{170.6863bp}{192.0000bp}{168.0000bp}{189.3137bp}{168.0000bp}{186.0000bp}
    \pgfpathqcurveto{168.0000bp}{182.6863bp}{170.6863bp}{180.0000bp}{174.0000bp}{180.0000bp}
    \pgfpathqcurveto{177.3137bp}{180.0000bp}{180.0000bp}{182.6863bp}{180.0000bp}{186.0000bp}
    \pgfpathclose
    \pgfusepathqfillstroke
  \end{pgfscope}
  \begin{pgfscope}
    \definecolor{fc}{rgb}{0.0000,0.0000,0.0000}
    \pgfsetfillcolor{fc}
    \pgftransformshift{\pgfqpoint{174.0000bp}{186.0000bp}}
    \pgftransformscale{0.5250}
    \pgftext[]{$5$}
  \end{pgfscope}
  \begin{pgfscope}
    \pgfsetlinewidth{1.2000bp}
    \definecolor{sc}{rgb}{0.0000,0.0000,0.0000}
    \pgfsetstrokecolor{sc}
    \pgfsetmiterjoin
    \pgfsetbuttcap
    \pgfpathqmoveto{178.2426bp}{181.7574bp}
    \pgfpathqlineto{193.7574bp}{166.2426bp}
    \pgfusepathqstroke
  \end{pgfscope}
  \begin{pgfscope}
    \definecolor{fc}{rgb}{0.0000,0.0000,0.0000}
    \pgfsetfillcolor{fc}
    \pgfusepathqfill
  \end{pgfscope}
  \begin{pgfscope}
    \definecolor{fc}{rgb}{0.0000,0.0000,0.0000}
    \pgfsetfillcolor{fc}
    \pgfusepathqfill
  \end{pgfscope}
  \begin{pgfscope}
    \definecolor{fc}{rgb}{0.0000,0.0000,0.0000}
    \pgfsetfillcolor{fc}
    \pgfusepathqfill
  \end{pgfscope}
  \begin{pgfscope}
    \definecolor{fc}{rgb}{0.0000,0.0000,0.0000}
    \pgfsetfillcolor{fc}
    \pgfusepathqfill
  \end{pgfscope}
  \begin{pgfscope}
    \pgfsetlinewidth{1.2000bp}
    \definecolor{sc}{rgb}{0.0000,0.0000,0.0000}
    \pgfsetstrokecolor{sc}
    \pgfsetmiterjoin
    \pgfsetbuttcap
    \pgfpathqmoveto{174.0000bp}{180.0000bp}
    \pgfpathqlineto{174.0000bp}{168.0000bp}
    \pgfusepathqstroke
  \end{pgfscope}
  \begin{pgfscope}
    \definecolor{fc}{rgb}{0.0000,0.0000,0.0000}
    \pgfsetfillcolor{fc}
    \pgfusepathqfill
  \end{pgfscope}
  \begin{pgfscope}
    \definecolor{fc}{rgb}{0.0000,0.0000,0.0000}
    \pgfsetfillcolor{fc}
    \pgfusepathqfill
  \end{pgfscope}
  \begin{pgfscope}
    \definecolor{fc}{rgb}{0.0000,0.0000,0.0000}
    \pgfsetfillcolor{fc}
    \pgfusepathqfill
  \end{pgfscope}
  \begin{pgfscope}
    \definecolor{fc}{rgb}{0.0000,0.0000,0.0000}
    \pgfsetfillcolor{fc}
    \pgfusepathqfill
  \end{pgfscope}
  \begin{pgfscope}
    \definecolor{fc}{rgb}{0.0000,0.0000,0.0000}
    \pgfsetfillcolor{fc}
    \pgfsetfillopacity{0.0000}
    \pgfsetlinewidth{1.2000bp}
    \definecolor{sc}{rgb}{0.0000,0.0000,0.0000}
    \pgfsetstrokecolor{sc}
    \pgfsetmiterjoin
    \pgfsetbuttcap
    \pgfpathqmoveto{144.0000bp}{114.0000bp}
    \pgfpathqcurveto{144.0000bp}{117.3137bp}{141.3137bp}{120.0000bp}{138.0000bp}{120.0000bp}
    \pgfpathqcurveto{134.6863bp}{120.0000bp}{132.0000bp}{117.3137bp}{132.0000bp}{114.0000bp}
    \pgfpathqcurveto{132.0000bp}{110.6863bp}{134.6863bp}{108.0000bp}{138.0000bp}{108.0000bp}
    \pgfpathqcurveto{141.3137bp}{108.0000bp}{144.0000bp}{110.6863bp}{144.0000bp}{114.0000bp}
    \pgfpathclose
    \pgfusepathqfillstroke
  \end{pgfscope}
  \begin{pgfscope}
    \definecolor{fc}{rgb}{0.0000,0.0000,0.0000}
    \pgfsetfillcolor{fc}
    \pgftransformshift{\pgfqpoint{138.0000bp}{114.0000bp}}
    \pgftransformscale{0.5250}
    \pgftext[]{$53$}
  \end{pgfscope}
  \begin{pgfscope}
    \definecolor{fc}{rgb}{0.0000,0.0000,0.0000}
    \pgfsetfillcolor{fc}
    \pgfsetfillopacity{0.0000}
    \pgfsetlinewidth{1.2000bp}
    \definecolor{sc}{rgb}{0.0000,0.0000,0.0000}
    \pgfsetstrokecolor{sc}
    \pgfsetmiterjoin
    \pgfsetbuttcap
    \pgfpathqmoveto{144.0000bp}{138.0000bp}
    \pgfpathqcurveto{144.0000bp}{141.3137bp}{141.3137bp}{144.0000bp}{138.0000bp}{144.0000bp}
    \pgfpathqcurveto{134.6863bp}{144.0000bp}{132.0000bp}{141.3137bp}{132.0000bp}{138.0000bp}
    \pgfpathqcurveto{132.0000bp}{134.6863bp}{134.6863bp}{132.0000bp}{138.0000bp}{132.0000bp}
    \pgfpathqcurveto{141.3137bp}{132.0000bp}{144.0000bp}{134.6863bp}{144.0000bp}{138.0000bp}
    \pgfpathclose
    \pgfusepathqfillstroke
  \end{pgfscope}
  \begin{pgfscope}
    \definecolor{fc}{rgb}{0.0000,0.0000,0.0000}
    \pgfsetfillcolor{fc}
    \pgftransformshift{\pgfqpoint{138.0000bp}{138.0000bp}}
    \pgftransformscale{0.5250}
    \pgftext[]{$33$}
  \end{pgfscope}
  \begin{pgfscope}
    \pgfsetlinewidth{1.2000bp}
    \definecolor{sc}{rgb}{0.0000,0.0000,0.0000}
    \pgfsetstrokecolor{sc}
    \pgfsetmiterjoin
    \pgfsetbuttcap
    \pgfpathqmoveto{138.0000bp}{132.0000bp}
    \pgfpathqlineto{138.0000bp}{120.0000bp}
    \pgfusepathqstroke
  \end{pgfscope}
  \begin{pgfscope}
    \definecolor{fc}{rgb}{0.0000,0.0000,0.0000}
    \pgfsetfillcolor{fc}
    \pgfusepathqfill
  \end{pgfscope}
  \begin{pgfscope}
    \definecolor{fc}{rgb}{0.0000,0.0000,0.0000}
    \pgfsetfillcolor{fc}
    \pgfusepathqfill
  \end{pgfscope}
  \begin{pgfscope}
    \definecolor{fc}{rgb}{0.0000,0.0000,0.0000}
    \pgfsetfillcolor{fc}
    \pgfusepathqfill
  \end{pgfscope}
  \begin{pgfscope}
    \definecolor{fc}{rgb}{0.0000,0.0000,0.0000}
    \pgfsetfillcolor{fc}
    \pgfusepathqfill
  \end{pgfscope}
  \begin{pgfscope}
    \definecolor{fc}{rgb}{0.0000,0.0000,0.0000}
    \pgfsetfillcolor{fc}
    \pgfsetfillopacity{0.0000}
    \pgfsetlinewidth{1.2000bp}
    \definecolor{sc}{rgb}{0.0000,0.0000,0.0000}
    \pgfsetstrokecolor{sc}
    \pgfsetmiterjoin
    \pgfsetbuttcap
    \pgfpathqmoveto{120.0000bp}{138.0000bp}
    \pgfpathqcurveto{120.0000bp}{141.3137bp}{117.3137bp}{144.0000bp}{114.0000bp}{144.0000bp}
    \pgfpathqcurveto{110.6863bp}{144.0000bp}{108.0000bp}{141.3137bp}{108.0000bp}{138.0000bp}
    \pgfpathqcurveto{108.0000bp}{134.6863bp}{110.6863bp}{132.0000bp}{114.0000bp}{132.0000bp}
    \pgfpathqcurveto{117.3137bp}{132.0000bp}{120.0000bp}{134.6863bp}{120.0000bp}{138.0000bp}
    \pgfpathclose
    \pgfusepathqfillstroke
  \end{pgfscope}
  \begin{pgfscope}
    \definecolor{fc}{rgb}{0.0000,0.0000,0.0000}
    \pgfsetfillcolor{fc}
    \pgftransformshift{\pgfqpoint{114.0000bp}{138.0000bp}}
    \pgftransformscale{0.5250}
    \pgftext[]{$24$}
  \end{pgfscope}
  \begin{pgfscope}
    \definecolor{fc}{rgb}{0.0000,0.0000,0.0000}
    \pgfsetfillcolor{fc}
    \pgfsetfillopacity{0.0000}
    \pgfsetlinewidth{1.2000bp}
    \definecolor{sc}{rgb}{0.0000,0.0000,0.0000}
    \pgfsetstrokecolor{sc}
    \pgfsetmiterjoin
    \pgfsetbuttcap
    \pgfpathqmoveto{120.0000bp}{162.0000bp}
    \pgfpathqcurveto{120.0000bp}{165.3137bp}{117.3137bp}{168.0000bp}{114.0000bp}{168.0000bp}
    \pgfpathqcurveto{110.6863bp}{168.0000bp}{108.0000bp}{165.3137bp}{108.0000bp}{162.0000bp}
    \pgfpathqcurveto{108.0000bp}{158.6863bp}{110.6863bp}{156.0000bp}{114.0000bp}{156.0000bp}
    \pgfpathqcurveto{117.3137bp}{156.0000bp}{120.0000bp}{158.6863bp}{120.0000bp}{162.0000bp}
    \pgfpathclose
    \pgfusepathqfillstroke
  \end{pgfscope}
  \begin{pgfscope}
    \definecolor{fc}{rgb}{0.0000,0.0000,0.0000}
    \pgfsetfillcolor{fc}
    \pgftransformshift{\pgfqpoint{114.0000bp}{162.0000bp}}
    \pgftransformscale{0.5250}
    \pgftext[]{$23$}
  \end{pgfscope}
  \begin{pgfscope}
    \pgfsetlinewidth{1.2000bp}
    \definecolor{sc}{rgb}{0.0000,0.0000,0.0000}
    \pgfsetstrokecolor{sc}
    \pgfsetmiterjoin
    \pgfsetbuttcap
    \pgfpathqmoveto{118.2426bp}{157.7574bp}
    \pgfpathqlineto{133.7574bp}{142.2426bp}
    \pgfusepathqstroke
  \end{pgfscope}
  \begin{pgfscope}
    \definecolor{fc}{rgb}{0.0000,0.0000,0.0000}
    \pgfsetfillcolor{fc}
    \pgfusepathqfill
  \end{pgfscope}
  \begin{pgfscope}
    \definecolor{fc}{rgb}{0.0000,0.0000,0.0000}
    \pgfsetfillcolor{fc}
    \pgfusepathqfill
  \end{pgfscope}
  \begin{pgfscope}
    \definecolor{fc}{rgb}{0.0000,0.0000,0.0000}
    \pgfsetfillcolor{fc}
    \pgfusepathqfill
  \end{pgfscope}
  \begin{pgfscope}
    \definecolor{fc}{rgb}{0.0000,0.0000,0.0000}
    \pgfsetfillcolor{fc}
    \pgfusepathqfill
  \end{pgfscope}
  \begin{pgfscope}
    \pgfsetlinewidth{1.2000bp}
    \definecolor{sc}{rgb}{0.0000,0.0000,0.0000}
    \pgfsetstrokecolor{sc}
    \pgfsetmiterjoin
    \pgfsetbuttcap
    \pgfpathqmoveto{114.0000bp}{156.0000bp}
    \pgfpathqlineto{114.0000bp}{144.0000bp}
    \pgfusepathqstroke
  \end{pgfscope}
  \begin{pgfscope}
    \definecolor{fc}{rgb}{0.0000,0.0000,0.0000}
    \pgfsetfillcolor{fc}
    \pgfusepathqfill
  \end{pgfscope}
  \begin{pgfscope}
    \definecolor{fc}{rgb}{0.0000,0.0000,0.0000}
    \pgfsetfillcolor{fc}
    \pgfusepathqfill
  \end{pgfscope}
  \begin{pgfscope}
    \definecolor{fc}{rgb}{0.0000,0.0000,0.0000}
    \pgfsetfillcolor{fc}
    \pgfusepathqfill
  \end{pgfscope}
  \begin{pgfscope}
    \definecolor{fc}{rgb}{0.0000,0.0000,0.0000}
    \pgfsetfillcolor{fc}
    \pgfusepathqfill
  \end{pgfscope}
  \begin{pgfscope}
    \definecolor{fc}{rgb}{0.0000,0.0000,0.0000}
    \pgfsetfillcolor{fc}
    \pgfsetfillopacity{0.0000}
    \pgfsetlinewidth{1.2000bp}
    \definecolor{sc}{rgb}{0.0000,0.0000,0.0000}
    \pgfsetstrokecolor{sc}
    \pgfsetmiterjoin
    \pgfsetbuttcap
    \pgfpathqmoveto{96.0000bp}{138.0000bp}
    \pgfpathqcurveto{96.0000bp}{141.3137bp}{93.3137bp}{144.0000bp}{90.0000bp}{144.0000bp}
    \pgfpathqcurveto{86.6863bp}{144.0000bp}{84.0000bp}{141.3137bp}{84.0000bp}{138.0000bp}
    \pgfpathqcurveto{84.0000bp}{134.6863bp}{86.6863bp}{132.0000bp}{90.0000bp}{132.0000bp}
    \pgfpathqcurveto{93.3137bp}{132.0000bp}{96.0000bp}{134.6863bp}{96.0000bp}{138.0000bp}
    \pgfpathclose
    \pgfusepathqfillstroke
  \end{pgfscope}
  \begin{pgfscope}
    \definecolor{fc}{rgb}{0.0000,0.0000,0.0000}
    \pgfsetfillcolor{fc}
    \pgftransformshift{\pgfqpoint{90.0000bp}{138.0000bp}}
    \pgftransformscale{0.5250}
    \pgftext[]{$28$}
  \end{pgfscope}
  \begin{pgfscope}
    \definecolor{fc}{rgb}{0.0000,0.0000,0.0000}
    \pgfsetfillcolor{fc}
    \pgfsetfillopacity{0.0000}
    \pgfsetlinewidth{1.2000bp}
    \definecolor{sc}{rgb}{0.0000,0.0000,0.0000}
    \pgfsetstrokecolor{sc}
    \pgfsetmiterjoin
    \pgfsetbuttcap
    \pgfpathqmoveto{96.0000bp}{162.0000bp}
    \pgfpathqcurveto{96.0000bp}{165.3137bp}{93.3137bp}{168.0000bp}{90.0000bp}{168.0000bp}
    \pgfpathqcurveto{86.6863bp}{168.0000bp}{84.0000bp}{165.3137bp}{84.0000bp}{162.0000bp}
    \pgfpathqcurveto{84.0000bp}{158.6863bp}{86.6863bp}{156.0000bp}{90.0000bp}{156.0000bp}
    \pgfpathqcurveto{93.3137bp}{156.0000bp}{96.0000bp}{158.6863bp}{96.0000bp}{162.0000bp}
    \pgfpathclose
    \pgfusepathqfillstroke
  \end{pgfscope}
  \begin{pgfscope}
    \definecolor{fc}{rgb}{0.0000,0.0000,0.0000}
    \pgfsetfillcolor{fc}
    \pgftransformshift{\pgfqpoint{90.0000bp}{162.0000bp}}
    \pgftransformscale{0.5250}
    \pgftext[]{$13$}
  \end{pgfscope}
  \begin{pgfscope}
    \pgfsetlinewidth{1.2000bp}
    \definecolor{sc}{rgb}{0.0000,0.0000,0.0000}
    \pgfsetstrokecolor{sc}
    \pgfsetmiterjoin
    \pgfsetbuttcap
    \pgfpathqmoveto{90.0000bp}{156.0000bp}
    \pgfpathqlineto{90.0000bp}{144.0000bp}
    \pgfusepathqstroke
  \end{pgfscope}
  \begin{pgfscope}
    \definecolor{fc}{rgb}{0.0000,0.0000,0.0000}
    \pgfsetfillcolor{fc}
    \pgfusepathqfill
  \end{pgfscope}
  \begin{pgfscope}
    \definecolor{fc}{rgb}{0.0000,0.0000,0.0000}
    \pgfsetfillcolor{fc}
    \pgfusepathqfill
  \end{pgfscope}
  \begin{pgfscope}
    \definecolor{fc}{rgb}{0.0000,0.0000,0.0000}
    \pgfsetfillcolor{fc}
    \pgfusepathqfill
  \end{pgfscope}
  \begin{pgfscope}
    \definecolor{fc}{rgb}{0.0000,0.0000,0.0000}
    \pgfsetfillcolor{fc}
    \pgfusepathqfill
  \end{pgfscope}
  \begin{pgfscope}
    \definecolor{fc}{rgb}{0.0000,0.0000,0.0000}
    \pgfsetfillcolor{fc}
    \pgfsetfillopacity{0.0000}
    \pgfsetlinewidth{1.2000bp}
    \definecolor{sc}{rgb}{0.0000,0.0000,0.0000}
    \pgfsetstrokecolor{sc}
    \pgfsetmiterjoin
    \pgfsetbuttcap
    \pgfpathqmoveto{72.0000bp}{162.0000bp}
    \pgfpathqcurveto{72.0000bp}{165.3137bp}{69.3137bp}{168.0000bp}{66.0000bp}{168.0000bp}
    \pgfpathqcurveto{62.6863bp}{168.0000bp}{60.0000bp}{165.3137bp}{60.0000bp}{162.0000bp}
    \pgfpathqcurveto{60.0000bp}{158.6863bp}{62.6863bp}{156.0000bp}{66.0000bp}{156.0000bp}
    \pgfpathqcurveto{69.3137bp}{156.0000bp}{72.0000bp}{158.6863bp}{72.0000bp}{162.0000bp}
    \pgfpathclose
    \pgfusepathqfillstroke
  \end{pgfscope}
  \begin{pgfscope}
    \definecolor{fc}{rgb}{0.0000,0.0000,0.0000}
    \pgfsetfillcolor{fc}
    \pgftransformshift{\pgfqpoint{66.0000bp}{162.0000bp}}
    \pgftransformscale{0.5250}
    \pgftext[]{$77$}
  \end{pgfscope}
  \begin{pgfscope}
    \definecolor{fc}{rgb}{0.0000,0.0000,0.0000}
    \pgfsetfillcolor{fc}
    \pgfsetfillopacity{0.0000}
    \pgfsetlinewidth{1.2000bp}
    \definecolor{sc}{rgb}{0.0000,0.0000,0.0000}
    \pgfsetstrokecolor{sc}
    \pgfsetmiterjoin
    \pgfsetbuttcap
    \pgfpathqmoveto{72.0000bp}{186.0000bp}
    \pgfpathqcurveto{72.0000bp}{189.3137bp}{69.3137bp}{192.0000bp}{66.0000bp}{192.0000bp}
    \pgfpathqcurveto{62.6863bp}{192.0000bp}{60.0000bp}{189.3137bp}{60.0000bp}{186.0000bp}
    \pgfpathqcurveto{60.0000bp}{182.6863bp}{62.6863bp}{180.0000bp}{66.0000bp}{180.0000bp}
    \pgfpathqcurveto{69.3137bp}{180.0000bp}{72.0000bp}{182.6863bp}{72.0000bp}{186.0000bp}
    \pgfpathclose
    \pgfusepathqfillstroke
  \end{pgfscope}
  \begin{pgfscope}
    \definecolor{fc}{rgb}{0.0000,0.0000,0.0000}
    \pgfsetfillcolor{fc}
    \pgftransformshift{\pgfqpoint{66.0000bp}{186.0000bp}}
    \pgftransformscale{0.5250}
    \pgftext[]{$12$}
  \end{pgfscope}
  \begin{pgfscope}
    \pgfsetlinewidth{1.2000bp}
    \definecolor{sc}{rgb}{0.0000,0.0000,0.0000}
    \pgfsetstrokecolor{sc}
    \pgfsetmiterjoin
    \pgfsetbuttcap
    \pgfpathqmoveto{71.3677bp}{183.3161bp}
    \pgfpathqlineto{108.6323bp}{164.6839bp}
    \pgfusepathqstroke
  \end{pgfscope}
  \begin{pgfscope}
    \definecolor{fc}{rgb}{0.0000,0.0000,0.0000}
    \pgfsetfillcolor{fc}
    \pgfusepathqfill
  \end{pgfscope}
  \begin{pgfscope}
    \definecolor{fc}{rgb}{0.0000,0.0000,0.0000}
    \pgfsetfillcolor{fc}
    \pgfusepathqfill
  \end{pgfscope}
  \begin{pgfscope}
    \definecolor{fc}{rgb}{0.0000,0.0000,0.0000}
    \pgfsetfillcolor{fc}
    \pgfusepathqfill
  \end{pgfscope}
  \begin{pgfscope}
    \definecolor{fc}{rgb}{0.0000,0.0000,0.0000}
    \pgfsetfillcolor{fc}
    \pgfusepathqfill
  \end{pgfscope}
  \begin{pgfscope}
    \pgfsetlinewidth{1.2000bp}
    \definecolor{sc}{rgb}{0.0000,0.0000,0.0000}
    \pgfsetstrokecolor{sc}
    \pgfsetmiterjoin
    \pgfsetbuttcap
    \pgfpathqmoveto{70.2426bp}{181.7574bp}
    \pgfpathqlineto{85.7574bp}{166.2426bp}
    \pgfusepathqstroke
  \end{pgfscope}
  \begin{pgfscope}
    \definecolor{fc}{rgb}{0.0000,0.0000,0.0000}
    \pgfsetfillcolor{fc}
    \pgfusepathqfill
  \end{pgfscope}
  \begin{pgfscope}
    \definecolor{fc}{rgb}{0.0000,0.0000,0.0000}
    \pgfsetfillcolor{fc}
    \pgfusepathqfill
  \end{pgfscope}
  \begin{pgfscope}
    \definecolor{fc}{rgb}{0.0000,0.0000,0.0000}
    \pgfsetfillcolor{fc}
    \pgfusepathqfill
  \end{pgfscope}
  \begin{pgfscope}
    \definecolor{fc}{rgb}{0.0000,0.0000,0.0000}
    \pgfsetfillcolor{fc}
    \pgfusepathqfill
  \end{pgfscope}
  \begin{pgfscope}
    \pgfsetlinewidth{1.2000bp}
    \definecolor{sc}{rgb}{0.0000,0.0000,0.0000}
    \pgfsetstrokecolor{sc}
    \pgfsetmiterjoin
    \pgfsetbuttcap
    \pgfpathqmoveto{66.0000bp}{180.0000bp}
    \pgfpathqlineto{66.0000bp}{168.0000bp}
    \pgfusepathqstroke
  \end{pgfscope}
  \begin{pgfscope}
    \definecolor{fc}{rgb}{0.0000,0.0000,0.0000}
    \pgfsetfillcolor{fc}
    \pgfusepathqfill
  \end{pgfscope}
  \begin{pgfscope}
    \definecolor{fc}{rgb}{0.0000,0.0000,0.0000}
    \pgfsetfillcolor{fc}
    \pgfusepathqfill
  \end{pgfscope}
  \begin{pgfscope}
    \definecolor{fc}{rgb}{0.0000,0.0000,0.0000}
    \pgfsetfillcolor{fc}
    \pgfusepathqfill
  \end{pgfscope}
  \begin{pgfscope}
    \definecolor{fc}{rgb}{0.0000,0.0000,0.0000}
    \pgfsetfillcolor{fc}
    \pgfusepathqfill
  \end{pgfscope}
  \begin{pgfscope}
    \pgfsetlinewidth{1.2000bp}
    \definecolor{sc}{rgb}{0.5020,0.5020,0.5020}
    \pgfsetstrokecolor{sc}
    \pgfsetmiterjoin
    \pgfsetbuttcap
    \pgfpathqmoveto{0.0000bp}{204.0000bp}
    \pgfpathqlineto{300.0000bp}{204.0000bp}
    \pgfusepathqstroke
  \end{pgfscope}
  \begin{pgfscope}
    \definecolor{fc}{rgb}{0.0000,0.0000,0.0000}
    \pgfsetfillcolor{fc}
    \pgfsetfillopacity{0.0000}
    \pgfsetlinewidth{1.2000bp}
    \definecolor{sc}{rgb}{0.0000,0.0000,0.0000}
    \pgfsetstrokecolor{sc}
    \pgfsetmiterjoin
    \pgfsetbuttcap
    \pgfpathqmoveto{240.0000bp}{294.0000bp}
    \pgfpathqcurveto{240.0000bp}{297.3137bp}{237.3137bp}{300.0000bp}{234.0000bp}{300.0000bp}
    \pgfpathqcurveto{230.6863bp}{300.0000bp}{228.0000bp}{297.3137bp}{228.0000bp}{294.0000bp}
    \pgfpathqcurveto{228.0000bp}{290.6863bp}{230.6863bp}{288.0000bp}{234.0000bp}{288.0000bp}
    \pgfpathqcurveto{237.3137bp}{288.0000bp}{240.0000bp}{290.6863bp}{240.0000bp}{294.0000bp}
    \pgfpathclose
    \pgfusepathqfillstroke
  \end{pgfscope}
  \begin{pgfscope}
    \definecolor{fc}{rgb}{0.0000,0.0000,0.0000}
    \pgfsetfillcolor{fc}
    \pgftransformshift{\pgfqpoint{234.0000bp}{294.0000bp}}
    \pgftransformscale{0.5250}
    \pgftext[]{$8$}
  \end{pgfscope}
  \begin{pgfscope}
    \definecolor{fc}{rgb}{0.0000,0.0000,0.0000}
    \pgfsetfillcolor{fc}
    \pgfsetfillopacity{0.0000}
    \pgfsetlinewidth{1.2000bp}
    \definecolor{sc}{rgb}{0.0000,0.0000,0.0000}
    \pgfsetstrokecolor{sc}
    \pgfsetmiterjoin
    \pgfsetbuttcap
    \pgfpathqmoveto{204.0000bp}{246.0000bp}
    \pgfpathqcurveto{204.0000bp}{249.3137bp}{201.3137bp}{252.0000bp}{198.0000bp}{252.0000bp}
    \pgfpathqcurveto{194.6863bp}{252.0000bp}{192.0000bp}{249.3137bp}{192.0000bp}{246.0000bp}
    \pgfpathqcurveto{192.0000bp}{242.6863bp}{194.6863bp}{240.0000bp}{198.0000bp}{240.0000bp}
    \pgfpathqcurveto{201.3137bp}{240.0000bp}{204.0000bp}{242.6863bp}{204.0000bp}{246.0000bp}
    \pgfpathclose
    \pgfusepathqfillstroke
  \end{pgfscope}
  \begin{pgfscope}
    \definecolor{fc}{rgb}{0.0000,0.0000,0.0000}
    \pgfsetfillcolor{fc}
    \pgftransformshift{\pgfqpoint{198.0000bp}{246.0000bp}}
    \pgftransformscale{0.5250}
    \pgftext[]{$99$}
  \end{pgfscope}
  \begin{pgfscope}
    \definecolor{fc}{rgb}{0.0000,0.0000,0.0000}
    \pgfsetfillcolor{fc}
    \pgfsetfillopacity{0.0000}
    \pgfsetlinewidth{1.2000bp}
    \definecolor{sc}{rgb}{0.0000,0.0000,0.0000}
    \pgfsetstrokecolor{sc}
    \pgfsetmiterjoin
    \pgfsetbuttcap
    \pgfpathqmoveto{204.0000bp}{270.0000bp}
    \pgfpathqcurveto{204.0000bp}{273.3137bp}{201.3137bp}{276.0000bp}{198.0000bp}{276.0000bp}
    \pgfpathqcurveto{194.6863bp}{276.0000bp}{192.0000bp}{273.3137bp}{192.0000bp}{270.0000bp}
    \pgfpathqcurveto{192.0000bp}{266.6863bp}{194.6863bp}{264.0000bp}{198.0000bp}{264.0000bp}
    \pgfpathqcurveto{201.3137bp}{264.0000bp}{204.0000bp}{266.6863bp}{204.0000bp}{270.0000bp}
    \pgfpathclose
    \pgfusepathqfillstroke
  \end{pgfscope}
  \begin{pgfscope}
    \definecolor{fc}{rgb}{0.0000,0.0000,0.0000}
    \pgfsetfillcolor{fc}
    \pgftransformshift{\pgfqpoint{198.0000bp}{270.0000bp}}
    \pgftransformscale{0.5250}
    \pgftext[]{$17$}
  \end{pgfscope}
  \begin{pgfscope}
    \pgfsetlinewidth{1.2000bp}
    \definecolor{sc}{rgb}{0.0000,0.0000,0.0000}
    \pgfsetstrokecolor{sc}
    \pgfsetmiterjoin
    \pgfsetbuttcap
    \pgfpathqmoveto{198.0000bp}{264.0000bp}
    \pgfpathqlineto{198.0000bp}{252.0000bp}
    \pgfusepathqstroke
  \end{pgfscope}
  \begin{pgfscope}
    \definecolor{fc}{rgb}{0.0000,0.0000,0.0000}
    \pgfsetfillcolor{fc}
    \pgfusepathqfill
  \end{pgfscope}
  \begin{pgfscope}
    \definecolor{fc}{rgb}{0.0000,0.0000,0.0000}
    \pgfsetfillcolor{fc}
    \pgfusepathqfill
  \end{pgfscope}
  \begin{pgfscope}
    \definecolor{fc}{rgb}{0.0000,0.0000,0.0000}
    \pgfsetfillcolor{fc}
    \pgfusepathqfill
  \end{pgfscope}
  \begin{pgfscope}
    \definecolor{fc}{rgb}{0.0000,0.0000,0.0000}
    \pgfsetfillcolor{fc}
    \pgfusepathqfill
  \end{pgfscope}
  \begin{pgfscope}
    \definecolor{fc}{rgb}{0.0000,0.0000,0.0000}
    \pgfsetfillcolor{fc}
    \pgfsetfillopacity{0.0000}
    \pgfsetlinewidth{1.2000bp}
    \definecolor{sc}{rgb}{0.0000,0.0000,0.0000}
    \pgfsetstrokecolor{sc}
    \pgfsetmiterjoin
    \pgfsetbuttcap
    \pgfpathqmoveto{180.0000bp}{270.0000bp}
    \pgfpathqcurveto{180.0000bp}{273.3137bp}{177.3137bp}{276.0000bp}{174.0000bp}{276.0000bp}
    \pgfpathqcurveto{170.6863bp}{276.0000bp}{168.0000bp}{273.3137bp}{168.0000bp}{270.0000bp}
    \pgfpathqcurveto{168.0000bp}{266.6863bp}{170.6863bp}{264.0000bp}{174.0000bp}{264.0000bp}
    \pgfpathqcurveto{177.3137bp}{264.0000bp}{180.0000bp}{266.6863bp}{180.0000bp}{270.0000bp}
    \pgfpathclose
    \pgfusepathqfillstroke
  \end{pgfscope}
  \begin{pgfscope}
    \definecolor{fc}{rgb}{0.0000,0.0000,0.0000}
    \pgfsetfillcolor{fc}
    \pgftransformshift{\pgfqpoint{174.0000bp}{270.0000bp}}
    \pgftransformscale{0.5250}
    \pgftext[]{$21$}
  \end{pgfscope}
  \begin{pgfscope}
    \definecolor{fc}{rgb}{0.0000,0.0000,0.0000}
    \pgfsetfillcolor{fc}
    \pgfsetfillopacity{0.0000}
    \pgfsetlinewidth{1.2000bp}
    \definecolor{sc}{rgb}{0.0000,0.0000,0.0000}
    \pgfsetstrokecolor{sc}
    \pgfsetmiterjoin
    \pgfsetbuttcap
    \pgfpathqmoveto{180.0000bp}{294.0000bp}
    \pgfpathqcurveto{180.0000bp}{297.3137bp}{177.3137bp}{300.0000bp}{174.0000bp}{300.0000bp}
    \pgfpathqcurveto{170.6863bp}{300.0000bp}{168.0000bp}{297.3137bp}{168.0000bp}{294.0000bp}
    \pgfpathqcurveto{168.0000bp}{290.6863bp}{170.6863bp}{288.0000bp}{174.0000bp}{288.0000bp}
    \pgfpathqcurveto{177.3137bp}{288.0000bp}{180.0000bp}{290.6863bp}{180.0000bp}{294.0000bp}
    \pgfpathclose
    \pgfusepathqfillstroke
  \end{pgfscope}
  \begin{pgfscope}
    \definecolor{fc}{rgb}{0.0000,0.0000,0.0000}
    \pgfsetfillcolor{fc}
    \pgftransformshift{\pgfqpoint{174.0000bp}{294.0000bp}}
    \pgftransformscale{0.5250}
    \pgftext[]{$5$}
  \end{pgfscope}
  \begin{pgfscope}
    \pgfsetlinewidth{1.2000bp}
    \definecolor{sc}{rgb}{0.0000,0.0000,0.0000}
    \pgfsetstrokecolor{sc}
    \pgfsetmiterjoin
    \pgfsetbuttcap
    \pgfpathqmoveto{178.2426bp}{289.7574bp}
    \pgfpathqlineto{193.7574bp}{274.2426bp}
    \pgfusepathqstroke
  \end{pgfscope}
  \begin{pgfscope}
    \definecolor{fc}{rgb}{0.0000,0.0000,0.0000}
    \pgfsetfillcolor{fc}
    \pgfusepathqfill
  \end{pgfscope}
  \begin{pgfscope}
    \definecolor{fc}{rgb}{0.0000,0.0000,0.0000}
    \pgfsetfillcolor{fc}
    \pgfusepathqfill
  \end{pgfscope}
  \begin{pgfscope}
    \definecolor{fc}{rgb}{0.0000,0.0000,0.0000}
    \pgfsetfillcolor{fc}
    \pgfusepathqfill
  \end{pgfscope}
  \begin{pgfscope}
    \definecolor{fc}{rgb}{0.0000,0.0000,0.0000}
    \pgfsetfillcolor{fc}
    \pgfusepathqfill
  \end{pgfscope}
  \begin{pgfscope}
    \pgfsetlinewidth{1.2000bp}
    \definecolor{sc}{rgb}{0.0000,0.0000,0.0000}
    \pgfsetstrokecolor{sc}
    \pgfsetmiterjoin
    \pgfsetbuttcap
    \pgfpathqmoveto{174.0000bp}{288.0000bp}
    \pgfpathqlineto{174.0000bp}{276.0000bp}
    \pgfusepathqstroke
  \end{pgfscope}
  \begin{pgfscope}
    \definecolor{fc}{rgb}{0.0000,0.0000,0.0000}
    \pgfsetfillcolor{fc}
    \pgfusepathqfill
  \end{pgfscope}
  \begin{pgfscope}
    \definecolor{fc}{rgb}{0.0000,0.0000,0.0000}
    \pgfsetfillcolor{fc}
    \pgfusepathqfill
  \end{pgfscope}
  \begin{pgfscope}
    \definecolor{fc}{rgb}{0.0000,0.0000,0.0000}
    \pgfsetfillcolor{fc}
    \pgfusepathqfill
  \end{pgfscope}
  \begin{pgfscope}
    \definecolor{fc}{rgb}{0.0000,0.0000,0.0000}
    \pgfsetfillcolor{fc}
    \pgfusepathqfill
  \end{pgfscope}
  \begin{pgfscope}
    \definecolor{fc}{rgb}{0.0000,0.0000,0.0000}
    \pgfsetfillcolor{fc}
    \pgfsetfillopacity{0.0000}
    \pgfsetlinewidth{1.2000bp}
    \definecolor{sc}{rgb}{0.0000,0.0000,0.0000}
    \pgfsetstrokecolor{sc}
    \pgfsetmiterjoin
    \pgfsetbuttcap
    \pgfpathqmoveto{144.0000bp}{222.0000bp}
    \pgfpathqcurveto{144.0000bp}{225.3137bp}{141.3137bp}{228.0000bp}{138.0000bp}{228.0000bp}
    \pgfpathqcurveto{134.6863bp}{228.0000bp}{132.0000bp}{225.3137bp}{132.0000bp}{222.0000bp}
    \pgfpathqcurveto{132.0000bp}{218.6863bp}{134.6863bp}{216.0000bp}{138.0000bp}{216.0000bp}
    \pgfpathqcurveto{141.3137bp}{216.0000bp}{144.0000bp}{218.6863bp}{144.0000bp}{222.0000bp}
    \pgfpathclose
    \pgfusepathqfillstroke
  \end{pgfscope}
  \begin{pgfscope}
    \definecolor{fc}{rgb}{0.0000,0.0000,0.0000}
    \pgfsetfillcolor{fc}
    \pgftransformshift{\pgfqpoint{138.0000bp}{222.0000bp}}
    \pgftransformscale{0.5250}
    \pgftext[]{$53$}
  \end{pgfscope}
  \begin{pgfscope}
    \definecolor{fc}{rgb}{0.0000,0.0000,0.0000}
    \pgfsetfillcolor{fc}
    \pgfsetfillopacity{0.0000}
    \pgfsetlinewidth{1.2000bp}
    \definecolor{sc}{rgb}{0.0000,0.0000,0.0000}
    \pgfsetstrokecolor{sc}
    \pgfsetmiterjoin
    \pgfsetbuttcap
    \pgfpathqmoveto{144.0000bp}{246.0000bp}
    \pgfpathqcurveto{144.0000bp}{249.3137bp}{141.3137bp}{252.0000bp}{138.0000bp}{252.0000bp}
    \pgfpathqcurveto{134.6863bp}{252.0000bp}{132.0000bp}{249.3137bp}{132.0000bp}{246.0000bp}
    \pgfpathqcurveto{132.0000bp}{242.6863bp}{134.6863bp}{240.0000bp}{138.0000bp}{240.0000bp}
    \pgfpathqcurveto{141.3137bp}{240.0000bp}{144.0000bp}{242.6863bp}{144.0000bp}{246.0000bp}
    \pgfpathclose
    \pgfusepathqfillstroke
  \end{pgfscope}
  \begin{pgfscope}
    \definecolor{fc}{rgb}{0.0000,0.0000,0.0000}
    \pgfsetfillcolor{fc}
    \pgftransformshift{\pgfqpoint{138.0000bp}{246.0000bp}}
    \pgftransformscale{0.5250}
    \pgftext[]{$33$}
  \end{pgfscope}
  \begin{pgfscope}
    \pgfsetlinewidth{1.2000bp}
    \definecolor{sc}{rgb}{0.0000,0.0000,0.0000}
    \pgfsetstrokecolor{sc}
    \pgfsetmiterjoin
    \pgfsetbuttcap
    \pgfpathqmoveto{138.0000bp}{240.0000bp}
    \pgfpathqlineto{138.0000bp}{228.0000bp}
    \pgfusepathqstroke
  \end{pgfscope}
  \begin{pgfscope}
    \definecolor{fc}{rgb}{0.0000,0.0000,0.0000}
    \pgfsetfillcolor{fc}
    \pgfusepathqfill
  \end{pgfscope}
  \begin{pgfscope}
    \definecolor{fc}{rgb}{0.0000,0.0000,0.0000}
    \pgfsetfillcolor{fc}
    \pgfusepathqfill
  \end{pgfscope}
  \begin{pgfscope}
    \definecolor{fc}{rgb}{0.0000,0.0000,0.0000}
    \pgfsetfillcolor{fc}
    \pgfusepathqfill
  \end{pgfscope}
  \begin{pgfscope}
    \definecolor{fc}{rgb}{0.0000,0.0000,0.0000}
    \pgfsetfillcolor{fc}
    \pgfusepathqfill
  \end{pgfscope}
  \begin{pgfscope}
    \definecolor{fc}{rgb}{0.0000,0.0000,0.0000}
    \pgfsetfillcolor{fc}
    \pgfsetfillopacity{0.0000}
    \pgfsetlinewidth{1.2000bp}
    \definecolor{sc}{rgb}{0.0000,0.0000,0.0000}
    \pgfsetstrokecolor{sc}
    \pgfsetmiterjoin
    \pgfsetbuttcap
    \pgfpathqmoveto{120.0000bp}{246.0000bp}
    \pgfpathqcurveto{120.0000bp}{249.3137bp}{117.3137bp}{252.0000bp}{114.0000bp}{252.0000bp}
    \pgfpathqcurveto{110.6863bp}{252.0000bp}{108.0000bp}{249.3137bp}{108.0000bp}{246.0000bp}
    \pgfpathqcurveto{108.0000bp}{242.6863bp}{110.6863bp}{240.0000bp}{114.0000bp}{240.0000bp}
    \pgfpathqcurveto{117.3137bp}{240.0000bp}{120.0000bp}{242.6863bp}{120.0000bp}{246.0000bp}
    \pgfpathclose
    \pgfusepathqfillstroke
  \end{pgfscope}
  \begin{pgfscope}
    \definecolor{fc}{rgb}{0.0000,0.0000,0.0000}
    \pgfsetfillcolor{fc}
    \pgftransformshift{\pgfqpoint{114.0000bp}{246.0000bp}}
    \pgftransformscale{0.5250}
    \pgftext[]{$24$}
  \end{pgfscope}
  \begin{pgfscope}
    \definecolor{fc}{rgb}{0.0000,0.0000,0.0000}
    \pgfsetfillcolor{fc}
    \pgfsetfillopacity{0.0000}
    \pgfsetlinewidth{1.2000bp}
    \definecolor{sc}{rgb}{0.0000,0.0000,0.0000}
    \pgfsetstrokecolor{sc}
    \pgfsetmiterjoin
    \pgfsetbuttcap
    \pgfpathqmoveto{120.0000bp}{270.0000bp}
    \pgfpathqcurveto{120.0000bp}{273.3137bp}{117.3137bp}{276.0000bp}{114.0000bp}{276.0000bp}
    \pgfpathqcurveto{110.6863bp}{276.0000bp}{108.0000bp}{273.3137bp}{108.0000bp}{270.0000bp}
    \pgfpathqcurveto{108.0000bp}{266.6863bp}{110.6863bp}{264.0000bp}{114.0000bp}{264.0000bp}
    \pgfpathqcurveto{117.3137bp}{264.0000bp}{120.0000bp}{266.6863bp}{120.0000bp}{270.0000bp}
    \pgfpathclose
    \pgfusepathqfillstroke
  \end{pgfscope}
  \begin{pgfscope}
    \definecolor{fc}{rgb}{0.0000,0.0000,0.0000}
    \pgfsetfillcolor{fc}
    \pgftransformshift{\pgfqpoint{114.0000bp}{270.0000bp}}
    \pgftransformscale{0.5250}
    \pgftext[]{$25$}
  \end{pgfscope}
  \begin{pgfscope}
    \pgfsetlinewidth{1.2000bp}
    \definecolor{sc}{rgb}{0.0000,0.0000,0.0000}
    \pgfsetstrokecolor{sc}
    \pgfsetmiterjoin
    \pgfsetbuttcap
    \pgfpathqmoveto{118.2426bp}{265.7574bp}
    \pgfpathqlineto{133.7574bp}{250.2426bp}
    \pgfusepathqstroke
  \end{pgfscope}
  \begin{pgfscope}
    \definecolor{fc}{rgb}{0.0000,0.0000,0.0000}
    \pgfsetfillcolor{fc}
    \pgfusepathqfill
  \end{pgfscope}
  \begin{pgfscope}
    \definecolor{fc}{rgb}{0.0000,0.0000,0.0000}
    \pgfsetfillcolor{fc}
    \pgfusepathqfill
  \end{pgfscope}
  \begin{pgfscope}
    \definecolor{fc}{rgb}{0.0000,0.0000,0.0000}
    \pgfsetfillcolor{fc}
    \pgfusepathqfill
  \end{pgfscope}
  \begin{pgfscope}
    \definecolor{fc}{rgb}{0.0000,0.0000,0.0000}
    \pgfsetfillcolor{fc}
    \pgfusepathqfill
  \end{pgfscope}
  \begin{pgfscope}
    \pgfsetlinewidth{1.2000bp}
    \definecolor{sc}{rgb}{0.0000,0.0000,0.0000}
    \pgfsetstrokecolor{sc}
    \pgfsetmiterjoin
    \pgfsetbuttcap
    \pgfpathqmoveto{114.0000bp}{264.0000bp}
    \pgfpathqlineto{114.0000bp}{252.0000bp}
    \pgfusepathqstroke
  \end{pgfscope}
  \begin{pgfscope}
    \definecolor{fc}{rgb}{0.0000,0.0000,0.0000}
    \pgfsetfillcolor{fc}
    \pgfusepathqfill
  \end{pgfscope}
  \begin{pgfscope}
    \definecolor{fc}{rgb}{0.0000,0.0000,0.0000}
    \pgfsetfillcolor{fc}
    \pgfusepathqfill
  \end{pgfscope}
  \begin{pgfscope}
    \definecolor{fc}{rgb}{0.0000,0.0000,0.0000}
    \pgfsetfillcolor{fc}
    \pgfusepathqfill
  \end{pgfscope}
  \begin{pgfscope}
    \definecolor{fc}{rgb}{0.0000,0.0000,0.0000}
    \pgfsetfillcolor{fc}
    \pgfusepathqfill
  \end{pgfscope}
  \begin{pgfscope}
    \definecolor{fc}{rgb}{0.0000,0.0000,0.0000}
    \pgfsetfillcolor{fc}
    \pgfsetfillopacity{0.0000}
    \pgfsetlinewidth{1.2000bp}
    \definecolor{sc}{rgb}{0.0000,0.0000,0.0000}
    \pgfsetstrokecolor{sc}
    \pgfsetmiterjoin
    \pgfsetbuttcap
    \pgfpathqmoveto{96.0000bp}{246.0000bp}
    \pgfpathqcurveto{96.0000bp}{249.3137bp}{93.3137bp}{252.0000bp}{90.0000bp}{252.0000bp}
    \pgfpathqcurveto{86.6863bp}{252.0000bp}{84.0000bp}{249.3137bp}{84.0000bp}{246.0000bp}
    \pgfpathqcurveto{84.0000bp}{242.6863bp}{86.6863bp}{240.0000bp}{90.0000bp}{240.0000bp}
    \pgfpathqcurveto{93.3137bp}{240.0000bp}{96.0000bp}{242.6863bp}{96.0000bp}{246.0000bp}
    \pgfpathclose
    \pgfusepathqfillstroke
  \end{pgfscope}
  \begin{pgfscope}
    \definecolor{fc}{rgb}{0.0000,0.0000,0.0000}
    \pgfsetfillcolor{fc}
    \pgftransformshift{\pgfqpoint{90.0000bp}{246.0000bp}}
    \pgftransformscale{0.5250}
    \pgftext[]{$28$}
  \end{pgfscope}
  \begin{pgfscope}
    \definecolor{fc}{rgb}{0.0000,0.0000,0.0000}
    \pgfsetfillcolor{fc}
    \pgfsetfillopacity{0.0000}
    \pgfsetlinewidth{1.2000bp}
    \definecolor{sc}{rgb}{0.0000,0.0000,0.0000}
    \pgfsetstrokecolor{sc}
    \pgfsetmiterjoin
    \pgfsetbuttcap
    \pgfpathqmoveto{96.0000bp}{270.0000bp}
    \pgfpathqcurveto{96.0000bp}{273.3137bp}{93.3137bp}{276.0000bp}{90.0000bp}{276.0000bp}
    \pgfpathqcurveto{86.6863bp}{276.0000bp}{84.0000bp}{273.3137bp}{84.0000bp}{270.0000bp}
    \pgfpathqcurveto{84.0000bp}{266.6863bp}{86.6863bp}{264.0000bp}{90.0000bp}{264.0000bp}
    \pgfpathqcurveto{93.3137bp}{264.0000bp}{96.0000bp}{266.6863bp}{96.0000bp}{270.0000bp}
    \pgfpathclose
    \pgfusepathqfillstroke
  \end{pgfscope}
  \begin{pgfscope}
    \definecolor{fc}{rgb}{0.0000,0.0000,0.0000}
    \pgfsetfillcolor{fc}
    \pgftransformshift{\pgfqpoint{90.0000bp}{270.0000bp}}
    \pgftransformscale{0.5250}
    \pgftext[]{$13$}
  \end{pgfscope}
  \begin{pgfscope}
    \pgfsetlinewidth{1.2000bp}
    \definecolor{sc}{rgb}{0.0000,0.0000,0.0000}
    \pgfsetstrokecolor{sc}
    \pgfsetmiterjoin
    \pgfsetbuttcap
    \pgfpathqmoveto{90.0000bp}{264.0000bp}
    \pgfpathqlineto{90.0000bp}{252.0000bp}
    \pgfusepathqstroke
  \end{pgfscope}
  \begin{pgfscope}
    \definecolor{fc}{rgb}{0.0000,0.0000,0.0000}
    \pgfsetfillcolor{fc}
    \pgfusepathqfill
  \end{pgfscope}
  \begin{pgfscope}
    \definecolor{fc}{rgb}{0.0000,0.0000,0.0000}
    \pgfsetfillcolor{fc}
    \pgfusepathqfill
  \end{pgfscope}
  \begin{pgfscope}
    \definecolor{fc}{rgb}{0.0000,0.0000,0.0000}
    \pgfsetfillcolor{fc}
    \pgfusepathqfill
  \end{pgfscope}
  \begin{pgfscope}
    \definecolor{fc}{rgb}{0.0000,0.0000,0.0000}
    \pgfsetfillcolor{fc}
    \pgfusepathqfill
  \end{pgfscope}
  \begin{pgfscope}
    \definecolor{fc}{rgb}{0.0000,0.0000,0.0000}
    \pgfsetfillcolor{fc}
    \pgfsetfillopacity{0.0000}
    \pgfsetlinewidth{1.2000bp}
    \definecolor{sc}{rgb}{0.0000,0.0000,0.0000}
    \pgfsetstrokecolor{sc}
    \pgfsetmiterjoin
    \pgfsetbuttcap
    \pgfpathqmoveto{72.0000bp}{270.0000bp}
    \pgfpathqcurveto{72.0000bp}{273.3137bp}{69.3137bp}{276.0000bp}{66.0000bp}{276.0000bp}
    \pgfpathqcurveto{62.6863bp}{276.0000bp}{60.0000bp}{273.3137bp}{60.0000bp}{270.0000bp}
    \pgfpathqcurveto{60.0000bp}{266.6863bp}{62.6863bp}{264.0000bp}{66.0000bp}{264.0000bp}
    \pgfpathqcurveto{69.3137bp}{264.0000bp}{72.0000bp}{266.6863bp}{72.0000bp}{270.0000bp}
    \pgfpathclose
    \pgfusepathqfillstroke
  \end{pgfscope}
  \begin{pgfscope}
    \definecolor{fc}{rgb}{0.0000,0.0000,0.0000}
    \pgfsetfillcolor{fc}
    \pgftransformshift{\pgfqpoint{66.0000bp}{270.0000bp}}
    \pgftransformscale{0.5250}
    \pgftext[]{$77$}
  \end{pgfscope}
  \begin{pgfscope}
    \definecolor{fc}{rgb}{0.0000,0.0000,0.0000}
    \pgfsetfillcolor{fc}
    \pgfsetfillopacity{0.0000}
    \pgfsetlinewidth{1.2000bp}
    \definecolor{sc}{rgb}{0.0000,0.0000,0.0000}
    \pgfsetstrokecolor{sc}
    \pgfsetmiterjoin
    \pgfsetbuttcap
    \pgfpathqmoveto{72.0000bp}{294.0000bp}
    \pgfpathqcurveto{72.0000bp}{297.3137bp}{69.3137bp}{300.0000bp}{66.0000bp}{300.0000bp}
    \pgfpathqcurveto{62.6863bp}{300.0000bp}{60.0000bp}{297.3137bp}{60.0000bp}{294.0000bp}
    \pgfpathqcurveto{60.0000bp}{290.6863bp}{62.6863bp}{288.0000bp}{66.0000bp}{288.0000bp}
    \pgfpathqcurveto{69.3137bp}{288.0000bp}{72.0000bp}{290.6863bp}{72.0000bp}{294.0000bp}
    \pgfpathclose
    \pgfusepathqfillstroke
  \end{pgfscope}
  \begin{pgfscope}
    \definecolor{fc}{rgb}{0.0000,0.0000,0.0000}
    \pgfsetfillcolor{fc}
    \pgftransformshift{\pgfqpoint{66.0000bp}{294.0000bp}}
    \pgftransformscale{0.5250}
    \pgftext[]{$12$}
  \end{pgfscope}
  \begin{pgfscope}
    \pgfsetlinewidth{1.2000bp}
    \definecolor{sc}{rgb}{0.0000,0.0000,0.0000}
    \pgfsetstrokecolor{sc}
    \pgfsetmiterjoin
    \pgfsetbuttcap
    \pgfpathqmoveto{71.3677bp}{291.3161bp}
    \pgfpathqlineto{108.6323bp}{272.6839bp}
    \pgfusepathqstroke
  \end{pgfscope}
  \begin{pgfscope}
    \definecolor{fc}{rgb}{0.0000,0.0000,0.0000}
    \pgfsetfillcolor{fc}
    \pgfusepathqfill
  \end{pgfscope}
  \begin{pgfscope}
    \definecolor{fc}{rgb}{0.0000,0.0000,0.0000}
    \pgfsetfillcolor{fc}
    \pgfusepathqfill
  \end{pgfscope}
  \begin{pgfscope}
    \definecolor{fc}{rgb}{0.0000,0.0000,0.0000}
    \pgfsetfillcolor{fc}
    \pgfusepathqfill
  \end{pgfscope}
  \begin{pgfscope}
    \definecolor{fc}{rgb}{0.0000,0.0000,0.0000}
    \pgfsetfillcolor{fc}
    \pgfusepathqfill
  \end{pgfscope}
  \begin{pgfscope}
    \pgfsetlinewidth{1.2000bp}
    \definecolor{sc}{rgb}{0.0000,0.0000,0.0000}
    \pgfsetstrokecolor{sc}
    \pgfsetmiterjoin
    \pgfsetbuttcap
    \pgfpathqmoveto{70.2426bp}{289.7574bp}
    \pgfpathqlineto{85.7574bp}{274.2426bp}
    \pgfusepathqstroke
  \end{pgfscope}
  \begin{pgfscope}
    \definecolor{fc}{rgb}{0.0000,0.0000,0.0000}
    \pgfsetfillcolor{fc}
    \pgfusepathqfill
  \end{pgfscope}
  \begin{pgfscope}
    \definecolor{fc}{rgb}{0.0000,0.0000,0.0000}
    \pgfsetfillcolor{fc}
    \pgfusepathqfill
  \end{pgfscope}
  \begin{pgfscope}
    \definecolor{fc}{rgb}{0.0000,0.0000,0.0000}
    \pgfsetfillcolor{fc}
    \pgfusepathqfill
  \end{pgfscope}
  \begin{pgfscope}
    \definecolor{fc}{rgb}{0.0000,0.0000,0.0000}
    \pgfsetfillcolor{fc}
    \pgfusepathqfill
  \end{pgfscope}
  \begin{pgfscope}
    \pgfsetlinewidth{1.2000bp}
    \definecolor{sc}{rgb}{0.0000,0.0000,0.0000}
    \pgfsetstrokecolor{sc}
    \pgfsetmiterjoin
    \pgfsetbuttcap
    \pgfpathqmoveto{66.0000bp}{288.0000bp}
    \pgfpathqlineto{66.0000bp}{276.0000bp}
    \pgfusepathqstroke
  \end{pgfscope}
  \begin{pgfscope}
    \definecolor{fc}{rgb}{0.0000,0.0000,0.0000}
    \pgfsetfillcolor{fc}
    \pgfusepathqfill
  \end{pgfscope}
  \begin{pgfscope}
    \definecolor{fc}{rgb}{0.0000,0.0000,0.0000}
    \pgfsetfillcolor{fc}
    \pgfusepathqfill
  \end{pgfscope}
  \begin{pgfscope}
    \definecolor{fc}{rgb}{0.0000,0.0000,0.0000}
    \pgfsetfillcolor{fc}
    \pgfusepathqfill
  \end{pgfscope}
  \begin{pgfscope}
    \definecolor{fc}{rgb}{0.0000,0.0000,0.0000}
    \pgfsetfillcolor{fc}
    \pgfusepathqfill
  \end{pgfscope}
\end{pgfpicture}

  \end{center}
  \label{binomial-heap}
\end{model}

\begin{questions}
\item Two of the binomial heaps shown above are invalid, and one is
  valid.  Which is which?  Cross out the invalid ones.

\item How many total nodes does the valid binomial heap in Model
  \ref{binomial-heap} contain?

\item Draw a valid binomial heap with\dots
  \begin{subquestions}
  \item 5 nodes
  \item 8 nodes
  \item 11 nodes
  \end{subquestions}

\item The table below displays total number of nodes ($n$) across the
  top row; each column records the number of trees of each order
  needed to make a binomial heap of a certain size.  The number of
  trees of each order needed to make binomial heaps of sizes $0$, $1$,
  and $2$ have already been filled in for you.  Fill in the rest of
  the table. \bigskip

  \begin{tabular}{c|ccccccccc}
    $n$ & $0$ & $1$ & $2$ & $3$ & $4$ & $5$ & $6$ & $7$ & $8$ \\[8pt]
    Order 0 trees & $0$ & $1$ & $0$ & & & & & & \\[8pt]
    Order 1 trees & $0$ & $0$ & $1$ & & & & & & \\[8pt]
    Order 2 trees & $0$ & $0$ & $0$ & & & & & & \\[8pt]
    Order 3 trees & $0$ & $0$ & $0$ & & & & & & \\[8pt]
  \end{tabular}

\newpage
\item What is the relationship between the number of elements in a
  binomial heap and the orders of the binomial trees that it contains?

\item If a binomial heap has a total of $n$ elements, what is the
  maximum number of binomial trees it could contain?

% \item The \verb|merge| operation takes two binomial heaps as parameters and returns a new binomial heap containing all of the values from the first two heaps. If two binomial heaps each have one tree of order 0, explain how \verb|merge| will create a new binomial heap.

% \item Next, imagine that we want to \verb|merge| two binomial heaps, one of which has an order 0 binomial tree, the other of which has an order 1 binomial tree. Explain how \verb|merge| will create a new binomial heap.

% \item Now imagine that we want to \verb|merge| two binomial heaps, each of which has an order 0 binomial tree and also an order 1 binomial tree. Explain how \verb|merge| will create a new binomial heap.

% \item Based on the insights gained from the previous three questions, write pseudocode for \verb|merge|.

% \item What is the relationship between the \verb|merge| algorithm and the binary counter from the \emph{Introduction to Amortized Analysis} POGIL activity? \label{counter}

% \item What is the best-case execution time for one call to \verb|merge|?

% \item What is the worst-case execution time for one call to \verb|merge|?

% \item What is the worst-case amortized execution time for $n$ calls to \verb|merge|? Feel free to use your answer to Question \ref{counter} to support your answer to this question. \label{amortizedN}

% \item Based on your answer to Question \ref{amortizedN}, what is the worst-case amortized execution time for one call to \verb|merge|?

% \item Priority queues have three main operations:
% \begin{itemize}
%     \item INSERT: Insert a new value into the heap.
%     \item FIND-MIN: Find the smallest value in the heap.
%     \item DELETE-MIN: Remove the smallest value from the heap.
% \end{itemize}

% Devise an algorithm for INSERT that uses \verb|merge| as a subroutine. 

% \item What is the amortized worst-case execution time for INSERT?

% \item Devise an algorithm for DELETE that uses \verb|merge| as a subroutine. 

% \item What is its worst-case execution time for DELETE?

% \item Devise a constant-time algorithm for FIND-MIN. You may want to make a small modification to the binomial heap data structure to ensure that it runs in constant time.

% \item What advantages does a binomial heap have in comparison to the standard binary heap?

% \item What advantages does a standard binary heap have in comparison to a binomial heap?

\end{questions}

\end{document}
