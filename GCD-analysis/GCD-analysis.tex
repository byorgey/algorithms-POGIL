% -*- compile-command: "rubber --unsafe -d GCD-analysis.tex" -*-

% History:
%   - 8a79ce3: used Fri, 25 Aug, 2017.

\documentclass{tufte-handout}

\usepackage{algo-activity}
\usepackage{acode}

\title{\thecourse: GCD analysis}
\date{}

\begin{document}

\maketitle

\section{Review questions (7 minutes)}

\begin{mdframed}
  \emph{Notice that section headings will often contain a target time,
like the one above.  The manager should set a timer to help
ensure your group is making good progress through the section in order
to finish on time.}
\end{mdframed}

\begin{questions}
  \item What is $27 \bmod 5$?
  \item What is $2 \bmod 5$?
  \item Which of the following statements is always true, assuming that $a$
    and $b$ are positive integers?
    \begin{itemize}
    \item $0 \leq a \bmod b < b$
    \item $0 \leq a \bmod b < a$
    \end{itemize}
  \item What is $5 \bmod 0$?
  \item Is $0$ divisible by $10$?
\end{questions}
\vfill
\begin{mdframed}[innerrightmargin=1cm]
  \emph{When you see a STOP sign like the one below, it means that you
    should wait until instructed to go on to the next page.  If you
    finish before other groups, take the time to go back over your
    answers on this section: are there any lingering questions or
    confusions you have? Do all team members agree and understand your
    answers?  Are there answers you could make more complete or more
    nuanced?  Anything you notice, or wonder about?}
\end{mdframed}
\pause

\begin{model*}{GCD (8 minutes)}{gcd}
\begin{defn}
  Recall that the \term{greatest common divisor}, or GCD, of two
  positive integers $a$ and $b$ is defined as the
  largest\footnotemark\ positive integer which evenly divides both $a$
  and $b$.  The GCD of $a$ and $b$ is denoted $\gcd(a,b)$.
\end{defn}
\end{model*}
\marginnote{\footnotemark[1]\ You may recall that technically, it's
    defined as the \emph{highest in the divisibility lattice}, not
    actually the largest; but for positive integers it amounts to the
    same thing.}

\begin{questions}
\item What is $\gcd(12,30)$?
\item What are the prime factorizations of $12$ and $30$?
\item What do the prime factorizations of $12$ and $30$ have to do
  with $\gcd(12,30)$?
\item Find the prime factorizations of $144$ and $690$ and use them to
  compute $\gcd(144, 690)$.
\item What if we extend the definition of GCD to apply to all
  nonnegative integers?  What should $\gcd(a,0)$ be when $a > 0$?
\item Have the Reflector take a minute to share one way that your
  group has worked well together, and one way that you could improve.
\end{questions}

\pause

\begin{model*}{The Euclidean Algorithm}{euclidean-algorithm}

  Consider the four algorithms specified below. They are all supposed
  to compute the GCD of nonnegative integers, but only two of them are
  correct. \bigskip

\begin{minipage}[t]{.5\textwidth}
\begin{acode}
\> GCDIterA($m$,$n$) = \\
\> \tb \> $a \gets m$ \\
\>     \> $b \gets n$ \\
\>     \> !while $(a \neq 0)$ \\
\>     \> \tb \> !if $a \leq b$ \\
\>     \>     \> \tb \> !then \> $b \gets b \bmod a$ \\
\>     \>     \>     \> !else \> $a \gets a \bmod b$ \\
\>     \> !if $a = 0$ !then !return $b$ !else !return $a$
\end{acode} \bigskip

\begin{acode}
\> GCDRecA($a$,$b$) = \\
\> \tb \> !if $b = 0$ \\
\>     \> \tb \> !then \> $a$ \\
\>     \>     \> !else \> GCDRecA($b$, $a \bmod b$)
\end{acode}
\end{minipage}
\begin{minipage}[t]{.5\textwidth}
\begin{acode}
\> GCDIterB($m$,$n$) = \\
\> \tb \> $a \gets m$ \\
\>     \> $b \gets n$ \\
\>     \> !while $(a \neq 0)$ !and $(b \neq 0)$ \\
\>     \> \tb \> !if $a \leq b$ \\
\>     \>     \> \tb \> !then \> $b \gets b \bmod a$ \\
\>     \>     \>     \> !else \> $a \gets a \bmod b$ \\
\>     \> !if $a = 0$ !then !return $b$ !else !return $a$
\end{acode} \bigskip

\begin{acode}
\> GCDRecB($a$,$b$) = \\
\> \tb \> !if $b = 0$ \\
\>     \> \tb \> !then \> $a$ \\
\>     \>     \> !else \> GCDRecB($a \bmod b$, $b$)
\end{acode}
\end{minipage}
\end{model*}

\begin{questions}
\item \label{q:trace} Trace the execution of each algorithm on the
  inputs $(144,690)$. \emph{It may be tempting to split these up and
    have one person do each independently.  However, that rarely makes
    things any faster, and I recommend resisting that temptation.
    Instead, work together as a group on one at a time, and build a
    shared understanding of what is going on.}
\end{questions}

\newpage
\begin{questions}
\item What do you think \textsf{Iter} and \textsf{Rec} stand for in
  the names of the functions?
\item List some similarities and differences among the algorithms.
\item Which algorithms are incorrect?  What is wrong with them?
\item For the correct algorithms, describe in a few sentences what
  happened to the values of $a$ and $b$ as the algorithm ran.  Can you
  explain why the algorithms will always stop eventually?
\item Look at one of your execution traces from \pref{q:trace}.  Find
  the $\gcd$ of $a$ and $b$ after each iteration of the algorithm.
  What do you notice?
\end{questions}

\newpage

\section{Facilitation plan}
\label{sec:facilitation}

\begin{itemize}
\item Review questions: 7 mins + 3 mins for discussion re: 5 mod 0.

\item After Model 1 (8 mins), just share out very quickly to make sure
  everyone is on the same page.

\item Model 2 (remaining time).

\end{itemize}


\end{document}
