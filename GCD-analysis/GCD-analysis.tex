% -*- compile-command: "rubber --unsafe -d GCD-analysis.tex" -*-

% History:
%   - 8a79ce3: used Fri, 25 Aug, 2017.

\documentclass{tufte-handout}

\usepackage{algo-activity}
\usepackage{acode}

\title{\thecourse: GCD analysis}
\date{}

\begin{document}

\maketitle

\section{Review questions}

\begin{questions}
  \item What is $27 \bmod 5$?
  \item What is $2 \bmod 5$?
  \item Which of the following statements is true, assuming that $a$
    and $b$ are positive integers?
    \begin{itemize}
    \item $0 \leq a \bmod b < b$
    \item $0 \leq a \bmod b < a$
    \end{itemize}
  \item What is $5 \bmod 0$?
  \item Is $0$ divisible by $10$?
\end{questions}

\newpage
\begin{model*}{GCD}{gcd}
\begin{defn}
  Recall that the \term{greatest common divisor}, or GCD, of two
  positive integers $a$ and $b$ is defined as the largest positive
  integer which evenly divides both $a$ and $b$.  The GCD of $a$ and
  $b$ is denoted $\gcd(a,b)$.
\end{defn}
\end{model*}

\begin{questions}
\item What is $\gcd(12,30)$?
\item What are the prime factorizations of $12$ and $30$?
\item What do the prime factorizations of $12$ and $30$ have to do
  with $\gcd(12,30)$?
\item What is $\gcd(144, 690)$?
\item What if we extend the definition of GCD to apply to all
  nonnegative integers?  What should $\gcd(a,0)$ be when $a > 0$?
\end{questions}

\newpage

\begin{model*}{The Euclidean Algorithm}{euclidean-algorithm}

  Consider the four algorithms specified below. They are all supposed
  to compute the GCD of nonnegative integers, but only two of them are
  correct. \bigskip

\begin{minipage}[t]{.5\textwidth}
\begin{acode}
\> GCDIa($m$,$n$) = \\
\> \tb \> $a \gets m$ \\
\>     \> $b \gets n$ \\
\>     \> !while $(a \neq 0)$ \\
\>     \> \tb \> !if $a \leq b$ \\
\>     \>     \> \tb \> !then \> $b \gets b \bmod a$ \\
\>     \>     \>     \> !else \> $a \gets a \bmod b$ \\
\>     \> !if $a = 0$ !then !return $b$ !else !return $a$
\end{acode} \bigskip

\begin{acode}
\> GCDRa($a$,$b$) = \\
\> \tb \> !if $b = 0$ \\
\>     \> \tb \> !then \> $a$ \\
\>     \>     \> !else \> GCDRa($b$, $a \bmod b$)
\end{acode}
\end{minipage}
\begin{minipage}[t]{.5\textwidth}
\begin{acode}
\> GCDIb($m$,$n$) = \\
\> \tb \> $a \gets m$ \\
\>     \> $b \gets n$ \\
\>     \> !while $(a \neq 0)$ !and $(b \neq 0)$ \\
\>     \> \tb \> !if $a \leq b$ \\
\>     \>     \> \tb \> !then \> $b \gets b \bmod a$ \\
\>     \>     \>     \> !else \> $a \gets a \bmod b$ \\
\>     \> !if $a = 0$ !then !return $b$ !else !return $a$
\end{acode} \bigskip

\begin{acode}
\> GCDRb($a$,$b$) = \\
\> \tb \> !if $b = 0$ \\
\>     \> \tb \> !then \> $a$ \\
\>     \>     \> !else \> GCDRb($a \bmod b$, $b$)
\end{acode}
\end{minipage}
\end{model*}

\begin{questions}
\item \label{q:trace} Trace the execution of each algorithm on the inputs
  $(144,690)$.
\end{questions}

\newpage
\begin{questions}
\item What do you think the \textsf{I} and \textsf{R} stand for in
  \textsf{GCDI} and \textsf{GCDR}?
\item List some similarities and differences among the algorithms.
\item Which algorithms are incorrect?  What is wrong with them?
\item For the correct algorithms, describe in a few sentences what
  happened to the values of $a$ and $b$ as the algorithm ran.  Can you
  explain why the algorithms will always stop eventually?
\item Look at one of your execution traces from \pref{q:trace}.  Find
  the $\gcd$ of $a$ and $b$ after each iteration of the algorithm.
  What do you notice?
\end{questions}

\end{document}
