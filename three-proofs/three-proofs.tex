% -*- compile-command: "stack exec -- rubber -d --unsafe three-proofs.tex" -*-
\documentclass{tufte-handout}

\usepackage{algo-activity}

\title{\thecourse: Some asymptotic sums}
\date{}

\begin{document}

\maketitle

\begin{model}{Three proofs}{proofs}
\begin{center}
\begin{diagram}[width=200]
  proofDots :: Int -> Diagram B
  proofDots n =
    [ [(x,y) | x <- [0 .. n-1] ] | y <- [n-1, n-2 .. 0] ]
    # map (map drawPt)
    # map (hcat' opts)
    # vcat' opts
    where
      opts = with & catMethod .~ Distrib & sep .~ 1
      drawPt (x,y)
        | y > x     = circle 0.1
        | x >= n `div` 2 && y < n `div` 2 = circle 0.2 # fc black
        | otherwise = circle 0.1 # fc black

  dia :: Diagram B
  dia = [proofDots 4, proofDots 8]
    # map centerY
    # hsep 3
\end{diagram}
\end{center} \bigskip

\vspace{0.7in}

\[ \begin{array}{ccccccccccc}
     1 &+& 2 &+& 3 &+& \dots &+& (n-1) &+& n \\
     n &+& (n-1) &+& (n-2) &+& \dots &+& 2 &+& 1 \\
     \hline
     (n+1) &+& (n+1) &+& (n+1) &+& \dots &+& (n+1) &+& (n+1)
   \end{array}
\]

\vspace{0.7in}

\begin{align}
  1 + 2 + \dots + n &< n + n + \dots + n \label{eq:ltns} \\
                    &= n^2 \label{eq:nsqr} \\[1em]
  1 + 2 + \dots + n &> n/2 + (n/2 + 1) + \dots + n \label{eq:halfzero} \\
                    &> n/2 + n/2 + \dots +
                      n/2 \label{eq:downtohalves} \\
                    &= (n/2)^2 \label{eq:sqrhalf} \\
                    &= n^2/4 \label{eq:quartersqr}
\end{align}
\end{model}

\begin{objective}
  Students will understand and prove the asymptotic behavior of $1 + 2
  + 3 + \dots + n$..
\end{objective}

\begin{objective}
  Students will apply geometric, algebraic, and inequational reasoning
  to asymptotic behavior.
\end{objective}

The first row of \pref{model:proofs} actually shows two similar
diagrams at different sizes, one $4 \times 4$ and one $8 \times 8$.
Each diagram consists of a bunch of dots---some hollow and some filled;
and the filled dots come in two varieties, big and small.
\begin{questions}
  \item How many dots are there in total in the first diagram?  In the
    second diagram?
  \item How many big dots are there (\ie the lower-right square) in the
    first diagram?  How many are in the second?
  \item How many filled dots are there in total (both big and small
    filled dots, \ie the lower-right triangle) in the first diagram?
    In the second?
\end{questions}
Now suppose that we abstract away the specific sizes of the diagrams
and imagine a generic $n \times n$ version of the same diagram.  To
make things slightly simpler, assume that $n$ is even.
\begin{questions}
\item In terms of $n$, how many dots would there be in total?
\item In terms of $n$, how many big dots would there be in the lower right?
\item Explain why the number of filled-in dots is equal to \[ 1 + 2 + 3
  + \dots + n. \]
\item Based on the diagrams, what can you say about the relationship
  between these three quantities?
  \newpage
\item In terms of concepts you explored on the previous activity, what
  does this prove about the sum $1 + 2 + 3 + \dots + n$?
\end{questions}
Now consider the second proof.
\begin{questions}
\item Notice that the top row is our friend $1 + 2 + 3 + \dots + n$.
  What is the second row?
\item Why does the bottom row consist of copies of $(n+1)$?
\item What is the sum of the bottom row?
\item Use this to derive a formula for $1 + 2 + \dots + n$ in
  terms of $n$.
\item What does this formula imply about the asymptotic behavior of $1
  + 2 + \dots + n$?  Justify your answer.
\end{questions}

Finally, consider the third proof.  Surprise!---once again it has to
do with the sum $1 + 2 + 3 + \dots + n$.  For this proof we will
again assume $n$ is even.\footnote{It is not hard to fix the proof to work
  for odd $n$ as well, but the details would end up obscuring the main
  idea somewhat.}
\begin{questions}
  \item Why is step \eqref{eq:ltns} true?
  \item Why is the right-hand side of \eqref{eq:ltns} equal to
    \eqref{eq:nsqr} ?
  \item What does this prove about $1 + 2 + \dots + n$?
  \item Now, what is happening in step \eqref{eq:halfzero}?
  \item Why is the right-hand side of \eqref{eq:halfzero} greater than
    \eqref{eq:downtohalves}?
  \item Why is \eqref{eq:downtohalves} equal to \eqref{eq:sqrhalf}?
  \item What does this prove about $1 + 2 + \dots + n$?
  \item Two of these three proofs are in some sense the same.  Which two?
  \end{questions}

\end{document}
