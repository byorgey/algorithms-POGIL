
\newpage
\begin{model*}{The Cut Property (15 mins)}{cut-property}
  \begin{defn}
    A \term{cut} in a graph $G = (V,E)$ is a partition of the vertices
    $V$ into two sets $S$ and $T$, that is, every vertex is in either
    $S$ or $T$ but not both.  We say that an edge $e$ \term{crosses}
    the cut $(S,T)$ if one vertex of $e$ is in $S$ and the other is in
    $T$.
  \end{defn}

  \begin{thm}[Cut Property]
    Given a weighted, undirected graph $G = (V,E)$, let $S$ and $T$ be
    any partition of $V$, and suppose $e$ is the smallest-weight edge
    crossing the $(S,T)$ cut.  Then every minimum spanning tree of $G$
    must include $e$.
  \end{thm}
\end{model*}

\begin{questions}
\item Give three examples of cuts in the graph from \pref{model:mwss}
  and identify the smallest edge crossing each cut.
\end{questions}

Let's prove the cut property.

\begin{proof}
  Let $G$ be a weighted, undirected graph $G = (V,E)$, let $S$ and $T$
  be an arbitrary partition of $V$ into two sets, and suppose
  $e = (x,y)$ is the smallest-weight edge with one endpoint in $S$ and
  one in $T$.  We wish to show that \emph{every MST of $G$ must
    include $e$}.

  Suppose $M$ is a spanning tree of $G$ which does \textbf{not}
  contain the edge $e$.  Since $M$ is a \emph{tree} it contains a unique
  \emph{path} between any two \emph{vertices}. So consider the unique
  \emph{path} in $M$ between \emph{$x$ and $y$}.  \marginnote{\emph{Hint}:
    draw a picture!} It must cross
  the cut at least once since \emph{$x \in S$ and $y \in T$}; suppose it
  crosses at $e' = (x',y')$, with $x' \in X$ and $y' \in Y$.
  We know that $w_e < w_{e'}$ since \emph{we assumed $e$ is the
    smallest-weight edge crossing the cut}. Now take $M$ and
  replace $e'$ with $e$; the resulting graph is still
  \emph{connected} because \emph{any path that went through $e'$ can
    detour through $e$}, but it has a smaller total \emph{weight}
  because \emph{we removed $e'$ and replaced it with $e$ which has a
    smaller weight}.

  So, we have shown that any spanning tree $M$ which does not contain
  the edge $e$ can be made into a \emph{smaller-weight spanning tree},
  which means that $M$ is not a \emph{minimum spanning tree}. This
  proves what we wanted to show, since \emph{any tree not containing
    $e$ is not an MST}.
\end{proof}

The cut property can be used to directly show the correctness of
several MST algorithms.  Let's prove the correctness of Kruskal's
Algorithm; the proofs for the other algorithms are similar.

\begin{thm}
  Kruskal's Algorithm is correct.
\end{thm}

\begin{proof}
  Suppose at some step the algorithm picks the edge $e = (x,y)$.  Let
  $X$ be the set of vertices connected to $x$ by edges which have been
  picked so far (not including $e$), and let $Y$ be all other
  vertices. $x \in X$ by definition.  We know that $y \notin X$ since
  if it was, $e$ would \emph{create a cycle} but then Kruskal's
  Algorithm wouldn't \emph{pick it}. $e$ therefore crosses the cut
  $(X,Y)$. No other edges which have been picked previously cross the
  cut, since \emph{by definition all the edges go between vertices in
    $X$}. Therefore $e$ must be the \emph{smallest edge crossing the
    cut} because \emph{the edges are chosen in order from smallest to
    biggest weight}. Therefore by the Cut Property $e$ must be in any
  MST and Kruskal's Algorithm is correct to pick it.
\end{proof}
