% -*- compile-command: "./build.sh" -*-
\documentclass{tufte-handout}

\usepackage{algo-activity}
\usepackage{algorithm, algorithmicx}
\usepackage[noend]{algpseudocode}

\title{\thecourse: Kruskal's Algorithm (key)}
\date{}

\begin{document}

\maketitle

In the previous activity you learned about minimum spanning trees and
experimented with several different algorithms for finding them.  In
today's activity we will focus on Kruskal's Algorithm and prove that
it works correctly.

\begin{model*}{Kruskal's Algorithm (12 mins)}{kruskal}
  \begin{center}
    \begin{diagram}[width=300]
      {-# LANGUAGE ViewPatterns #-}

      import GraphDiagrams
      import Data.Bifunctor
      import System.Random

      perturb :: P2 Double -> IO (P2 Double)
      perturb (coords -> (x :& y)) = do
        dx <- randomRIO (-0.3, 0.3)
        dy <- randomRIO (-0.3, 0.3)
        return ((x + dx) ^& (y + dy))

      mvertices :: [(Char, P2 Double)]
      mvertices =
        [ ('a', 1 ^& 2)
        , ('b', 0 ^& 1)
        , ('c', 1 ^& 0)
        , ('d', 2 ^& 1)
        , ('i', 3 ^& 0)
        , ('f', 3 ^& 2)
        , ('e', 4 ^& 1)
        , ('j', 4 ^& (-1))
        , ('g', 5 ^& 2)
        , ('h', 5 ^& 0)
        ]

      mvertices' :: IO [(Char, P2 Double)]
      mvertices' = (traverse . traverse) perturb mvertices

      medges =
        [ (('a','b'), 3)
        , (('b','c'), 17)
        , (('b','d'), 16)
        , (('a','f'), 2)
        , (('c','d'), 8)
        , (('c','i'), 18)
        , (('d','i'), 4)
        , (('d','e'), 11)
        , (('i','e'), 10)
        , (('f','e'), 1)
        , (('f','g'), 7)
        , (('e','g'), 6)
        , (('e','h'), 5)
        , (('i','h'), 12)
        , (('i','j'), 9)
        , (('j','h'), 13)
        , (('g','h'), 15)
        ]

      g :: IO (Graph Char Int)
      g = do
        vs <- mvertices'
        return $ wgraph vs medges

      dia :: IO (Diagram B)
      dia = drawGraph (drawVTeX . first (:[])) (drawUWE (tex . show)) <$> g
  \end{diagram}

  \end{center}

  \begin{algorithm}[H]
    \begin{algorithmic}[1]
      \Require Undirected, weighted graph $G = (V,E)$
      \State $T \gets \varnothing$   \Comment{$T$ holds the set of
        edges in the MST}
      \State Sort $E$ from smallest to biggest weight
      \For{each edge $e \in E$}
        \If{$e$ does not make a cycle with other edges in $T$}
          \State Add $e$ to $T$
        \EndIf
      \EndFor
    \end{algorithmic}
    \label{alg:kruskal}
  \end{algorithm}
\end{model*}

\begin{questions}
\item Simulate Kruskal's Algorithm on the graph in
  \pref{model:kruskal}.  What is the total weight of the resulting
  spanning tree?

  \emph{All edges from 1--10 except edge 7 are included; the total
    weight is $48$.}
\item The way the algorithm is written in \pref{model:kruskal}, one
  must iterate through every single edge in $E$.  However, this is not
  always necessary.  Can you think of a simple way to tell when we can
  stop the loop early?

  \emph{Since a tree with $V$ vertices has exactly $V-1$ edges, we can
  just count how many edges have been picked so far and stop when we
  get to $V-1$.}
\item Explain why even in the worst case,
  $\Theta(\lg V) = \Theta(\lg E)$ in any graph.

  \emph{In the worst case, $E$ is $O(V^2)$, in which case $\Theta(\lg
    E) = \Theta(\lg V^2) = \Theta(2 \lg V) = \Theta(\lg V)$.}
\item In the above algorithm, how long does line 2 take?  Simplify
  your answer using the observation from the previous question.

  \emph{We are sorting a list of $E$ edges; if we use an efficient
    $\Theta(n \lg n)$ sorting algorithm, it will take $\Theta(E \lg E)
    = \Theta(E \lg V)$ time.}
\item Can you think of a way to implement line 4?  How long would it
  take?

  \emph{Do a DFS (or BFS) in $T$ from one endpoint of $e$.  $e$
    makes a cycle if and only if the other endpoint of $e$ is
    reachable.  This would take $\Theta(V)$ time (BFS or DFS in
    general take $\Theta(V+E)$ time, but in this case since $T$ has no
    cycles it can't have more edges than vertices).  Hence the total
    time for the whole algorithm would be $\Theta(E \lg V) + \Theta(E
    \cdot V) = \Theta(VE)$.}
\end{questions}

\pause

\begin{model*}{The Cut Property (20 mins)}{cut-property}
  \begin{center}
    \begin{diagram}[width=300]
      {-# LANGUAGE ViewPatterns #-}

      import GraphDiagrams
      import Data.Bifunctor
      import System.Random

      perturb :: P2 Double -> IO (P2 Double)
      perturb (coords -> (x :& y)) = do
        dx <- randomRIO (-0.3, 0.3)
        dy <- randomRIO (-0.3, 0.3)
        return ((x + dx) ^& (y + dy))

      mvertices :: [(Char, P2 Double)]
      mvertices =
        [ ('a', 1 ^& 2)
        , ('b', 0 ^& 1)
        , ('c', 1 ^& 0)
        , ('d', 2 ^& 1)
        , ('i', 3 ^& 0)
        , ('f', 3 ^& 2)
        , ('e', 4 ^& 1)
        , ('j', 4 ^& (-1))
        , ('g', 5 ^& 2)
        , ('h', 5 ^& 0)
        ]

      mvertices' :: IO [(Char, P2 Double)]
      mvertices' = (traverse . traverse) perturb mvertices

      medges =
        [ (('a','b'), 3)
        , (('b','c'), 17)
        , (('b','d'), 16)
        , (('a','f'), 2)
        , (('c','d'), 8)
        , (('c','i'), 18)
        , (('d','i'), 4)
        , (('d','e'), 11)
        , (('i','e'), 10)
        , (('f','e'), 1)
        , (('f','g'), 7)
        , (('e','g'), 6)
        , (('e','h'), 5)
        , (('i','h'), 12)
        , (('i','j'), 9)
        , (('j','h'), 13)
        , (('g','h'), 15)
        ]

      g :: IO (Graph Char Int)
      g = do
        vs <- mvertices'
        return $ wgraph vs medges

      dia :: IO (Diagram B)
      dia = drawGraph (drawVTeX . first (:[])) (drawUWE (tex . show)) <$> g
  \end{diagram}

  \end{center}

  \begin{defn}
    A \term{cut} in a graph $G = (V,E)$ is a partition of the vertices
    $V$ into two sets $S$ and $T$, that is, every vertex is in either
    $S$ or $T$ but not both.  We say that an edge $e$ \term{crosses}
    the cut $(S,T)$ if one vertex of $e$ is in $S$ and the other is in
    $T$.
  \end{defn}

  \begin{thm}[Cut Property]
    Given a weighted, undirected graph $G = (V,E)$, let $S$ and $T$ be
    any partition of $V$, and suppose $e$ is some edge crossing the
    $(S,T)$ cut, such that the weight of $e$ is strictly smaller than
    the weight of any other edge crossing the $(S,T)$ cut.  Then every
    minimum spanning tree of $G$ must include $e$.
  \end{thm}
\end{model*}

\begin{questions}
\item Give three examples of cuts in the graph from
  \pref{model:cut-property} and identify the smallest edge crossing
  each cut.
\end{questions}

Let's prove the cut property.

\newcommand{\filled}[1]{\underline{#1}}

\begin{proof}
  Let $G$ be a weighted, undirected graph $G = (V,E)$, let $X$ and $Y$
  be an arbitrary partition of $V$ into two sets, and suppose
  $e = (x,y)$ is the smallest-weight edge with one endpoint in $X$ and
  one in $Y$.  We wish to show that \filled{every MST of $G$ must
    include $e$}.

  We will prove the contrapositive. Suppose $M$ is a spanning tree of
  $G$ which does \textbf{not} contain the edge $e$.  Since $M$ is a
  \filled{tree} it contains a unique \filled{path} between any two
  \filled{vertices}. So consider the unique \filled{path} in $M$
  between \filled{$x$ and $y$}. It must cross the cut at least once
  since \filled{$x \in X$ and $y \in Y$}; suppose it crosses at
  $e' = (x',y')$, with $x' \in X$ and $y' \in Y$.  We know that the
  weight of $e$ is smaller than the weight of $e'$, since \filled{we
    assumed $e$ is the smallest-weight edge crossing the cut}. Now
  take $M$ and replace \filled{$e'$} with \filled{$e$}; the result is
  still \filled{connected} because \filled{any path that went through
    $e'$ can detour around through $e$}, but it has a smaller total
  \filled{weight} because \filled{we removed $e'$ and replaced it with
  $e$ which has a smaller weight}.

So, we have shown that any spanning tree $M$ which does not contain
the edge $e$ can be made into a \filled{smaller-weight spanning tree},
which means that $M$ is not a \filled{minimum spanning tree}.
\end{proof}

The cut property can be used to directly show the correctness of
several MST algorithms.  Let's prove the correctness of Kruskal's
Algorithm; the proofs for the other algorithms are similar.

\begin{thm}
  Kruskal's Algorithm is correct.
\end{thm}

\begin{proof}
  Suppose at some step the algorithm picks the edge $e = (x,y)$.  Let
  $X$ be the set of vertices connected to $x$ by edges which have been
  picked so far (not including $e$), and let $Y$ be all other
  vertices. $x \in X$ by definition.  We know that $y \notin X$ since
  if it was, $e$ would make a \filled{cycle} but then Kruskal's
  Algorithm wouldn't \filled{have picked it}. $e$ therefore crosses
  the cut $(X,Y)$. No other edges which have been picked previously
  cross the cut, since \filled{by definition the vertices in $Y$
    aren't connected to $x$}. Therefore $e$ must be the smallest
  \filled{edge crossing the cut} because \filled{the edges are
    considered in order from smallest to biggest weight}. Therefore by
  the Cut Property $e$ must be in any MST and Kruskal's Algorithm is
  correct to pick it.
\end{proof}

\end{document}
