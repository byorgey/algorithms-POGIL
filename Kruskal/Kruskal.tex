% -*- compile-command: "rubber -d --unsafe Kruskal.tex" -*-
\documentclass{tufte-handout}

\usepackage{algo-activity}

\title{\thecourse: Kruskal's Algorithm}
\date{}

\begin{document}

\maketitle

XXX we will focus on Kruskal's Algorithm, prove it correct.
XXX remind what Kruskal's Algorithm is.
XXX include graph from previous activity
XXX put in an aside about exchange properties, used to prove greedy algorithms.
XXX add something at the end about analysis of time complexity?  Or
just do that as a bit of lecture at the end?

\begin{model*}{The Cut Property (15 mins)}{cut-property}
  \begin{defn}
    A \term{cut} in a graph $G = (V,E)$ is a partition of the vertices
    $V$ into two sets $S$ and $T$, that is, every vertex is in either
    $S$ or $T$ but not both.  We say that an edge $e$ \term{crosses}
    the cut $(S,T)$ if one vertex of $e$ is in $S$ and the other is in
    $T$.
  \end{defn}

  \begin{thm}[Cut Property]
    Given a weighted, undirected graph $G = (V,E)$, let $S$ and $T$ be
    any partition of $V$, and suppose $e$ is the smallest-weight edge
    crossing the $(S,T)$ cut.  Then every minimum spanning tree of $G$
    must include $e$.
  \end{thm}
\end{model*}

\begin{questions}
\item Give three examples of cuts in the graph from \pref{model:mwss}
  and identify the smallest edge crossing each cut.
\end{questions}

Let's prove the cut property.

\begin{proof}
  Let $G$ be a weighted, undirected graph $G = (V,E)$, let $S$ and $T$
  be an arbitrary partition of $V$ into two sets, and suppose
  $e = (x,y)$ is the smallest-weight edge with one endpoint in $S$ and
  one in $T$.  We wish to show that \blank.

  Suppose $M$ is a spanning tree of $G$ which does \textbf{not}
  contain the edge $e$.  Since $M$ is a \blank it contains a unique
  \blank\linebreak between any two \blank. So consider the unique
  \blank\linebreak in $M$ between \blank.  \marginnote{\emph{Hint}:
    draw a picture!} It must cross
  the cut at least once since\linebreak \mbox{} \blank; suppose it
  crosses at $e' = (x',y')$,\linebreak with $x' \in X$ and $y' \in Y$.
  We know that $w_e < w_{e'}$ since \blank.\linebreak Now take $M$ and
  replace \blank with \blank; the resulting graph is\linebreak still
  \blank because \blank,\linebreak but it has a smaller total \blank
  because \blank.

  So, we have shown that any spanning tree $M$ which does not contain
  the edge $e$ can be made into a \blank,\linebreak which means that
  $M$ is not a \blank. This proves what\linebreak we wanted to show, since
  \blank.
\end{proof}

The cut property can be used to directly show the correctness of
several MST algorithms.  Let's prove the correctness of Kruskal's
Algorithm; the proofs for the other algorithms are similar.

\begin{thm}
  Kruskal's Algorithm is correct.
\end{thm}

\begin{proof}
  Suppose at some step the algorithm picks the edge $e = (x,y)$.  Let
  $X$ be the set of vertices connected to $x$ by edges which have been
  picked so far (not including $e$), and let $Y$ be all other
  vertices. $x \in X$ by definition.  We know that $y \notin X$ since
  if it was, $e$ would \blank\linebreak but then Kruskal's Algorithm
  wouldn't \blank.\linebreak $e$ therefore crosses the cut $(X,Y)$. No
  other edges which have been picked previously cross the cut, since
  \blank.\linebreak Therefore $e$ must be the \blank\linebreak because
  \blank.\linebreak Therefore by the Cut Property $e$ must
  be in any MST and Kruskal's Algorithm is correct to pick it.
\end{proof}

\end{document}