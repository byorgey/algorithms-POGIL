% -*- compile-command: "pdflatex AA-limits.tex" -*-
\documentclass{tufte-handout}

\usepackage{algo-activity}

\title{\thecourse: Asymptotic analysis: limit theorems}
\date{}

\begin{document}

\maketitle

\begin{objective}
  Students will determine the asymptotic behavior of functions using
  limit theorems.
\end{objective}
As you probably found on the previous activity, it can be somewhat
tedious to directly apply the formal definitions of $O$, $\Omega$, and
$\Theta$.  Fortunately, there is often an easier way.  Consider again
the functions
\begin{align*}
  f(n) &= (n^2 + 2)/n, \\ g(n) &= n^2/2 - n, \text{and} \\ h(n) &= n^3/1000.
\end{align*}

\begin{questions}
\item (Review) Say whether each of $f$, $g$, and $h$ is $O(n^2)$ only,
  $\Omega(n^2)$ only, or $\Theta(n^2)$ (\ie both).
\item What is \[ \lim_{n \to \infty} \frac{f(n)}{n^2}? \]
\item What is \[ \lim_{n \to \infty} \frac{g(n)}{n^2}? \]
\item What is \[ \lim_{n \to \infty} \frac{h(n)}{n^2}? \]
\item In general, consider the limit \[ \lim_{n \to \infty} T(n)/g(n). \]
  Intuitively, what can you say about the long-term behavior of $T(n)$
  relative to $g(n)$ if\dots
  \begin{subquestions}
    \item \dots the limit exists and is equal to $0$?  Draw a picture.
    \item \dots the limit exists and is equal to some positive
      constant $c$?  Draw a picture.
    \item \dots the limit does not exist since $T(n)/g(n)$ diverges to
      $+\infty$?  Draw a picture.
    \end{subquestions}
\item Fill in the statements of the following theorems:\marginnote{We
    will not formally prove these, although the proofs are not hard;
    you might like to try proving them yourself, based on the formal
    definitions of $O$ and $\Omega$.}
\begin{thm}
  If \[ 0 \leq \lim_{n \to \infty} \frac{T(n)}{g(n)} < \infty, \]
  \\[2em] then $T(n)$ \uline{\hfill}.
\end{thm} \vspace{0.2in}
\marginnote{Optional challenge problem to think about later: why do
  these theorems say ``if'' and not ``if and only if''?  \emph{Hint}:
  consider a function like \[ f(n) = \begin{cases} n^2 & \text{$n$ is
        even} \\ 0 & \text{$n$ is odd} \end{cases}. \] }
\begin{thm}
  If \\[6em] then $T(n)$ is $\Omega(g(n))$.
\end{thm} \vspace{0.2in}
\begin{thm}
  If the limit \[ \lim_{n \to \infty} \frac{T(n)}{g(n)} \] \\[2em]
  exists and \uline{\hfill}, then $T(n)$ is $\Theta(g(n))$.
\end{thm}

\item Describe the asymptotic behavior of \[ f(n) = 2n + \sqrt{3n} + 2 \]
  using big-$\Theta$ notation.  Justify your answer.

\end{questions}

\end{document}
