% -*- compile-command: "rubber -d AA-definitions.tex" -*-
\documentclass{tufte-handout}

\usepackage{algo-activity}

\usepackage{pgfplots}
\pgfplotsset{width=10cm,compat=1.9}
\usepgfplotslibrary{external}
\tikzexternalize

\title{\thecourse: Asymptotic Analysis: definitions}
\date{}

\begin{document}

\maketitle

\begin{model*}{Definitions}{defns}
  \begin{defn}[Big-O] \mlabel{defn:bigO}
    $T(n)$ is $O(g(n))$ if there exist a real number $c > 0$ and an
    integer $n_0 \geq 0$ such that for all $n \geq n_0$,
    \[ T(n) \leq c \cdot g(n). \]
  \end{defn}

  \begin{defn}[Big-Omega] \mlabel{defn:bigOmega}
    $T(n)$ is $\Omega(g(n))$ if there exist a real number $c > 0$ and an
    integer $n_0 \geq 0$ such that
    for all $n \geq n_0$, \[ T(n) \geq c \cdot g(n). \]
  \end{defn}

  \begin{defn}[Big-Theta] \mlabel{defn:bigTheta}
    $T(n)$ is $\Theta(g(n))$ if it is both $O(g(n))$ and $\Omega(g(n))$.
  \end{defn}

  \emph{Sample proof} that $n^2 + 2n$ is $\Theta(n^2)$:
  \begin{itemize}
  \item First, $n^2 + 2n \leq n^2 + 2n^2 = 3n^2$ for $n \geq 1$ (since
    $n^2 \geq n$ when $n \geq 1$).  Hence $n^2 + 2n$ is $O(n^2)$
    according to the definition if we pick $c = 3$ and $n_0 = 1$.
  \item Next, $n^2 + 2n \geq n^2$ as long as $n \geq 0$.  So by
    picking $c = 1$ and $n_0 = 0$, we see that $n^2 + 2n$ is also
    $\Omega(n^2)$.
  \end{itemize}
\end{model*}

\begin{objective}
  Students will describe asymptotic behavior of functions
  using big-$O$, big-$\Theta$, and big-$\Omega$ notation.
\end{objective}

\begin{questions}
\item Compare our class consensus definition of $O(n^2)$ from the
  previous activity with the formal definition of $O(g(n))$ above.
  List one way in which they are similar, and one way in which they
  are different. \vspace{0.5in}
\item \label{q:phrasing} Consider the following three more intuitive
  phrasings.  Match each one with its corresponding definition.
    \begin{itemize}
    \item $T(n)$ eventually has some constant multiple of $g(n)$ as a
      lower bound.
    \item $T(n)$ is eventually bounded between two constant multiples
      of $g(n)$.
    \item $T(n)$ eventually has some constant multiple of $g(n)$ as an
      upper bound.
    \end{itemize}

  \item Which part of the definitions corresponds to the word
    ``eventually'' in \pref{q:phrasing}?

  \item \label{q:alt-proof} In the sample proof that $n^2 + 2n$ is $O(n^2)$, the given
    values of $c$ and $n_0$ are not the only values that would work.
    Given an alternate proof that $n^2 + 2n$ is $O(n^2)$ using
    different values of $c$ and $n_0$.
  \item Prove that $f(n) = 20n - 1$ is $O(n^2)$ by applying the formal
    definition.
    \vspace{0.5in}
  \item Prove that $f(n) = n^3/10$ is $\Omega(n^2)$ by applying the
    formal definition.
    \vspace{0.5in}
  \item \label{q:last-formal-proof} Prove that $f(n) = 3n^2 - n + 1$ is
    $\Theta(n^2)$ by applying the formal definition.  \vspace{0.5in}

  \item Consider this definition\sidenote{Unlike the definitions of
      $O$, $\Omega$, and $\Theta$, this definition is not standard; I
      just made it up.}: $T(n)$ is $\aleph(g(n))$ if there
    exist a real number $c > 0$ and an integer $n_0 \geq 0$ such that
    for all $n \geq n_0$, \[ T(n) = c \cdot g(n). \]
    \begin{subquestions}
    \item True or false: if $T(n)$ is $\aleph(g(n))$, then $T(n)$ is
      $\Theta(g(n))$.  Justify your answer.
    \item True or false: if $T(n)$ is $\Theta(g(n))$, then $T(n)$ is
      $\aleph(g(n))$.  Justify your answer.
    \end{subquestions}
\end{questions}

\end{document}
