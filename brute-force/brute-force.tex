% -*- compile-command: "pdflatex brute-force.tex" -*-

\documentclass{tufte-handout}

\usepackage{../algo-activity}

\title{\thecourse: Brute Force}
\date{}

\begin{document}

\maketitle

\begin{model}{Some Problems and Algorithms}{probs-algos}
Consider the following scenarios:

\begin{enumerate}[label=(\alph*)]
\item You are given a list of $64$-bit integers, and asked to sort
  them in increasing order.
% \item You have a subway map of New York City and want to find
%   the shortest route between two given stations.
\item You are given a collection of 2D points, and asked to find the
  two points which are closest to each other.
\item You are given a set of (positive and negative)
  integers, and asked to find a subset which sums to zero (if there is
  one).
\item You are given two positive integers $a$ and $b$ and asked to
  find their greatest common divisor, that is, the largest positive
  integer which evenly divides both $a$ and $b$.
% \item Given an amount in US dollars and cents, you want to find the
%   minimum number of coins (using only pennies, nickels, dimes, and
%   quarters) needed to make the given amount.
\item You are asked whether there is any way to visit all $48$
  contiguous US states by car without entering the same state twice.
\end{enumerate}

\vspace{0.5in}
Here are some algorithms intended to solve the above scenarios:
  \begin{enumerate}[label=(\alph*)]
  \item List every possible ordering of the given integers.  Test each
    one to see whether it is sorted, and stop as soon as a sorted
    ordering is found.
  \item List each possible pair of points, and compute the distance
    between each.  Return the points with the smallest distance.
  \item List every subset of the given set, and find the sum of each.
    If any subset is found with a sum of zero, return it.
  % \item List all possible routes (that never retrace any segments)
  %   between the two stations.  Compute the length of each route and
  %   output the shortest one.
  \item (\emph{You will fill in this algorithm in \pref{q:gcd-brute}.})
  % \item List all possible ways to make the given amount using US
  %   coins.  Output the one that uses the fewest coins.
  \item List every possible ordering of the 48 states.  Check each
    ordering to see whether every pair of adjacent states in the
    ordering is also geographically adjacent.  If such an ordering is
    found, output ``yes''; else output ``no''.
  \end{enumerate}
\end{model}

\begin{objective}
  Students will analyze problems in terms of inputs and outputs.
\end{objective}

\begin{objective}
  Students will write brute-force algorithms to solve search problems.
\end{objective}

\begin{questions}
% \item Sort these integers in increasing order:
% \[ \begin{array}{rrrr}
%      2850315545522213899 & 1319900231253041735 & 6671615477926962880 & 816074892707851151 \\
%      3153324888213964873 & 6478282629679038038 & 524678787506768624 & 3299307200078362735
%    \end{array}
% \]
% \item Which two of these points are closest to each other?
%   \[ (2,5) \quad (3,1) \quad (4,0) \quad (0,0) \quad (5,7) \quad (6,2) \]
% \item Is there a subset of these integers that sums to zero?
%   \[ \{ 2, 3, -9, -4, 7, -2, -6 \} \]
% \item What is the greatest common divisor of 90 and 525?

% \item What is the minimum number of coins needed to make \$1.67?

%   \newpage

% \item What do you think?  \emph{Is} there a way to visit all 48
%   contiguous US states by car without entering the same state
%   twice?
%   \begin{fullwidth}
%     \begin{center}
%       \includegraphics[width=1.4\textwidth]{US-outline.pdf}
%     \end{center}
%   \end{fullwidth}
% \newpage

\item For each scenario, identify the \emph{input(s)} to the problem.
  \begin{subquestions}
  \item \mbox{}
  \item \mbox{}
  \item \mbox{}
  \item \mbox{}
  \item \mbox{}
  \end{subquestions}

\item For each scenario, identify the desired \emph{output(s)}.
  \begin{subquestions}
  \item \mbox{}
  \item \mbox{}
  \item \mbox{}
  \item \mbox{}
  \item \mbox{}
  \end{subquestions}

  \newpage
  \item All the algorithms in \pref{model:probs-algos} are called
    \term{brute force algorithms}.  Using one or more \textbf{complete
      English sentences}, write down a definition of what a
    \term{brute force algorithm} is.

  \item \label{q:gcd-brute} Fill in this description of a brute force
    algorithm for scenario (d): \vspace{0.3in}

    \begin{fullwidth}
    \begin{quote}
      \emph{List all the \uline{\hfill} \\[2em] and output
        \uline{\hfill}.}
    \end{quote}
    \end{fullwidth}
  \item For which of the scenarios (a)--(e) do you think the given
    algorithm is the fastest possible algorithm?  For which ones do
    you think a faster algorithm is possible?\marginnote[-1em]{I don't
      expect you to know the right answer for all of these!}
\end{questions}

\end{document}