% -*- compile-command: "pdflatex --enable-write18 graphs.tex" -*-
\documentclass{tufte-handout}

\usepackage{algo-activity}

\title{\thecourse: Graphs}
\date{}

\begin{document}

\maketitle

\begin{model}{Graphs and graph terms}{graphs}
  \begin{center}
  \begin{minipage}{0.45\textwidth}
    \begin{pgfpicture}
\pgfpathrectangle{\pgfpointorigin}{\pgfqpoint{150.0000bp}{150.0000bp}}
\pgfusepath{use as bounding box}
\begin{pgfscope}
\pgfsetlinewidth{0.5655bp}
\definecolor{sc}{rgb}{0.0000,0.0000,0.0000}
\pgfsetstrokecolor{sc}
\pgfsetmiterjoin
\pgfsetbuttcap
\pgfpathqmoveto{110.8756bp}{99.7024bp}
\pgfpathqlineto{104.4652bp}{71.6757bp}
\pgfusepathqstroke
\end{pgfscope}
\begin{pgfscope}
\pgfsetlinewidth{0.5655bp}
\definecolor{sc}{rgb}{0.0000,0.0000,0.0000}
\pgfsetstrokecolor{sc}
\pgfsetmiterjoin
\pgfsetbuttcap
\pgfpathqmoveto{110.8756bp}{99.7024bp}
\pgfpathqlineto{82.2135bp}{95.6883bp}
\pgfusepathqstroke
\end{pgfscope}
\begin{pgfscope}
\pgfsetlinewidth{0.5655bp}
\definecolor{sc}{rgb}{0.0000,0.0000,0.0000}
\pgfsetstrokecolor{sc}
\pgfsetmiterjoin
\pgfsetbuttcap
\pgfpathqmoveto{104.4652bp}{71.6757bp}
\pgfpathqlineto{82.2135bp}{95.6883bp}
\pgfusepathqstroke
\end{pgfscope}
\begin{pgfscope}
\pgfsetlinewidth{0.5655bp}
\definecolor{sc}{rgb}{0.0000,0.0000,0.0000}
\pgfsetstrokecolor{sc}
\pgfsetmiterjoin
\pgfsetbuttcap
\pgfpathqmoveto{104.4652bp}{71.6757bp}
\pgfpathqlineto{117.3643bp}{40.4385bp}
\pgfusepathqstroke
\end{pgfscope}
\begin{pgfscope}
\pgfsetlinewidth{0.5655bp}
\definecolor{sc}{rgb}{0.0000,0.0000,0.0000}
\pgfsetstrokecolor{sc}
\pgfsetmiterjoin
\pgfsetbuttcap
\pgfpathqmoveto{117.3643bp}{40.4385bp}
\pgfpathqlineto{103.4723bp}{13.4302bp}
\pgfusepathqstroke
\end{pgfscope}
\begin{pgfscope}
\pgfsetlinewidth{0.5655bp}
\definecolor{sc}{rgb}{0.0000,0.0000,0.0000}
\pgfsetstrokecolor{sc}
\pgfsetmiterjoin
\pgfsetbuttcap
\pgfpathqmoveto{82.2135bp}{95.6883bp}
\pgfpathqlineto{51.3110bp}{110.7828bp}
\pgfusepathqstroke
\end{pgfscope}
\begin{pgfscope}
\pgfsetlinewidth{0.5655bp}
\definecolor{sc}{rgb}{0.0000,0.0000,0.0000}
\pgfsetstrokecolor{sc}
\pgfsetmiterjoin
\pgfsetbuttcap
\pgfpathqmoveto{51.3110bp}{110.7828bp}
\pgfpathqlineto{33.3304bp}{136.2671bp}
\pgfusepathqstroke
\end{pgfscope}
\begin{pgfscope}
\pgfsetlinewidth{0.5655bp}
\definecolor{sc}{rgb}{0.0000,0.0000,0.0000}
\pgfsetstrokecolor{sc}
\pgfsetmiterjoin
\pgfsetbuttcap
\pgfpathqmoveto{33.3304bp}{136.2671bp}
\pgfpathqlineto{5.3272bp}{128.3841bp}
\pgfusepathqstroke
\end{pgfscope}
\begin{pgfscope}
\pgfsetlinewidth{0.5655bp}
\definecolor{sc}{rgb}{0.0000,0.0000,0.0000}
\pgfsetstrokecolor{sc}
\pgfsetmiterjoin
\pgfsetbuttcap
\pgfpathqmoveto{5.3272bp}{128.3841bp}
\pgfpathqlineto{21.0578bp}{104.1213bp}
\pgfusepathqstroke
\end{pgfscope}
\begin{pgfscope}
\pgfsetlinewidth{0.5655bp}
\definecolor{sc}{rgb}{0.0000,0.0000,0.0000}
\pgfsetstrokecolor{sc}
\pgfsetmiterjoin
\pgfsetbuttcap
\pgfpathqmoveto{21.0578bp}{104.1213bp}
\pgfpathqlineto{51.3110bp}{110.7828bp}
\pgfusepathqstroke
\end{pgfscope}
\begin{pgfscope}
\pgfsetlinewidth{0.5655bp}
\definecolor{sc}{rgb}{0.0000,0.0000,0.0000}
\pgfsetstrokecolor{sc}
\pgfsetmiterjoin
\pgfsetbuttcap
\pgfpathqmoveto{117.3643bp}{40.4385bp}
\pgfpathqlineto{144.8617bp}{27.3170bp}
\pgfusepathqstroke
\end{pgfscope}
\begin{pgfscope}
\definecolor{fc}{rgb}{0.0000,0.0000,0.0000}
\pgfsetfillcolor{fc}
\pgftransformshift{\pgfqpoint{150.0000bp}{24.8650bp}}
\pgftransformscale{1.0000}
\pgftext[]{$9$}
\end{pgfscope}
\begin{pgfscope}
\definecolor{fc}{rgb}{0.0000,0.0000,0.0000}
\pgfsetfillcolor{fc}
\pgfsetlinewidth{0.5655bp}
\definecolor{sc}{rgb}{0.0000,0.0000,0.0000}
\pgfsetstrokecolor{sc}
\pgfsetmiterjoin
\pgfsetbuttcap
\pgfpathqmoveto{146.0004bp}{27.3170bp}
\pgfpathqcurveto{146.0004bp}{27.9458bp}{145.4906bp}{28.4556bp}{144.8617bp}{28.4556bp}
\pgfpathqcurveto{144.2328bp}{28.4556bp}{143.7230bp}{27.9458bp}{143.7230bp}{27.3170bp}
\pgfpathqcurveto{143.7230bp}{26.6881bp}{144.2328bp}{26.1783bp}{144.8617bp}{26.1783bp}
\pgfpathqcurveto{145.4906bp}{26.1783bp}{146.0004bp}{26.6881bp}{146.0004bp}{27.3170bp}
\pgfpathclose
\pgfusepathqfillstroke
\end{pgfscope}
\begin{pgfscope}
\definecolor{fc}{rgb}{0.0000,0.0000,0.0000}
\pgfsetfillcolor{fc}
\pgftransformshift{\pgfqpoint{18.8963bp}{98.8542bp}}
\pgftransformscale{1.0000}
\pgftext[]{$8$}
\end{pgfscope}
\begin{pgfscope}
\definecolor{fc}{rgb}{0.0000,0.0000,0.0000}
\pgfsetfillcolor{fc}
\pgfsetlinewidth{0.5655bp}
\definecolor{sc}{rgb}{0.0000,0.0000,0.0000}
\pgfsetstrokecolor{sc}
\pgfsetmiterjoin
\pgfsetbuttcap
\pgfpathqmoveto{22.1965bp}{104.1213bp}
\pgfpathqcurveto{22.1965bp}{104.7501bp}{21.6867bp}{105.2599bp}{21.0578bp}{105.2599bp}
\pgfpathqcurveto{20.4289bp}{105.2599bp}{19.9191bp}{104.7501bp}{19.9191bp}{104.1213bp}
\pgfpathqcurveto{19.9191bp}{103.4924bp}{20.4289bp}{102.9826bp}{21.0578bp}{102.9826bp}
\pgfpathqcurveto{21.6867bp}{102.9826bp}{22.1965bp}{103.4924bp}{22.1965bp}{104.1213bp}
\pgfpathclose
\pgfusepathqfillstroke
\end{pgfscope}
\begin{pgfscope}
\definecolor{fc}{rgb}{0.0000,0.0000,0.0000}
\pgfsetfillcolor{fc}
\pgftransformshift{\pgfqpoint{0.0000bp}{130.3929bp}}
\pgftransformscale{1.0000}
\pgftext[]{$7$}
\end{pgfscope}
\begin{pgfscope}
\definecolor{fc}{rgb}{0.0000,0.0000,0.0000}
\pgfsetfillcolor{fc}
\pgfsetlinewidth{0.5655bp}
\definecolor{sc}{rgb}{0.0000,0.0000,0.0000}
\pgfsetstrokecolor{sc}
\pgfsetmiterjoin
\pgfsetbuttcap
\pgfpathqmoveto{6.4659bp}{128.3841bp}
\pgfpathqcurveto{6.4659bp}{129.0130bp}{5.9561bp}{129.5228bp}{5.3272bp}{129.5228bp}
\pgfpathqcurveto{4.6983bp}{129.5228bp}{4.1885bp}{129.0130bp}{4.1885bp}{128.3841bp}
\pgfpathqcurveto{4.1885bp}{127.7552bp}{4.6983bp}{127.2454bp}{5.3272bp}{127.2454bp}
\pgfpathqcurveto{5.9561bp}{127.2454bp}{6.4659bp}{127.7552bp}{6.4659bp}{128.3841bp}
\pgfpathclose
\pgfusepathqfillstroke
\end{pgfscope}
\begin{pgfscope}
\definecolor{fc}{rgb}{0.0000,0.0000,0.0000}
\pgfsetfillcolor{fc}
\pgftransformshift{\pgfqpoint{35.2343bp}{141.6327bp}}
\pgftransformscale{1.0000}
\pgftext[]{$6$}
\end{pgfscope}
\begin{pgfscope}
\definecolor{fc}{rgb}{0.0000,0.0000,0.0000}
\pgfsetfillcolor{fc}
\pgfsetlinewidth{0.5655bp}
\definecolor{sc}{rgb}{0.0000,0.0000,0.0000}
\pgfsetstrokecolor{sc}
\pgfsetmiterjoin
\pgfsetbuttcap
\pgfpathqmoveto{34.4690bp}{136.2671bp}
\pgfpathqcurveto{34.4690bp}{136.8960bp}{33.9592bp}{137.4058bp}{33.3304bp}{137.4058bp}
\pgfpathqcurveto{32.7015bp}{137.4058bp}{32.1917bp}{136.8960bp}{32.1917bp}{136.2671bp}
\pgfpathqcurveto{32.1917bp}{135.6382bp}{32.7015bp}{135.1284bp}{33.3304bp}{135.1284bp}
\pgfpathqcurveto{33.9592bp}{135.1284bp}{34.4690bp}{135.6382bp}{34.4690bp}{136.2671bp}
\pgfpathclose
\pgfusepathqfillstroke
\end{pgfscope}
\begin{pgfscope}
\definecolor{fc}{rgb}{0.0000,0.0000,0.0000}
\pgfsetfillcolor{fc}
\pgftransformshift{\pgfqpoint{55.0020bp}{115.1176bp}}
\pgftransformscale{1.0000}
\pgftext[]{$5$}
\end{pgfscope}
\begin{pgfscope}
\definecolor{fc}{rgb}{0.0000,0.0000,0.0000}
\pgfsetfillcolor{fc}
\pgfsetlinewidth{0.5655bp}
\definecolor{sc}{rgb}{0.0000,0.0000,0.0000}
\pgfsetstrokecolor{sc}
\pgfsetmiterjoin
\pgfsetbuttcap
\pgfpathqmoveto{52.4496bp}{110.7828bp}
\pgfpathqcurveto{52.4496bp}{111.4117bp}{51.9398bp}{111.9215bp}{51.3110bp}{111.9215bp}
\pgfpathqcurveto{50.6821bp}{111.9215bp}{50.1723bp}{111.4117bp}{50.1723bp}{110.7828bp}
\pgfpathqcurveto{50.1723bp}{110.1539bp}{50.6821bp}{109.6441bp}{51.3110bp}{109.6441bp}
\pgfpathqcurveto{51.9398bp}{109.6441bp}{52.4496bp}{110.1539bp}{52.4496bp}{110.7828bp}
\pgfpathclose
\pgfusepathqfillstroke
\end{pgfscope}
\begin{pgfscope}
\definecolor{fc}{rgb}{0.0000,0.0000,0.0000}
\pgfsetfillcolor{fc}
\pgftransformshift{\pgfqpoint{100.8682bp}{8.3673bp}}
\pgftransformscale{1.0000}
\pgftext[]{$4$}
\end{pgfscope}
\begin{pgfscope}
\definecolor{fc}{rgb}{0.0000,0.0000,0.0000}
\pgfsetfillcolor{fc}
\pgfsetlinewidth{0.5655bp}
\definecolor{sc}{rgb}{0.0000,0.0000,0.0000}
\pgfsetstrokecolor{sc}
\pgfsetmiterjoin
\pgfsetbuttcap
\pgfpathqmoveto{104.6110bp}{13.4302bp}
\pgfpathqcurveto{104.6110bp}{14.0591bp}{104.1012bp}{14.5689bp}{103.4723bp}{14.5689bp}
\pgfpathqcurveto{102.8434bp}{14.5689bp}{102.3336bp}{14.0591bp}{102.3336bp}{13.4302bp}
\pgfpathqcurveto{102.3336bp}{12.8013bp}{102.8434bp}{12.2915bp}{103.4723bp}{12.2915bp}
\pgfpathqcurveto{104.1012bp}{12.2915bp}{104.6110bp}{12.8013bp}{104.6110bp}{13.4302bp}
\pgfpathclose
\pgfusepathqfillstroke
\end{pgfscope}
\begin{pgfscope}
\definecolor{fc}{rgb}{0.0000,0.0000,0.0000}
\pgfsetfillcolor{fc}
\pgftransformshift{\pgfqpoint{121.4966bp}{44.3550bp}}
\pgftransformscale{1.0000}
\pgftext[]{$3$}
\end{pgfscope}
\begin{pgfscope}
\definecolor{fc}{rgb}{0.0000,0.0000,0.0000}
\pgfsetfillcolor{fc}
\pgfsetlinewidth{0.5655bp}
\definecolor{sc}{rgb}{0.0000,0.0000,0.0000}
\pgfsetstrokecolor{sc}
\pgfsetmiterjoin
\pgfsetbuttcap
\pgfpathqmoveto{118.5030bp}{40.4385bp}
\pgfpathqcurveto{118.5030bp}{41.0674bp}{117.9932bp}{41.5772bp}{117.3643bp}{41.5772bp}
\pgfpathqcurveto{116.7354bp}{41.5772bp}{116.2256bp}{41.0674bp}{116.2256bp}{40.4385bp}
\pgfpathqcurveto{116.2256bp}{39.8096bp}{116.7354bp}{39.2998bp}{117.3643bp}{39.2998bp}
\pgfpathqcurveto{117.9932bp}{39.2998bp}{118.5030bp}{39.8096bp}{118.5030bp}{40.4385bp}
\pgfpathclose
\pgfusepathqfillstroke
\end{pgfscope}
\begin{pgfscope}
\definecolor{fc}{rgb}{0.0000,0.0000,0.0000}
\pgfsetfillcolor{fc}
\pgftransformshift{\pgfqpoint{78.8185bp}{91.1179bp}}
\pgftransformscale{1.0000}
\pgftext[]{$2$}
\end{pgfscope}
\begin{pgfscope}
\definecolor{fc}{rgb}{0.0000,0.0000,0.0000}
\pgfsetfillcolor{fc}
\pgfsetlinewidth{0.5655bp}
\definecolor{sc}{rgb}{0.0000,0.0000,0.0000}
\pgfsetstrokecolor{sc}
\pgfsetmiterjoin
\pgfsetbuttcap
\pgfpathqmoveto{83.3522bp}{95.6883bp}
\pgfpathqcurveto{83.3522bp}{96.3171bp}{82.8424bp}{96.8269bp}{82.2135bp}{96.8269bp}
\pgfpathqcurveto{81.5846bp}{96.8269bp}{81.0748bp}{96.3171bp}{81.0748bp}{95.6883bp}
\pgfpathqcurveto{81.0748bp}{95.0594bp}{81.5846bp}{94.5496bp}{82.2135bp}{94.5496bp}
\pgfpathqcurveto{82.8424bp}{94.5496bp}{83.3522bp}{95.0594bp}{83.3522bp}{95.6883bp}
\pgfpathclose
\pgfusepathqfillstroke
\end{pgfscope}
\begin{pgfscope}
\definecolor{fc}{rgb}{0.0000,0.0000,0.0000}
\pgfsetfillcolor{fc}
\pgftransformshift{\pgfqpoint{99.6704bp}{68.6059bp}}
\pgftransformscale{1.0000}
\pgftext[]{$1$}
\end{pgfscope}
\begin{pgfscope}
\definecolor{fc}{rgb}{0.0000,0.0000,0.0000}
\pgfsetfillcolor{fc}
\pgfsetlinewidth{0.5655bp}
\definecolor{sc}{rgb}{0.0000,0.0000,0.0000}
\pgfsetstrokecolor{sc}
\pgfsetmiterjoin
\pgfsetbuttcap
\pgfpathqmoveto{105.6039bp}{71.6757bp}
\pgfpathqcurveto{105.6039bp}{72.3046bp}{105.0941bp}{72.8144bp}{104.4652bp}{72.8144bp}
\pgfpathqcurveto{103.8364bp}{72.8144bp}{103.3266bp}{72.3046bp}{103.3266bp}{71.6757bp}
\pgfpathqcurveto{103.3266bp}{71.0469bp}{103.8364bp}{70.5371bp}{104.4652bp}{70.5371bp}
\pgfpathqcurveto{105.0941bp}{70.5371bp}{105.6039bp}{71.0469bp}{105.6039bp}{71.6757bp}
\pgfpathclose
\pgfusepathqfillstroke
\end{pgfscope}
\begin{pgfscope}
\definecolor{fc}{rgb}{0.0000,0.0000,0.0000}
\pgfsetfillcolor{fc}
\pgftransformshift{\pgfqpoint{115.0702bp}{103.5521bp}}
\pgftransformscale{1.0000}
\pgftext[]{$0$}
\end{pgfscope}
\begin{pgfscope}
\definecolor{fc}{rgb}{0.0000,0.0000,0.0000}
\pgfsetfillcolor{fc}
\pgfsetlinewidth{0.5655bp}
\definecolor{sc}{rgb}{0.0000,0.0000,0.0000}
\pgfsetstrokecolor{sc}
\pgfsetmiterjoin
\pgfsetbuttcap
\pgfpathqmoveto{112.0143bp}{99.7024bp}
\pgfpathqcurveto{112.0143bp}{100.3313bp}{111.5045bp}{100.8411bp}{110.8756bp}{100.8411bp}
\pgfpathqcurveto{110.2467bp}{100.8411bp}{109.7369bp}{100.3313bp}{109.7369bp}{99.7024bp}
\pgfpathqcurveto{109.7369bp}{99.0736bp}{110.2467bp}{98.5638bp}{110.8756bp}{98.5638bp}
\pgfpathqcurveto{111.5045bp}{98.5638bp}{112.0143bp}{99.0736bp}{112.0143bp}{99.7024bp}
\pgfpathclose
\pgfusepathqfillstroke
\end{pgfscope}
\end{pgfpicture}

  \end{minipage}
  \begin{minipage}{0.45\textwidth}
    \begin{pgfpicture}
\pgfpathrectangle{\pgfpointorigin}{\pgfqpoint{150.0000bp}{150.0000bp}}
\pgfusepath{use as bounding box}
\begin{pgfscope}
\pgfsetlinewidth{0.5987bp}
\definecolor{sc}{rgb}{0.0000,0.0000,0.0000}
\pgfsetstrokecolor{sc}
\pgfsetmiterjoin
\pgfsetbuttcap
\pgfpathqmoveto{144.0766bp}{67.2085bp}
\pgfpathqlineto{112.0079bp}{88.2694bp}
\pgfusepathqstroke
\end{pgfscope}
\begin{pgfscope}
\pgfsetlinewidth{0.5987bp}
\definecolor{sc}{rgb}{0.0000,0.0000,0.0000}
\pgfsetstrokecolor{sc}
\pgfsetmiterjoin
\pgfsetbuttcap
\pgfpathqmoveto{137.4218bp}{117.1461bp}
\pgfpathqlineto{112.0079bp}{88.2694bp}
\pgfusepathqstroke
\end{pgfscope}
\begin{pgfscope}
\pgfsetlinewidth{0.5987bp}
\definecolor{sc}{rgb}{0.0000,0.0000,0.0000}
\pgfsetstrokecolor{sc}
\pgfsetmiterjoin
\pgfsetbuttcap
\pgfpathqmoveto{112.0079bp}{88.2694bp}
\pgfpathqlineto{68.6020bp}{80.1602bp}
\pgfusepathqstroke
\end{pgfscope}
\begin{pgfscope}
\pgfsetlinewidth{0.5987bp}
\definecolor{sc}{rgb}{0.0000,0.0000,0.0000}
\pgfsetstrokecolor{sc}
\pgfsetmiterjoin
\pgfsetbuttcap
\pgfpathqmoveto{68.6020bp}{80.1602bp}
\pgfpathqlineto{45.7604bp}{115.0036bp}
\pgfusepathqstroke
\end{pgfscope}
\begin{pgfscope}
\pgfsetlinewidth{0.5987bp}
\definecolor{sc}{rgb}{0.0000,0.0000,0.0000}
\pgfsetstrokecolor{sc}
\pgfsetmiterjoin
\pgfsetbuttcap
\pgfpathqmoveto{45.7604bp}{115.0036bp}
\pgfpathqlineto{21.8889bp}{144.1840bp}
\pgfusepathqstroke
\end{pgfscope}
\begin{pgfscope}
\pgfsetlinewidth{0.5987bp}
\definecolor{sc}{rgb}{0.0000,0.0000,0.0000}
\pgfsetstrokecolor{sc}
\pgfsetmiterjoin
\pgfsetbuttcap
\pgfpathqmoveto{68.6020bp}{80.1602bp}
\pgfpathqlineto{45.1884bp}{40.9871bp}
\pgfusepathqstroke
\end{pgfscope}
\begin{pgfscope}
\pgfsetlinewidth{0.5987bp}
\definecolor{sc}{rgb}{0.0000,0.0000,0.0000}
\pgfsetstrokecolor{sc}
\pgfsetmiterjoin
\pgfsetbuttcap
\pgfpathqmoveto{45.1884bp}{40.9871bp}
\pgfpathqlineto{6.9699bp}{48.0099bp}
\pgfusepathqstroke
\end{pgfscope}
\begin{pgfscope}
\pgfsetlinewidth{0.5987bp}
\definecolor{sc}{rgb}{0.0000,0.0000,0.0000}
\pgfsetstrokecolor{sc}
\pgfsetmiterjoin
\pgfsetbuttcap
\pgfpathqmoveto{45.1884bp}{40.9871bp}
\pgfpathqlineto{25.5620bp}{6.4905bp}
\pgfusepathqstroke
\end{pgfscope}
\begin{pgfscope}
\pgfsetlinewidth{0.5987bp}
\definecolor{sc}{rgb}{0.0000,0.0000,0.0000}
\pgfsetstrokecolor{sc}
\pgfsetmiterjoin
\pgfsetbuttcap
\pgfpathqmoveto{45.1884bp}{40.9871bp}
\pgfpathqlineto{70.6877bp}{11.9140bp}
\pgfusepathqstroke
\end{pgfscope}
\begin{pgfscope}
\definecolor{fc}{rgb}{0.0000,0.0000,0.0000}
\pgfsetfillcolor{fc}
\pgftransformcm{1.0000}{0.0000}{0.0000}{1.0000}{\pgfqpoint{75.3605bp}{6.5863bp}}
\pgftransformscale{1.0000}
\pgftext[]{j}
\end{pgfscope}
\begin{pgfscope}
\definecolor{fc}{rgb}{0.0000,0.0000,0.0000}
\pgfsetfillcolor{fc}
\pgfsetlinewidth{0.5987bp}
\definecolor{sc}{rgb}{0.0000,0.0000,0.0000}
\pgfsetstrokecolor{sc}
\pgfsetmiterjoin
\pgfsetbuttcap
\pgfpathqmoveto{72.1050bp}{11.9140bp}
\pgfpathqcurveto{72.1050bp}{12.6968bp}{71.4705bp}{13.3313bp}{70.6877bp}{13.3313bp}
\pgfpathqcurveto{69.9050bp}{13.3313bp}{69.2704bp}{12.6968bp}{69.2704bp}{11.9140bp}
\pgfpathqcurveto{69.2704bp}{11.1312bp}{69.9050bp}{10.4967bp}{70.6877bp}{10.4967bp}
\pgfpathqcurveto{71.4705bp}{10.4967bp}{72.1050bp}{11.1312bp}{72.1050bp}{11.9140bp}
\pgfpathclose
\pgfusepathqfillstroke
\end{pgfscope}
\begin{pgfscope}
\definecolor{fc}{rgb}{0.0000,0.0000,0.0000}
\pgfsetfillcolor{fc}
\pgftransformcm{1.0000}{0.0000}{0.0000}{1.0000}{\pgfqpoint{22.0576bp}{0.3310bp}}
\pgftransformscale{1.0000}
\pgftext[]{i}
\end{pgfscope}
\begin{pgfscope}
\definecolor{fc}{rgb}{0.0000,0.0000,0.0000}
\pgfsetfillcolor{fc}
\pgfsetlinewidth{0.5987bp}
\definecolor{sc}{rgb}{0.0000,0.0000,0.0000}
\pgfsetstrokecolor{sc}
\pgfsetmiterjoin
\pgfsetbuttcap
\pgfpathqmoveto{26.9793bp}{6.4905bp}
\pgfpathqcurveto{26.9793bp}{7.2733bp}{26.3447bp}{7.9078bp}{25.5620bp}{7.9078bp}
\pgfpathqcurveto{24.7792bp}{7.9078bp}{24.1446bp}{7.2733bp}{24.1446bp}{6.4905bp}
\pgfpathqcurveto{24.1446bp}{5.7077bp}{24.7792bp}{5.0732bp}{25.5620bp}{5.0732bp}
\pgfpathqcurveto{26.3447bp}{5.0732bp}{26.9793bp}{5.7077bp}{26.9793bp}{6.4905bp}
\pgfpathclose
\pgfusepathqfillstroke
\end{pgfscope}
\begin{pgfscope}
\definecolor{fc}{rgb}{0.0000,0.0000,0.0000}
\pgfsetfillcolor{fc}
\pgftransformcm{1.0000}{0.0000}{0.0000}{1.0000}{\pgfqpoint{0.0000bp}{49.2906bp}}
\pgftransformscale{1.0000}
\pgftext[]{h}
\end{pgfscope}
\begin{pgfscope}
\definecolor{fc}{rgb}{0.0000,0.0000,0.0000}
\pgfsetfillcolor{fc}
\pgfsetlinewidth{0.5987bp}
\definecolor{sc}{rgb}{0.0000,0.0000,0.0000}
\pgfsetstrokecolor{sc}
\pgfsetmiterjoin
\pgfsetbuttcap
\pgfpathqmoveto{8.3872bp}{48.0099bp}
\pgfpathqcurveto{8.3872bp}{48.7926bp}{7.7527bp}{49.4272bp}{6.9699bp}{49.4272bp}
\pgfpathqcurveto{6.1871bp}{49.4272bp}{5.5526bp}{48.7926bp}{5.5526bp}{48.0099bp}
\pgfpathqcurveto{5.5526bp}{47.2271bp}{6.1871bp}{46.5925bp}{6.9699bp}{46.5925bp}
\pgfpathqcurveto{7.7527bp}{46.5925bp}{8.3872bp}{47.2271bp}{8.3872bp}{48.0099bp}
\pgfpathclose
\pgfusepathqfillstroke
\end{pgfscope}
\begin{pgfscope}
\definecolor{fc}{rgb}{0.0000,0.0000,0.0000}
\pgfsetfillcolor{fc}
\pgftransformcm{1.0000}{0.0000}{0.0000}{1.0000}{\pgfqpoint{42.2654bp}{47.4427bp}}
\pgftransformscale{1.0000}
\pgftext[]{g}
\end{pgfscope}
\begin{pgfscope}
\definecolor{fc}{rgb}{0.0000,0.0000,0.0000}
\pgfsetfillcolor{fc}
\pgfsetlinewidth{0.5987bp}
\definecolor{sc}{rgb}{0.0000,0.0000,0.0000}
\pgfsetstrokecolor{sc}
\pgfsetmiterjoin
\pgfsetbuttcap
\pgfpathqmoveto{46.6057bp}{40.9871bp}
\pgfpathqcurveto{46.6057bp}{41.7698bp}{45.9712bp}{42.4044bp}{45.1884bp}{42.4044bp}
\pgfpathqcurveto{44.4057bp}{42.4044bp}{43.7711bp}{41.7698bp}{43.7711bp}{40.9871bp}
\pgfpathqcurveto{43.7711bp}{40.2043bp}{44.4057bp}{39.5698bp}{45.1884bp}{39.5698bp}
\pgfpathqcurveto{45.9712bp}{39.5698bp}{46.6057bp}{40.2043bp}{46.6057bp}{40.9871bp}
\pgfpathclose
\pgfusepathqfillstroke
\end{pgfscope}
\begin{pgfscope}
\definecolor{fc}{rgb}{0.0000,0.0000,0.0000}
\pgfsetfillcolor{fc}
\pgftransformcm{1.0000}{0.0000}{0.0000}{1.0000}{\pgfqpoint{17.4017bp}{149.6690bp}}
\pgftransformscale{1.0000}
\pgftext[]{f}
\end{pgfscope}
\begin{pgfscope}
\definecolor{fc}{rgb}{0.0000,0.0000,0.0000}
\pgfsetfillcolor{fc}
\pgfsetlinewidth{0.5987bp}
\definecolor{sc}{rgb}{0.0000,0.0000,0.0000}
\pgfsetstrokecolor{sc}
\pgfsetmiterjoin
\pgfsetbuttcap
\pgfpathqmoveto{23.3062bp}{144.1840bp}
\pgfpathqcurveto{23.3062bp}{144.9667bp}{22.6716bp}{145.6013bp}{21.8889bp}{145.6013bp}
\pgfpathqcurveto{21.1061bp}{145.6013bp}{20.4715bp}{144.9667bp}{20.4715bp}{144.1840bp}
\pgfpathqcurveto{20.4715bp}{143.4012bp}{21.1061bp}{142.7666bp}{21.8889bp}{142.7666bp}
\pgfpathqcurveto{22.6716bp}{142.7666bp}{23.3062bp}{143.4012bp}{23.3062bp}{144.1840bp}
\pgfpathclose
\pgfusepathqfillstroke
\end{pgfscope}
\begin{pgfscope}
\definecolor{fc}{rgb}{0.0000,0.0000,0.0000}
\pgfsetfillcolor{fc}
\pgftransformcm{1.0000}{0.0000}{0.0000}{1.0000}{\pgfqpoint{51.4742bp}{119.1956bp}}
\pgftransformscale{1.0000}
\pgftext[]{e}
\end{pgfscope}
\begin{pgfscope}
\definecolor{fc}{rgb}{0.0000,0.0000,0.0000}
\pgfsetfillcolor{fc}
\pgfsetlinewidth{0.5987bp}
\definecolor{sc}{rgb}{0.0000,0.0000,0.0000}
\pgfsetstrokecolor{sc}
\pgfsetmiterjoin
\pgfsetbuttcap
\pgfpathqmoveto{47.1778bp}{115.0036bp}
\pgfpathqcurveto{47.1778bp}{115.7864bp}{46.5432bp}{116.4209bp}{45.7604bp}{116.4209bp}
\pgfpathqcurveto{44.9777bp}{116.4209bp}{44.3431bp}{115.7864bp}{44.3431bp}{115.0036bp}
\pgfpathqcurveto{44.3431bp}{114.2209bp}{44.9777bp}{113.5863bp}{45.7604bp}{113.5863bp}
\pgfpathqcurveto{46.5432bp}{113.5863bp}{47.1778bp}{114.2209bp}{47.1778bp}{115.0036bp}
\pgfpathclose
\pgfusepathqfillstroke
\end{pgfscope}
\begin{pgfscope}
\definecolor{fc}{rgb}{0.0000,0.0000,0.0000}
\pgfsetfillcolor{fc}
\pgftransformcm{1.0000}{0.0000}{0.0000}{1.0000}{\pgfqpoint{72.6523bp}{74.3451bp}}
\pgftransformscale{1.0000}
\pgftext[]{d}
\end{pgfscope}
\begin{pgfscope}
\definecolor{fc}{rgb}{0.0000,0.0000,0.0000}
\pgfsetfillcolor{fc}
\pgfsetlinewidth{0.5987bp}
\definecolor{sc}{rgb}{0.0000,0.0000,0.0000}
\pgfsetstrokecolor{sc}
\pgfsetmiterjoin
\pgfsetbuttcap
\pgfpathqmoveto{70.0193bp}{80.1602bp}
\pgfpathqcurveto{70.0193bp}{80.9429bp}{69.3848bp}{81.5775bp}{68.6020bp}{81.5775bp}
\pgfpathqcurveto{67.8193bp}{81.5775bp}{67.1847bp}{80.9429bp}{67.1847bp}{80.1602bp}
\pgfpathqcurveto{67.1847bp}{79.3774bp}{67.8193bp}{78.7428bp}{68.6020bp}{78.7428bp}
\pgfpathqcurveto{69.3848bp}{78.7428bp}{70.0193bp}{79.3774bp}{70.0193bp}{80.1602bp}
\pgfpathclose
\pgfusepathqfillstroke
\end{pgfscope}
\begin{pgfscope}
\definecolor{fc}{rgb}{0.0000,0.0000,0.0000}
\pgfsetfillcolor{fc}
\pgftransformcm{1.0000}{0.0000}{0.0000}{1.0000}{\pgfqpoint{108.5059bp}{94.4302bp}}
\pgftransformscale{1.0000}
\pgftext[]{c}
\end{pgfscope}
\begin{pgfscope}
\definecolor{fc}{rgb}{0.0000,0.0000,0.0000}
\pgfsetfillcolor{fc}
\pgfsetlinewidth{0.5987bp}
\definecolor{sc}{rgb}{0.0000,0.0000,0.0000}
\pgfsetstrokecolor{sc}
\pgfsetmiterjoin
\pgfsetbuttcap
\pgfpathqmoveto{113.4253bp}{88.2694bp}
\pgfpathqcurveto{113.4253bp}{89.0521bp}{112.7907bp}{89.6867bp}{112.0079bp}{89.6867bp}
\pgfpathqcurveto{111.2252bp}{89.6867bp}{110.5906bp}{89.0521bp}{110.5906bp}{88.2694bp}
\pgfpathqcurveto{110.5906bp}{87.4866bp}{111.2252bp}{86.8520bp}{112.0079bp}{86.8520bp}
\pgfpathqcurveto{112.7907bp}{86.8520bp}{113.4253bp}{87.4866bp}{113.4253bp}{88.2694bp}
\pgfpathclose
\pgfusepathqfillstroke
\end{pgfscope}
\begin{pgfscope}
\definecolor{fc}{rgb}{0.0000,0.0000,0.0000}
\pgfsetfillcolor{fc}
\pgftransformcm{1.0000}{0.0000}{0.0000}{1.0000}{\pgfqpoint{142.1036bp}{122.4659bp}}
\pgftransformscale{1.0000}
\pgftext[]{b}
\end{pgfscope}
\begin{pgfscope}
\definecolor{fc}{rgb}{0.0000,0.0000,0.0000}
\pgfsetfillcolor{fc}
\pgfsetlinewidth{0.5987bp}
\definecolor{sc}{rgb}{0.0000,0.0000,0.0000}
\pgfsetstrokecolor{sc}
\pgfsetmiterjoin
\pgfsetbuttcap
\pgfpathqmoveto{138.8391bp}{117.1461bp}
\pgfpathqcurveto{138.8391bp}{117.9288bp}{138.2046bp}{118.5634bp}{137.4218bp}{118.5634bp}
\pgfpathqcurveto{136.6390bp}{118.5634bp}{136.0045bp}{117.9288bp}{136.0045bp}{117.1461bp}
\pgfpathqcurveto{136.0045bp}{116.3633bp}{136.6390bp}{115.7288bp}{137.4218bp}{115.7288bp}
\pgfpathqcurveto{138.2046bp}{115.7288bp}{138.8391bp}{116.3633bp}{138.8391bp}{117.1461bp}
\pgfpathclose
\pgfusepathqfillstroke
\end{pgfscope}
\begin{pgfscope}
\definecolor{fc}{rgb}{0.0000,0.0000,0.0000}
\pgfsetfillcolor{fc}
\pgftransformcm{1.0000}{0.0000}{0.0000}{1.0000}{\pgfqpoint{150.0000bp}{63.3183bp}}
\pgftransformscale{1.0000}
\pgftext[]{a}
\end{pgfscope}
\begin{pgfscope}
\definecolor{fc}{rgb}{0.0000,0.0000,0.0000}
\pgfsetfillcolor{fc}
\pgfsetlinewidth{0.5987bp}
\definecolor{sc}{rgb}{0.0000,0.0000,0.0000}
\pgfsetstrokecolor{sc}
\pgfsetmiterjoin
\pgfsetbuttcap
\pgfpathqmoveto{145.4939bp}{67.2085bp}
\pgfpathqcurveto{145.4939bp}{67.9912bp}{144.8594bp}{68.6258bp}{144.0766bp}{68.6258bp}
\pgfpathqcurveto{143.2938bp}{68.6258bp}{142.6593bp}{67.9912bp}{142.6593bp}{67.2085bp}
\pgfpathqcurveto{142.6593bp}{66.4257bp}{143.2938bp}{65.7911bp}{144.0766bp}{65.7911bp}
\pgfpathqcurveto{144.8594bp}{65.7911bp}{145.4939bp}{66.4257bp}{145.4939bp}{67.2085bp}
\pgfpathclose
\pgfusepathqfillstroke
\end{pgfscope}
\end{pgfpicture}

  \end{minipage}

  \vspace{1em}
  \begin{minipage}{0.45\textwidth}
    \begin{pgfpicture}
\pgfpathrectangle{\pgfpointorigin}{\pgfqpoint{150.0000bp}{150.0000bp}}
\pgfusepath{use as bounding box}
\begin{pgfscope}
\pgfsetlinewidth{0.5899bp}
\definecolor{sc}{rgb}{0.0000,0.0000,0.0000}
\pgfsetstrokecolor{sc}
\pgfsetmiterjoin
\pgfsetbuttcap
\pgfpathqmoveto{138.0334bp}{48.2895bp}
\pgfpathqlineto{138.0334bp}{100.0997bp}
\pgfusepathqstroke
\end{pgfscope}
\begin{pgfscope}
\pgfsetlinewidth{0.5899bp}
\definecolor{sc}{rgb}{0.0000,0.0000,0.0000}
\pgfsetstrokecolor{sc}
\pgfsetmiterjoin
\pgfsetbuttcap
\pgfpathqmoveto{138.0334bp}{100.0997bp}
\pgfpathqlineto{101.3981bp}{136.7350bp}
\pgfusepathqstroke
\end{pgfscope}
\begin{pgfscope}
\pgfsetlinewidth{0.5899bp}
\definecolor{sc}{rgb}{0.0000,0.0000,0.0000}
\pgfsetstrokecolor{sc}
\pgfsetmiterjoin
\pgfsetbuttcap
\pgfpathqmoveto{101.3981bp}{136.7350bp}
\pgfpathqlineto{138.0334bp}{48.2895bp}
\pgfusepathqstroke
\end{pgfscope}
\begin{pgfscope}
\pgfsetlinewidth{0.5899bp}
\definecolor{sc}{rgb}{0.0000,0.0000,0.0000}
\pgfsetstrokecolor{sc}
\pgfsetmiterjoin
\pgfsetbuttcap
\pgfpathqmoveto{49.5879bp}{136.7350bp}
\pgfpathqlineto{12.9525bp}{100.0997bp}
\pgfusepathqstroke
\end{pgfscope}
\begin{pgfscope}
\pgfsetlinewidth{0.5899bp}
\definecolor{sc}{rgb}{0.0000,0.0000,0.0000}
\pgfsetstrokecolor{sc}
\pgfsetmiterjoin
\pgfsetbuttcap
\pgfpathqmoveto{12.9525bp}{100.0997bp}
\pgfpathqlineto{101.3981bp}{11.6542bp}
\pgfusepathqstroke
\end{pgfscope}
\begin{pgfscope}
\pgfsetlinewidth{0.5899bp}
\definecolor{sc}{rgb}{0.0000,0.0000,0.0000}
\pgfsetstrokecolor{sc}
\pgfsetmiterjoin
\pgfsetbuttcap
\pgfpathqmoveto{12.9525bp}{48.2895bp}
\pgfpathqlineto{12.9525bp}{48.2895bp}
\pgfusepathqstroke
\end{pgfscope}
\begin{pgfscope}
\pgfsetlinewidth{0.5899bp}
\definecolor{sc}{rgb}{0.0000,0.0000,0.0000}
\pgfsetstrokecolor{sc}
\pgfsetmiterjoin
\pgfsetbuttcap
\pgfpathqmoveto{49.5879bp}{11.6542bp}
\pgfpathqlineto{49.5879bp}{11.6542bp}
\pgfusepathqstroke
\end{pgfscope}
\begin{pgfscope}
\definecolor{fc}{rgb}{0.0000,0.0000,0.0000}
\pgfsetfillcolor{fc}
\pgftransformshift{\pgfqpoint{36.6353bp}{11.6542bp}}
\pgftransformscale{1.0000}
\pgftext[]{H}
\end{pgfscope}
\begin{pgfscope}
\definecolor{fc}{rgb}{0.0000,0.0000,0.0000}
\pgfsetfillcolor{fc}
\pgfsetlinewidth{0.5899bp}
\definecolor{sc}{rgb}{0.0000,0.0000,0.0000}
\pgfsetstrokecolor{sc}
\pgfsetmiterjoin
\pgfsetbuttcap
\pgfpathqmoveto{52.1784bp}{11.6542bp}
\pgfpathqcurveto{52.1784bp}{13.0849bp}{51.0186bp}{14.2447bp}{49.5879bp}{14.2447bp}
\pgfpathqcurveto{48.1572bp}{14.2447bp}{46.9974bp}{13.0849bp}{46.9974bp}{11.6542bp}
\pgfpathqcurveto{46.9974bp}{10.2235bp}{48.1572bp}{9.0637bp}{49.5879bp}{9.0637bp}
\pgfpathqcurveto{51.0186bp}{9.0637bp}{52.1784bp}{10.2235bp}{52.1784bp}{11.6542bp}
\pgfpathclose
\pgfusepathqfillstroke
\end{pgfscope}
\begin{pgfscope}
\definecolor{fc}{rgb}{0.0000,0.0000,0.0000}
\pgfsetfillcolor{fc}
\pgftransformshift{\pgfqpoint{0.0000bp}{48.2895bp}}
\pgftransformscale{1.0000}
\pgftext[]{G}
\end{pgfscope}
\begin{pgfscope}
\definecolor{fc}{rgb}{0.0000,0.0000,0.0000}
\pgfsetfillcolor{fc}
\pgfsetlinewidth{0.5899bp}
\definecolor{sc}{rgb}{0.0000,0.0000,0.0000}
\pgfsetstrokecolor{sc}
\pgfsetmiterjoin
\pgfsetbuttcap
\pgfpathqmoveto{15.5431bp}{48.2895bp}
\pgfpathqcurveto{15.5431bp}{49.7202bp}{14.3832bp}{50.8800bp}{12.9525bp}{50.8800bp}
\pgfpathqcurveto{11.5218bp}{50.8800bp}{10.3620bp}{49.7202bp}{10.3620bp}{48.2895bp}
\pgfpathqcurveto{10.3620bp}{46.8588bp}{11.5218bp}{45.6990bp}{12.9525bp}{45.6990bp}
\pgfpathqcurveto{14.3832bp}{45.6990bp}{15.5431bp}{46.8588bp}{15.5431bp}{48.2895bp}
\pgfpathclose
\pgfusepathqfillstroke
\end{pgfscope}
\begin{pgfscope}
\definecolor{fc}{rgb}{0.0000,0.0000,0.0000}
\pgfsetfillcolor{fc}
\pgftransformshift{\pgfqpoint{110.5569bp}{2.4953bp}}
\pgftransformscale{1.0000}
\pgftext[]{F}
\end{pgfscope}
\begin{pgfscope}
\definecolor{fc}{rgb}{0.0000,0.0000,0.0000}
\pgfsetfillcolor{fc}
\pgfsetlinewidth{0.5899bp}
\definecolor{sc}{rgb}{0.0000,0.0000,0.0000}
\pgfsetstrokecolor{sc}
\pgfsetmiterjoin
\pgfsetbuttcap
\pgfpathqmoveto{103.9886bp}{11.6542bp}
\pgfpathqcurveto{103.9886bp}{13.0849bp}{102.8288bp}{14.2447bp}{101.3981bp}{14.2447bp}
\pgfpathqcurveto{99.9674bp}{14.2447bp}{98.8076bp}{13.0849bp}{98.8076bp}{11.6542bp}
\pgfpathqcurveto{98.8076bp}{10.2235bp}{99.9674bp}{9.0637bp}{101.3981bp}{9.0637bp}
\pgfpathqcurveto{102.8288bp}{9.0637bp}{103.9886bp}{10.2235bp}{103.9886bp}{11.6542bp}
\pgfpathclose
\pgfusepathqfillstroke
\end{pgfscope}
\begin{pgfscope}
\definecolor{fc}{rgb}{0.0000,0.0000,0.0000}
\pgfsetfillcolor{fc}
\pgftransformshift{\pgfqpoint{0.0000bp}{100.0997bp}}
\pgftransformscale{1.0000}
\pgftext[]{E}
\end{pgfscope}
\begin{pgfscope}
\definecolor{fc}{rgb}{0.0000,0.0000,0.0000}
\pgfsetfillcolor{fc}
\pgfsetlinewidth{0.5899bp}
\definecolor{sc}{rgb}{0.0000,0.0000,0.0000}
\pgfsetstrokecolor{sc}
\pgfsetmiterjoin
\pgfsetbuttcap
\pgfpathqmoveto{15.5431bp}{100.0997bp}
\pgfpathqcurveto{15.5431bp}{101.5304bp}{14.3832bp}{102.6902bp}{12.9525bp}{102.6902bp}
\pgfpathqcurveto{11.5218bp}{102.6902bp}{10.3620bp}{101.5304bp}{10.3620bp}{100.0997bp}
\pgfpathqcurveto{10.3620bp}{98.6690bp}{11.5218bp}{97.5092bp}{12.9525bp}{97.5092bp}
\pgfpathqcurveto{14.3832bp}{97.5092bp}{15.5431bp}{98.6690bp}{15.5431bp}{100.0997bp}
\pgfpathclose
\pgfusepathqfillstroke
\end{pgfscope}
\begin{pgfscope}
\definecolor{fc}{rgb}{0.0000,0.0000,0.0000}
\pgfsetfillcolor{fc}
\pgftransformshift{\pgfqpoint{58.7467bp}{145.8939bp}}
\pgftransformscale{1.0000}
\pgftext[]{D}
\end{pgfscope}
\begin{pgfscope}
\definecolor{fc}{rgb}{0.0000,0.0000,0.0000}
\pgfsetfillcolor{fc}
\pgfsetlinewidth{0.5899bp}
\definecolor{sc}{rgb}{0.0000,0.0000,0.0000}
\pgfsetstrokecolor{sc}
\pgfsetmiterjoin
\pgfsetbuttcap
\pgfpathqmoveto{52.1784bp}{136.7350bp}
\pgfpathqcurveto{52.1784bp}{138.1657bp}{51.0186bp}{139.3255bp}{49.5879bp}{139.3255bp}
\pgfpathqcurveto{48.1572bp}{139.3255bp}{46.9974bp}{138.1657bp}{46.9974bp}{136.7350bp}
\pgfpathqcurveto{46.9974bp}{135.3043bp}{48.1572bp}{134.1445bp}{49.5879bp}{134.1445bp}
\pgfpathqcurveto{51.0186bp}{134.1445bp}{52.1784bp}{135.3043bp}{52.1784bp}{136.7350bp}
\pgfpathclose
\pgfusepathqfillstroke
\end{pgfscope}
\begin{pgfscope}
\definecolor{fc}{rgb}{0.0000,0.0000,0.0000}
\pgfsetfillcolor{fc}
\pgftransformshift{\pgfqpoint{94.2020bp}{147.5047bp}}
\pgftransformscale{1.0000}
\pgftext[]{C}
\end{pgfscope}
\begin{pgfscope}
\definecolor{fc}{rgb}{0.0000,0.0000,0.0000}
\pgfsetfillcolor{fc}
\pgfsetlinewidth{0.5899bp}
\definecolor{sc}{rgb}{0.0000,0.0000,0.0000}
\pgfsetstrokecolor{sc}
\pgfsetmiterjoin
\pgfsetbuttcap
\pgfpathqmoveto{103.9886bp}{136.7350bp}
\pgfpathqcurveto{103.9886bp}{138.1657bp}{102.8288bp}{139.3255bp}{101.3981bp}{139.3255bp}
\pgfpathqcurveto{99.9674bp}{139.3255bp}{98.8076bp}{138.1657bp}{98.8076bp}{136.7350bp}
\pgfpathqcurveto{98.8076bp}{135.3043bp}{99.9674bp}{134.1445bp}{101.3981bp}{134.1445bp}
\pgfpathqcurveto{102.8288bp}{134.1445bp}{103.9886bp}{135.3043bp}{103.9886bp}{136.7350bp}
\pgfpathclose
\pgfusepathqfillstroke
\end{pgfscope}
\begin{pgfscope}
\definecolor{fc}{rgb}{0.0000,0.0000,0.0000}
\pgfsetfillcolor{fc}
\pgftransformshift{\pgfqpoint{150.0000bp}{105.0564bp}}
\pgftransformscale{1.0000}
\pgftext[]{B}
\end{pgfscope}
\begin{pgfscope}
\definecolor{fc}{rgb}{0.0000,0.0000,0.0000}
\pgfsetfillcolor{fc}
\pgfsetlinewidth{0.5899bp}
\definecolor{sc}{rgb}{0.0000,0.0000,0.0000}
\pgfsetstrokecolor{sc}
\pgfsetmiterjoin
\pgfsetbuttcap
\pgfpathqmoveto{140.6239bp}{100.0997bp}
\pgfpathqcurveto{140.6239bp}{101.5304bp}{139.4641bp}{102.6902bp}{138.0334bp}{102.6902bp}
\pgfpathqcurveto{136.6027bp}{102.6902bp}{135.4429bp}{101.5304bp}{135.4429bp}{100.0997bp}
\pgfpathqcurveto{135.4429bp}{98.6690bp}{136.6027bp}{97.5092bp}{138.0334bp}{97.5092bp}
\pgfpathqcurveto{139.4641bp}{97.5092bp}{140.6239bp}{98.6690bp}{140.6239bp}{100.0997bp}
\pgfpathclose
\pgfusepathqfillstroke
\end{pgfscope}
\begin{pgfscope}
\definecolor{fc}{rgb}{0.0000,0.0000,0.0000}
\pgfsetfillcolor{fc}
\pgftransformshift{\pgfqpoint{140.5603bp}{35.5858bp}}
\pgftransformscale{1.0000}
\pgftext[]{A}
\end{pgfscope}
\begin{pgfscope}
\definecolor{fc}{rgb}{0.0000,0.0000,0.0000}
\pgfsetfillcolor{fc}
\pgfsetlinewidth{0.5899bp}
\definecolor{sc}{rgb}{0.0000,0.0000,0.0000}
\pgfsetstrokecolor{sc}
\pgfsetmiterjoin
\pgfsetbuttcap
\pgfpathqmoveto{140.6239bp}{48.2895bp}
\pgfpathqcurveto{140.6239bp}{49.7202bp}{139.4641bp}{50.8800bp}{138.0334bp}{50.8800bp}
\pgfpathqcurveto{136.6027bp}{50.8800bp}{135.4429bp}{49.7202bp}{135.4429bp}{48.2895bp}
\pgfpathqcurveto{135.4429bp}{46.8588bp}{136.6027bp}{45.6990bp}{138.0334bp}{45.6990bp}
\pgfpathqcurveto{139.4641bp}{45.6990bp}{140.6239bp}{46.8588bp}{140.6239bp}{48.2895bp}
\pgfpathclose
\pgfusepathqfillstroke
\end{pgfscope}
\end{pgfpicture}

  \end{minipage}
  \begin{minipage}{0.45\textwidth}
    \begin{align*}
      G &= (V,E) \\
      V &= \{\alpha, \beta, \gamma, \delta, \epsilon, \zeta, \eta,
          \theta\} \\
      E &= \{\{\alpha, \delta\}, \{\theta, \eta\}, \{\beta, \alpha\},
          \{\zeta, \delta\}, \{\epsilon, \eta\}, \{\gamma, \alpha\}\}
    \end{align*}
    {\scriptsize
      Key: $\alpha$ = alpha, $\beta$ = beta, $\gamma$ = gamma, $\delta$ =
      delta, $\epsilon$ = epsilon, \\ $\zeta$ = zeta, $\eta$ = eta,
      $\theta$ = theta
    }
  \end{minipage}
  \end{center}

  \vspace{1em}
  \begin{defn}
    A (simple) \term{graph} $G = (V,E)$ is a set of \term{vertices} $V$
    together with a set $E$ of \term{edges}, where each edge consists
    of a set of two vertices.
  \end{defn}

  Above are shown four example graphs.  In clockwise order from the
  upper-left, we will refer to them as the numeric graph, the
  lowercase graph, the Greek graph, and the uppercase graph.  Here are
  some facts about the graphs:

  \begin{itemize}
  \item $C$ and $B$ are \term{adjacent} (aka \term{neighbors}).  So
    are $6$ and $5$.  $C$ and $F$ are not adjacent.  $f$ and $d$ are
    not adjacent.  $H$ is not adjacent to itself.
  \item The \term{degree} of vertex $C$ is 2.  The degree of vertex $H$ is
    0.  The degree of vertex $g$ is 4.
  \item Vertices $D$, $F$, $4$, $9$, $a$, $b$, $f$, $h$, $i$, $j$ are
    \term{leaves}.  No other vertex is a leaf.
  \end{itemize}
\end{model}

Don't worry if you don't remember all these graph terms!  In fact you
should probably try to forget what you think you might remember and
just focus on the information in the model. Part of the point of this
exercise is to help you either recall these terms, or learn them for
the first time.

\begin{objective}
  Students will understand and apply graph terms \term{edge},
  \term{vertex}, \term{adjacent}/\term{neighbor}, \term{degree}, \term{leaf},
  \term{path}, \term{connected}, \term{connected component},
  \term{cycle}, \term{cyclic}, \term{acyclic}, and \term{tree}.
\end{objective}

% XXX probably need more exploration questions first!  Currently this
% leads off with an invention question.
\begin{questions}
\item What does it mean for two vertices to be \term{adjacent}?
\item Are $\alpha$ and $\beta$ adjacent?   What about $\alpha$ and $\zeta$?
\item What is the \term{degree} of a vertex?  Use the term
  \term{adjacent} in your answer.
\item What is the degree of $\alpha$?
\item What is the definition of a \term{leaf}?  Use the term
  \term{degree} in your answer.
\item Which vertices are leaves in the Greek graph?
\item In the space below, draw a graph with at least three leaves, one
  vertex with degree five, and at least one pair of vertices that are
  not adjacent.
\end{questions}

\newpage
\begin{model*}{More graph terms}{graph-terms}
Here are a few more facts about the graphs from
Model~\ref{model:graphs}:

\begin{itemize}
\item Vertices $7$--$8$--$5$--$2$--$0$ are a \term{path}.
  $2$--$1$--$0$--$2$--$1$ is also a path.  $H$ by itself is a
  path. $H$--$F$--$B$ is not a path.  $8$--$2$--$1$ is also not a path.
\item Vertices $f$ and $g$ are \term{connected}.  So are $7$ and $8$.
  $C$ and $F$ are not connected.  $H$ is connected to itself.
\item The graph with numbers is a \term{connected graph}.  So is the
  graph with lowercase letters.  The graph with uppercase letters is
  not a connected graph (it is \term{disconnected}).
\item The numeric and lowercase graphs have one \term{connected
    component} each.  The uppercase graph has four connected
  components.
\item $8$--$7$--$6$--$5$--$8$ is a \term{cycle}.  So is $C$--$A$--$B$--$C$.
  $h$--$g$--$i$--$h$ is not a cycle.  $h$--$g$--$h$ is not a cycle
  either.  Nor is $H$--$H$.
\item The numeric graph and uppercase graph are \term{cyclic} graphs.
  The lowercase graph is \term{acyclic}.
\item The lowercase graph is a \term{tree}.  None of the other graphs
  are trees (not even the Greek one).
\end{itemize}
\end{model*}

\begin{questions}
\item How many vertices can be in a \term{path}?
\item What do you think is the definition of a \term{path}?
\item Give an example of a path in the Greek graph.
\item Can two vertices be connected but not adjacent?  If so, give an
  example.
\item Can two vertices be adjacent but not connected?  If so, give an
  example.
\item What do you think it means for two vertices to be connected?  Be
  sure to use the term \term{path} in your answer.
\item Is the Greek graph connected?
\item What do you think is the definition of a \term{connected graph}?
\item How many vertices can be in a connected component?
\item How many \term{connected components} does the Greek graph have?
\item Is the set of vertices $\{E, F\}$ a connected component?  Why or
  why not?
\item Write a definition for \term{connected component}.
\item Write an ``if and only if'' statement using the terms
  \term{connected graph} and \term{connected component}.
\item What is a \term{cycle}?  Use the term \term{path} in your
  answer.
\item Does your definition for \term{cycle} correctly explain why
  $h$--$g$--$h$ is not a cycle?  If not, revise it so it does.
\item Does the Greek graph have a cycle?
\item Is the Greek graph cyclic or acyclic?
\item What do you think is the definition of a tree?  You should use
  two of the other graph terms in your
  definition.\marginnote{Warning---a tree graph is not quite the same
    thing as a tree data structure!}
\end{questions}

\pause

\section{Some proofs about graphs}

\begin{objective}
  Students will write proofs about graphs.
\end{objective}

\begin{thm}[Trees]
  Let $G = (V,E)$ be a graph with $|V| = n \geq 1$.  Any two of the following
  imply the third:
  \begin{enumerate}
  \item $G$ is connected.
  \item $G$ is acyclic.
  \item $G$ has $n-1$ edges.
  \end{enumerate}
\end{thm}

We will take each pair of statements in turn and show that they imply
the third.  Fill in the blanks to complete the following proofs!  Note
that the size of a blank does not necessarily correspond to the
amount of stuff you should write in it.

\begin{lem} \label{lem:onetwothree}
  $(1), (2) \implies (3)$.  That is: let $G = (V,E)$ be a graph with $|V|
  = n \geq 1$.  If \blank\linebreak and \blank,\linebreak then \blank.
\end{lem}

\begin{proof}
  Let $P(n)$ denote the statement ``Any graph $G$ with $n$ vertices
  which is \blank and \blank\linebreak
  must have \blank.''\linebreak  We wish to show
  that $P(n)$ holds for all $n \geq 1$.

  The proof is by \blank.
  \begin{itemize}
  \item The base case is when \blank.\linebreak In this case, $G$ must
    be \blank\linebreak which indeed \blank.
  \item For the induction step, suppose $P(k)$ holds for some $k \geq
    1$.  That is, suppose that any graph with \blank vertices\linebreak which is
    \blank\linebreak must have \blank.\linebreak  Then we wish to show
    $P(k+1)$, that is, any graph with \blank\linebreak vertices which is
    connected and acyclic must have \blank.

    So, let $G$ be a graph with \blank vertices which is\linebreak
    \mbox{}\blank and \blank.\linebreak  We claim that $G$
    must have some vertex which is a leaf, that is, a vertex of degree
    \blank,\newline which we can show as follows:
    \begin{itemize}
    \item $G$ cannot have any vertices of degree \blank\linebreak because
      \blank.
    \item It also cannot be the case that every vertex of $G$ has
      degree $\geq$ \blank.  If they did, then we could find a \blank
      by starting at any\linebreak vertex and walking along edges
      randomly until \blank;\linebreak we would never get stuck
      because \blank.\linebreak However, this is impossible because we
      assumed \blank.
    \end{itemize}
    Hence, $G$ must have some vertex which \blank.\linebreak If we
    delete this vertex along with the edge adjacent to it, it results
    in a graph $G'$ with only \blank vertices;\linebreak we note that
    $G'$ is still \blank\linebreak because \blank\linebreak and also
    \blank\linebreak because \blank.\linebreak  Hence we may apply the
    inductive hypothesis to conclude that $G'$\linebreak \mbox{}\blank.
    Adding the deleted vertex and edge\linebreak back to $G'$ shows that $G$
    \blank,\linebreak which is what we wanted to show.
  \end{itemize}
\end{proof}

Let's do one more!  (You will do the third on your HW.)

\begin{lem}
  $(2),(3) \implies (1)$, that is, \blank\linebreak \mbox{}\blank.
\end{lem}

\begin{proof}
  This proof uses a \term{counting argument}: we will show what we
  wish to show by counting things in multiple ways.

  Let $c$ denote the number of connected components of $G$.  We want
  to show that \blank.

  Number the components of $G$ from $1 \dots c$, and say that
  component $i$ has $n_i$ vertices.  Then \[ \sum_{i=1}^c n_i =
    \underline{\phantom{XXXXXXXXXX}} \] because \blank.\linebreak  Each
  connected component is by definition a \blank graph;\linebreak each
  component must also be \blank\linebreak since we assumed that $G$ is.  Hence
  we may apply \pref{lem:onetwothree} to conclude that component $i$
  \blank.\linebreak
  Adding these up, the total number of edges in $G$ is
  \[ |E| = \sum_{i=1}^c \underline{\phantom{XXXXXX}} =
    \underline{\phantom{XXXXXXXXXXXXX}} \] But we already assumed the
  number of edges in $G$ is \blank,\linebreak and hence \blank as desired.
\end{proof}

\end{document}
