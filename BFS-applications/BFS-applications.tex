% -*- compile-command: "pdflatex BFS-applications.tex" -*-
\documentclass{tufte-handout}

\usepackage{algo-activity}

\title{\thecourse: Applications of BFS}
\date{}

\begin{document}

\maketitle

Suppose we have a graph $G = (V,E)$.  A given graph could have few
edges, or lots of edges, or anything in between.  Let's think about
the range of possible relationships between $V$ and $E$.

\begin{questions}
\item The smallest possible value of $|E|$ is \blank.
\item $|E|$ is $O\Big( \qquad\qquad \Big)$ because \blank.
\item When $G$ is a tree, $|E|$ is $\Theta\Big( \qquad\qquad \Big)$
  because \blank.
\end{questions}

Now, recall from last class that we showed breadth-first search (BFS) can
be implemented to run in $\Theta(|V| + |E|)$ time.

\begin{questions}
\item In terms of $\Theta$, how fast does BFS run, as a function of
  $|V|$, when $G$ is a tree?
\item How fast does BFS run, as a function of $|V|$, when $G$ is very
  dense, \ie it contains some constant fraction (say, half) of all
  possible edges?
\end{questions}

\newpage
\section{A first application of BFS}

\begin{questions}
  \item Describe an algorithm to find the connected components of a
    graph $G$.

    \textbf{Input}: a graph $G = (V,E)$ \\
    \textbf{Output}: a set of sets of vertices,
    \texttt{Set<Set<Vertex{>}>}, where each set contains the vertices in
    some (maximal) connected component.  That is, all the vertices
    within each set should be connected; no vertex should be
    connected to vertices in any other set; and every vertex in $V$
    should be contained in exactly one of the sets.

    For example, given the graph below, the algorithm should return
    $\{\{D,E,F\}, \{C,B,A\}, \{G\}, \{H\}\}$.

    \begin{center}
      \begin{pgfpicture}
\pgfpathrectangle{\pgfpointorigin}{\pgfqpoint{200.0000bp}{193.0000bp}}
\pgfusepath{use as bounding box}
\begin{pgfscope}
\pgfsetlinewidth{0.7866bp}
\definecolor{sc}{rgb}{0.0000,0.0000,0.0000}
\pgfsetstrokecolor{sc}
\pgfsetmiterjoin
\pgfsetbuttcap
\pgfpathqmoveto{184.0445bp}{61.0589bp}
\pgfpathqlineto{184.0445bp}{130.1391bp}
\pgfusepathqstroke
\end{pgfscope}
\begin{pgfscope}
\pgfsetlinewidth{0.7866bp}
\definecolor{sc}{rgb}{0.0000,0.0000,0.0000}
\pgfsetstrokecolor{sc}
\pgfsetmiterjoin
\pgfsetbuttcap
\pgfpathqmoveto{184.0445bp}{130.1391bp}
\pgfpathqlineto{135.1974bp}{178.9863bp}
\pgfusepathqstroke
\end{pgfscope}
\begin{pgfscope}
\pgfsetlinewidth{0.7866bp}
\definecolor{sc}{rgb}{0.0000,0.0000,0.0000}
\pgfsetstrokecolor{sc}
\pgfsetmiterjoin
\pgfsetbuttcap
\pgfpathqmoveto{135.1974bp}{178.9863bp}
\pgfpathqlineto{184.0445bp}{61.0589bp}
\pgfusepathqstroke
\end{pgfscope}
\begin{pgfscope}
\pgfsetlinewidth{0.7866bp}
\definecolor{sc}{rgb}{0.0000,0.0000,0.0000}
\pgfsetstrokecolor{sc}
\pgfsetmiterjoin
\pgfsetbuttcap
\pgfpathqmoveto{66.1172bp}{178.9863bp}
\pgfpathqlineto{17.2701bp}{130.1391bp}
\pgfusepathqstroke
\end{pgfscope}
\begin{pgfscope}
\pgfsetlinewidth{0.7866bp}
\definecolor{sc}{rgb}{0.0000,0.0000,0.0000}
\pgfsetstrokecolor{sc}
\pgfsetmiterjoin
\pgfsetbuttcap
\pgfpathqmoveto{17.2701bp}{130.1391bp}
\pgfpathqlineto{135.1974bp}{12.2118bp}
\pgfusepathqstroke
\end{pgfscope}
\begin{pgfscope}
\definecolor{fc}{rgb}{0.0000,0.0000,0.0000}
\pgfsetfillcolor{fc}
\pgftransformshift{\pgfqpoint{48.8471bp}{12.2118bp}}
\pgftransformscale{1.0000}
\pgftext[]{H}
\end{pgfscope}
\begin{pgfscope}
\definecolor{fc}{rgb}{0.0000,0.0000,0.0000}
\pgfsetfillcolor{fc}
\pgfsetlinewidth{0.7866bp}
\definecolor{sc}{rgb}{0.0000,0.0000,0.0000}
\pgfsetstrokecolor{sc}
\pgfsetmiterjoin
\pgfsetbuttcap
\pgfpathqmoveto{69.5712bp}{12.2118bp}
\pgfpathqcurveto{69.5712bp}{14.1194bp}{68.0248bp}{15.6658bp}{66.1172bp}{15.6658bp}
\pgfpathqcurveto{64.2096bp}{15.6658bp}{62.6632bp}{14.1194bp}{62.6632bp}{12.2118bp}
\pgfpathqcurveto{62.6632bp}{10.3042bp}{64.2096bp}{8.7578bp}{66.1172bp}{8.7578bp}
\pgfpathqcurveto{68.0248bp}{8.7578bp}{69.5712bp}{10.3042bp}{69.5712bp}{12.2118bp}
\pgfpathclose
\pgfusepathqfillstroke
\end{pgfscope}
\begin{pgfscope}
\definecolor{fc}{rgb}{0.0000,0.0000,0.0000}
\pgfsetfillcolor{fc}
\pgftransformshift{\pgfqpoint{-0.0000bp}{61.0589bp}}
\pgftransformscale{1.0000}
\pgftext[]{G}
\end{pgfscope}
\begin{pgfscope}
\definecolor{fc}{rgb}{0.0000,0.0000,0.0000}
\pgfsetfillcolor{fc}
\pgfsetlinewidth{0.7866bp}
\definecolor{sc}{rgb}{0.0000,0.0000,0.0000}
\pgfsetstrokecolor{sc}
\pgfsetmiterjoin
\pgfsetbuttcap
\pgfpathqmoveto{20.7241bp}{61.0589bp}
\pgfpathqcurveto{20.7241bp}{62.9665bp}{19.1777bp}{64.5129bp}{17.2701bp}{64.5129bp}
\pgfpathqcurveto{15.3625bp}{64.5129bp}{13.8161bp}{62.9665bp}{13.8161bp}{61.0589bp}
\pgfpathqcurveto{13.8161bp}{59.1513bp}{15.3625bp}{57.6049bp}{17.2701bp}{57.6049bp}
\pgfpathqcurveto{19.1777bp}{57.6049bp}{20.7241bp}{59.1513bp}{20.7241bp}{61.0589bp}
\pgfpathclose
\pgfusepathqfillstroke
\end{pgfscope}
\begin{pgfscope}
\definecolor{fc}{rgb}{0.0000,0.0000,0.0000}
\pgfsetfillcolor{fc}
\pgftransformshift{\pgfqpoint{147.4092bp}{0.0000bp}}
\pgftransformscale{1.0000}
\pgftext[]{F}
\end{pgfscope}
\begin{pgfscope}
\definecolor{fc}{rgb}{0.0000,0.0000,0.0000}
\pgfsetfillcolor{fc}
\pgfsetlinewidth{0.7866bp}
\definecolor{sc}{rgb}{0.0000,0.0000,0.0000}
\pgfsetstrokecolor{sc}
\pgfsetmiterjoin
\pgfsetbuttcap
\pgfpathqmoveto{138.6514bp}{12.2118bp}
\pgfpathqcurveto{138.6514bp}{14.1194bp}{137.1050bp}{15.6658bp}{135.1974bp}{15.6658bp}
\pgfpathqcurveto{133.2898bp}{15.6658bp}{131.7434bp}{14.1194bp}{131.7434bp}{12.2118bp}
\pgfpathqcurveto{131.7434bp}{10.3042bp}{133.2898bp}{8.7578bp}{135.1974bp}{8.7578bp}
\pgfpathqcurveto{137.1050bp}{8.7578bp}{138.6514bp}{10.3042bp}{138.6514bp}{12.2118bp}
\pgfpathclose
\pgfusepathqfillstroke
\end{pgfscope}
\begin{pgfscope}
\definecolor{fc}{rgb}{0.0000,0.0000,0.0000}
\pgfsetfillcolor{fc}
\pgftransformshift{\pgfqpoint{-0.0000bp}{130.1391bp}}
\pgftransformscale{1.0000}
\pgftext[]{E}
\end{pgfscope}
\begin{pgfscope}
\definecolor{fc}{rgb}{0.0000,0.0000,0.0000}
\pgfsetfillcolor{fc}
\pgfsetlinewidth{0.7866bp}
\definecolor{sc}{rgb}{0.0000,0.0000,0.0000}
\pgfsetstrokecolor{sc}
\pgfsetmiterjoin
\pgfsetbuttcap
\pgfpathqmoveto{20.7241bp}{130.1391bp}
\pgfpathqcurveto{20.7241bp}{132.0467bp}{19.1777bp}{133.5932bp}{17.2701bp}{133.5932bp}
\pgfpathqcurveto{15.3625bp}{133.5932bp}{13.8161bp}{132.0467bp}{13.8161bp}{130.1391bp}
\pgfpathqcurveto{13.8161bp}{128.2315bp}{15.3625bp}{126.6851bp}{17.2701bp}{126.6851bp}
\pgfpathqcurveto{19.1777bp}{126.6851bp}{20.7241bp}{128.2315bp}{20.7241bp}{130.1391bp}
\pgfpathclose
\pgfusepathqfillstroke
\end{pgfscope}
\begin{pgfscope}
\definecolor{fc}{rgb}{0.0000,0.0000,0.0000}
\pgfsetfillcolor{fc}
\pgftransformshift{\pgfqpoint{78.3290bp}{191.1980bp}}
\pgftransformscale{1.0000}
\pgftext[]{D}
\end{pgfscope}
\begin{pgfscope}
\definecolor{fc}{rgb}{0.0000,0.0000,0.0000}
\pgfsetfillcolor{fc}
\pgfsetlinewidth{0.7866bp}
\definecolor{sc}{rgb}{0.0000,0.0000,0.0000}
\pgfsetstrokecolor{sc}
\pgfsetmiterjoin
\pgfsetbuttcap
\pgfpathqmoveto{69.5712bp}{178.9863bp}
\pgfpathqcurveto{69.5712bp}{180.8939bp}{68.0248bp}{182.4403bp}{66.1172bp}{182.4403bp}
\pgfpathqcurveto{64.2096bp}{182.4403bp}{62.6632bp}{180.8939bp}{62.6632bp}{178.9863bp}
\pgfpathqcurveto{62.6632bp}{177.0787bp}{64.2096bp}{175.5322bp}{66.1172bp}{175.5322bp}
\pgfpathqcurveto{68.0248bp}{175.5322bp}{69.5712bp}{177.0787bp}{69.5712bp}{178.9863bp}
\pgfpathclose
\pgfusepathqfillstroke
\end{pgfscope}
\begin{pgfscope}
\definecolor{fc}{rgb}{0.0000,0.0000,0.0000}
\pgfsetfillcolor{fc}
\pgftransformshift{\pgfqpoint{125.6027bp}{193.3458bp}}
\pgftransformscale{1.0000}
\pgftext[]{C}
\end{pgfscope}
\begin{pgfscope}
\definecolor{fc}{rgb}{0.0000,0.0000,0.0000}
\pgfsetfillcolor{fc}
\pgfsetlinewidth{0.7866bp}
\definecolor{sc}{rgb}{0.0000,0.0000,0.0000}
\pgfsetstrokecolor{sc}
\pgfsetmiterjoin
\pgfsetbuttcap
\pgfpathqmoveto{138.6514bp}{178.9863bp}
\pgfpathqcurveto{138.6514bp}{180.8939bp}{137.1050bp}{182.4403bp}{135.1974bp}{182.4403bp}
\pgfpathqcurveto{133.2898bp}{182.4403bp}{131.7434bp}{180.8939bp}{131.7434bp}{178.9863bp}
\pgfpathqcurveto{131.7434bp}{177.0787bp}{133.2898bp}{175.5322bp}{135.1974bp}{175.5322bp}
\pgfpathqcurveto{137.1050bp}{175.5322bp}{138.6514bp}{177.0787bp}{138.6514bp}{178.9863bp}
\pgfpathclose
\pgfusepathqfillstroke
\end{pgfscope}
\begin{pgfscope}
\definecolor{fc}{rgb}{0.0000,0.0000,0.0000}
\pgfsetfillcolor{fc}
\pgftransformshift{\pgfqpoint{200.0000bp}{136.7481bp}}
\pgftransformscale{1.0000}
\pgftext[]{B}
\end{pgfscope}
\begin{pgfscope}
\definecolor{fc}{rgb}{0.0000,0.0000,0.0000}
\pgfsetfillcolor{fc}
\pgfsetlinewidth{0.7866bp}
\definecolor{sc}{rgb}{0.0000,0.0000,0.0000}
\pgfsetstrokecolor{sc}
\pgfsetmiterjoin
\pgfsetbuttcap
\pgfpathqmoveto{187.4986bp}{130.1391bp}
\pgfpathqcurveto{187.4986bp}{132.0467bp}{185.9521bp}{133.5932bp}{184.0445bp}{133.5932bp}
\pgfpathqcurveto{182.1369bp}{133.5932bp}{180.5905bp}{132.0467bp}{180.5905bp}{130.1391bp}
\pgfpathqcurveto{180.5905bp}{128.2315bp}{182.1369bp}{126.6851bp}{184.0445bp}{126.6851bp}
\pgfpathqcurveto{185.9521bp}{126.6851bp}{187.4986bp}{128.2315bp}{187.4986bp}{130.1391bp}
\pgfpathclose
\pgfusepathqfillstroke
\end{pgfscope}
\begin{pgfscope}
\definecolor{fc}{rgb}{0.0000,0.0000,0.0000}
\pgfsetfillcolor{fc}
\pgftransformshift{\pgfqpoint{187.4138bp}{44.1207bp}}
\pgftransformscale{1.0000}
\pgftext[]{A}
\end{pgfscope}
\begin{pgfscope}
\definecolor{fc}{rgb}{0.0000,0.0000,0.0000}
\pgfsetfillcolor{fc}
\pgfsetlinewidth{0.7866bp}
\definecolor{sc}{rgb}{0.0000,0.0000,0.0000}
\pgfsetstrokecolor{sc}
\pgfsetmiterjoin
\pgfsetbuttcap
\pgfpathqmoveto{187.4986bp}{61.0589bp}
\pgfpathqcurveto{187.4986bp}{62.9665bp}{185.9521bp}{64.5129bp}{184.0445bp}{64.5129bp}
\pgfpathqcurveto{182.1369bp}{64.5129bp}{180.5905bp}{62.9665bp}{180.5905bp}{61.0589bp}
\pgfpathqcurveto{180.5905bp}{59.1513bp}{182.1369bp}{57.6049bp}{184.0445bp}{57.6049bp}
\pgfpathqcurveto{185.9521bp}{57.6049bp}{187.4986bp}{59.1513bp}{187.4986bp}{61.0589bp}
\pgfpathclose
\pgfusepathqfillstroke
\end{pgfscope}
\end{pgfpicture}

    \end{center}

    Describe your algorithm (using informal prose or pseudocode) and
    analyze its asymptotic running time.
\end{questions}

\pause

\section{A second application of BFS}

\begin{model}{Directed graphs}{directed}
  \begin{center}
    \begin{minipage}{0.45\textwidth}
      \begin{pgfpicture}
\pgfpathrectangle{\pgfpointorigin}{\pgfqpoint{150.0000bp}{133.0000bp}}
\pgfusepath{use as bounding box}
\begin{pgfscope}
\definecolor{fc}{rgb}{0.0000,0.0000,0.0000}
\pgfsetfillcolor{fc}
\pgftransformshift{\pgfqpoint{150.0000bp}{16.4977bp}}
\pgftransformscale{1.0000}
\pgftext[]{$9$}
\end{pgfscope}
\begin{pgfscope}
\definecolor{fc}{rgb}{0.0000,0.0000,0.0000}
\pgfsetfillcolor{fc}
\pgfsetlinewidth{0.5655bp}
\definecolor{sc}{rgb}{0.0000,0.0000,0.0000}
\pgfsetstrokecolor{sc}
\pgfsetmiterjoin
\pgfsetbuttcap
\pgfpathqmoveto{146.0004bp}{18.9497bp}
\pgfpathqcurveto{146.0004bp}{19.5786bp}{145.4906bp}{20.0884bp}{144.8617bp}{20.0884bp}
\pgfpathqcurveto{144.2328bp}{20.0884bp}{143.7230bp}{19.5786bp}{143.7230bp}{18.9497bp}
\pgfpathqcurveto{143.7230bp}{18.3208bp}{144.2328bp}{17.8110bp}{144.8617bp}{17.8110bp}
\pgfpathqcurveto{145.4906bp}{17.8110bp}{146.0004bp}{18.3208bp}{146.0004bp}{18.9497bp}
\pgfpathclose
\pgfusepathqfillstroke
\end{pgfscope}
\begin{pgfscope}
\definecolor{fc}{rgb}{0.0000,0.0000,0.0000}
\pgfsetfillcolor{fc}
\pgftransformshift{\pgfqpoint{18.8963bp}{90.4869bp}}
\pgftransformscale{1.0000}
\pgftext[]{$8$}
\end{pgfscope}
\begin{pgfscope}
\definecolor{fc}{rgb}{0.0000,0.0000,0.0000}
\pgfsetfillcolor{fc}
\pgfsetlinewidth{0.5655bp}
\definecolor{sc}{rgb}{0.0000,0.0000,0.0000}
\pgfsetstrokecolor{sc}
\pgfsetmiterjoin
\pgfsetbuttcap
\pgfpathqmoveto{22.1965bp}{95.7540bp}
\pgfpathqcurveto{22.1965bp}{96.3828bp}{21.6867bp}{96.8926bp}{21.0578bp}{96.8926bp}
\pgfpathqcurveto{20.4289bp}{96.8926bp}{19.9191bp}{96.3828bp}{19.9191bp}{95.7540bp}
\pgfpathqcurveto{19.9191bp}{95.1251bp}{20.4289bp}{94.6153bp}{21.0578bp}{94.6153bp}
\pgfpathqcurveto{21.6867bp}{94.6153bp}{22.1965bp}{95.1251bp}{22.1965bp}{95.7540bp}
\pgfpathclose
\pgfusepathqfillstroke
\end{pgfscope}
\begin{pgfscope}
\definecolor{fc}{rgb}{0.0000,0.0000,0.0000}
\pgfsetfillcolor{fc}
\pgftransformshift{\pgfqpoint{0.0000bp}{122.0256bp}}
\pgftransformscale{1.0000}
\pgftext[]{$7$}
\end{pgfscope}
\begin{pgfscope}
\definecolor{fc}{rgb}{0.0000,0.0000,0.0000}
\pgfsetfillcolor{fc}
\pgfsetlinewidth{0.5655bp}
\definecolor{sc}{rgb}{0.0000,0.0000,0.0000}
\pgfsetstrokecolor{sc}
\pgfsetmiterjoin
\pgfsetbuttcap
\pgfpathqmoveto{6.4659bp}{120.0168bp}
\pgfpathqcurveto{6.4659bp}{120.6457bp}{5.9561bp}{121.1555bp}{5.3272bp}{121.1555bp}
\pgfpathqcurveto{4.6983bp}{121.1555bp}{4.1885bp}{120.6457bp}{4.1885bp}{120.0168bp}
\pgfpathqcurveto{4.1885bp}{119.3879bp}{4.6983bp}{118.8781bp}{5.3272bp}{118.8781bp}
\pgfpathqcurveto{5.9561bp}{118.8781bp}{6.4659bp}{119.3879bp}{6.4659bp}{120.0168bp}
\pgfpathclose
\pgfusepathqfillstroke
\end{pgfscope}
\begin{pgfscope}
\definecolor{fc}{rgb}{0.0000,0.0000,0.0000}
\pgfsetfillcolor{fc}
\pgftransformshift{\pgfqpoint{35.2343bp}{133.2654bp}}
\pgftransformscale{1.0000}
\pgftext[]{$6$}
\end{pgfscope}
\begin{pgfscope}
\definecolor{fc}{rgb}{0.0000,0.0000,0.0000}
\pgfsetfillcolor{fc}
\pgfsetlinewidth{0.5655bp}
\definecolor{sc}{rgb}{0.0000,0.0000,0.0000}
\pgfsetstrokecolor{sc}
\pgfsetmiterjoin
\pgfsetbuttcap
\pgfpathqmoveto{34.4690bp}{127.8998bp}
\pgfpathqcurveto{34.4690bp}{128.5287bp}{33.9592bp}{129.0385bp}{33.3304bp}{129.0385bp}
\pgfpathqcurveto{32.7015bp}{129.0385bp}{32.1917bp}{128.5287bp}{32.1917bp}{127.8998bp}
\pgfpathqcurveto{32.1917bp}{127.2709bp}{32.7015bp}{126.7611bp}{33.3304bp}{126.7611bp}
\pgfpathqcurveto{33.9592bp}{126.7611bp}{34.4690bp}{127.2709bp}{34.4690bp}{127.8998bp}
\pgfpathclose
\pgfusepathqfillstroke
\end{pgfscope}
\begin{pgfscope}
\definecolor{fc}{rgb}{0.0000,0.0000,0.0000}
\pgfsetfillcolor{fc}
\pgftransformshift{\pgfqpoint{55.0020bp}{106.7503bp}}
\pgftransformscale{1.0000}
\pgftext[]{$5$}
\end{pgfscope}
\begin{pgfscope}
\definecolor{fc}{rgb}{0.0000,0.0000,0.0000}
\pgfsetfillcolor{fc}
\pgfsetlinewidth{0.5655bp}
\definecolor{sc}{rgb}{0.0000,0.0000,0.0000}
\pgfsetstrokecolor{sc}
\pgfsetmiterjoin
\pgfsetbuttcap
\pgfpathqmoveto{52.4496bp}{102.4155bp}
\pgfpathqcurveto{52.4496bp}{103.0444bp}{51.9398bp}{103.5542bp}{51.3110bp}{103.5542bp}
\pgfpathqcurveto{50.6821bp}{103.5542bp}{50.1723bp}{103.0444bp}{50.1723bp}{102.4155bp}
\pgfpathqcurveto{50.1723bp}{101.7866bp}{50.6821bp}{101.2768bp}{51.3110bp}{101.2768bp}
\pgfpathqcurveto{51.9398bp}{101.2768bp}{52.4496bp}{101.7866bp}{52.4496bp}{102.4155bp}
\pgfpathclose
\pgfusepathqfillstroke
\end{pgfscope}
\begin{pgfscope}
\definecolor{fc}{rgb}{0.0000,0.0000,0.0000}
\pgfsetfillcolor{fc}
\pgftransformshift{\pgfqpoint{100.8682bp}{0.0000bp}}
\pgftransformscale{1.0000}
\pgftext[]{$4$}
\end{pgfscope}
\begin{pgfscope}
\definecolor{fc}{rgb}{0.0000,0.0000,0.0000}
\pgfsetfillcolor{fc}
\pgfsetlinewidth{0.5655bp}
\definecolor{sc}{rgb}{0.0000,0.0000,0.0000}
\pgfsetstrokecolor{sc}
\pgfsetmiterjoin
\pgfsetbuttcap
\pgfpathqmoveto{104.6110bp}{5.0629bp}
\pgfpathqcurveto{104.6110bp}{5.6918bp}{104.1012bp}{6.2016bp}{103.4723bp}{6.2016bp}
\pgfpathqcurveto{102.8434bp}{6.2016bp}{102.3336bp}{5.6918bp}{102.3336bp}{5.0629bp}
\pgfpathqcurveto{102.3336bp}{4.4340bp}{102.8434bp}{3.9242bp}{103.4723bp}{3.9242bp}
\pgfpathqcurveto{104.1012bp}{3.9242bp}{104.6110bp}{4.4340bp}{104.6110bp}{5.0629bp}
\pgfpathclose
\pgfusepathqfillstroke
\end{pgfscope}
\begin{pgfscope}
\definecolor{fc}{rgb}{0.0000,0.0000,0.0000}
\pgfsetfillcolor{fc}
\pgftransformshift{\pgfqpoint{121.4966bp}{35.9877bp}}
\pgftransformscale{1.0000}
\pgftext[]{$3$}
\end{pgfscope}
\begin{pgfscope}
\definecolor{fc}{rgb}{0.0000,0.0000,0.0000}
\pgfsetfillcolor{fc}
\pgfsetlinewidth{0.5655bp}
\definecolor{sc}{rgb}{0.0000,0.0000,0.0000}
\pgfsetstrokecolor{sc}
\pgfsetmiterjoin
\pgfsetbuttcap
\pgfpathqmoveto{118.5030bp}{32.0712bp}
\pgfpathqcurveto{118.5030bp}{32.7001bp}{117.9932bp}{33.2099bp}{117.3643bp}{33.2099bp}
\pgfpathqcurveto{116.7354bp}{33.2099bp}{116.2256bp}{32.7001bp}{116.2256bp}{32.0712bp}
\pgfpathqcurveto{116.2256bp}{31.4424bp}{116.7354bp}{30.9326bp}{117.3643bp}{30.9326bp}
\pgfpathqcurveto{117.9932bp}{30.9326bp}{118.5030bp}{31.4424bp}{118.5030bp}{32.0712bp}
\pgfpathclose
\pgfusepathqfillstroke
\end{pgfscope}
\begin{pgfscope}
\definecolor{fc}{rgb}{0.0000,0.0000,0.0000}
\pgfsetfillcolor{fc}
\pgftransformshift{\pgfqpoint{78.8185bp}{82.7506bp}}
\pgftransformscale{1.0000}
\pgftext[]{$2$}
\end{pgfscope}
\begin{pgfscope}
\definecolor{fc}{rgb}{0.0000,0.0000,0.0000}
\pgfsetfillcolor{fc}
\pgfsetlinewidth{0.5655bp}
\definecolor{sc}{rgb}{0.0000,0.0000,0.0000}
\pgfsetstrokecolor{sc}
\pgfsetmiterjoin
\pgfsetbuttcap
\pgfpathqmoveto{83.3522bp}{87.3210bp}
\pgfpathqcurveto{83.3522bp}{87.9499bp}{82.8424bp}{88.4597bp}{82.2135bp}{88.4597bp}
\pgfpathqcurveto{81.5846bp}{88.4597bp}{81.0748bp}{87.9499bp}{81.0748bp}{87.3210bp}
\pgfpathqcurveto{81.0748bp}{86.6921bp}{81.5846bp}{86.1823bp}{82.2135bp}{86.1823bp}
\pgfpathqcurveto{82.8424bp}{86.1823bp}{83.3522bp}{86.6921bp}{83.3522bp}{87.3210bp}
\pgfpathclose
\pgfusepathqfillstroke
\end{pgfscope}
\begin{pgfscope}
\definecolor{fc}{rgb}{0.0000,0.0000,0.0000}
\pgfsetfillcolor{fc}
\pgftransformshift{\pgfqpoint{99.6704bp}{60.2386bp}}
\pgftransformscale{1.0000}
\pgftext[]{$1$}
\end{pgfscope}
\begin{pgfscope}
\definecolor{fc}{rgb}{0.0000,0.0000,0.0000}
\pgfsetfillcolor{fc}
\pgfsetlinewidth{0.5655bp}
\definecolor{sc}{rgb}{0.0000,0.0000,0.0000}
\pgfsetstrokecolor{sc}
\pgfsetmiterjoin
\pgfsetbuttcap
\pgfpathqmoveto{105.6039bp}{63.3084bp}
\pgfpathqcurveto{105.6039bp}{63.9373bp}{105.0941bp}{64.4471bp}{104.4652bp}{64.4471bp}
\pgfpathqcurveto{103.8364bp}{64.4471bp}{103.3266bp}{63.9373bp}{103.3266bp}{63.3084bp}
\pgfpathqcurveto{103.3266bp}{62.6796bp}{103.8364bp}{62.1698bp}{104.4652bp}{62.1698bp}
\pgfpathqcurveto{105.0941bp}{62.1698bp}{105.6039bp}{62.6796bp}{105.6039bp}{63.3084bp}
\pgfpathclose
\pgfusepathqfillstroke
\end{pgfscope}
\begin{pgfscope}
\definecolor{fc}{rgb}{0.0000,0.0000,0.0000}
\pgfsetfillcolor{fc}
\pgftransformshift{\pgfqpoint{115.0702bp}{95.1848bp}}
\pgftransformscale{1.0000}
\pgftext[]{$0$}
\end{pgfscope}
\begin{pgfscope}
\definecolor{fc}{rgb}{0.0000,0.0000,0.0000}
\pgfsetfillcolor{fc}
\pgfsetlinewidth{0.5655bp}
\definecolor{sc}{rgb}{0.0000,0.0000,0.0000}
\pgfsetstrokecolor{sc}
\pgfsetmiterjoin
\pgfsetbuttcap
\pgfpathqmoveto{112.0143bp}{91.3351bp}
\pgfpathqcurveto{112.0143bp}{91.9640bp}{111.5045bp}{92.4738bp}{110.8756bp}{92.4738bp}
\pgfpathqcurveto{110.2467bp}{92.4738bp}{109.7369bp}{91.9640bp}{109.7369bp}{91.3351bp}
\pgfpathqcurveto{109.7369bp}{90.7063bp}{110.2467bp}{90.1965bp}{110.8756bp}{90.1965bp}
\pgfpathqcurveto{111.5045bp}{90.1965bp}{112.0143bp}{90.7063bp}{112.0143bp}{91.3351bp}
\pgfpathclose
\pgfusepathqfillstroke
\end{pgfscope}
\begin{pgfscope}
\pgfsetlinewidth{0.5655bp}
\definecolor{sc}{rgb}{0.0000,0.0000,0.0000}
\pgfsetstrokecolor{sc}
\pgfsetmiterjoin
\pgfsetbuttcap
\pgfpathqmoveto{118.3922bp}{31.5807bp}
\pgfpathqlineto{139.3677bp}{21.5714bp}
\pgfusepathqstroke
\end{pgfscope}
\begin{pgfscope}
\definecolor{fc}{rgb}{0.0000,0.0000,0.0000}
\pgfsetfillcolor{fc}
\pgfusepathqfill
\end{pgfscope}
\begin{pgfscope}
\definecolor{fc}{rgb}{0.0000,0.0000,0.0000}
\pgfsetfillcolor{fc}
\pgfusepathqfill
\end{pgfscope}
\begin{pgfscope}
\definecolor{fc}{rgb}{0.0000,0.0000,0.0000}
\pgfsetfillcolor{fc}
\pgfpathqmoveto{143.8338bp}{19.4402bp}
\pgfpathqlineto{139.4195bp}{23.6307bp}
\pgfpathqlineto{139.5019bp}{21.5073bp}
\pgfpathqlineto{137.7994bp}{20.2357bp}
\pgfpathqlineto{143.8338bp}{19.4402bp}
\pgfpathclose
\pgfusepathqfill
\end{pgfscope}
\begin{pgfscope}
\definecolor{fc}{rgb}{0.0000,0.0000,0.0000}
\pgfsetfillcolor{fc}
\pgfpathqmoveto{139.5019bp}{21.5073bp}
\pgfpathqlineto{139.3677bp}{21.5714bp}
\pgfpathqlineto{139.4895bp}{21.8266bp}
\pgfpathqlineto{139.5019bp}{21.5073bp}
\pgfpathqlineto{139.3677bp}{21.5714bp}
\pgfpathqlineto{139.2459bp}{21.3162bp}
\pgfpathqlineto{139.5019bp}{21.5073bp}
\pgfpathclose
\pgfusepathqfill
\end{pgfscope}
\begin{pgfscope}
\pgfsetlinewidth{0.5655bp}
\definecolor{sc}{rgb}{0.0000,0.0000,0.0000}
\pgfsetstrokecolor{sc}
\pgfsetmiterjoin
\pgfsetbuttcap
\pgfpathqmoveto{22.1701bp}{95.9989bp}
\pgfpathqlineto{45.3660bp}{101.1065bp}
\pgfusepathqstroke
\end{pgfscope}
\begin{pgfscope}
\definecolor{fc}{rgb}{0.0000,0.0000,0.0000}
\pgfsetfillcolor{fc}
\pgfusepathqfill
\end{pgfscope}
\begin{pgfscope}
\definecolor{fc}{rgb}{0.0000,0.0000,0.0000}
\pgfsetfillcolor{fc}
\pgfusepathqfill
\end{pgfscope}
\begin{pgfscope}
\definecolor{fc}{rgb}{0.0000,0.0000,0.0000}
\pgfsetfillcolor{fc}
\pgfpathqmoveto{50.1987bp}{102.1706bp}
\pgfpathqlineto{44.1410bp}{102.7626bp}
\pgfpathqlineto{45.5112bp}{101.1384bp}
\pgfpathqlineto{44.9499bp}{99.0890bp}
\pgfpathqlineto{50.1987bp}{102.1706bp}
\pgfpathclose
\pgfusepathqfill
\end{pgfscope}
\begin{pgfscope}
\definecolor{fc}{rgb}{0.0000,0.0000,0.0000}
\pgfsetfillcolor{fc}
\pgfpathqmoveto{45.5112bp}{101.1384bp}
\pgfpathqlineto{45.3660bp}{101.1065bp}
\pgfpathqlineto{45.3052bp}{101.3826bp}
\pgfpathqlineto{45.5112bp}{101.1384bp}
\pgfpathqlineto{45.3660bp}{101.1065bp}
\pgfpathqlineto{45.4268bp}{100.8303bp}
\pgfpathqlineto{45.5112bp}{101.1384bp}
\pgfpathclose
\pgfusepathqfill
\end{pgfscope}
\begin{pgfscope}
\pgfsetlinewidth{0.5655bp}
\definecolor{sc}{rgb}{0.0000,0.0000,0.0000}
\pgfsetstrokecolor{sc}
\pgfsetmiterjoin
\pgfsetbuttcap
\pgfpathqmoveto{5.9467bp}{119.0613bp}
\pgfpathqlineto{17.7463bp}{100.8617bp}
\pgfusepathqstroke
\end{pgfscope}
\begin{pgfscope}
\definecolor{fc}{rgb}{0.0000,0.0000,0.0000}
\pgfsetfillcolor{fc}
\pgfusepathqfill
\end{pgfscope}
\begin{pgfscope}
\definecolor{fc}{rgb}{0.0000,0.0000,0.0000}
\pgfsetfillcolor{fc}
\pgfusepathqfill
\end{pgfscope}
\begin{pgfscope}
\definecolor{fc}{rgb}{0.0000,0.0000,0.0000}
\pgfsetfillcolor{fc}
\pgfpathqmoveto{20.4383bp}{96.7095bp}
\pgfpathqlineto{18.8674bp}{102.5898bp}
\pgfpathqlineto{17.8271bp}{100.7370bp}
\pgfpathqlineto{15.7110bp}{100.5435bp}
\pgfpathqlineto{20.4383bp}{96.7095bp}
\pgfpathclose
\pgfusepathqfill
\end{pgfscope}
\begin{pgfscope}
\definecolor{fc}{rgb}{0.0000,0.0000,0.0000}
\pgfsetfillcolor{fc}
\pgfpathqmoveto{17.8271bp}{100.7370bp}
\pgfpathqlineto{17.7463bp}{100.8617bp}
\pgfpathqlineto{17.9835bp}{101.0155bp}
\pgfpathqlineto{17.8271bp}{100.7370bp}
\pgfpathqlineto{17.7463bp}{100.8617bp}
\pgfpathqlineto{17.5090bp}{100.7079bp}
\pgfpathqlineto{17.8271bp}{100.7370bp}
\pgfpathclose
\pgfusepathqfill
\end{pgfscope}
\begin{pgfscope}
\pgfsetlinewidth{0.5655bp}
\definecolor{sc}{rgb}{0.0000,0.0000,0.0000}
\pgfsetstrokecolor{sc}
\pgfsetmiterjoin
\pgfsetbuttcap
\pgfpathqmoveto{32.2340bp}{127.5912bp}
\pgfpathqlineto{11.1869bp}{121.6663bp}
\pgfusepathqstroke
\end{pgfscope}
\begin{pgfscope}
\definecolor{fc}{rgb}{0.0000,0.0000,0.0000}
\pgfsetfillcolor{fc}
\pgfusepathqfill
\end{pgfscope}
\begin{pgfscope}
\definecolor{fc}{rgb}{0.0000,0.0000,0.0000}
\pgfsetfillcolor{fc}
\pgfusepathqfill
\end{pgfscope}
\begin{pgfscope}
\definecolor{fc}{rgb}{0.0000,0.0000,0.0000}
\pgfsetfillcolor{fc}
\pgfpathqmoveto{6.4236bp}{120.3254bp}
\pgfpathqlineto{12.5053bp}{120.0835bp}
\pgfpathqlineto{11.0438bp}{121.6261bp}
\pgfpathqlineto{11.4860bp}{123.7045bp}
\pgfpathqlineto{6.4236bp}{120.3254bp}
\pgfpathclose
\pgfusepathqfill
\end{pgfscope}
\begin{pgfscope}
\definecolor{fc}{rgb}{0.0000,0.0000,0.0000}
\pgfsetfillcolor{fc}
\pgfpathqmoveto{11.0438bp}{121.6261bp}
\pgfpathqlineto{11.1869bp}{121.6663bp}
\pgfpathqlineto{11.2635bp}{121.3942bp}
\pgfpathqlineto{11.0438bp}{121.6261bp}
\pgfpathqlineto{11.1869bp}{121.6663bp}
\pgfpathqlineto{11.1103bp}{121.9385bp}
\pgfpathqlineto{11.0438bp}{121.6261bp}
\pgfpathclose
\pgfusepathqfill
\end{pgfscope}
\begin{pgfscope}
\pgfsetlinewidth{0.5655bp}
\definecolor{sc}{rgb}{0.0000,0.0000,0.0000}
\pgfsetstrokecolor{sc}
\pgfsetmiterjoin
\pgfsetbuttcap
\pgfpathqmoveto{50.6545bp}{103.3460bp}
\pgfpathqlineto{36.8397bp}{122.9260bp}
\pgfusepathqstroke
\end{pgfscope}
\begin{pgfscope}
\definecolor{fc}{rgb}{0.0000,0.0000,0.0000}
\pgfsetfillcolor{fc}
\pgfusepathqfill
\end{pgfscope}
\begin{pgfscope}
\definecolor{fc}{rgb}{0.0000,0.0000,0.0000}
\pgfsetfillcolor{fc}
\pgfusepathqfill
\end{pgfscope}
\begin{pgfscope}
\definecolor{fc}{rgb}{0.0000,0.0000,0.0000}
\pgfsetfillcolor{fc}
\pgfpathqmoveto{33.9869bp}{126.9693bp}
\pgfpathqlineto{35.7872bp}{121.1552bp}
\pgfpathqlineto{36.7540bp}{123.0474bp}
\pgfpathqlineto{38.8609bp}{123.3238bp}
\pgfpathqlineto{33.9869bp}{126.9693bp}
\pgfpathclose
\pgfusepathqfill
\end{pgfscope}
\begin{pgfscope}
\definecolor{fc}{rgb}{0.0000,0.0000,0.0000}
\pgfsetfillcolor{fc}
\pgfpathqmoveto{36.7540bp}{123.0474bp}
\pgfpathqlineto{36.8397bp}{122.9260bp}
\pgfpathqlineto{36.6087bp}{122.7629bp}
\pgfpathqlineto{36.7540bp}{123.0474bp}
\pgfpathqlineto{36.8397bp}{122.9260bp}
\pgfpathqlineto{37.0708bp}{123.0890bp}
\pgfpathqlineto{36.7540bp}{123.0474bp}
\pgfpathclose
\pgfusepathqfill
\end{pgfscope}
\begin{pgfscope}
\pgfsetlinewidth{0.5655bp}
\definecolor{sc}{rgb}{0.0000,0.0000,0.0000}
\pgfsetstrokecolor{sc}
\pgfsetmiterjoin
\pgfsetbuttcap
\pgfpathqmoveto{81.1901bp}{87.8209bp}
\pgfpathqlineto{56.7807bp}{99.7438bp}
\pgfusepathqstroke
\end{pgfscope}
\begin{pgfscope}
\definecolor{fc}{rgb}{0.0000,0.0000,0.0000}
\pgfsetfillcolor{fc}
\pgfusepathqfill
\end{pgfscope}
\begin{pgfscope}
\definecolor{fc}{rgb}{0.0000,0.0000,0.0000}
\pgfsetfillcolor{fc}
\pgfusepathqfill
\end{pgfscope}
\begin{pgfscope}
\definecolor{fc}{rgb}{0.0000,0.0000,0.0000}
\pgfsetfillcolor{fc}
\pgfpathqmoveto{52.3343bp}{101.9156bp}
\pgfpathqlineto{56.7102bp}{97.6850bp}
\pgfpathqlineto{56.6472bp}{99.8090bp}
\pgfpathqlineto{58.3612bp}{101.0650bp}
\pgfpathqlineto{52.3343bp}{101.9156bp}
\pgfpathclose
\pgfusepathqfill
\end{pgfscope}
\begin{pgfscope}
\definecolor{fc}{rgb}{0.0000,0.0000,0.0000}
\pgfsetfillcolor{fc}
\pgfpathqmoveto{56.6472bp}{99.8090bp}
\pgfpathqlineto{56.7807bp}{99.7438bp}
\pgfpathqlineto{56.6566bp}{99.4897bp}
\pgfpathqlineto{56.6472bp}{99.8090bp}
\pgfpathqlineto{56.7807bp}{99.7438bp}
\pgfpathqlineto{56.9048bp}{99.9978bp}
\pgfpathqlineto{56.6472bp}{99.8090bp}
\pgfpathclose
\pgfusepathqfill
\end{pgfscope}
\begin{pgfscope}
\pgfsetlinewidth{0.5655bp}
\definecolor{sc}{rgb}{0.0000,0.0000,0.0000}
\pgfsetstrokecolor{sc}
\pgfsetmiterjoin
\pgfsetbuttcap
\pgfpathqmoveto{116.8434bp}{31.0584bp}
\pgfpathqlineto{106.2567bp}{10.4762bp}
\pgfusepathqstroke
\end{pgfscope}
\begin{pgfscope}
\definecolor{fc}{rgb}{0.0000,0.0000,0.0000}
\pgfsetfillcolor{fc}
\pgfusepathqfill
\end{pgfscope}
\begin{pgfscope}
\definecolor{fc}{rgb}{0.0000,0.0000,0.0000}
\pgfsetfillcolor{fc}
\pgfusepathqfill
\end{pgfscope}
\begin{pgfscope}
\definecolor{fc}{rgb}{0.0000,0.0000,0.0000}
\pgfsetfillcolor{fc}
\pgfpathqmoveto{103.9933bp}{6.0757bp}
\pgfpathqlineto{108.3135bp}{10.3630bp}
\pgfpathqlineto{106.1887bp}{10.3440bp}
\pgfpathqlineto{104.9684bp}{12.0836bp}
\pgfpathqlineto{103.9933bp}{6.0757bp}
\pgfpathclose
\pgfusepathqfill
\end{pgfscope}
\begin{pgfscope}
\definecolor{fc}{rgb}{0.0000,0.0000,0.0000}
\pgfsetfillcolor{fc}
\pgfpathqmoveto{106.1887bp}{10.3440bp}
\pgfpathqlineto{106.2567bp}{10.4762bp}
\pgfpathqlineto{106.5081bp}{10.3468bp}
\pgfpathqlineto{106.1887bp}{10.3440bp}
\pgfpathqlineto{106.2567bp}{10.4762bp}
\pgfpathqlineto{106.0052bp}{10.6055bp}
\pgfpathqlineto{106.1887bp}{10.3440bp}
\pgfpathclose
\pgfusepathqfill
\end{pgfscope}
\begin{pgfscope}
\pgfsetlinewidth{0.5655bp}
\definecolor{sc}{rgb}{0.0000,0.0000,0.0000}
\pgfsetstrokecolor{sc}
\pgfsetmiterjoin
\pgfsetbuttcap
\pgfpathqmoveto{104.9000bp}{62.2557bp}
\pgfpathqlineto{115.0408bp}{37.6978bp}
\pgfusepathqstroke
\end{pgfscope}
\begin{pgfscope}
\definecolor{fc}{rgb}{0.0000,0.0000,0.0000}
\pgfsetfillcolor{fc}
\pgfusepathqfill
\end{pgfscope}
\begin{pgfscope}
\definecolor{fc}{rgb}{0.0000,0.0000,0.0000}
\pgfsetfillcolor{fc}
\pgfusepathqfill
\end{pgfscope}
\begin{pgfscope}
\definecolor{fc}{rgb}{0.0000,0.0000,0.0000}
\pgfsetfillcolor{fc}
\pgfpathqmoveto{116.9296bp}{33.1240bp}
\pgfpathqlineto{116.4586bp}{39.1923bp}
\pgfpathqlineto{115.0976bp}{37.5604bp}
\pgfpathqlineto{112.9817bp}{37.7565bp}
\pgfpathqlineto{116.9296bp}{33.1240bp}
\pgfpathclose
\pgfusepathqfill
\end{pgfscope}
\begin{pgfscope}
\definecolor{fc}{rgb}{0.0000,0.0000,0.0000}
\pgfsetfillcolor{fc}
\pgfpathqmoveto{115.0976bp}{37.5604bp}
\pgfpathqlineto{115.0408bp}{37.6978bp}
\pgfpathqlineto{115.3022bp}{37.8058bp}
\pgfpathqlineto{115.0976bp}{37.5604bp}
\pgfpathqlineto{115.0408bp}{37.6978bp}
\pgfpathqlineto{114.7795bp}{37.5899bp}
\pgfpathqlineto{115.0976bp}{37.5604bp}
\pgfpathclose
\pgfusepathqfill
\end{pgfscope}
\begin{pgfscope}
\pgfsetlinewidth{0.5655bp}
\definecolor{sc}{rgb}{0.0000,0.0000,0.0000}
\pgfsetstrokecolor{sc}
\pgfsetmiterjoin
\pgfsetbuttcap
\pgfpathqmoveto{103.6913bp}{64.1436bp}
\pgfpathqlineto{86.3510bp}{82.8561bp}
\pgfusepathqstroke
\end{pgfscope}
\begin{pgfscope}
\definecolor{fc}{rgb}{0.0000,0.0000,0.0000}
\pgfsetfillcolor{fc}
\pgfusepathqfill
\end{pgfscope}
\begin{pgfscope}
\definecolor{fc}{rgb}{0.0000,0.0000,0.0000}
\pgfsetfillcolor{fc}
\pgfusepathqfill
\end{pgfscope}
\begin{pgfscope}
\definecolor{fc}{rgb}{0.0000,0.0000,0.0000}
\pgfsetfillcolor{fc}
\pgfpathqmoveto{82.9875bp}{86.4858bp}
\pgfpathqlineto{85.5424bp}{80.9615bp}
\pgfpathqlineto{86.2499bp}{82.9652bp}
\pgfpathqlineto{88.3016bp}{83.5183bp}
\pgfpathqlineto{82.9875bp}{86.4858bp}
\pgfpathclose
\pgfusepathqfill
\end{pgfscope}
\begin{pgfscope}
\definecolor{fc}{rgb}{0.0000,0.0000,0.0000}
\pgfsetfillcolor{fc}
\pgfpathqmoveto{86.2499bp}{82.9652bp}
\pgfpathqlineto{86.3510bp}{82.8561bp}
\pgfpathqlineto{86.1436bp}{82.6639bp}
\pgfpathqlineto{86.2499bp}{82.9652bp}
\pgfpathqlineto{86.3510bp}{82.8561bp}
\pgfpathqlineto{86.5584bp}{83.0483bp}
\pgfpathqlineto{86.2499bp}{82.9652bp}
\pgfpathclose
\pgfusepathqfill
\end{pgfscope}
\begin{pgfscope}
\pgfsetlinewidth{0.5655bp}
\definecolor{sc}{rgb}{0.0000,0.0000,0.0000}
\pgfsetstrokecolor{sc}
\pgfsetmiterjoin
\pgfsetbuttcap
\pgfpathqmoveto{109.7478bp}{91.1772bp}
\pgfpathqlineto{88.2420bp}{88.1653bp}
\pgfusepathqstroke
\end{pgfscope}
\begin{pgfscope}
\definecolor{fc}{rgb}{0.0000,0.0000,0.0000}
\pgfsetfillcolor{fc}
\pgfusepathqfill
\end{pgfscope}
\begin{pgfscope}
\definecolor{fc}{rgb}{0.0000,0.0000,0.0000}
\pgfsetfillcolor{fc}
\pgfusepathqfill
\end{pgfscope}
\begin{pgfscope}
\definecolor{fc}{rgb}{0.0000,0.0000,0.0000}
\pgfsetfillcolor{fc}
\pgfpathqmoveto{83.3413bp}{87.4789bp}
\pgfpathqlineto{89.3349bp}{86.4191bp}
\pgfpathqlineto{88.0947bp}{88.1447bp}
\pgfpathqlineto{88.8131bp}{90.1445bp}
\pgfpathqlineto{83.3413bp}{87.4789bp}
\pgfpathclose
\pgfusepathqfill
\end{pgfscope}
\begin{pgfscope}
\definecolor{fc}{rgb}{0.0000,0.0000,0.0000}
\pgfsetfillcolor{fc}
\pgfpathqmoveto{88.0947bp}{88.1447bp}
\pgfpathqlineto{88.2420bp}{88.1653bp}
\pgfpathqlineto{88.2812bp}{87.8852bp}
\pgfpathqlineto{88.0947bp}{88.1447bp}
\pgfpathqlineto{88.2420bp}{88.1653bp}
\pgfpathqlineto{88.2028bp}{88.4453bp}
\pgfpathqlineto{88.0947bp}{88.1447bp}
\pgfpathclose
\pgfusepathqfill
\end{pgfscope}
\begin{pgfscope}
\pgfsetlinewidth{0.5655bp}
\definecolor{sc}{rgb}{0.0000,0.0000,0.0000}
\pgfsetstrokecolor{sc}
\pgfsetmiterjoin
\pgfsetbuttcap
\pgfpathqmoveto{110.6217bp}{90.2249bp}
\pgfpathqlineto{105.8225bp}{69.2426bp}
\pgfusepathqstroke
\end{pgfscope}
\begin{pgfscope}
\definecolor{fc}{rgb}{0.0000,0.0000,0.0000}
\pgfsetfillcolor{fc}
\pgfusepathqfill
\end{pgfscope}
\begin{pgfscope}
\definecolor{fc}{rgb}{0.0000,0.0000,0.0000}
\pgfsetfillcolor{fc}
\pgfusepathqfill
\end{pgfscope}
\begin{pgfscope}
\definecolor{fc}{rgb}{0.0000,0.0000,0.0000}
\pgfsetfillcolor{fc}
\pgfpathqmoveto{104.7192bp}{64.4187bp}
\pgfpathqlineto{107.8434bp}{69.6422bp}
\pgfpathqlineto{105.7894bp}{69.0977bp}
\pgfpathqlineto{104.1764bp}{70.4810bp}
\pgfpathqlineto{104.7192bp}{64.4187bp}
\pgfpathclose
\pgfusepathqfill
\end{pgfscope}
\begin{pgfscope}
\definecolor{fc}{rgb}{0.0000,0.0000,0.0000}
\pgfsetfillcolor{fc}
\pgfpathqmoveto{105.7894bp}{69.0977bp}
\pgfpathqlineto{105.8225bp}{69.2426bp}
\pgfpathqlineto{106.0982bp}{69.1796bp}
\pgfpathqlineto{105.7894bp}{69.0977bp}
\pgfpathqlineto{105.8225bp}{69.2426bp}
\pgfpathqlineto{105.5469bp}{69.3057bp}
\pgfpathqlineto{105.7894bp}{69.0977bp}
\pgfpathclose
\pgfusepathqfill
\end{pgfscope}
\end{pgfpicture}

    \end{minipage}
    \begin{minipage}{0.45\textwidth}
      \begin{pgfpicture}
\pgfpathrectangle{\pgfpointorigin}{\pgfqpoint{150.0000bp}{149.0000bp}}
\pgfusepath{use as bounding box}
\begin{pgfscope}
\definecolor{fc}{rgb}{0.0000,0.0000,0.0000}
\pgfsetfillcolor{fc}
\pgftransformcm{1.0000}{0.0000}{0.0000}{1.0000}{\pgfqpoint{75.3605bp}{6.2553bp}}
\pgftransformscale{1.0000}
\pgftext[]{j}
\end{pgfscope}
\begin{pgfscope}
\definecolor{fc}{rgb}{0.0000,0.0000,0.0000}
\pgfsetfillcolor{fc}
\pgfsetlinewidth{0.5987bp}
\definecolor{sc}{rgb}{0.0000,0.0000,0.0000}
\pgfsetstrokecolor{sc}
\pgfsetmiterjoin
\pgfsetbuttcap
\pgfpathqmoveto{72.1050bp}{11.5830bp}
\pgfpathqcurveto{72.1050bp}{12.3657bp}{71.4705bp}{13.0003bp}{70.6877bp}{13.0003bp}
\pgfpathqcurveto{69.9050bp}{13.0003bp}{69.2704bp}{12.3657bp}{69.2704bp}{11.5830bp}
\pgfpathqcurveto{69.2704bp}{10.8002bp}{69.9050bp}{10.1657bp}{70.6877bp}{10.1657bp}
\pgfpathqcurveto{71.4705bp}{10.1657bp}{72.1050bp}{10.8002bp}{72.1050bp}{11.5830bp}
\pgfpathclose
\pgfusepathqfillstroke
\end{pgfscope}
\begin{pgfscope}
\definecolor{fc}{rgb}{0.0000,0.0000,0.0000}
\pgfsetfillcolor{fc}
\pgftransformcm{1.0000}{0.0000}{0.0000}{1.0000}{\pgfqpoint{22.0576bp}{0.0000bp}}
\pgftransformscale{1.0000}
\pgftext[]{i}
\end{pgfscope}
\begin{pgfscope}
\definecolor{fc}{rgb}{0.0000,0.0000,0.0000}
\pgfsetfillcolor{fc}
\pgfsetlinewidth{0.5987bp}
\definecolor{sc}{rgb}{0.0000,0.0000,0.0000}
\pgfsetstrokecolor{sc}
\pgfsetmiterjoin
\pgfsetbuttcap
\pgfpathqmoveto{26.9793bp}{6.1595bp}
\pgfpathqcurveto{26.9793bp}{6.9422bp}{26.3447bp}{7.5768bp}{25.5620bp}{7.5768bp}
\pgfpathqcurveto{24.7792bp}{7.5768bp}{24.1446bp}{6.9422bp}{24.1446bp}{6.1595bp}
\pgfpathqcurveto{24.1446bp}{5.3767bp}{24.7792bp}{4.7422bp}{25.5620bp}{4.7422bp}
\pgfpathqcurveto{26.3447bp}{4.7422bp}{26.9793bp}{5.3767bp}{26.9793bp}{6.1595bp}
\pgfpathclose
\pgfusepathqfillstroke
\end{pgfscope}
\begin{pgfscope}
\definecolor{fc}{rgb}{0.0000,0.0000,0.0000}
\pgfsetfillcolor{fc}
\pgftransformcm{1.0000}{0.0000}{0.0000}{1.0000}{\pgfqpoint{0.0000bp}{48.9596bp}}
\pgftransformscale{1.0000}
\pgftext[]{h}
\end{pgfscope}
\begin{pgfscope}
\definecolor{fc}{rgb}{0.0000,0.0000,0.0000}
\pgfsetfillcolor{fc}
\pgfsetlinewidth{0.5987bp}
\definecolor{sc}{rgb}{0.0000,0.0000,0.0000}
\pgfsetstrokecolor{sc}
\pgfsetmiterjoin
\pgfsetbuttcap
\pgfpathqmoveto{8.3872bp}{47.6788bp}
\pgfpathqcurveto{8.3872bp}{48.4616bp}{7.7527bp}{49.0961bp}{6.9699bp}{49.0961bp}
\pgfpathqcurveto{6.1871bp}{49.0961bp}{5.5526bp}{48.4616bp}{5.5526bp}{47.6788bp}
\pgfpathqcurveto{5.5526bp}{46.8961bp}{6.1871bp}{46.2615bp}{6.9699bp}{46.2615bp}
\pgfpathqcurveto{7.7527bp}{46.2615bp}{8.3872bp}{46.8961bp}{8.3872bp}{47.6788bp}
\pgfpathclose
\pgfusepathqfillstroke
\end{pgfscope}
\begin{pgfscope}
\definecolor{fc}{rgb}{0.0000,0.0000,0.0000}
\pgfsetfillcolor{fc}
\pgftransformcm{1.0000}{0.0000}{0.0000}{1.0000}{\pgfqpoint{42.2654bp}{47.1117bp}}
\pgftransformscale{1.0000}
\pgftext[]{g}
\end{pgfscope}
\begin{pgfscope}
\definecolor{fc}{rgb}{0.0000,0.0000,0.0000}
\pgfsetfillcolor{fc}
\pgfsetlinewidth{0.5987bp}
\definecolor{sc}{rgb}{0.0000,0.0000,0.0000}
\pgfsetstrokecolor{sc}
\pgfsetmiterjoin
\pgfsetbuttcap
\pgfpathqmoveto{46.6057bp}{40.6560bp}
\pgfpathqcurveto{46.6057bp}{41.4388bp}{45.9712bp}{42.0734bp}{45.1884bp}{42.0734bp}
\pgfpathqcurveto{44.4057bp}{42.0734bp}{43.7711bp}{41.4388bp}{43.7711bp}{40.6560bp}
\pgfpathqcurveto{43.7711bp}{39.8733bp}{44.4057bp}{39.2387bp}{45.1884bp}{39.2387bp}
\pgfpathqcurveto{45.9712bp}{39.2387bp}{46.6057bp}{39.8733bp}{46.6057bp}{40.6560bp}
\pgfpathclose
\pgfusepathqfillstroke
\end{pgfscope}
\begin{pgfscope}
\definecolor{fc}{rgb}{0.0000,0.0000,0.0000}
\pgfsetfillcolor{fc}
\pgftransformcm{1.0000}{0.0000}{0.0000}{1.0000}{\pgfqpoint{17.4017bp}{149.3379bp}}
\pgftransformscale{1.0000}
\pgftext[]{f}
\end{pgfscope}
\begin{pgfscope}
\definecolor{fc}{rgb}{0.0000,0.0000,0.0000}
\pgfsetfillcolor{fc}
\pgfsetlinewidth{0.5987bp}
\definecolor{sc}{rgb}{0.0000,0.0000,0.0000}
\pgfsetstrokecolor{sc}
\pgfsetmiterjoin
\pgfsetbuttcap
\pgfpathqmoveto{23.3062bp}{143.8529bp}
\pgfpathqcurveto{23.3062bp}{144.6357bp}{22.6716bp}{145.2702bp}{21.8889bp}{145.2702bp}
\pgfpathqcurveto{21.1061bp}{145.2702bp}{20.4715bp}{144.6357bp}{20.4715bp}{143.8529bp}
\pgfpathqcurveto{20.4715bp}{143.0702bp}{21.1061bp}{142.4356bp}{21.8889bp}{142.4356bp}
\pgfpathqcurveto{22.6716bp}{142.4356bp}{23.3062bp}{143.0702bp}{23.3062bp}{143.8529bp}
\pgfpathclose
\pgfusepathqfillstroke
\end{pgfscope}
\begin{pgfscope}
\definecolor{fc}{rgb}{0.0000,0.0000,0.0000}
\pgfsetfillcolor{fc}
\pgftransformcm{1.0000}{0.0000}{0.0000}{1.0000}{\pgfqpoint{51.4742bp}{118.8646bp}}
\pgftransformscale{1.0000}
\pgftext[]{e}
\end{pgfscope}
\begin{pgfscope}
\definecolor{fc}{rgb}{0.0000,0.0000,0.0000}
\pgfsetfillcolor{fc}
\pgfsetlinewidth{0.5987bp}
\definecolor{sc}{rgb}{0.0000,0.0000,0.0000}
\pgfsetstrokecolor{sc}
\pgfsetmiterjoin
\pgfsetbuttcap
\pgfpathqmoveto{47.1778bp}{114.6726bp}
\pgfpathqcurveto{47.1778bp}{115.4554bp}{46.5432bp}{116.0899bp}{45.7604bp}{116.0899bp}
\pgfpathqcurveto{44.9777bp}{116.0899bp}{44.3431bp}{115.4554bp}{44.3431bp}{114.6726bp}
\pgfpathqcurveto{44.3431bp}{113.8898bp}{44.9777bp}{113.2553bp}{45.7604bp}{113.2553bp}
\pgfpathqcurveto{46.5432bp}{113.2553bp}{47.1778bp}{113.8898bp}{47.1778bp}{114.6726bp}
\pgfpathclose
\pgfusepathqfillstroke
\end{pgfscope}
\begin{pgfscope}
\definecolor{fc}{rgb}{0.0000,0.0000,0.0000}
\pgfsetfillcolor{fc}
\pgftransformcm{1.0000}{0.0000}{0.0000}{1.0000}{\pgfqpoint{72.6523bp}{74.0141bp}}
\pgftransformscale{1.0000}
\pgftext[]{d}
\end{pgfscope}
\begin{pgfscope}
\definecolor{fc}{rgb}{0.0000,0.0000,0.0000}
\pgfsetfillcolor{fc}
\pgfsetlinewidth{0.5987bp}
\definecolor{sc}{rgb}{0.0000,0.0000,0.0000}
\pgfsetstrokecolor{sc}
\pgfsetmiterjoin
\pgfsetbuttcap
\pgfpathqmoveto{70.0193bp}{79.8291bp}
\pgfpathqcurveto{70.0193bp}{80.6119bp}{69.3848bp}{81.2465bp}{68.6020bp}{81.2465bp}
\pgfpathqcurveto{67.8193bp}{81.2465bp}{67.1847bp}{80.6119bp}{67.1847bp}{79.8291bp}
\pgfpathqcurveto{67.1847bp}{79.0464bp}{67.8193bp}{78.4118bp}{68.6020bp}{78.4118bp}
\pgfpathqcurveto{69.3848bp}{78.4118bp}{70.0193bp}{79.0464bp}{70.0193bp}{79.8291bp}
\pgfpathclose
\pgfusepathqfillstroke
\end{pgfscope}
\begin{pgfscope}
\definecolor{fc}{rgb}{0.0000,0.0000,0.0000}
\pgfsetfillcolor{fc}
\pgftransformcm{1.0000}{0.0000}{0.0000}{1.0000}{\pgfqpoint{108.5059bp}{94.0991bp}}
\pgftransformscale{1.0000}
\pgftext[]{c}
\end{pgfscope}
\begin{pgfscope}
\definecolor{fc}{rgb}{0.0000,0.0000,0.0000}
\pgfsetfillcolor{fc}
\pgfsetlinewidth{0.5987bp}
\definecolor{sc}{rgb}{0.0000,0.0000,0.0000}
\pgfsetstrokecolor{sc}
\pgfsetmiterjoin
\pgfsetbuttcap
\pgfpathqmoveto{113.4253bp}{87.9383bp}
\pgfpathqcurveto{113.4253bp}{88.7211bp}{112.7907bp}{89.3557bp}{112.0079bp}{89.3557bp}
\pgfpathqcurveto{111.2252bp}{89.3557bp}{110.5906bp}{88.7211bp}{110.5906bp}{87.9383bp}
\pgfpathqcurveto{110.5906bp}{87.1556bp}{111.2252bp}{86.5210bp}{112.0079bp}{86.5210bp}
\pgfpathqcurveto{112.7907bp}{86.5210bp}{113.4253bp}{87.1556bp}{113.4253bp}{87.9383bp}
\pgfpathclose
\pgfusepathqfillstroke
\end{pgfscope}
\begin{pgfscope}
\definecolor{fc}{rgb}{0.0000,0.0000,0.0000}
\pgfsetfillcolor{fc}
\pgftransformcm{1.0000}{0.0000}{0.0000}{1.0000}{\pgfqpoint{142.1036bp}{122.1348bp}}
\pgftransformscale{1.0000}
\pgftext[]{b}
\end{pgfscope}
\begin{pgfscope}
\definecolor{fc}{rgb}{0.0000,0.0000,0.0000}
\pgfsetfillcolor{fc}
\pgfsetlinewidth{0.5987bp}
\definecolor{sc}{rgb}{0.0000,0.0000,0.0000}
\pgfsetstrokecolor{sc}
\pgfsetmiterjoin
\pgfsetbuttcap
\pgfpathqmoveto{138.8391bp}{116.8151bp}
\pgfpathqcurveto{138.8391bp}{117.5978bp}{138.2046bp}{118.2324bp}{137.4218bp}{118.2324bp}
\pgfpathqcurveto{136.6390bp}{118.2324bp}{136.0045bp}{117.5978bp}{136.0045bp}{116.8151bp}
\pgfpathqcurveto{136.0045bp}{116.0323bp}{136.6390bp}{115.3977bp}{137.4218bp}{115.3977bp}
\pgfpathqcurveto{138.2046bp}{115.3977bp}{138.8391bp}{116.0323bp}{138.8391bp}{116.8151bp}
\pgfpathclose
\pgfusepathqfillstroke
\end{pgfscope}
\begin{pgfscope}
\definecolor{fc}{rgb}{0.0000,0.0000,0.0000}
\pgfsetfillcolor{fc}
\pgftransformcm{1.0000}{0.0000}{0.0000}{1.0000}{\pgfqpoint{150.0000bp}{62.9873bp}}
\pgftransformscale{1.0000}
\pgftext[]{a}
\end{pgfscope}
\begin{pgfscope}
\definecolor{fc}{rgb}{0.0000,0.0000,0.0000}
\pgfsetfillcolor{fc}
\pgfsetlinewidth{0.5987bp}
\definecolor{sc}{rgb}{0.0000,0.0000,0.0000}
\pgfsetstrokecolor{sc}
\pgfsetmiterjoin
\pgfsetbuttcap
\pgfpathqmoveto{145.4939bp}{66.8774bp}
\pgfpathqcurveto{145.4939bp}{67.6602bp}{144.8594bp}{68.2947bp}{144.0766bp}{68.2947bp}
\pgfpathqcurveto{143.2938bp}{68.2947bp}{142.6593bp}{67.6602bp}{142.6593bp}{66.8774bp}
\pgfpathqcurveto{142.6593bp}{66.0947bp}{143.2938bp}{65.4601bp}{144.0766bp}{65.4601bp}
\pgfpathqcurveto{144.8594bp}{65.4601bp}{145.4939bp}{66.0947bp}{145.4939bp}{66.8774bp}
\pgfpathclose
\pgfusepathqfillstroke
\end{pgfscope}
\begin{pgfscope}
\pgfsetlinewidth{0.5987bp}
\definecolor{sc}{rgb}{0.0000,0.0000,0.0000}
\pgfsetstrokecolor{sc}
\pgfsetmiterjoin
\pgfsetbuttcap
\pgfpathqmoveto{46.1230bp}{39.5905bp}
\pgfpathqlineto{66.2990bp}{16.5868bp}
\pgfusepathqstroke
\end{pgfscope}
\begin{pgfscope}
\definecolor{fc}{rgb}{0.0000,0.0000,0.0000}
\pgfsetfillcolor{fc}
\pgfusepathqfill
\end{pgfscope}
\begin{pgfscope}
\definecolor{fc}{rgb}{0.0000,0.0000,0.0000}
\pgfsetfillcolor{fc}
\pgfusepathqfill
\end{pgfscope}
\begin{pgfscope}
\definecolor{fc}{rgb}{0.0000,0.0000,0.0000}
\pgfsetfillcolor{fc}
\pgfpathqmoveto{69.7531bp}{12.6485bp}
\pgfpathqlineto{67.2094bp}{18.5683bp}
\pgfpathqlineto{66.4028bp}{16.4685bp}
\pgfpathqlineto{64.2157bp}{15.9425bp}
\pgfpathqlineto{69.7531bp}{12.6485bp}
\pgfpathclose
\pgfusepathqfill
\end{pgfscope}
\begin{pgfscope}
\definecolor{fc}{rgb}{0.0000,0.0000,0.0000}
\pgfsetfillcolor{fc}
\pgfpathqmoveto{66.4028bp}{16.4685bp}
\pgfpathqlineto{66.2990bp}{16.5868bp}
\pgfpathqlineto{66.5241bp}{16.7842bp}
\pgfpathqlineto{66.4028bp}{16.4685bp}
\pgfpathqlineto{66.2990bp}{16.5868bp}
\pgfpathqlineto{66.0740bp}{16.3894bp}
\pgfpathqlineto{66.4028bp}{16.4685bp}
\pgfpathclose
\pgfusepathqfill
\end{pgfscope}
\begin{pgfscope}
\pgfsetlinewidth{0.5987bp}
\definecolor{sc}{rgb}{0.0000,0.0000,0.0000}
\pgfsetstrokecolor{sc}
\pgfsetmiterjoin
\pgfsetbuttcap
\pgfpathqmoveto{44.4874bp}{39.4239bp}
\pgfpathqlineto{28.8534bp}{11.9447bp}
\pgfusepathqstroke
\end{pgfscope}
\begin{pgfscope}
\definecolor{fc}{rgb}{0.0000,0.0000,0.0000}
\pgfsetfillcolor{fc}
\pgfusepathqfill
\end{pgfscope}
\begin{pgfscope}
\definecolor{fc}{rgb}{0.0000,0.0000,0.0000}
\pgfsetfillcolor{fc}
\pgfusepathqfill
\end{pgfscope}
\begin{pgfscope}
\definecolor{fc}{rgb}{0.0000,0.0000,0.0000}
\pgfsetfillcolor{fc}
\pgfpathqmoveto{26.2629bp}{7.3916bp}
\pgfpathqlineto{31.0237bp}{11.7331bp}
\pgfpathqlineto{28.7756bp}{11.8079bp}
\pgfpathqlineto{27.5626bp}{13.7023bp}
\pgfpathqlineto{26.2629bp}{7.3916bp}
\pgfpathclose
\pgfusepathqfill
\end{pgfscope}
\begin{pgfscope}
\definecolor{fc}{rgb}{0.0000,0.0000,0.0000}
\pgfsetfillcolor{fc}
\pgfpathqmoveto{28.7756bp}{11.8079bp}
\pgfpathqlineto{28.8534bp}{11.9447bp}
\pgfpathqlineto{29.1136bp}{11.7966bp}
\pgfpathqlineto{28.7756bp}{11.8079bp}
\pgfpathqlineto{28.8534bp}{11.9447bp}
\pgfpathqlineto{28.5932bp}{12.0927bp}
\pgfpathqlineto{28.7756bp}{11.8079bp}
\pgfpathclose
\pgfusepathqfill
\end{pgfscope}
\begin{pgfscope}
\pgfsetlinewidth{0.5987bp}
\definecolor{sc}{rgb}{0.0000,0.0000,0.0000}
\pgfsetstrokecolor{sc}
\pgfsetmiterjoin
\pgfsetbuttcap
\pgfpathqmoveto{43.7942bp}{40.9122bp}
\pgfpathqlineto{13.5163bp}{46.4759bp}
\pgfusepathqstroke
\end{pgfscope}
\begin{pgfscope}
\definecolor{fc}{rgb}{0.0000,0.0000,0.0000}
\pgfsetfillcolor{fc}
\pgfusepathqfill
\end{pgfscope}
\begin{pgfscope}
\definecolor{fc}{rgb}{0.0000,0.0000,0.0000}
\pgfsetfillcolor{fc}
\pgfusepathqfill
\end{pgfscope}
\begin{pgfscope}
\definecolor{fc}{rgb}{0.0000,0.0000,0.0000}
\pgfsetfillcolor{fc}
\pgfpathqmoveto{8.3641bp}{47.4226bp}
\pgfpathqlineto{14.0312bp}{44.3569bp}
\pgfpathqlineto{13.3615bp}{46.5044bp}
\pgfpathqlineto{14.7508bp}{48.2734bp}
\pgfpathqlineto{8.3641bp}{47.4226bp}
\pgfpathclose
\pgfusepathqfill
\end{pgfscope}
\begin{pgfscope}
\definecolor{fc}{rgb}{0.0000,0.0000,0.0000}
\pgfsetfillcolor{fc}
\pgfpathqmoveto{13.3615bp}{46.5044bp}
\pgfpathqlineto{13.5163bp}{46.4759bp}
\pgfpathqlineto{13.4622bp}{46.1815bp}
\pgfpathqlineto{13.3615bp}{46.5044bp}
\pgfpathqlineto{13.5163bp}{46.4759bp}
\pgfpathqlineto{13.5704bp}{46.7703bp}
\pgfpathqlineto{13.3615bp}{46.5044bp}
\pgfpathclose
\pgfusepathqfill
\end{pgfscope}
\begin{pgfscope}
\pgfsetlinewidth{0.5987bp}
\definecolor{sc}{rgb}{0.0000,0.0000,0.0000}
\pgfsetstrokecolor{sc}
\pgfsetmiterjoin
\pgfsetbuttcap
\pgfpathqmoveto{67.8748bp}{78.6124bp}
\pgfpathqlineto{48.6032bp}{46.3693bp}
\pgfusepathqstroke
\end{pgfscope}
\begin{pgfscope}
\definecolor{fc}{rgb}{0.0000,0.0000,0.0000}
\pgfsetfillcolor{fc}
\pgfusepathqfill
\end{pgfscope}
\begin{pgfscope}
\definecolor{fc}{rgb}{0.0000,0.0000,0.0000}
\pgfsetfillcolor{fc}
\pgfusepathqfill
\end{pgfscope}
\begin{pgfscope}
\definecolor{fc}{rgb}{0.0000,0.0000,0.0000}
\pgfsetfillcolor{fc}
\pgfpathqmoveto{45.9157bp}{41.8728bp}
\pgfpathqlineto{50.7685bp}{46.1112bp}
\pgfpathqlineto{48.5225bp}{46.2342bp}
\pgfpathqlineto{47.3504bp}{48.1542bp}
\pgfpathqlineto{45.9157bp}{41.8728bp}
\pgfpathclose
\pgfusepathqfill
\end{pgfscope}
\begin{pgfscope}
\definecolor{fc}{rgb}{0.0000,0.0000,0.0000}
\pgfsetfillcolor{fc}
\pgfpathqmoveto{48.5225bp}{46.2342bp}
\pgfpathqlineto{48.6032bp}{46.3693bp}
\pgfpathqlineto{48.8601bp}{46.2157bp}
\pgfpathqlineto{48.5225bp}{46.2342bp}
\pgfpathqlineto{48.6032bp}{46.3693bp}
\pgfpathqlineto{48.3463bp}{46.5228bp}
\pgfpathqlineto{48.5225bp}{46.2342bp}
\pgfpathclose
\pgfusepathqfill
\end{pgfscope}
\begin{pgfscope}
\pgfsetlinewidth{0.5987bp}
\definecolor{sc}{rgb}{0.0000,0.0000,0.0000}
\pgfsetstrokecolor{sc}
\pgfsetmiterjoin
\pgfsetbuttcap
\pgfpathqmoveto{44.8630bp}{115.7696bp}
\pgfpathqlineto{26.1032bp}{138.7014bp}
\pgfusepathqstroke
\end{pgfscope}
\begin{pgfscope}
\definecolor{fc}{rgb}{0.0000,0.0000,0.0000}
\pgfsetfillcolor{fc}
\pgfusepathqfill
\end{pgfscope}
\begin{pgfscope}
\definecolor{fc}{rgb}{0.0000,0.0000,0.0000}
\pgfsetfillcolor{fc}
\pgfusepathqfill
\end{pgfscope}
\begin{pgfscope}
\definecolor{fc}{rgb}{0.0000,0.0000,0.0000}
\pgfsetfillcolor{fc}
\pgfpathqmoveto{22.7863bp}{142.7559bp}
\pgfpathqlineto{25.1253bp}{136.7523bp}
\pgfpathqlineto{26.0035bp}{138.8232bp}
\pgfpathqlineto{28.2074bp}{139.2737bp}
\pgfpathqlineto{22.7863bp}{142.7559bp}
\pgfpathclose
\pgfusepathqfill
\end{pgfscope}
\begin{pgfscope}
\definecolor{fc}{rgb}{0.0000,0.0000,0.0000}
\pgfsetfillcolor{fc}
\pgfpathqmoveto{26.0035bp}{138.8232bp}
\pgfpathqlineto{26.1032bp}{138.7014bp}
\pgfpathqlineto{25.8715bp}{138.5118bp}
\pgfpathqlineto{26.0035bp}{138.8232bp}
\pgfpathqlineto{26.1032bp}{138.7014bp}
\pgfpathqlineto{26.3349bp}{138.8909bp}
\pgfpathqlineto{26.0035bp}{138.8232bp}
\pgfpathclose
\pgfusepathqfill
\end{pgfscope}
\begin{pgfscope}
\pgfsetlinewidth{0.5987bp}
\definecolor{sc}{rgb}{0.0000,0.0000,0.0000}
\pgfsetstrokecolor{sc}
\pgfsetmiterjoin
\pgfsetbuttcap
\pgfpathqmoveto{67.8249bp}{81.0146bp}
\pgfpathqlineto{49.4095bp}{109.1062bp}
\pgfusepathqstroke
\end{pgfscope}
\begin{pgfscope}
\definecolor{fc}{rgb}{0.0000,0.0000,0.0000}
\pgfsetfillcolor{fc}
\pgfusepathqfill
\end{pgfscope}
\begin{pgfscope}
\definecolor{fc}{rgb}{0.0000,0.0000,0.0000}
\pgfsetfillcolor{fc}
\pgfusepathqfill
\end{pgfscope}
\begin{pgfscope}
\definecolor{fc}{rgb}{0.0000,0.0000,0.0000}
\pgfsetfillcolor{fc}
\pgfpathqmoveto{46.5376bp}{113.4871bp}
\pgfpathqlineto{48.2320bp}{107.2708bp}
\pgfpathqlineto{49.3232bp}{109.2378bp}
\pgfpathqlineto{51.5622bp}{109.4539bp}
\pgfpathqlineto{46.5376bp}{113.4871bp}
\pgfpathclose
\pgfusepathqfill
\end{pgfscope}
\begin{pgfscope}
\definecolor{fc}{rgb}{0.0000,0.0000,0.0000}
\pgfsetfillcolor{fc}
\pgfpathqmoveto{49.3232bp}{109.2378bp}
\pgfpathqlineto{49.4095bp}{109.1062bp}
\pgfpathqlineto{49.1592bp}{108.9421bp}
\pgfpathqlineto{49.3232bp}{109.2378bp}
\pgfpathqlineto{49.4095bp}{109.1062bp}
\pgfpathqlineto{49.6598bp}{109.2703bp}
\pgfpathqlineto{49.3232bp}{109.2378bp}
\pgfpathclose
\pgfusepathqfill
\end{pgfscope}
\begin{pgfscope}
\pgfsetlinewidth{0.5987bp}
\definecolor{sc}{rgb}{0.0000,0.0000,0.0000}
\pgfsetstrokecolor{sc}
\pgfsetmiterjoin
\pgfsetbuttcap
\pgfpathqmoveto{110.6145bp}{87.6780bp}
\pgfpathqlineto{75.1448bp}{81.0515bp}
\pgfusepathqstroke
\end{pgfscope}
\begin{pgfscope}
\definecolor{fc}{rgb}{0.0000,0.0000,0.0000}
\pgfsetfillcolor{fc}
\pgfusepathqfill
\end{pgfscope}
\begin{pgfscope}
\definecolor{fc}{rgb}{0.0000,0.0000,0.0000}
\pgfsetfillcolor{fc}
\pgfusepathqfill
\end{pgfscope}
\begin{pgfscope}
\definecolor{fc}{rgb}{0.0000,0.0000,0.0000}
\pgfsetfillcolor{fc}
\pgfpathqmoveto{69.9955bp}{80.0895bp}
\pgfpathqlineto{76.3847bp}{79.2576bp}
\pgfpathqlineto{74.9901bp}{81.0226bp}
\pgfpathqlineto{75.6534bp}{83.1720bp}
\pgfpathqlineto{69.9955bp}{80.0895bp}
\pgfpathclose
\pgfusepathqfill
\end{pgfscope}
\begin{pgfscope}
\definecolor{fc}{rgb}{0.0000,0.0000,0.0000}
\pgfsetfillcolor{fc}
\pgfpathqmoveto{74.9901bp}{81.0226bp}
\pgfpathqlineto{75.1448bp}{81.0515bp}
\pgfpathqlineto{75.1998bp}{80.7572bp}
\pgfpathqlineto{74.9901bp}{81.0226bp}
\pgfpathqlineto{75.1448bp}{81.0515bp}
\pgfpathqlineto{75.0898bp}{81.3457bp}
\pgfpathqlineto{74.9901bp}{81.0226bp}
\pgfpathclose
\pgfusepathqfill
\end{pgfscope}
\begin{pgfscope}
\pgfsetlinewidth{0.5987bp}
\definecolor{sc}{rgb}{0.0000,0.0000,0.0000}
\pgfsetstrokecolor{sc}
\pgfsetmiterjoin
\pgfsetbuttcap
\pgfpathqmoveto{136.4854bp}{115.7511bp}
\pgfpathqlineto{116.4051bp}{92.9347bp}
\pgfusepathqstroke
\end{pgfscope}
\begin{pgfscope}
\definecolor{fc}{rgb}{0.0000,0.0000,0.0000}
\pgfsetfillcolor{fc}
\pgfusepathqfill
\end{pgfscope}
\begin{pgfscope}
\definecolor{fc}{rgb}{0.0000,0.0000,0.0000}
\pgfsetfillcolor{fc}
\pgfusepathqfill
\end{pgfscope}
\begin{pgfscope}
\definecolor{fc}{rgb}{0.0000,0.0000,0.0000}
\pgfsetfillcolor{fc}
\pgfpathqmoveto{112.9443bp}{89.0023bp}
\pgfpathqlineto{118.4874bp}{92.2869bp}
\pgfpathqlineto{116.3012bp}{92.8166bp}
\pgfpathqlineto{115.4981bp}{94.9177bp}
\pgfpathqlineto{112.9443bp}{89.0023bp}
\pgfpathclose
\pgfusepathqfill
\end{pgfscope}
\begin{pgfscope}
\definecolor{fc}{rgb}{0.0000,0.0000,0.0000}
\pgfsetfillcolor{fc}
\pgfpathqmoveto{116.3012bp}{92.8166bp}
\pgfpathqlineto{116.4051bp}{92.9347bp}
\pgfpathqlineto{116.6298bp}{92.7369bp}
\pgfpathqlineto{116.3012bp}{92.8166bp}
\pgfpathqlineto{116.4051bp}{92.9347bp}
\pgfpathqlineto{116.1804bp}{93.1325bp}
\pgfpathqlineto{116.3012bp}{92.8166bp}
\pgfpathclose
\pgfusepathqfill
\end{pgfscope}
\begin{pgfscope}
\pgfsetlinewidth{0.5987bp}
\definecolor{sc}{rgb}{0.0000,0.0000,0.0000}
\pgfsetstrokecolor{sc}
\pgfsetmiterjoin
\pgfsetbuttcap
\pgfpathqmoveto{142.8918bp}{67.6555bp}
\pgfpathqlineto{117.5713bp}{84.2846bp}
\pgfusepathqstroke
\end{pgfscope}
\begin{pgfscope}
\definecolor{fc}{rgb}{0.0000,0.0000,0.0000}
\pgfsetfillcolor{fc}
\pgfusepathqfill
\end{pgfscope}
\begin{pgfscope}
\definecolor{fc}{rgb}{0.0000,0.0000,0.0000}
\pgfsetfillcolor{fc}
\pgfusepathqfill
\end{pgfscope}
\begin{pgfscope}
\definecolor{fc}{rgb}{0.0000,0.0000,0.0000}
\pgfsetfillcolor{fc}
\pgfpathqmoveto{113.1928bp}{87.1602bp}
\pgfpathqlineto{117.2217bp}{82.1322bp}
\pgfpathqlineto{117.4398bp}{84.3710bp}
\pgfpathqlineto{119.4077bp}{85.4606bp}
\pgfpathqlineto{113.1928bp}{87.1602bp}
\pgfpathclose
\pgfusepathqfill
\end{pgfscope}
\begin{pgfscope}
\definecolor{fc}{rgb}{0.0000,0.0000,0.0000}
\pgfsetfillcolor{fc}
\pgfpathqmoveto{117.4398bp}{84.3710bp}
\pgfpathqlineto{117.5713bp}{84.2846bp}
\pgfpathqlineto{117.4070bp}{84.0344bp}
\pgfpathqlineto{117.4398bp}{84.3710bp}
\pgfpathqlineto{117.5713bp}{84.2846bp}
\pgfpathqlineto{117.7356bp}{84.5348bp}
\pgfpathqlineto{117.4398bp}{84.3710bp}
\pgfpathclose
\pgfusepathqfill
\end{pgfscope}
\end{pgfpicture}

    \end{minipage}

    \vspace{1em}
    \begin{minipage}{0.45\textwidth}
      \begin{pgfpicture}
\pgfpathrectangle{\pgfpointorigin}{\pgfqpoint{150.0000bp}{145.0000bp}}
\pgfusepath{use as bounding box}
\begin{pgfscope}
\definecolor{fc}{rgb}{0.0000,0.0000,0.0000}
\pgfsetfillcolor{fc}
\pgftransformshift{\pgfqpoint{36.6353bp}{9.1588bp}}
\pgftransformscale{1.0000}
\pgftext[]{H}
\end{pgfscope}
\begin{pgfscope}
\definecolor{fc}{rgb}{0.0000,0.0000,0.0000}
\pgfsetfillcolor{fc}
\pgfsetlinewidth{0.5899bp}
\definecolor{sc}{rgb}{0.0000,0.0000,0.0000}
\pgfsetstrokecolor{sc}
\pgfsetmiterjoin
\pgfsetbuttcap
\pgfpathqmoveto{52.1784bp}{9.1588bp}
\pgfpathqcurveto{52.1784bp}{10.5895bp}{51.0186bp}{11.7493bp}{49.5879bp}{11.7493bp}
\pgfpathqcurveto{48.1572bp}{11.7493bp}{46.9974bp}{10.5895bp}{46.9974bp}{9.1588bp}
\pgfpathqcurveto{46.9974bp}{7.7281bp}{48.1572bp}{6.5683bp}{49.5879bp}{6.5683bp}
\pgfpathqcurveto{51.0186bp}{6.5683bp}{52.1784bp}{7.7281bp}{52.1784bp}{9.1588bp}
\pgfpathclose
\pgfusepathqfillstroke
\end{pgfscope}
\begin{pgfscope}
\definecolor{fc}{rgb}{0.0000,0.0000,0.0000}
\pgfsetfillcolor{fc}
\pgftransformshift{\pgfqpoint{0.0000bp}{45.7942bp}}
\pgftransformscale{1.0000}
\pgftext[]{G}
\end{pgfscope}
\begin{pgfscope}
\definecolor{fc}{rgb}{0.0000,0.0000,0.0000}
\pgfsetfillcolor{fc}
\pgfsetlinewidth{0.5899bp}
\definecolor{sc}{rgb}{0.0000,0.0000,0.0000}
\pgfsetstrokecolor{sc}
\pgfsetmiterjoin
\pgfsetbuttcap
\pgfpathqmoveto{15.5431bp}{45.7942bp}
\pgfpathqcurveto{15.5431bp}{47.2249bp}{14.3832bp}{48.3847bp}{12.9525bp}{48.3847bp}
\pgfpathqcurveto{11.5218bp}{48.3847bp}{10.3620bp}{47.2249bp}{10.3620bp}{45.7942bp}
\pgfpathqcurveto{10.3620bp}{44.3635bp}{11.5218bp}{43.2037bp}{12.9525bp}{43.2037bp}
\pgfpathqcurveto{14.3832bp}{43.2037bp}{15.5431bp}{44.3635bp}{15.5431bp}{45.7942bp}
\pgfpathclose
\pgfusepathqfillstroke
\end{pgfscope}
\begin{pgfscope}
\definecolor{fc}{rgb}{0.0000,0.0000,0.0000}
\pgfsetfillcolor{fc}
\pgftransformshift{\pgfqpoint{110.5569bp}{0.0000bp}}
\pgftransformscale{1.0000}
\pgftext[]{F}
\end{pgfscope}
\begin{pgfscope}
\definecolor{fc}{rgb}{0.0000,0.0000,0.0000}
\pgfsetfillcolor{fc}
\pgfsetlinewidth{0.5899bp}
\definecolor{sc}{rgb}{0.0000,0.0000,0.0000}
\pgfsetstrokecolor{sc}
\pgfsetmiterjoin
\pgfsetbuttcap
\pgfpathqmoveto{103.9886bp}{9.1588bp}
\pgfpathqcurveto{103.9886bp}{10.5895bp}{102.8288bp}{11.7493bp}{101.3981bp}{11.7493bp}
\pgfpathqcurveto{99.9674bp}{11.7493bp}{98.8076bp}{10.5895bp}{98.8076bp}{9.1588bp}
\pgfpathqcurveto{98.8076bp}{7.7281bp}{99.9674bp}{6.5683bp}{101.3981bp}{6.5683bp}
\pgfpathqcurveto{102.8288bp}{6.5683bp}{103.9886bp}{7.7281bp}{103.9886bp}{9.1588bp}
\pgfpathclose
\pgfusepathqfillstroke
\end{pgfscope}
\begin{pgfscope}
\definecolor{fc}{rgb}{0.0000,0.0000,0.0000}
\pgfsetfillcolor{fc}
\pgftransformshift{\pgfqpoint{0.0000bp}{97.6044bp}}
\pgftransformscale{1.0000}
\pgftext[]{E}
\end{pgfscope}
\begin{pgfscope}
\definecolor{fc}{rgb}{0.0000,0.0000,0.0000}
\pgfsetfillcolor{fc}
\pgfsetlinewidth{0.5899bp}
\definecolor{sc}{rgb}{0.0000,0.0000,0.0000}
\pgfsetstrokecolor{sc}
\pgfsetmiterjoin
\pgfsetbuttcap
\pgfpathqmoveto{15.5431bp}{97.6044bp}
\pgfpathqcurveto{15.5431bp}{99.0351bp}{14.3832bp}{100.1949bp}{12.9525bp}{100.1949bp}
\pgfpathqcurveto{11.5218bp}{100.1949bp}{10.3620bp}{99.0351bp}{10.3620bp}{97.6044bp}
\pgfpathqcurveto{10.3620bp}{96.1737bp}{11.5218bp}{95.0138bp}{12.9525bp}{95.0138bp}
\pgfpathqcurveto{14.3832bp}{95.0138bp}{15.5431bp}{96.1737bp}{15.5431bp}{97.6044bp}
\pgfpathclose
\pgfusepathqfillstroke
\end{pgfscope}
\begin{pgfscope}
\definecolor{fc}{rgb}{0.0000,0.0000,0.0000}
\pgfsetfillcolor{fc}
\pgftransformshift{\pgfqpoint{58.7467bp}{143.3985bp}}
\pgftransformscale{1.0000}
\pgftext[]{D}
\end{pgfscope}
\begin{pgfscope}
\definecolor{fc}{rgb}{0.0000,0.0000,0.0000}
\pgfsetfillcolor{fc}
\pgfsetlinewidth{0.5899bp}
\definecolor{sc}{rgb}{0.0000,0.0000,0.0000}
\pgfsetstrokecolor{sc}
\pgfsetmiterjoin
\pgfsetbuttcap
\pgfpathqmoveto{52.1784bp}{134.2397bp}
\pgfpathqcurveto{52.1784bp}{135.6704bp}{51.0186bp}{136.8302bp}{49.5879bp}{136.8302bp}
\pgfpathqcurveto{48.1572bp}{136.8302bp}{46.9974bp}{135.6704bp}{46.9974bp}{134.2397bp}
\pgfpathqcurveto{46.9974bp}{132.8090bp}{48.1572bp}{131.6492bp}{49.5879bp}{131.6492bp}
\pgfpathqcurveto{51.0186bp}{131.6492bp}{52.1784bp}{132.8090bp}{52.1784bp}{134.2397bp}
\pgfpathclose
\pgfusepathqfillstroke
\end{pgfscope}
\begin{pgfscope}
\definecolor{fc}{rgb}{0.0000,0.0000,0.0000}
\pgfsetfillcolor{fc}
\pgftransformshift{\pgfqpoint{94.2020bp}{145.0093bp}}
\pgftransformscale{1.0000}
\pgftext[]{C}
\end{pgfscope}
\begin{pgfscope}
\definecolor{fc}{rgb}{0.0000,0.0000,0.0000}
\pgfsetfillcolor{fc}
\pgfsetlinewidth{0.5899bp}
\definecolor{sc}{rgb}{0.0000,0.0000,0.0000}
\pgfsetstrokecolor{sc}
\pgfsetmiterjoin
\pgfsetbuttcap
\pgfpathqmoveto{103.9886bp}{134.2397bp}
\pgfpathqcurveto{103.9886bp}{135.6704bp}{102.8288bp}{136.8302bp}{101.3981bp}{136.8302bp}
\pgfpathqcurveto{99.9674bp}{136.8302bp}{98.8076bp}{135.6704bp}{98.8076bp}{134.2397bp}
\pgfpathqcurveto{98.8076bp}{132.8090bp}{99.9674bp}{131.6492bp}{101.3981bp}{131.6492bp}
\pgfpathqcurveto{102.8288bp}{131.6492bp}{103.9886bp}{132.8090bp}{103.9886bp}{134.2397bp}
\pgfpathclose
\pgfusepathqfillstroke
\end{pgfscope}
\begin{pgfscope}
\definecolor{fc}{rgb}{0.0000,0.0000,0.0000}
\pgfsetfillcolor{fc}
\pgftransformshift{\pgfqpoint{150.0000bp}{102.5611bp}}
\pgftransformscale{1.0000}
\pgftext[]{B}
\end{pgfscope}
\begin{pgfscope}
\definecolor{fc}{rgb}{0.0000,0.0000,0.0000}
\pgfsetfillcolor{fc}
\pgfsetlinewidth{0.5899bp}
\definecolor{sc}{rgb}{0.0000,0.0000,0.0000}
\pgfsetstrokecolor{sc}
\pgfsetmiterjoin
\pgfsetbuttcap
\pgfpathqmoveto{140.6239bp}{97.6044bp}
\pgfpathqcurveto{140.6239bp}{99.0351bp}{139.4641bp}{100.1949bp}{138.0334bp}{100.1949bp}
\pgfpathqcurveto{136.6027bp}{100.1949bp}{135.4429bp}{99.0351bp}{135.4429bp}{97.6044bp}
\pgfpathqcurveto{135.4429bp}{96.1737bp}{136.6027bp}{95.0138bp}{138.0334bp}{95.0138bp}
\pgfpathqcurveto{139.4641bp}{95.0138bp}{140.6239bp}{96.1737bp}{140.6239bp}{97.6044bp}
\pgfpathclose
\pgfusepathqfillstroke
\end{pgfscope}
\begin{pgfscope}
\definecolor{fc}{rgb}{0.0000,0.0000,0.0000}
\pgfsetfillcolor{fc}
\pgftransformshift{\pgfqpoint{140.5603bp}{33.0905bp}}
\pgftransformscale{1.0000}
\pgftext[]{A}
\end{pgfscope}
\begin{pgfscope}
\definecolor{fc}{rgb}{0.0000,0.0000,0.0000}
\pgfsetfillcolor{fc}
\pgfsetlinewidth{0.5899bp}
\definecolor{sc}{rgb}{0.0000,0.0000,0.0000}
\pgfsetstrokecolor{sc}
\pgfsetmiterjoin
\pgfsetbuttcap
\pgfpathqmoveto{140.6239bp}{45.7942bp}
\pgfpathqcurveto{140.6239bp}{47.2249bp}{139.4641bp}{48.3847bp}{138.0334bp}{48.3847bp}
\pgfpathqcurveto{136.6027bp}{48.3847bp}{135.4429bp}{47.2249bp}{135.4429bp}{45.7942bp}
\pgfpathqcurveto{135.4429bp}{44.3635bp}{136.6027bp}{43.2037bp}{138.0334bp}{43.2037bp}
\pgfpathqcurveto{139.4641bp}{43.2037bp}{140.6239bp}{44.3635bp}{140.6239bp}{45.7942bp}
\pgfpathclose
\pgfusepathqfillstroke
\end{pgfscope}
\begin{pgfscope}
\pgfsetlinewidth{0.5899bp}
\definecolor{sc}{rgb}{0.0000,0.0000,0.0000}
\pgfsetstrokecolor{sc}
\pgfsetmiterjoin
\pgfsetbuttcap
\pgfpathqmoveto{14.7843bp}{95.7726bp}
\pgfpathqlineto{95.9163bp}{14.6406bp}
\pgfusepathqstroke
\end{pgfscope}
\begin{pgfscope}
\definecolor{fc}{rgb}{0.0000,0.0000,0.0000}
\pgfsetfillcolor{fc}
\pgfusepathqfill
\end{pgfscope}
\begin{pgfscope}
\definecolor{fc}{rgb}{0.0000,0.0000,0.0000}
\pgfsetfillcolor{fc}
\pgfusepathqfill
\end{pgfscope}
\begin{pgfscope}
\definecolor{fc}{rgb}{0.0000,0.0000,0.0000}
\pgfsetfillcolor{fc}
\pgfpathqmoveto{99.5663bp}{10.9906bp}
\pgfpathqlineto{96.6839bp}{16.6477bp}
\pgfpathqlineto{96.0259bp}{14.5310bp}
\pgfpathqlineto{93.9092bp}{13.8730bp}
\pgfpathqlineto{99.5663bp}{10.9906bp}
\pgfpathclose
\pgfusepathqfill
\end{pgfscope}
\begin{pgfscope}
\definecolor{fc}{rgb}{0.0000,0.0000,0.0000}
\pgfsetfillcolor{fc}
\pgfpathqmoveto{96.0259bp}{14.5310bp}
\pgfpathqlineto{95.9163bp}{14.6406bp}
\pgfpathqlineto{96.1248bp}{14.8492bp}
\pgfpathqlineto{96.0259bp}{14.5310bp}
\pgfpathqlineto{95.9163bp}{14.6406bp}
\pgfpathqlineto{95.7077bp}{14.4321bp}
\pgfpathqlineto{96.0259bp}{14.5310bp}
\pgfpathclose
\pgfusepathqfill
\end{pgfscope}
\begin{pgfscope}
\pgfsetlinewidth{0.5899bp}
\definecolor{sc}{rgb}{0.0000,0.0000,0.0000}
\pgfsetstrokecolor{sc}
\pgfsetmiterjoin
\pgfsetbuttcap
\pgfpathqmoveto{47.7561bp}{132.4079bp}
\pgfpathqlineto{18.4343bp}{103.0862bp}
\pgfusepathqstroke
\end{pgfscope}
\begin{pgfscope}
\definecolor{fc}{rgb}{0.0000,0.0000,0.0000}
\pgfsetfillcolor{fc}
\pgfusepathqfill
\end{pgfscope}
\begin{pgfscope}
\definecolor{fc}{rgb}{0.0000,0.0000,0.0000}
\pgfsetfillcolor{fc}
\pgfusepathqfill
\end{pgfscope}
\begin{pgfscope}
\definecolor{fc}{rgb}{0.0000,0.0000,0.0000}
\pgfsetfillcolor{fc}
\pgfpathqmoveto{14.7843bp}{99.4361bp}
\pgfpathqlineto{20.4414bp}{102.3185bp}
\pgfpathqlineto{18.3247bp}{102.9765bp}
\pgfpathqlineto{17.6667bp}{105.0932bp}
\pgfpathqlineto{14.7843bp}{99.4361bp}
\pgfpathclose
\pgfusepathqfill
\end{pgfscope}
\begin{pgfscope}
\definecolor{fc}{rgb}{0.0000,0.0000,0.0000}
\pgfsetfillcolor{fc}
\pgfpathqmoveto{18.3247bp}{102.9765bp}
\pgfpathqlineto{18.4343bp}{103.0862bp}
\pgfpathqlineto{18.6429bp}{102.8776bp}
\pgfpathqlineto{18.3247bp}{102.9765bp}
\pgfpathqlineto{18.4343bp}{103.0862bp}
\pgfpathqlineto{18.2258bp}{103.2947bp}
\pgfpathqlineto{18.3247bp}{102.9765bp}
\pgfpathclose
\pgfusepathqfill
\end{pgfscope}
\begin{pgfscope}
\pgfsetlinewidth{0.5899bp}
\definecolor{sc}{rgb}{0.0000,0.0000,0.0000}
\pgfsetstrokecolor{sc}
\pgfsetmiterjoin
\pgfsetbuttcap
\pgfpathqmoveto{102.3897bp}{131.8457bp}
\pgfpathqlineto{135.0664bp}{52.9571bp}
\pgfusepathqstroke
\end{pgfscope}
\begin{pgfscope}
\definecolor{fc}{rgb}{0.0000,0.0000,0.0000}
\pgfsetfillcolor{fc}
\pgfusepathqfill
\end{pgfscope}
\begin{pgfscope}
\definecolor{fc}{rgb}{0.0000,0.0000,0.0000}
\pgfsetfillcolor{fc}
\pgfusepathqfill
\end{pgfscope}
\begin{pgfscope}
\definecolor{fc}{rgb}{0.0000,0.0000,0.0000}
\pgfsetfillcolor{fc}
\pgfpathqmoveto{137.0418bp}{48.1881bp}
\pgfpathqlineto{136.5437bp}{54.5176bp}
\pgfpathqlineto{135.1258bp}{52.8138bp}
\pgfpathqlineto{132.9184bp}{53.0160bp}
\pgfpathqlineto{137.0418bp}{48.1881bp}
\pgfpathclose
\pgfusepathqfill
\end{pgfscope}
\begin{pgfscope}
\definecolor{fc}{rgb}{0.0000,0.0000,0.0000}
\pgfsetfillcolor{fc}
\pgfpathqmoveto{135.1258bp}{52.8138bp}
\pgfpathqlineto{135.0664bp}{52.9571bp}
\pgfpathqlineto{135.3389bp}{53.0700bp}
\pgfpathqlineto{135.1258bp}{52.8138bp}
\pgfpathqlineto{135.0664bp}{52.9571bp}
\pgfpathqlineto{134.7939bp}{52.8442bp}
\pgfpathqlineto{135.1258bp}{52.8138bp}
\pgfpathclose
\pgfusepathqfill
\end{pgfscope}
\begin{pgfscope}
\pgfsetlinewidth{0.5899bp}
\definecolor{sc}{rgb}{0.0000,0.0000,0.0000}
\pgfsetstrokecolor{sc}
\pgfsetmiterjoin
\pgfsetbuttcap
\pgfpathqmoveto{136.2016bp}{99.4361bp}
\pgfpathqlineto{106.8799bp}{128.7579bp}
\pgfusepathqstroke
\end{pgfscope}
\begin{pgfscope}
\definecolor{fc}{rgb}{0.0000,0.0000,0.0000}
\pgfsetfillcolor{fc}
\pgfusepathqfill
\end{pgfscope}
\begin{pgfscope}
\definecolor{fc}{rgb}{0.0000,0.0000,0.0000}
\pgfsetfillcolor{fc}
\pgfusepathqfill
\end{pgfscope}
\begin{pgfscope}
\definecolor{fc}{rgb}{0.0000,0.0000,0.0000}
\pgfsetfillcolor{fc}
\pgfpathqmoveto{103.2298bp}{132.4079bp}
\pgfpathqlineto{106.1123bp}{126.7509bp}
\pgfpathqlineto{106.7702bp}{128.8675bp}
\pgfpathqlineto{108.8869bp}{129.5255bp}
\pgfpathqlineto{103.2298bp}{132.4079bp}
\pgfpathclose
\pgfusepathqfill
\end{pgfscope}
\begin{pgfscope}
\definecolor{fc}{rgb}{0.0000,0.0000,0.0000}
\pgfsetfillcolor{fc}
\pgfpathqmoveto{106.7702bp}{128.8675bp}
\pgfpathqlineto{106.8799bp}{128.7579bp}
\pgfpathqlineto{106.6713bp}{128.5493bp}
\pgfpathqlineto{106.7702bp}{128.8675bp}
\pgfpathqlineto{106.8799bp}{128.7579bp}
\pgfpathqlineto{107.0884bp}{128.9665bp}
\pgfpathqlineto{106.7702bp}{128.8675bp}
\pgfpathclose
\pgfusepathqfill
\end{pgfscope}
\begin{pgfscope}
\pgfsetlinewidth{0.5899bp}
\definecolor{sc}{rgb}{0.0000,0.0000,0.0000}
\pgfsetstrokecolor{sc}
\pgfsetmiterjoin
\pgfsetbuttcap
\pgfpathqmoveto{138.0334bp}{48.3847bp}
\pgfpathqlineto{138.0334bp}{89.8519bp}
\pgfusepathqstroke
\end{pgfscope}
\begin{pgfscope}
\definecolor{fc}{rgb}{0.0000,0.0000,0.0000}
\pgfsetfillcolor{fc}
\pgfusepathqfill
\end{pgfscope}
\begin{pgfscope}
\definecolor{fc}{rgb}{0.0000,0.0000,0.0000}
\pgfsetfillcolor{fc}
\pgfusepathqfill
\end{pgfscope}
\begin{pgfscope}
\definecolor{fc}{rgb}{0.0000,0.0000,0.0000}
\pgfsetfillcolor{fc}
\pgfpathqmoveto{138.0334bp}{95.0138bp}
\pgfpathqlineto{136.0714bp}{88.9755bp}
\pgfpathqlineto{138.0334bp}{90.0070bp}
\pgfpathqlineto{139.9954bp}{88.9755bp}
\pgfpathqlineto{138.0334bp}{95.0138bp}
\pgfpathclose
\pgfusepathqfill
\end{pgfscope}
\begin{pgfscope}
\definecolor{fc}{rgb}{0.0000,0.0000,0.0000}
\pgfsetfillcolor{fc}
\pgfpathqmoveto{138.0334bp}{90.0070bp}
\pgfpathqlineto{138.0334bp}{89.8519bp}
\pgfpathqlineto{137.7384bp}{89.8519bp}
\pgfpathqlineto{138.0334bp}{90.0070bp}
\pgfpathqlineto{138.0334bp}{89.8519bp}
\pgfpathqlineto{138.3284bp}{89.8519bp}
\pgfpathqlineto{138.0334bp}{90.0070bp}
\pgfpathclose
\pgfusepathqfill
\end{pgfscope}
\end{pgfpicture}

    \end{minipage}
  \begin{minipage}{0.45\textwidth}
    \begin{align*}
      G &= (V,E) \\
      V &= \{\alpha, \beta, \gamma, \delta, \epsilon, \zeta, \eta,
          \theta\} \\
      E &= \{(\alpha, \delta), (\theta, \eta), (\beta, \alpha),
          (\zeta, \delta), (\epsilon, \eta), (\gamma, \alpha)\}
    \end{align*}
    {\scriptsize
      Key: $\alpha$ = alpha, $\beta$ = beta, $\gamma$ = gamma, $\delta$ =
      delta, $\epsilon$ = epsilon, \\ $\zeta$ = zeta, $\eta$ = eta,
      $\theta$ = theta
    }
  \end{minipage}
  \end{center}

  \begin{itemize}
  \item The \term{indegree} of vertex $C$ is 1.  The \term{outdegree}
    of vertex $C$ is also 1.  The \term{indegree} of vertex 5 is $2$.
    The \term{outdegree} of vertex $g$ is $3$.
  \item $\{C,B,A\}$ is a \term{strongly connected component}.  So is
    $\{5,6,7,8\}$.  $\{D,E,F\}$ is a \term{weakly connected component}
    but not a strongly connected one.
  \item $b,c,d,e,f$ is a path.  $0,1,2,5,6$ is a path.  So is $D,E,F$.
    $0,1,2,5,8$ is not a path.  Neither is $F,E,D$.
  \end{itemize}
\end{model}

\begin{questions}
  \item What is the difference between directed graphs and the
    (undirected) graphs we saw on a previous activity?
  \item The previous activity defined graphs as consisting of a set
    $V$ of vertices and a set $E$ of edges, where each edge is a set
    of two vertices.  How would you modify this definition to allow
    for directed graphs?
  \item For each of the following graph terms/concepts, say whether
    you think its definition needs to be modified for directed graphs;
    if so, say what the new definition should be.
    \begin{questions}
    \item \term{vertex}
    \item \term{degree}
    \item \term{path}
    \item \term{cycle}
    \end{questions}
  \item What (if anything) about our implementation of BFS needs to be
    modified for BFS to work sensibly on directed graphs?
\end{questions}

\newpage
\begin{defn}
  A directed graph $G = (V,E)$ is \term{strongly connected} if for any two
  vertices $u,v \in V$ there is a (directed) path from $u$ to $v$,
  \emph{and also} from $v$ to $u$.
\end{defn}

\begin{questions}
  \item Describe a brute force algorithm for determining whether a
    given directed graph $G$ is strongly connected.
  \item Analyze the running time of your algorithm. Express your
    answer using $\Theta$.
\end{questions}

% \pause

% \begin{thm} \label{thm:strong-conn}
%   A directed graph $G = (V,E)$ is strongly connected if and only if
%   for any vertex $s \in V$, every other vertex in $G$
%   is mutually reachable with $s$ (that is, for each $v \in V$ there is
%   a directed path from $s$ to $v$ and another directed path from $v$
%   to $s$).
% \end{thm}

\begin{model*}{Reverse graphs and strong connectivity}{reverse-conn}

\begin{defn}
  Given a directed graph $G$, its \term{reverse graph}
  $G^{\mathrm{rev}}$ is the graph with the same vertices and edges,
  except with all the edges reversed.
\end{defn}

\begin{thm} \label{thm:conn-rev}
  A directed graph $G = (V,E)$ is strongly connected if and only if
  given any $s \in V$,
  \begin{itemize}
  \item all vertices are reachable from $s$ in $G$, and
  \item all vertices are reachable from $s$ in $G^{\mathrm{rev}}$.
  \end{itemize}

\end{thm}
\end{model*}

% \begin{questions}
%   \item In order to prove this ``if and only if'' statement, we must
%     prove both \blank\linebreak and \blank.
% \end{questions}
% \begin{proof}\marginnote{Hint: draw a picture!}
%   \mbox{} \vfill \mbox{}
% \end{proof}

\begin{questions}
\item Based on the above theorem, describe an algorithm to determine
  whether a given directed graph $G = (V,E)$ is strongly connected,
  and analyze its running time. \vspace{2in}

\item Can you give an informal, intuitive explanation why the theorem
  is true? (\emph{Hint}: if all vertices are reachable from $s$ in
  $G^{\mathrm{rev}}$, what does it tell us about $G$?)
\end{questions}

%% Did this as an activity in Fall '17 but it didn't really go very
%% well.  It's really not obvious what property I'm trying to get at
%% just by looking at the examples.  A good activity would have to
%% somehow introduce the idea via a lot of examples, e.g. of
%% situations that can be modeled by a bipartite graph.

%% ALTERNATIVE IDEA: extend this activity to lead them through
%% algorithm to decide whether a directed graph is strongly connected
%% (using BFS in G + BFS in Grev).

% \newpage

% \begin{model*}{Some graphs}{bipartite}
% There is a mystery property $X$.  Each graph either has property $X$
% or not.

%   These graphs have property $X$:

%   \begin{center}
%   \begin{diagram}[width=400]
%     import Graphs
%     dia = hsep 4 . map centerXY $ -- $
%       [ replicate 2 (replicate 3 pip)
%         # zipWith zip [[0,1,2],[3,4,5 :: Int]]
%         # map (map (uncurry named))
%         # map (vsep 3) # hsep 4
%         # applyAll
%           [connect' opts i j | (i,j) <- [(0 :: Int, 4 :: Int),(1,3),(1,4),(1,5),(2,5)]]
%       , [replicate 4 pip, replicate 2 pip]
%         # zipWith zip [[0,1,2,3],[4,5 :: Int]]
%         # map (map (uncurry named))
%         # map (centerY . vsep 2) # hsep 4
%         # applyAll
%           [connect' opts i j | (i,j) <- [(1 :: Int, 5 :: Int)]]
%       , graph [(0,1),(1,2),(2,3),(3,0)] (const "")
%       , graph (zip [0..5] ([1..5] ++ [0])) (const "")
%       , graph [(0,1),(1,2),(2,3),(3,0),(2,4),(4,5),(5,3)] (const "")
%       ]
%       where
%         opts = with & arrowHead .~ noHead
%   \end{diagram}
%   \end{center}

%   \hrule \bigskip

%   These graphs do not have property $X$:

%   \begin{center}
%   \begin{diagram}[width=400]
%     import Graphs
%     dia = hsep 4 . map centerXY $ -- $
%       [ replicate 2 (replicate 3 pip)
%         # zipWith zip [[0,1,2],[3,4,5 :: Int]]
%         # map (map (uncurry named))
%         # map (vsep 3) # hsep 4
%         # applyAll
%           [connect' opts i j | (i,j) <- [(0 :: Int, 4 :: Int),(1,3),(1,4),(1,5),(2,5),(0,1)]]
%       , graph [(0,1),(1,2),(2,0)] (const "")
%       , graph (zip [0..4] ([1..4] ++ [0])) (const "")
%       , graph [(0,1),(1,2),(2,3),(3,0),(2,4),(4,3)] (const "")
%       ]
%       where
%         opts = with & arrowHead .~ noHead
%   \end{diagram}
%   \end{center}
% \end{model*}

% \begin{questions}
% \item For each graph below, say whether you think it has property $X$.
%   \begin{subquestions}
%   \item
%     \begin{diagram}[width=50]
%       import Graphs
%       dia :: Diagram B
%       dia = replicate 2 (replicate 3 pip)
%         # zipWith zip [[0,1,2],[3,4,5 :: Int]]
%         # map (map (uncurry named))
%         # map (vsep 3) # hsep 4
%         # applyAll
%           [ connect' opts i j
%           | (i,j) <- [(0 :: Int, 4 :: Int),(1,3),(1,4),(2,5),(2,4),(0,3)]
%           ]
%         where
%           opts = with & arrowHead .~ noHead
%     \end{diagram}

%   \item
%     \begin{diagram}[width=50]
%       import Graphs
%       dia :: Diagram B
%       dia = replicate 2 (replicate 3 pip)
%         # zipWith zip [[0,1,2],[3,4,5 :: Int]]
%         # map (map (uncurry named))
%         # map (vsep 3) # hsep 4
%         # applyAll
%           [ connect' opts i j
%           | (i,j) <- [(0 :: Int, 4 :: Int),(1,3),(1,4),(1,5),(2,5),(4,5)]
%           ]
%         where
%           opts = with & arrowHead .~ noHead
%     \end{diagram}

%     \item
%       \begin{diagram}[width=75]
%         import Graphs
%         dia = graph (zip [0..6] ([1..6] ++ [0])) (const "")
%       \end{diagram}

%     \item
%       \begin{diagram}[width=75]
%         import Graphs
%         dia = graph (zip [0..5] ([1..5])) (const "")
%       \end{diagram}

%     \item
%       \begin{diagram}[width=75]
%         import Graphs
%         dia = graph [(0,1),(1,2),(2,3),(3,0),(1,4),(4,5),(5,2),(5,6),(6,7),(7,2)] (const "")
%       \end{diagram}

%     \item
%       \begin{diagram}[width=50]
%       import Graphs
%       dia :: Diagram B
%       dia = replicate 2 (replicate 3 pip)
%         # zipWith zip [[0,1,2],[3,4,5 :: Int]]
%         # map (map (uncurry named))
%         # map (vsep 3) # hsep 4
%         # applyAll
%           [ connect' opts i j
%           | (i,j) <- [(1 :: Int,3 :: Int),(0,1),(3,4),(2,5),(4,2),(0,5)]
%           ]
%         where
%           opts = with & arrowHead .~ noHead
%     \end{diagram}

%     \end{subquestions}
% \item What do you think is the definition of property $X$?
% \item Make a conjecture of the form: a graph $G$ has property $X$ if
%   and only if $G$ \blank.
% \end{questions}

\end{document}
