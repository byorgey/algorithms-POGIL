% -*- compile-command: "./build.sh" -*-
\documentclass{tufte-handout}

\usepackage{algo-activity}

\title{\thecourse: Applications of BFS}
\date{}

\begin{document}

\maketitle

Suppose we have a graph $G = (V,E)$.  A given graph could have few
edges, or lots of edges, or anything in between.  Let's think about
the range of possible relationships between $V$ and $E$.

\begin{questions}
\item The smallest possible value of $|E|$ is \blank.
\item $|E|$ is $O\Big( \qquad\qquad \Big)$ because \blank.
\item When $G$ is a tree, $|E|$ is $\Theta\Big( \qquad\qquad \Big)$
  because \blank.
\end{questions}

Now, recall from last class that we showed breadth-first search (BFS) can
be implemented to run in $\Theta(|V| + |E|)$ time.

\begin{questions}
\item In terms of $\Theta$, how fast does BFS run, as a function of
  $|V|$, when $G$ is a tree?
\item How fast does BFS run, as a function of $|V|$, when $G$ is very
  dense, \ie it contains some constant fraction (say, half) of all
  possible edges?
\end{questions}

\newpage
\section{A first application of BFS}

\begin{questions}
  \item Describe an algorithm to find the connected components of a
    graph $G$.

    \textbf{Input}: a graph $G = (V,E)$ \\
    \textbf{Output}: a set of sets of vertices,
    \texttt{Set<Set<Vertex{>}>}, where each set contains the vertices in
    some (maximal) connected component.  That is, all the vertices
    within each set should be connected; no vertex should be
    connected to vertices in any other set; and every vertex in $V$
    should be contained in exactly one of the sets.

    For example, given the graph below, the algorithm should return
    $\{\{D,E,F\}, \{C,B,A\}, \{G\}, \{H\}\}$.

    \begin{center}
        \begin{diagram}[width=150]
      import Graphs
      import Data.Char (chr, ord)
      dia = graph
        [ (0,1), (1,2), (2,0)
        , (3,4), (4,5)
        , (6,6), (7,7)
        ]
        (\i -> [chr (i + ord 'A')])
      \end{diagram}
    \end{center}

    Describe your algorithm (using informal prose or pseudocode) and
    analyze its asymptotic running time.
\end{questions}

\pause

\section{A second application of BFS}

\begin{model}{Directed graphs}{directed}
  \begin{center}
  \begin{minipage}{0.45\textwidth}
  \begin{diagram}[width=150]
    import Graphs
    dia :: Diagram B
    dia = graph' True
      [(0,1),(0,2),(1,2), (1,3),(3,4),(2,5),(5,6),(6,7),(7,8),(8,5),(3,9)]
      (\i -> "$" ++ show i ++ "$")
    \end{diagram}
  \end{minipage}
  \begin{minipage}{0.45\textwidth}
  \begin{diagram}[width=150]
    import Graphs
    import Data.Char (chr, ord)
    dia :: Diagram B
    dia = graph' True
      [(0,2), (1,2), (2,3), (3,4), (4,5), (3,6), (6,7), (6,8), (6,9)]
      (\i -> [chr (i + ord 'a')])
    \end{diagram}
  \end{minipage}

  \vspace{1em}
  \begin{minipage}{0.45\textwidth}
    \begin{diagram}[width=150]
      import Graphs
      import Data.Char (chr, ord)
      dia :: Diagram B
      dia = graph' True
        [ (0,1), (1,2), (2,0)
        , (3,4), (4,5)
        , (6,6), (7,7)
        ]
        (\i -> [chr (i + ord 'A')])
    \end{diagram}
  \end{minipage}
  \begin{minipage}{0.45\textwidth}
    \begin{align*}
      G &= (V,E) \\
      V &= \{\alpha, \beta, \gamma, \delta, \epsilon, \zeta, \eta,
          \theta\} \\
      E &= \{(\alpha, \delta), (\theta, \eta), (\beta, \alpha),
          (\zeta, \delta), (\epsilon, \eta), (\gamma, \alpha)\}
    \end{align*}
    {\scriptsize
      Key: $\alpha$ = alpha, $\beta$ = beta, $\gamma$ = gamma, $\delta$ =
      delta, $\epsilon$ = epsilon, \\ $\zeta$ = zeta, $\eta$ = eta,
      $\theta$ = theta
    }
  \end{minipage}
  \end{center}

  \begin{itemize}
  \item The \term{indegree} of vertex $C$ is 1.  The \term{outdegree}
    of vertex $C$ is also 1.  The \term{indegree} of vertex 5 is $2$.
    The \term{outdegree} of vertex $g$ is $3$.
  \item $\{C,B,A\}$ is a \term{strongly connected component}.  So is
    $\{5,6,7,8\}$.  $\{D,E,F\}$ is a \term{weakly connected component}
    but not a strongly connected one.
  \item $b,c,d,e,f$ is a path.  $0,1,2,5,6$ is a path.  So is $D,E,F$.
    $0,1,2,5,8$ is not a path.  Neither is $F,E,D$.
  \end{itemize}
\end{model}

\begin{questions}
  \item What is the difference between directed graphs and the
    (undirected) graphs we saw on a previous activity?
  \item The previous activity defined graphs as consisting of a set
    $V$ of vertices and a set $E$ of edges, where each edge is a set
    of two vertices.  How would you modify this definition to allow
    for directed graphs?
  \item For each of the following graph terms/concepts, say whether
    you think its definition needs to be modified for directed graphs;
    if so, say what the new definition should be.
    \begin{questions}
    \item \term{vertex}
    \item \term{degree}
    \item \term{path}
    \item \term{cycle}
    \end{questions}
  \item What (if anything) about our implementation of BFS needs to be
    modified for BFS to work sensibly on directed graphs?
\end{questions}

\newpage
\begin{defn}
  A directed graph $G = (V,E)$ is \term{strongly connected} if for any two
  vertices $u,v \in V$ there is a (directed) path from $u$ to $v$,
  \emph{and also} from $v$ to $u$.
\end{defn}

\begin{questions}
  \item Describe a brute force algorithm for determining whether a
    given directed graph $G$ is strongly connected.
  \item Analyze the running time of your algorithm. Express your
    answer using $\Theta$.
\end{questions}

% \pause

% \begin{thm} \label{thm:strong-conn}
%   A directed graph $G = (V,E)$ is strongly connected if and only if
%   for any vertex $s \in V$, every other vertex in $G$
%   is mutually reachable with $s$ (that is, for each $v \in V$ there is
%   a directed path from $s$ to $v$ and another directed path from $v$
%   to $s$).
% \end{thm}

\begin{model*}{Reverse graphs and strong connectivity}{reverse-conn}

\begin{defn}
  Given a directed graph $G$, its \term{reverse graph}
  $G^{\mathrm{rev}}$ is the graph with the same vertices and edges,
  except with all the edges reversed.
\end{defn}

\begin{thm} \label{thm:conn-rev}
  A directed graph $G = (V,E)$ is strongly connected if and only if
  given any $s \in V$,
  \begin{itemize}
  \item all vertices are reachable from $s$ in $G$, and
  \item all vertices are reachable from $s$ in $G^{\mathrm{rev}}$.
  \end{itemize}

\end{thm}
\end{model*}

% \begin{questions}
%   \item In order to prove this ``if and only if'' statement, we must
%     prove both \blank\linebreak and \blank.
% \end{questions}
% \begin{proof}\marginnote{Hint: draw a picture!}
%   \mbox{} \vfill \mbox{}
% \end{proof}

\begin{questions}
\item Based on the above theorem, describe an algorithm to determine
  whether a given directed graph $G = (V,E)$ is strongly connected,
  and analyze its running time. \vspace{2in}

\item Can you give an informal, intuitive explanation why the theorem
  is true? (\emph{Hint}: if all vertices are reachable from $s$ in
  $G^{\mathrm{rev}}$, what does it tell us about $G$?)
\end{questions}

%% Did this as an activity in Fall '17 but it didn't really go very
%% well.  It's really not obvious what property I'm trying to get at
%% just by looking at the examples.  A good activity would have to
%% somehow introduce the idea via a lot of examples, e.g. of
%% situations that can be modeled by a bipartite graph.

%% ALTERNATIVE IDEA: extend this activity to lead them through
%% algorithm to decide whether a directed graph is strongly connected
%% (using BFS in G + BFS in Grev).

% \newpage

% \begin{model*}{Some graphs}{bipartite}
% There is a mystery property $X$.  Each graph either has property $X$
% or not.

%   These graphs have property $X$:

%   \begin{center}
%   \begin{diagram}[width=400]
%     import Graphs
%     dia = hsep 4 . map centerXY $ -- $
%       [ replicate 2 (replicate 3 pip)
%         # zipWith zip [[0,1,2],[3,4,5 :: Int]]
%         # map (map (uncurry named))
%         # map (vsep 3) # hsep 4
%         # applyAll
%           [connect' opts i j | (i,j) <- [(0 :: Int, 4 :: Int),(1,3),(1,4),(1,5),(2,5)]]
%       , [replicate 4 pip, replicate 2 pip]
%         # zipWith zip [[0,1,2,3],[4,5 :: Int]]
%         # map (map (uncurry named))
%         # map (centerY . vsep 2) # hsep 4
%         # applyAll
%           [connect' opts i j | (i,j) <- [(1 :: Int, 5 :: Int)]]
%       , graph [(0,1),(1,2),(2,3),(3,0)] (const "")
%       , graph (zip [0..5] ([1..5] ++ [0])) (const "")
%       , graph [(0,1),(1,2),(2,3),(3,0),(2,4),(4,5),(5,3)] (const "")
%       ]
%       where
%         opts = with & arrowHead .~ noHead
%   \end{diagram}
%   \end{center}

%   \hrule \bigskip

%   These graphs do not have property $X$:

%   \begin{center}
%   \begin{diagram}[width=400]
%     import Graphs
%     dia = hsep 4 . map centerXY $ -- $
%       [ replicate 2 (replicate 3 pip)
%         # zipWith zip [[0,1,2],[3,4,5 :: Int]]
%         # map (map (uncurry named))
%         # map (vsep 3) # hsep 4
%         # applyAll
%           [connect' opts i j | (i,j) <- [(0 :: Int, 4 :: Int),(1,3),(1,4),(1,5),(2,5),(0,1)]]
%       , graph [(0,1),(1,2),(2,0)] (const "")
%       , graph (zip [0..4] ([1..4] ++ [0])) (const "")
%       , graph [(0,1),(1,2),(2,3),(3,0),(2,4),(4,3)] (const "")
%       ]
%       where
%         opts = with & arrowHead .~ noHead
%   \end{diagram}
%   \end{center}
% \end{model*}

% \begin{questions}
% \item For each graph below, say whether you think it has property $X$.
%   \begin{subquestions}
%   \item
%     \begin{diagram}[width=50]
%       import Graphs
%       dia :: Diagram B
%       dia = replicate 2 (replicate 3 pip)
%         # zipWith zip [[0,1,2],[3,4,5 :: Int]]
%         # map (map (uncurry named))
%         # map (vsep 3) # hsep 4
%         # applyAll
%           [ connect' opts i j
%           | (i,j) <- [(0 :: Int, 4 :: Int),(1,3),(1,4),(2,5),(2,4),(0,3)]
%           ]
%         where
%           opts = with & arrowHead .~ noHead
%     \end{diagram}

%   \item
%     \begin{diagram}[width=50]
%       import Graphs
%       dia :: Diagram B
%       dia = replicate 2 (replicate 3 pip)
%         # zipWith zip [[0,1,2],[3,4,5 :: Int]]
%         # map (map (uncurry named))
%         # map (vsep 3) # hsep 4
%         # applyAll
%           [ connect' opts i j
%           | (i,j) <- [(0 :: Int, 4 :: Int),(1,3),(1,4),(1,5),(2,5),(4,5)]
%           ]
%         where
%           opts = with & arrowHead .~ noHead
%     \end{diagram}

%     \item
%       \begin{diagram}[width=75]
%         import Graphs
%         dia = graph (zip [0..6] ([1..6] ++ [0])) (const "")
%       \end{diagram}

%     \item
%       \begin{diagram}[width=75]
%         import Graphs
%         dia = graph (zip [0..5] ([1..5])) (const "")
%       \end{diagram}

%     \item
%       \begin{diagram}[width=75]
%         import Graphs
%         dia = graph [(0,1),(1,2),(2,3),(3,0),(1,4),(4,5),(5,2),(5,6),(6,7),(7,2)] (const "")
%       \end{diagram}

%     \item
%       \begin{diagram}[width=50]
%       import Graphs
%       dia :: Diagram B
%       dia = replicate 2 (replicate 3 pip)
%         # zipWith zip [[0,1,2],[3,4,5 :: Int]]
%         # map (map (uncurry named))
%         # map (vsep 3) # hsep 4
%         # applyAll
%           [ connect' opts i j
%           | (i,j) <- [(1 :: Int,3 :: Int),(0,1),(3,4),(2,5),(4,2),(0,5)]
%           ]
%         where
%           opts = with & arrowHead .~ noHead
%     \end{diagram}

%     \end{subquestions}
% \item What do you think is the definition of property $X$?
% \item Make a conjecture of the form: a graph $G$ has property $X$ if
%   and only if $G$ \blank.
% \end{questions}

\end{document}
