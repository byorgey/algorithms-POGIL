% -*- compile-command: "./build.sh" -*-
\documentclass{tufte-handout}

\usepackage{algo-activity}

\title{\thecourse: Applications of BFS}
\date{}

\begin{document}

\maketitle

Suppose we have a graph $G = (V,E)$.  A given graph could have few
edges, or lots of edges, or anything in between.  Let's think about
the range of possible relationships between $V$ and $E$.

\begin{questions}
  \item How big can $|E|$ be, relative to $|V|$?
  \begin{subquestions}
  \item The smallest possible value of $|E|$ is \blank.
  \item $|E|$ is $O\Big( \qquad\qquad \Big)$ because \blank.
  \item When $G$ is a tree, $|E|$ is $\Theta\Big( \qquad\qquad \Big)$
    because \blank.
  \end{subquestions}
\end{questions}

Now, recall from last class that we showed breadth-first search (BFS) can
be implemented to run in $\Theta(|E|)$ time.

\begin{questions}
\item In terms of $\Theta$, how fast does BFS run, as a function of
  $|V|$, when $G$ is a tree?
\item How fast does BFS run, as a function of $|V|$, when $G$ is very
  dense, \ie it contains some constant fraction (say, half) of all
  possible edges?
\end{questions}

\newpage
\section{A first application of BFS}

\begin{questions}
  \item Describe an algorithm to find the connected components of a
    graph $G$.

    \textbf{Input}: a graph $G = (V,E)$ \\
    \textbf{Output}: a set of sets of vertices,
    \texttt{Set<Set<Vertex{>}>}, where each set contains the vertices in
    some (maximal) connected component.  That is, all the vertices
    within each set should be connected; no vertex should be
    connected to vertices in any other sets; and every vertex in $V$
    should be contained in exactly one of the sets.

    For example, given the graph below, the algorithm should return
    $\{\{D,E,F\}, \{C,B,A\}, \{G\}, \{H\}\}$.

    \begin{center}
        \begin{diagram}[width=150]
      import Graphs
      import Data.Char (chr, ord)
      dia = graph
        [ (0,1), (1,2), (2,0)
        , (3,4), (4,5)
        , (6,6), (7,7)
        ]
        (\i -> [chr (i + ord 'A')])
      \end{diagram}
    \end{center}

    Describe your algorithm (using informal prose or pseudocode) and
    analyze its asymptotic running time.
\end{questions}

\pause

\section{A second application of BFS}

\begin{model*}{Directed graphs}{directed}

  See the board for examples of \term{directed} graphs.

  % \begin{center}
  % \begin{diagram}[width=400]
  %   import Graphs
  %   dia = graph True [(0,1), (0,2), (1,2)]
  % \end{diagram}
  % \end{center}

  % XXX vertex Y has degree nnn, indegree yyy, outdegree zzz.
\end{model*}

\begin{questions}
  \item What is the difference between directed graphs and the
    (undirected) graphs we saw on a previous activity?
  \item The previous activity defined graphs as consisting of a set
    $V$ of vertices and a set $E$ of edges, where each edge is a set
    of two vertices.  How would you modify this definition to allow
    for directed graphs?
  \item For each of the following graph terms/concepts, say whether
    you think its definition needs to be modified for directed graphs;
    if so, say what the new definition should be.
    \begin{questions}
    \item \term{vertex}
    \item \term{degree}
    \item \term{path}
    \item \term{cycle}
    \end{questions}
    \newpage
  \item What (if anything) about our implementation of BFS needs to be
    modified for BFS to work sensibly on directed graphs?
\end{questions}

\begin{defn}
  A directed graph $G = (V,E)$ is \term{strongly connected} if for any two
  vertices $u,v \in V$ there is a (directed) path from $u$ to $v$,
  \emph{and also} from $v$ to $u$.
\end{defn}

\begin{questions}
  \item Describe a brute force algorithm for determining whether a
    given directed graph $G$ is strongly connected.
  \item Analyze the running time of your algorithm in terms of $\Theta$.
\end{questions}

\pause

\noindent Let's see if we can do better!

\begin{thm} \label{thm:strong-conn}
  A directed graph $G = (V,E)$ is strongly connected if and only if
  for any vertex $s \in V$, every other vertex in $G$
  is mutually reachable with $s$ (that is, for each $v \in V$ there is
  a directed path from $s$ to $v$ and another directed path from $v$
  to $s$).
\end{thm}

\begin{questions}
  \item In order to prove this ``if and only if'' statement, we must
    prove both \blank\linebreak and \blank.
\end{questions}
\begin{proof}\marginnote{Hint: draw a picture!}
  \mbox{} \vfill \mbox{}
\end{proof}

\newpage
\begin{defn}
  Given a directed graph $G$, its \term{reverse graph}
  $G^{\mathrm{rev}}$ is the graph with the same vertices and all edges
  reversed.
\end{defn}

\begin{thm}
  A directed graph $G = (V,E)$ is strongly connected if and only if
  given any $s \in V$, all vertices are reachable from $s$ in $G$, and
  all vertices are reachable from $s$ in $G^{\mathrm{rev}}$.
\end{thm}

\begin{proof} \marginnote{Hint: what is the relationship between this
    theorem and \pref{thm:strong-conn}?}
  \mbox{} \vspace{2in}

  \mbox{}
\end{proof}

\begin{questions}
\item Based on the above theorem, describe an algorithm to determine
  whether a given directed graph $G = (V,E)$ is strongly connected,
  and analyze its running time.
\end{questions}

%% Did this as an activity in Fall '17 but it didn't really go very
%% well.  It's really not obvious what property I'm trying to get at
%% just by looking at the examples.  A good activity would have to
%% somehow introduce the idea via a lot of examples, e.g. of
%% situations that can be modeled by a bipartite graph.

%% ALTERNATIVE IDEA: extend this activity to lead them through
%% algorithm to decide whether a directed graph is strongly connected
%% (using BFS in G + BFS in Grev).

% \newpage

% \begin{model*}{Some graphs}{bipartite}
% There is a mystery property $X$.  Each graph either has property $X$
% or not.

%   These graphs have property $X$:

%   \begin{center}
%   \begin{diagram}[width=400]
%     import Graphs
%     dia = hsep 4 . map centerXY $ -- $
%       [ replicate 2 (replicate 3 pip)
%         # zipWith zip [[0,1,2],[3,4,5 :: Int]]
%         # map (map (uncurry named))
%         # map (vsep 3) # hsep 4
%         # applyAll
%           [connect' opts i j | (i,j) <- [(0 :: Int, 4 :: Int),(1,3),(1,4),(1,5),(2,5)]]
%       , [replicate 4 pip, replicate 2 pip]
%         # zipWith zip [[0,1,2,3],[4,5 :: Int]]
%         # map (map (uncurry named))
%         # map (centerY . vsep 2) # hsep 4
%         # applyAll
%           [connect' opts i j | (i,j) <- [(1 :: Int, 5 :: Int)]]
%       , graph [(0,1),(1,2),(2,3),(3,0)] (const "")
%       , graph (zip [0..5] ([1..5] ++ [0])) (const "")
%       , graph [(0,1),(1,2),(2,3),(3,0),(2,4),(4,5),(5,3)] (const "")
%       ]
%       where
%         opts = with & arrowHead .~ noHead
%   \end{diagram}
%   \end{center}

%   \hrule \bigskip

%   These graphs do not have property $X$:

%   \begin{center}
%   \begin{diagram}[width=400]
%     import Graphs
%     dia = hsep 4 . map centerXY $ -- $
%       [ replicate 2 (replicate 3 pip)
%         # zipWith zip [[0,1,2],[3,4,5 :: Int]]
%         # map (map (uncurry named))
%         # map (vsep 3) # hsep 4
%         # applyAll
%           [connect' opts i j | (i,j) <- [(0 :: Int, 4 :: Int),(1,3),(1,4),(1,5),(2,5),(0,1)]]
%       , graph [(0,1),(1,2),(2,0)] (const "")
%       , graph (zip [0..4] ([1..4] ++ [0])) (const "")
%       , graph [(0,1),(1,2),(2,3),(3,0),(2,4),(4,3)] (const "")
%       ]
%       where
%         opts = with & arrowHead .~ noHead
%   \end{diagram}
%   \end{center}
% \end{model*}

% \begin{questions}
% \item For each graph below, say whether you think it has property $X$.
%   \begin{subquestions}
%   \item
%     \begin{diagram}[width=50]
%       import Graphs
%       dia :: Diagram B
%       dia = replicate 2 (replicate 3 pip)
%         # zipWith zip [[0,1,2],[3,4,5 :: Int]]
%         # map (map (uncurry named))
%         # map (vsep 3) # hsep 4
%         # applyAll
%           [ connect' opts i j
%           | (i,j) <- [(0 :: Int, 4 :: Int),(1,3),(1,4),(2,5),(2,4),(0,3)]
%           ]
%         where
%           opts = with & arrowHead .~ noHead
%     \end{diagram}

%   \item
%     \begin{diagram}[width=50]
%       import Graphs
%       dia :: Diagram B
%       dia = replicate 2 (replicate 3 pip)
%         # zipWith zip [[0,1,2],[3,4,5 :: Int]]
%         # map (map (uncurry named))
%         # map (vsep 3) # hsep 4
%         # applyAll
%           [ connect' opts i j
%           | (i,j) <- [(0 :: Int, 4 :: Int),(1,3),(1,4),(1,5),(2,5),(4,5)]
%           ]
%         where
%           opts = with & arrowHead .~ noHead
%     \end{diagram}

%     \item
%       \begin{diagram}[width=75]
%         import Graphs
%         dia = graph (zip [0..6] ([1..6] ++ [0])) (const "")
%       \end{diagram}

%     \item
%       \begin{diagram}[width=75]
%         import Graphs
%         dia = graph (zip [0..5] ([1..5])) (const "")
%       \end{diagram}

%     \item
%       \begin{diagram}[width=75]
%         import Graphs
%         dia = graph [(0,1),(1,2),(2,3),(3,0),(1,4),(4,5),(5,2),(5,6),(6,7),(7,2)] (const "")
%       \end{diagram}

%     \item
%       \begin{diagram}[width=50]
%       import Graphs
%       dia :: Diagram B
%       dia = replicate 2 (replicate 3 pip)
%         # zipWith zip [[0,1,2],[3,4,5 :: Int]]
%         # map (map (uncurry named))
%         # map (vsep 3) # hsep 4
%         # applyAll
%           [ connect' opts i j
%           | (i,j) <- [(1 :: Int,3 :: Int),(0,1),(3,4),(2,5),(4,2),(0,5)]
%           ]
%         where
%           opts = with & arrowHead .~ noHead
%     \end{diagram}

%     \end{subquestions}
% \item What do you think is the definition of property $X$?
% \item Make a conjecture of the form: a graph $G$ has property $X$ if
%   and only if $G$ \blank.
% \end{questions}

\end{document}
