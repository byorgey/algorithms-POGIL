% -*- compile-command: "rubber -d --unsafe 2D-dynamic-programming.tex" -*-
\documentclass{tufte-handout}

\usepackage{algo-activity}

\graphicspath{{images/}{../images/}}

\titleformat{\section}%
  [display]% shape
  {\relax\ifthenelse{\NOT\boolean{@tufte@symmetric}}{\begin{fullwidth}}{}\normalfont\Large\itshape}% format applied to label+text
  {\thesection}% label
  {1em}% horizontal separation between label and title body
  {}% before the title body
  [\ifthenelse{\NOT\boolean{@tufte@symmetric}}{\end{fullwidth}}{}]% after the title body

\title{\thecourse: Subset Sum (Dynamic Programming)}
\date{}

\begin{document}

\maketitle

\begin{model*}{Some sets}{sets}
  \begin{align*}
    A &= \{ 1, 2, 3, 5, 7 \} \\
    B &= \{ 4, 16, 19, 23, 25, 72, 103 \} \\
    C &= \{ 3, 34, 6, 17 \} \\
    D &= \{ \}
  \end{align*}
\end{model*}

\begin{questions}
\item For each number below, say whether each set has some subset
  which adds up to the given number.  For example, $A$ and $C$ have
  subsets which add up to $7$ ($\{7\}$ and $\{5,2\}$ respectively),
  but $B$ and $D$ do not.
  \begin{enumerate}[label=(\alph*)]
  \item 9
  \item 16
  \item 0
  \end{enumerate}
\end{questions}

In general, consider the following problem, called the \textsc{Subset Sum}
problem:
\begin{itemize}
\item \textbf{Input}:
  \begin{itemize}
  \item a set $\{x_1, \dots, x_n\}$ of $n$ positive
    integers,\marginnote{Yes, the first element is $x_1$, not $x_0$.
      This is a deliberate choice which will come in handy later.} and
  \item a positive integer $S$.
  \end{itemize}

\item \textbf{Output}: is there a subset of $\{x_1, \dots, x_n\}$
  whose sum is exactly $S$?
\end{itemize}

\begin{questions}
\item Describe a brute-force algorithm for solving this problem.
\item \label{q:brute} What is the running time of your brute-force
  algorithm?
\item Use your brute-force algorithm to decide whether there is any
  subset of $C$ which adds up to $54$.  What about $55$?
\end{questions}

Let's see how to attack this problem using dynamic programming.

\noindent
\textbf{Step 1: Break the problem into subproblems and make a
  recurrence.}

\begin{itemize}
\item We can make the problem simpler by restricting ourselves to only
  using \emph{some} of the $x_i$.  For example, a subproblem might
  look like ``Can we find a subset of only $\{x_1, \dots, x_k\}$ that
  adds up to $S$?'' for some $k \leq n$.
\item However, by itself this doesn't help: just knowing whether we
  can add up to $S$ using only $x_1, \dots, x_k$ doesn't tell us
  whether we can add up to $S$ using $x_1, \dots, x_n$.  In
  particular, in order to add up to $S$ we might need to use some of
  the elements from $x_1, \dots, x_k$ in addition to some of the other
  elements.  We can fix this by generalizing along another dimension
  as well: we need to know whether we can add up not just to $S$
  itself, but to \emph{any} sum $0 \leq s \leq S$.  That is, a
  subproblem now looks like ``Can we find a subset of only $\{x_1,
  \dots, x_k\}$ that adds up to $s$?'' for some $k \leq n$ and $s \leq S$.
\end{itemize}

\newcommand{\canAddTo}{\mathit{canAddTo}}

Define $\canAddTo(k,s)$ to be a true or false value which is the
answer to the question, ``Is there a subset of only the first $k$
elements $\{x_1, \dots, x_k\}$ which adds up to exactly $s$?''

\begin{questions}
\item Consider set $A = \{1,2,3,5,7\}$ from Model 1 again.  Number the
  elements starting from $1$, that is, $x_1 = 1$, $x_2 = 2$, \dots,
  $x_5 = 7$.  Evaluate each expression below as true or false,
  and give a brief justification for each.
  \begin{subquestions}
  \item $\canAddTo(5, 16)$
  \item $\canAddTo(4, 16)$
  \item $\canAddTo(4, 9)$
  \item $\canAddTo(3, 1)$
  \item $\canAddTo(3, 0)$
  \item $\canAddTo(0, 2)$
  \item $\canAddTo(0, 0)$
  \end{subquestions}
\end{questions}
Now let's come up with a recurrence for $\canAddTo$ in the general
case of determining whether there is a subset of $X = \{x_1, \dots,
x_n\}$ which adds up to $S$.
\begin{questions}
  \item Fill in base cases for $\canAddTo$:
    \begin{itemize}
    \item $\canAddTo(k,0) = $ \blank for all $k \geq 0$, \\ because we
      can always
      \blank.
    \item $\canAddTo(0,s) = $ \blank for all $s > 0$, \\ because
      there's no way to \blank.
    \end{itemize}
  \item Now consider $\canAddTo(k,s)$ in the general case, when
    $k > 0$ and $s > 0$.  That is, we are trying to find whether we
    can add up to exactly $s$ if we're only allowed to use
    $x_1, \dots, x_k$.  In order to break this problem down into
    subproblems, we would need to decrease $k$ and/or $s$.  Fill in
    the following steps.
    \begin{itemize}
    \item If \blank, then we definitely cannot use $x_k$ as part of a \\
      subset adding to $s$, because it is too \blank.  \\ In this
      case, we would get the same result if we only allowed ourselves
      to use $\{x_1, \dots, x_{k-1}\}$, that is, $\canAddTo(k,s)$ is the
      same as \blank.
    \item Otherwise, we have two choices: we can try to use $x_k$ as
      part of our subset or not.  If we don't use it, it is the same
      as the previous case.  If we do use it, then in order to
      complete a subset adding to $s$ we have to make a subset using
      only \blank \\ which adds up to \blank.
    \end{itemize}
  \item Use your reasoning above to write down a complete recursive
    definition of $\canAddTo$.  Don't forget the base cases!
\end{questions}

\pause
\noindent
\textbf{Step 2: Memoize.}

\begin{questions}
  \item Explain why it would be extremely slow to directly evaluate
    $\canAddTo(n,S)$ as a recursive function.
  \item $\canAddTo$ takes a \emph{pair} of values as input:
    $0 \leq k \leq n$ and $0 \leq s \leq S$.  How many possible such
    pairs are there?\marginnote{Of course there are infinitely many
      pairs of numbers; this question is really asking about how many
      different inputs to recursive calls we might possibly see after
      calling $\canAddTo(n,S)$.}
  \item If we wanted to memoize the results of $\canAddTo$ by storing
    the output corresponding to each possible input, what data
    structure should we use?  Draw a picture.
  \item How big is this data structure?
  \item In what order can we fill in the data structure, so that we
    never try to fill in a value before filling in other values it
    depends on?
  \item How long does it take to fill in each value?
  \item Therefore, what is the running time of this dynamic
    programming algorithm?
  \item Is this faster than your answer to
    \pref{q:brute}?\marginnote{Hint: this is a trick question.}
    \newpage
  \item Write some code (using either pseudocode or a language of your
    choice) to compute $\canAddTo(n,S)$ using the approach outlined
    here.
\end{questions}

\end{document}
