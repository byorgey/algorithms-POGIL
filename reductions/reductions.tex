% -*- compile-command: "./build.sh" -*-
\documentclass{tufte-handout}

\usepackage{algo-activity}

\title{\thecourse: Introduction to reductions}
\date{}

\begin{document}

\maketitle

% \begin{objective}
%   Students will XXX.
% \end{objective}

\begin{model*}{Independent sets}{IS}
  \begin{defn}
    An \emph{independent set} in an undirected graph $G = (V,E)$ is a
    subset of vertices $S \subseteq V$ such that no two vertices in
    $S$ are adjacent.
  \end{defn}
  \begin{defn}
    A \emph{vertex cover} in an undirected graph $G = (V,E)$ is a
    subset of vertices $C \subseteq V$ such that every edge $e \in E$
    has at least one endpoint in (is ``covered by'') $C$.
  \end{defn}
  \begin{center}
  \begin{diagram}[width=300]
    import ReductionGraphs

    dia = hsep 2
      [ drawGraph (tex.show) exampleGraph1
      , drawGraph (tex.(:[])) exampleGraph2
      ]
  \end{diagram}
  \end{center}
\end{model*}

\begin{questions}
\item \label{q:whichIS} Which of the following are independent sets?
  \begin{enumerate}[label=(\alph*)]
  \item $\{1,2\}$
  \item $\{1,5\}$
  \item $\{c,a\}$
  \item $\{e,a,i,g\}$
  \item $\{7\}$
  \item $\varnothing$
  \end{enumerate}

\item For each graph, list at least three other examples of
  independent sets.

\item Given an arbitrary graph $G$, does $G$ always have at least one
  independent set?  Why or why not?

\item Intuitively, which is harder: to find big independent sets, or
  small ones?  Why?

\item Based on the previous observation, an interesting question to
  ask about a given graph $G$ is to find the \blank.

\item Try to answer your interesting question for the given example
  graphs (but don't spend more than a few minutes).  How sure are you
  about your answer?

\item Describe a brute-force algorithm to answer this question.  What
  is its big-$\Theta$ running time in terms of $|V|$ and $|E|$?

\item Guess the running time (in terms of $|V|$ and $|E|$) of the
  fastest known algorithm to solve this problem. (You do not have to
  come up with an algorithm; just guess how fast you think this
  problem can be solved.)

\item \label{q:whichVC} Which of the following are vertex covers?
  \begin{enumerate}[label=(\alph*)]
  \item $\{3,4,5,6,7\}$
  \item $\{2,3,4,6,7\}$
  \item $\{b,d,e,f,g,h,i,j\}$
  \item $\{b,c,d,f,h,j\}$
  \item $\{1,2,3,4,5,6\}$
  \item $\{1,2,3,4,5,6,7\}$
  \end{enumerate}

\item For each graph, list at least three other examples of
  vertex covers.

\item Given an arbitrary graph $G$, does $G$ always have at least one
  vertex cover?  Why or why not?

\item Intuitively, which is harder: to find small vertex covers, or
  big ones?  Why?

\item Based on the previous observation, an interesting question to
  ask about a given graph $G$ is to find the \blank.

\item Answer your interesting question for the given example graphs.
  How sure are you about your answer?

\item Describe a brute-force algorithm to answer this question.  What
  is its big-$\Theta$ running time in terms of $|V|$ and $|E|$?

\item Compare your answers to questions \ref{q:whichIS} and
  \ref{q:whichVC}.  What do you notice?
\end{questions}

Make a conjecture based on your observations in the previous question,
and prove it:

\begin{thm}
  Let $G = (V,E)$ be an undirected graph, and $S \subseteq V$ a subset
  of its vertices.  Then $S$ is an independent set if and only if \blank.
\end{thm}

\begin{proof}
  ($\Longrightarrow$) Let $S$ be an independent set.  We must show \\
  \blank. So pick an arbitrary edge $e = (u,v) \in E$; \\ by
  definition we must show that at least one of $u$ or $v$ \blank, \\
  that is, at least one of $u$ or $v$ is not \blank. \\
  Since $S$ is an independent set and $u$ and $v$ are connected by an \\
  edge, $u$ and $v$ can't both \blank, \\
  and therefore \blank. \medskip

  ($\Longleftarrow$) (You fill in the proof for this direction!)\marginnote{Write down
    what you get to assume and what you are trying to prove, and
    expand definitions.}
\end{proof}

\end{document}
