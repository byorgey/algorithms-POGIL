% -*- compile-command: "rubber -d why-POGIL.tex" -*-

\documentclass{tufte-handout}

\usepackage{algo-activity}

\title{\thecourse: Why POGIL?}
\date{}

\begin{document}

\maketitle

\section{Process Skills and Guided Inquiry}

Although we will not use it every time, many of our class meetings
will use an approach to learning called POGIL, which stands for
Process-Oriented Guided Inquiry Learning.  What does that mean?  Let's
start with ``Guided Inquiry Learning''.

\begin{objective}
  Students will explain the benefits of a guided-inquiry approach to
  learning.
\end{objective}

\begin{questions}
\item Get the role cards from your instructor.  Take the role card you
  had last time, and hand it to your right.  Everyone should now have
  a different role.  Take a minute to review your new role.

\item (Review) Take \textbf{60 seconds} to brainstorm a list of as
  many details as you can remember from the class syllabus and
  academic integrity policy (don't peek!).  \vspace{1in}

\item How much were you able to remember?  What if the instructor had
  gone over the syllabus instead of having you answer questions about
  it as a team---do you think you would you have remembered more,
  less, or about the same? \vspace{0.5in}

\item This type of format, where you work together to answer questions
  that guide you through a learning process, is known as \emph{guided
    inquiry}.  What do you think are some of the benefits of guided
  inquiry (as compared to listening to a lecture)?  \marginnote{Don't
    just put what you think I want to hear---what do you honestly
    think are the benefits?  If you don't think there are any
    benefits, go ahead and say that!}

\newpage

\item (3 minutes) Imagine that your team is tasked with building a
\begin{objective}
  Students will identify process skills and explain how they are
  developed in a POGIL classroom.
\end{objective}
  large, complex piece of software.  There is no way you can do it by
  yourselves in any reasonable amount of time.  Fortunately, you have
  a large budget, so you are going to hire people to help
  you. Brainstorm a list of qualities and skills you would look for in
  people that you hire.

\item Most likely, some of the skills you identified were
  content-based and specific to computer science, whereas other skills
  are more general ``soft'' skills that have to do with the
  \emph{process} of solving problems, being productive, and working
  well with other people.  We call these \emph{process skills}.  In
  what ways, if any, do you think working on guided inquiry activities
  in a learning team with assigned roles will help you develop these
  skills?  Which process skills will it help you develop?

\end{questions}

\end{document}
