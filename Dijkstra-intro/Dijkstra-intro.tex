% -*- compile-command: "./build.sh" -*-
\documentclass{tufte-handout}

\usepackage{algo-activity}

\title{\thecourse: Introduction to Dijkstra's Algorithm}
\date{}

\begin{document}

\maketitle

\begin{model}{Some graphs}{graphs}
  \begin{center}
    \begin{diagram}[width=300]
    import Graphs

    g1 :: Graph
    g1 = Graph
      (zip (map (:[]) "sabtc") [ (0,4), (0,1), (2,0), (4,1), (3,4) ])
      [ ("s", "a", Just 3), ("s", "c", Just 9), ("a", "b", Just 2)
      , ("b", "t", Just 4), ("c", "t", Just 2)
      ]

    g2 :: Graph
    g2 = Graph
      (zip (map (:[]) "s234567t") [(0,0), (3,3), (15,3), (13,0), (9,-1), (6,0), (3, -3), (15,-3)])
      [ ("s", "2", Just 9), ("s", "7", Just 15), ("s", "6", Just 14)
      , ("6", "7", Just 5), ("2", "3", Just 24), ("6", "3", Just 18)
      , ("6", "5", Just 30), ("5", "3", Just 2), ("5", "4", Just 11)
      , ("4", "3", Just 6), ("4", "t", Just 6), ("3", "t", Just 19)
      , ("7", "t", Just 44), ("7", "5", Just 20), ("5", "t", Just 16)
      ]

    dia :: Diagram B
    dia = vsep 5
      [ hsep 2 [drawG (unweight g1), drawG g1] # center # fontSizeL 0.6
      , drawG g2 # center # fontSizeL 0.5
      ]
  \end{diagram}

  \end{center}
\end{model}

(Review) Begin by considering Graph 1 (top left).
\begin{questions}
\item What is the shortest path from $s$ to $t$?  How long is it?
\item \label{q:bfs} What algorithm would you use to find it?
\end{questions}

Now consider Graph 2 (top right). It looks just like Graph 1 except that each edge
is labelled with a \emph{weight}.
\begin{questions}
  \item What do you think the ``length'' of a path means in this
    graph?
  \item According to your definition of length, what is the shortest
    path from $s$ to $t$?  How long is it?
  \item Why wouldn't your answer to \pref{q:bfs} work here?
\end{questions}

Now consider Graph 3.  Imagine that each edge is a pipe that only
allows water through in the direction the arrow is pointing.  The
number on the edge indicates how many seconds it takes for water to
flow from one end of the pipe to the other.  Now imagine that we hook
up an (infinite) source of water to vertex $s$ and watch the water
start flowing through the network of pipes.  Of course, water always
flows in all possible directions.  For example, as soon as we hook up
the water source to vertex $s$, water immediately begins flowing
along all three pipes leaving from $s$.
\begin{questions}
  \item What is the first vertex (besides $s$) the water reaches?  At
    what time does this happen (counting in seconds from the moment we
    hook up the water to vertex $s$)?
  \item What is the second vertex the water reaches?  At what time
    does it happen?
  \item What is the third vertex the water reaches?  At what time?
  \item Suppose we changed the length of the pipe $s \to 7$ from $15$
    to $20$.  How (if at all) would this change your answers to the
    previous questions?
  \item Draw the situation after $32$ seconds.  (You may wish to draw
    directly on the graph in the model; or you can make a separate
    copy.)  Which vertices has the water reached?  Which pipes are
    full?
  \item After $32$ seconds, which new vertex will the water reach next?
  \item How does thinking about water flooding the graph help us solve
    the problem of finding shortest paths?  Make a conjecture relating water in
    the graph to shortest paths between $s$ and other vertices.
\end{questions}

\pause

\begin{questions}
\item What similarities or differences do you see between BFS and
  Dijkstra's algorithm?
\item If you had only a black box that could run Dijkstra's algorithm,
  and someone gave you an \emph{unweighted}, directed graph and asked
  for the shortest path between two vertices, what would you do?
\item How about vice versa?  That is, imagine you have only a black
  box that can run BFS, and someone gives you a directed graph with
  positive integer weights on the edges, and asks for the shortest
  path between two vertices.  What should you do?
\item Does Dijkstra's algorithm work if there are edges with negative
  weight?  Explain why it works, or draw an example graph to illustrate
  why it doesn't.
\end{questions}

\end{document}
