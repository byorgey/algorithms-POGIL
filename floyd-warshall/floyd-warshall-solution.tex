% -*- compile-command: "pdflatex --enable-write18 XXX.tex" -*-
\documentclass{tufte-handout}

\usepackage{algo-activity}
\usepackage{xcolor}

\title{Algorithms: Floyd-Warshall}
\date{}

\begin{document}

\maketitle

\begin{questions}

\item Consider the \emph{adjacency matrix} in Model \ref{adj_matrix}. (In an adjacency matrix for a directed graph, each entry is the weight of an edge in the graph.) Nodes on the left are originating vertices. Nodes on the top are destination vertices.

\begin{objective}
  Students will derive an efficient algorithm for finding the shortest paths between all pairs of vertices in a directed, weighted graph.
\end{objective}

Examining the graph, list all pairs of vertices for which  there is a path through an intermediate vertex that is shorter than tracing the edge from the first to the second vertex. For example, it is shorter to travel from $v_1$ to $v_2$ through intermediate vertex $v_4$ (total length: 2 ($v1$ to $v_4$) + 1 ($v_4$ to $v_2$) = 3) than directly from $v_1$ to $v_2$ (edge weight: 6). 

Note that the graph is not symmetric; the shortest path from $v_1$ to $v_2$ need not be the same as the shortest path from $v_2$ to $v_1$. \label{inter_1}

Be sure to list the pair, the intermediate vertex, and the total length of the path with the intermediate vertex. You should find a total of five pairs.

{\color{red}
 $v_1$, $v_2$: $v_1$, $v_4$, $v_2$: 3
 $v_3$, $v_1$: $v_3$, $v_2$, $v_1$: 6
 $v_4$, $v_1$: $v_4$, $v_2$, $v_1$: 4
 $v_2$, $v_4$: $v_2$, $v_1$, $v_4$: 5
 $v_4$, $v_3$: $v_4$, $v_2$, $v_3$: 4
}

\item Find an instance where taking a path that goes through a second intermediate vertex is shorter than both the path going through just one intermediate vertex as well as the edge between the starting and ending vertices. There is only one such instance. \label{inter_2}

{\color{red}
 $v_1$, $v_3$: $v_1$, $v_4$, $v_2$, $v_3$: 6
}

%The table below lists each pair of vertices and the weight of the edge from the first vertex to the second. It then lists the possible intermediate vertices. Find the distance for each path. Then write the minimum distance in the final column. The first row is completed for you as an example.\label{by_hand}
%\\
%{\color{red} 
%\begin{tabular}{|c|c|c|c|c|c|c|c|c|}
%\hline
%Pair & $w_{v_i v_j}$ & $x_1$ & $x_2$ & $v_i$ $x_1$ $v_j$ & %$v_i$ $x_2$ $v_j$ &$v_i$ $x_1$ $x_2$ $v_j$ &$v_i$ $x_2$ %$x_1$ $v_j$ & Minimum\\
%\hline
%$v_1$ $v_2$ & 8 & $v_3$ & $v_4$ & 10 & 3 & 10 & 10 & 3 \\
%$v_2$ $v_1$ & 3 & $v_3$ & $v_4$ & 11 & 12 & 11 & 19 & 3 \\
%$v_1$ $v_3$ & 7 & $v_2$ & $v_4$ & 11 & 7 & 19 & 6 & 6 \\ 
%$v_3$ $v_1$ & 8 & $v_2$ & $v_4$ & 6 & 8 & 15 & 6 & 6 \\ 
%$v_1$ $v_4$ & 2 & $v_2$ & $v_3$ & 14 & 9 & 13 & 16 & 2\\ 
%$v_4$ $v_1$ & 6 & $v_2$ & $v_3$ & 4 & 13 & 12 & 11 & 4\\ 
%$v_2$ $v_3$ & 3 & $v_1$ & $v_4$ & 10 & 11 & 10 & 19 & 3\\ 
%$v_3$ $v_2$ & 3 & $v_1$ & $v_4$ & 16 & 3 & 11 & 16 & 3\\ 
%$v_2$ $v_4$ & 6 & $v_1$ & $v_3$ & 5 & 5 & 12 & 13 & 5 \\ 
%$v_4$ $v_2$ & 1 & $v_1$ & $v_3$ & 14 & 8 & 16 & 21 & 1\\
%$v_3$ $v_4$ & 2 & $v_1$ & $v_2$ & 10 & 9 & 22 & 8 & 2\\
%$v_4$ $v_3$ & 5 & $v_1$ & $v_2$ & 13 & 4 & 17 & 11 & 4\\
%%\end{tabular}
%}

\begin{model}{Adjacency Matrix}{adj_matrix}
\begin{tabular}{ l r r r r }
      & $v_1$ & $v_2$ & $v_3$ & $v_4$ \\
$v_1$ &       & 8     & 7     & 2 \\
$v_2$ & 3     &       & 3     & 6 \\
$v_3$ & 8     & 3     &       & 2 \\
$v_4$ & 6     & 1     & 5     &   \\
\end{tabular}
\label{adj_matrix}
\end{model}

\item Inspired by your answers to Questions \ref{inter_1} and \ref{inter_2}, fill in the base case and recursive cases of the algorithm below to find the shortest path distance between any pairs of vertices for a graph with $n$ vertices. 

The parameter \verb|g| refers to the graph being explored, and \verb|g.edge_weight(i, j)| returns the weight of the edge in graph $g$ from $v_i$ to $v_j$. The \verb|start| and \verb|end| parameters represent the indices of the vertices between which we seek the shortest path. The $k$ parameter represents the maximum vertex index under consideration as an intermediate node. When $k = 0$, no intermediate nodes will be considered; only the edge weight from \verb|start| to \verb|end| can be used. When $k=1$, $v_1$ can be considered. When $k=2$, $v_1$ and $v_2$ can be considered, and so forth. \label{f-w-1}

{\color{red}
\begin{verbatim}
shortest_path_distance(g, start, end, k)
  if k == 0
    return g.edge_weight(start, end)
  else 
    let distance_ignoring_vertex_k = 
       shortest_path_distance(g, start, end, k - 1)
    let distance_using_vertex_k = 
       shortest_path_distance(g, start, k, k - 1) 
       + shortest_path_distance(g, k, end, k - 1)
    if start == k or end == k or 
       distance_ignoring_vertex_k < distance_using_vertex_k
      return distance_ignoring_vertex_k
    else 
      return distance_using_vertex_k
\end{verbatim}
}

\item List every distinct base-case recursive call encountered when calling the algorithm on Model \ref{adj_matrix} with the call \verb|shortest_path_distance(model_1, 1, 4, 4)|.

{\color{red}
1, 4, 0
1, 3, 0
1, 2, 0
2, 4, 0
2, 3, 0
2, 1, 0
3, 4, 0
3, 2, 0
3, 1, 0
4, 3, 0
4, 2, 0
4, 1, 0
}

\item To prove the algorithm from Question \ref{f-w-1} correctly finds the shortest path from \verb|start| to \verb|end|, we need to use \emph{strong induction} due to its multiple distinct recursive calls.  What would all the base cases be for a correctness proof by strong induction?  {\color{red} For every pair of vertices, we return the weight of the edge between them.}

\item How many base cases would there be?  {\color{red} There will be $n^2$ base cases, as we need a base case for every possible edge.}

\item What would be the inductive hypothesis? {\color{red} 
For any pair of vertices \verb|start| and \verb|end|, and any $i \le k - 1$, \verb|shortest_path_distance(g, start, end, k - 1)| returns the distance of the shortest path from \verb|start| to \verb|end|.}

\item What would we need to prove in the inductive step?

{\color{red} 
For any pair of vertices \verb|start| and \verb|end|, \verb|shortest_path_distance(g, start, end, k)| returns the distance of the shortest path from \verb|start| to \verb|end|.}

\item Complete the proof by strong induction that this algorithm finds the shortest path from \verb|start| to \verb|end|.

{\color{red} We know \verb|shortest_path_distance(g, start, end, k - 1)| is the shortest path if we exclude vertex $k$ by the inductive hypothesis. If we include vertex $k$, the shortest path distance is the sum of distances between \verb|start| and $k$ and $k$ and \verb|end|, as those recursively generated distances are also the shortest by the inductive hypothesis. Therefore, whichever of the two is lower is the shortest distance, which is what we return.
}

\item Write a recurrence for the asymptotic time complexity of the algorithm you wrote in Question \ref{f-w-1}. \label{recurrence}

{\color{red} \[T(k) = 3T(k-1) + O(1) \]}

\item Estimate the asymptotic time complexity of your algorithm based on the recurrence from Question \ref{recurrence}.

{\color{red} $O(3^n)$}

\item If an edge is missing between two vertices, how might the algorithm (or adjacency matrix) from Question \ref{f-w-1} be modified to handle this?

{\color{red} Give it an infinite weight. It will then always be safely ignored by the min() function.}

\item If we were to rewrite the algorithm from Question \ref{f-w-1} as a bottom-up dynamic programming algorithm, how many dimensions would the table need to have? What would each dimension represent?

{\color{red} It would need two dimensions. The first dimension is the originating vertex and the second dimension is the destination vertex. Each entry is the shortest known distance between them.}

\item Write pseudocode for a bottom-up dynamic programming version of the algorithm. (This is known as the Floyd-Warshall algorithm.) \label{f-w-2}

{\color{red}
\begin{verbatim}
floyd_warshall(g)
  for i in g.vertices
    for j in g.vertices
      table[i][j] = g.edge(i, j)
  for k in g.vertices
    alternative = table[i][k] + table[k][j]
    if alternative < table[i][j]
      table[i][j] = alternative
  return table
\end{verbatim}
}

\item What is the asymptotic time complexity of the algorithm from Question \ref{f-w-2}?

{\color{red} $O(n^3)$}

\item How might we augment the dynamic programming table so that we can reconstruct the path between two vertices?

{\color{red} For the entry for $(i, j)$, have an additional field indicating the next vertex to visit as part of the shortest path to get there. Initialize this to $j$ at the start. Update it to $k$ whenever we update \verb|table[i][j]|.}

\item Under what circumstances is Dijkstra's algorithm preferable to the Floyd-Warshall algorithm? Why?

{\color{red} If you only care about one pair of vertices, Dijkstra's is a lot more efficient.}

\item Under what circumstances is the Floyd-Warshall algorithm preferable to Dijkstra's algorithm? Why?

{\color{red} If you want lots of shortest-paths in a dense graph where $E$ is close to $V^2$, it is more efficient to do $O(V^3)$ work once rather than $O(V^2 log V) * V = O(V^3 log V)$.}
\end{questions}

\end{document}
