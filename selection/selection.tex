% -*- compile-command: "pdflatex --enable-write18 XXX.tex" -*-
\documentclass{tufte-handout}

\usepackage{algo-activity}

\title{Algorithms: Selection}
\date{}

\begin{document}

\maketitle

\begin{figure}
\caption{The Master Theorem}
Let $T(n) = a T(\frac{n}{b}) + O(n^d)$ 

Case 1: If $a < b$, $T(n) \in O(n^d)$

Case 2: If $a = b$, $T(n) \in O(n^d \log n)$

Case 3: If $a > b$, $T(n) \in O(n^{\log_b a})$
\end{figure}

\begin{objective}
  Students will derive an efficient algorithm for finding the $k$th largest item in an array. Students will practice using the Master Theorem to solve practical divide-and-conquer recurrences.
\end{objective}


\begin{questions}
\item We can find the median of a sorted array of $n$ elements in $O(1)$ time by accessing the $\lfloor \frac{n}{2} \rfloor$ element. What is a straightforward algorithm for  finding the median of an unsorted array of $n$ elements? What is its time complexity? \label{start}

\item If, instead of finding the median, we wish to find the $k$th smallest item in the array, how might our answer to Question \ref{start} change?

\item Consider the arrays $a$ and $b$ below:

\begin{verbatim}
a = [7, 1, 0, 2, 6, 5, 4, 3]
b = [5, 4, 2, 1, 7, 3, 6, 0]
\end{verbatim}

Rearrange $a$ and $b$ according to the following algorithm, called \emph{partition}:
\begin{itemize}
    \item Let $i = 0$
    \item Let $j = 6$ (i.e., the second-to-last element index)
    \item While $i \le j$
    \begin{itemize}
        \item if element $i$ > element $7$, swap elements $i$ and $j$, and decrement $j$.
        \item Otherwise, increment $i$.
    \end{itemize}
    \item Swap element $i$ and the final element in the array.
\end{itemize}

\item After rearranging the array, what elements of the array are in the position they would be in after fully sorting the array?

\item Formulate a statement about what array element will be in its correct sorted position after running \emph{partition}. Give an informal argument as to why this will always be true after  \emph{partition} runs.

\item Consider the following pseudocode for the \emph{partition} function:
\begin{verbatim}
partition(items, start, end)
  i = start
  j = end - 1
  while i <= j
    if items[i] > items[end]
      swap(items, i, j)
      j -= 1
    else
      i += 1
  swap(items, i, end)
  return i
\end{verbatim}

How can we build a simple divide-and-conquer sorting algorithm using \emph{partition}? 

\item Write a recurrence relation for the number of key comparisons performed by this algorithm, assuming that \emph{partition} always returns $\lfloor \frac{start + end}{2} \rfloor$ as the index. Solve the recurrence relation by using the Master Theorem. 

\item Finding the $k$th smallest item in the array need not require sorting the entire array. How can we use \emph{partition} to build a recursive algorithm that correctly finds the $k$th smallest item without completely sorting the array? \label{select_question}

\item Give an informal argument for the correctness of your algorithm.

\item Write a recurrence relation for the number of key comparisons performed by this algorithm, assuming that partition always returns $\lfloor \frac{start + end}{2} \rfloor$ as the index. Solve the recurrence relation by using the Master Theorem. \label{select_recurrence}

\item How much of an improvement in asymptotic time complexity do we gain by avoiding a complete sort of the array?

\item What is the worst-case behavior of this algorithm? When does this behavior emerge?

\item Imagine adding some code to the start of \emph{partition} that swaps the last element with a randomly chosen element. On average, how would you expect this to affect the fraction of each part of the array? \label{unbalanced}

\item Rewrite the recurrence for Question \ref{select_recurrence} to take your answer to Question \ref{unbalanced} into account, and solve it using the Master Theorem.

\item What can we conclude about the impact on the expected time complexity of selection of randomly selecting the partition?


\end{questions}

\end{document}
