% -*- compile-command: "./build.sh" -*-
\documentclass{tufte-handout}

\usepackage{algo-activity}

\title{\thecourse: Linear-time Sorting}
\date{}

\begin{document}

\maketitle

\vspace{0.1in}
\begin{objective}
  Students will prove an $\Omega(n \lg n)$ lower bound on the runtime
  of a comparison-based sort.
\end{objective}

\begin{model*}{Decision trees}{decisions}
  \begin{center}
  \begin{diagram}[width=300]
    import Diagrams.TwoD.Layout.Tree
    import Diagrams.Prelude hiding (Empty)

    drawT = maybe mempty (renderTree renderNode renderEdge)
      . symmLayoutBin' (with & slVSep .~ 4 & slHSep .~ 8)

    renderNode t = mconcat
      [ text t
      , rect 5 1.5 # lw none # fc white
      ]

    renderEdge p q = mconcat
      [ p ~~ q
      , (if (p ^. _x > q ^. _x) then text "Y" else text "N")
        # fontSizeL 0.75
        # moveTo (lerp 0.5 p q .+^ ((1.2 * signum (p^._x - q^._x)) *^ unit_X))
      ]

    t1 =
      BNode "Is $a < b$?"
        (BNode "Is $b < c$?"
          (leaf "$[a,b,c]$")
          (BNode "Is $a < c$?"
            (leaf "$[a,c,b]$")
            (leaf "$[c,a,b]$")
          )
        )
        (BNode "Is $a < c$?"
          (leaf "$[b,a,c]$")
          (BNode "Is $b < c$?"
            (leaf "$[b,c,a]$")
            (leaf "$[c,b,a]$")
          )
        )

    dia = drawT t1
  \end{diagram}
  \bigskip
  \end{center}

  Above is shown a \emph{decision tree}.  To use the tree, start at
  the root and answer the yes-or-no question at each node you
  encounter, then take the appropriate path depending on the answer,
  either to the left (Yes) or right (No).  Repeat until reaching a leaf.
\end{model*}

\begin{questions}
\item Use the decision tree shown above with $a = 2$, $b = 9$,
  $c = 1$.  What is the output?
\item Now try with $a = 7$, $b = 4$, $c = 12$.
\item In general, what does this decision tree accomplish?
\item \label{q:height} In the worst case, how many questions will you
  have to answer when using this decision tree?  How can you tell?
\item How many leaves does this decision tree have?
\item Why does it need that many leaves?
\item What is the highest number of leaves it could have without changing
  your answer to \pref{q:height}?
\item Suppose there is some other decision tree such that when using
  it you have to answer a maximum of seven questions.  What is the
  maximum number of leaves it could have?\marginnote{Note this could
    be \emph{any} decision tree, not necessarily one that sorts.}
\end{questions}

The tree from the model---call it $T_3$---concerns a set of three
numbers $a$, $b$, and $c$. Let's consider making a similar decision
tree $T_4$ for a set of four numbers $a$, $b$, $c$, and $d$.  (You do
\textbf{not} have to actually draw it, unless you are in the mood for
some tedium!)

\begin{questions}
  \item How many leaves will $T_4$ need? Why?
  \item Suppose we try to make $T_4$ as balanced as possible.  Given
    such a balanced $T_4$, how many questions might we need to answer,
    in the worst case, to use it?  How do you know?
  \item Now consider making a decision tree $T_n$, which takes $n$
    numbers and asks questions until it is able to put them in the
    correct order.  How many leaves does $T_n$ need?
  \item Assuming $T_n$ is as balanced as possible, what is the
    worst-case number of questions one would need to answer when using
    it? \label{q:tn}
  \end{questions}
\pause
\begin{model*}{Lower bound proof}{lbproof}
  \begin{sproof}
    \stmt{\log (n!)}
    \reason{=}{(a)}
    \stmt{\log (n \cdot (n-1) \cdot (n-2) \cdot \dots \cdot 3 \cdot 2 \cdot 1)}
    \reason{=}{(b)}
    \stmt{\log n + \log (n-1) + \log (n-2) + \dots + \log (n/2) + \log
      (n/2 - 1) + \dots + \log 3 + \log 2 + \log 1}
    \reason{\geq}{(c)}
    \stmt{\log n + \log (n-1) + \log (n-2) + \dots + \log (n/2)}
    \reason{\geq}{(d)}
    \stmt{\log (n/2) + \log (n/2) + \log (n/2) + \dots + \log (n/2)}
    \reason{\geq}{(e)}
    \stmt{\frac{n}{2} \log (n/2)}
  \end{sproof}
\end{model*}
\begin{questions}
\item Fill in a reason for each step in the proof above.
  \begin{subquestions}
  \item \mbox{}
  \item \mbox{}
  \item \mbox{}
  \item \mbox{}
  \item \mbox{}
  \end{subquestions}
\item Explain why this shows that $\log(n!)$ is $\Omega(n \log n)$.
\newpage
\item Explain what this has to do with decision trees.\marginnote{\emph{Hint}: look at
  your answer to Question~\ref{q:tn}.}
\item Explain (not prove) why this implies the following theorem:
\end{questions}
\begin{thm} \label{thm:compsort}
  Any sorting algorithm which works by only performing comparisons
  between elements takes $\Omega(n \log n)$ time on a list of length
  $n$.
\end{thm}
\pause

Suppose you are given a list of $n$ records, where each record
\begin{objective}
  Students will describe and analyze bucket sort.
\end{objective}
contains information about a single person. (Assume $n$ could be very
large, say, $100$ million.)  You are tasked with sorting the records
by birthday.  That is, everyone whose birthday is January 1
(regardless of year) should come first, then everyone with a birthday
of January 2, and so on.

\begin{questions}
\item Describe the most efficient algorithm possible to accomplish
  this task. \vspace{1in}
\item What is the running time of your algorithm in terms of $n$?
\item Why doesn't this contradict \pref{thm:compsort}?
\end{questions}

Suppose we introduce the additional requirement that the birthday
sorting is \emph{stable}, that is, whenever $a$ and $b$ have the same
birthday, if $a$ comes before $b$ in the original input list, then $a$
still comes before $b$ in the sorted output list.  Put another way,
for any particular birthday, all the people with that birthday have to
stay in the same order that they started in.

  \begin{questions}
  \item Is the sorting algorithm you described above stable?  If so,
    explain why; if not, explain how to modify it to make it so.
  \end{questions}
\end{document}
