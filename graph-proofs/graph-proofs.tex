% -*- compile-command: "rubber -d graph-proofs.tex" -*-
\documentclass{tufte-handout}

\usepackage{algo-activity}

\title{\thecourse: Some Proofs about Trees}
\date{}

\begin{document}

\maketitle

\begin{objective}
  Students will write proofs about graphs.
\end{objective}

\begin{model*}{A theorem about trees}{trees}
\begin{thm}[Trees]
  Let $G = (V,E)$ be a graph with $|V| = n \geq 1$.  Any two of the following
  imply the third:
  \begin{enumerate}
  \item $G$ is connected.
  \item $G$ is acyclic.
  \item $G$ has $n-1$ edges.
  \end{enumerate}
\end{thm}
\end{model*}

\begin{questions}
  \item (Review) From the previous activity, what is the definition of
    a tree?
  \item How does the given theorem relate to the definition of a tree?
  \item The theorem implies that there are two other alternative, yet
    equivalent, ways we could have defined trees.  What are they?
\end{questions}

We will take each pair of statements in turn and show that they imply
the third.  Fill in the blanks to complete the following proofs!  Note
that the size of a blank does not necessarily correspond to the
amount of stuff you should write in it.

\begin{lem} \label{lem:onetwothree}
  $(1), (2) \implies (3)$.  That is: let $G = (V,E)$ be a graph with $|V|
  = n \geq 1$.  If \blank\linebreak and \blank,\linebreak then \blank.
\end{lem}

\begin{proof}
  Let $P(n)$ denote the statement ``Any graph $G$ with $n$ vertices
  which is \blank and \blank\linebreak
  must have \blank.''\linebreak  We wish to show
  that $P(n)$ holds for all $n \geq 1$.

  The proof is by \blank.
  \begin{itemize}
  \item The base case is when \blank.\linebreak In this case, $G$ must
    be \blank\linebreak which indeed \blank.
  \item For the induction step, suppose $P(k)$ holds for some $k \geq
    1$.  That is, suppose that any graph with \blank vertices\linebreak which is
    \blank\linebreak must have \blank.\linebreak  Then we wish to show
    $P(k+1)$, that is, any graph with \blank\linebreak vertices which is
    connected and acyclic must have \blank.

    So, let $G$ be a graph with \blank vertices which is\linebreak
    \mbox{}\blank and \blank.\linebreak  We claim that $G$
    must have some vertex which is a leaf, that is, a vertex of degree
    \blank,\newline which we can show as follows:
    \begin{itemize}
    \item $G$ cannot have any vertices of degree \blank\linebreak because
      \blank.
    \item It also cannot be the case that every vertex of $G$ has
      degree $\geq$ \blank.  If they did, then we could find a \blank
      by starting at any\linebreak vertex and walking along edges
      randomly until \blank;\linebreak we would never get stuck
      because \blank.\linebreak However, this is impossible because we
      assumed \blank.
    \end{itemize}
    Hence, $G$ must have some vertex which \blank.\linebreak If we
    delete this vertex along with the edge adjacent to it, it results
    in a graph $G'$ with only \blank vertices;\linebreak we note that
    $G'$ is still \blank\linebreak because \blank\linebreak and also
    \blank\linebreak because \blank.\linebreak  Hence we may apply the
    inductive hypothesis to conclude that $G'$\linebreak \mbox{}\blank.
    Adding the deleted vertex and edge\linebreak back to $G'$ shows that $G$
    \blank,\linebreak which is what we wanted to show.
  \end{itemize}
\end{proof}

Let's do one more!  (You will do the third on your HW.)

\begin{lem}
  $(2),(3) \implies (1)$, that is, \blank\linebreak \mbox{}\blank.
\end{lem}

\begin{proof}
  This proof uses a \term{counting argument}: we will show what we
  wish to show by counting things in multiple ways.

  Let $c$ denote the number of connected components of $G$.  We want
  to show that \blank.

  Number the components of $G$ from $1 \dots c$, and say that
  component $i$ has $n_i$ vertices.  Then \[ \sum_{i=1}^c n_i =
    \underline{\phantom{XXXXXXXXXX}} \] because \blank.\linebreak  Each
  connected component is by definition a \blank graph;\linebreak each
  component must also be \blank\linebreak since we assumed that $G$ is.  Hence
  we may apply \pref{lem:onetwothree} to conclude that component $i$
  \blank.\linebreak
  Adding these up, the total number of edges in $G$ is
  \[ |E| = \sum_{i=1}^c \underline{\phantom{XXXXXX}} =
    \underline{\phantom{XXXXXXXXXXXXX}} \] But we already assumed the
  number of edges in $G$ is \blank,\linebreak and hence \blank as desired.
\end{proof}

\end{document}