% -*- compile-command: "rubber -d graph-proofs-filled.tex" -*-
\documentclass{tufte-handout}

\usepackage{algo-activity}

\title{\thecourse: Some Proofs about Trees (Key)}
\date{}

\begin{document}

\maketitle

\begin{model*}{A theorem about trees}{trees}
\begin{thm}[Trees]
  Let $G = (V,E)$ be a graph with $|V| = n \geq 1$.  Any two of the following
  imply the third:
  \begin{enumerate}
  \item $G$ is connected.
  \item $G$ is acyclic.
  \item $G$ has $n-1$ edges.
  \end{enumerate}
\end{thm}
\end{model*}

\begin{lem} \label{lem:onetwothree} $(1), (2) \implies (3)$.  That is:
  let $G = (V,E)$ be a graph with $|V| = n \geq 1$.  If $G$ is
  connected and acyclic, then $G$ has $n-1$ edges.
\end{lem}

\begin{proof}
  Let $P(n)$ denote the statement ``Any graph $G$ with $n$ vertices
  which is connected and acyclic must have $n-1$ edges.''  We wish to
  show that $P(n)$ holds for all $n \geq 1$.

  The proof is by induction on $n$.
  \begin{itemize}
  \item The base case is when $n = 1$. In this case, $G$ is just
    a single vertex, so it is indeed connected, acyclic, and has $0$
    edges.
  \item For the induction step, suppose $P(k)$ holds for some $k \geq
    1$.  That is, suppose that any graph with $k$ vertices which is
    connected and acyclic must have $k-1$ edges.  Then we wish to show
    $P(k+1)$, that is, any graph with $k+1$ vertices which is
    connected and acyclic must have $k$ edges.

    So, let $G$ be a graph with $k+1$ vertices which is connected and
    acyclic.  We claim that $G$ must have some vertex which is a leaf,
    that is, a vertex of degree $1$, which we can show as
    follows:
    \begin{itemize}
    \item $G$ cannot have any vertices of degree $0$, because
      it is connected (and has at least two vertices).
    \item It also cannot be the case that every vertex of $G$ has
      degree $\geq 2$.  If they did, then we could find a cycle by
      starting at any vertex and walking along edges randomly until
      encountering a repeated vertex; we would never get stuck because
      every vertex has degree $\geq 2$, that is, if we come in along
      one edge there must always be a different edge along which we
      can leave. However, this is impossible because we assumed
      $G$ is acyclic.
    \end{itemize}
    Hence, $G$ must have some vertex which is a leaf. If we delete
    this vertex along with the edge adjacent to it, it results in a
    graph $G'$ with only $k$ vertices; we note that $G'$ is still
    connected (because $G$ was connected and we deleted a leaf) and
    also acyclic (because deleting something from an acyclic graph
    cannot create a cycle).  Hence we may apply the inductive
    hypothesis to conclude that $G'$ has $k-1$ edges.  Adding the
    deleted vertex and edge back to $G'$ shows that $G$ has $k$ edges,
    which is what we wanted to show.
  \end{itemize}
\end{proof}

\begin{lem}
  $(2),(3) \implies (1)$, that is, any acyclic graph with $n$ vertices
  and $n-1$ edges must be connected.
\end{lem}

\begin{proof}
  This proof uses a \term{counting argument}: we will show what we
  wish to show by counting things in multiple ways.

  Let $c$ denote the number of connected components of $G$.  We want
  to show that $c = 1$.

  Number the components of $G$ from $1 \dots c$, and say that
  component $i$ has $n_i$ vertices.  Then \[ \sum_{i=1}^c n_i = n
  \] because adding up the number of vertices in each component gives
  the total number of vertices.  Each connected component is by
  definition a connected graph; each component must also be acyclic
  since we assumed that $G$ is acyclic.  Hence we may apply
  \pref{lem:onetwothree} to conclude that component $i$ has $n_i - 1$
  edges.  Adding these up, the total number of edges in $G$ is
  \[ |E| = \sum_{i=1}^c (n_i - 1) = \left(\sum_{i=1}^c n_i \right) -
    \left(\sum_{i=1}^c 1 \right) = n - c. \] But we already assumed the
  number of edges in $G$ is $n-1$, and hence $c = 1$ as desired.
\end{proof}

\end{document}
