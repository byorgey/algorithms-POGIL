% -*- compile-command: "pdflatex --enable-write18 15-SAT.tex" -*-
\documentclass{tufte-handout}

\usepackage{../algo-activity}
\usepackage{booktabs}

\title{\thecourse\ Activity 15: SAT}
\date{}

\begin{document}

\maketitle

XXX THIS ACTIVITY NEEDS REVISING

% \begin{objective}
%   Students will XXX.
% \end{objective}

\begin{model*}{SAT}{SAT}
  Recall that $\land$ denotes logical AND, and $\lor$ denotes logical
  OR. $\overline{x}$ denotes logical negation. \bigskip

  \begin{tabular}{l|p{2.5in}|p{2.5in}}
    & Examples & Non-examples \\ \hline
    \emph{Term}
    & $x_1$ \newline $x_3$ \newline $\overline{x_3}$
    & $x_1 \land x_2$ \newline $x_2 \lor x_4$ \\ \hline
    \emph{Clause}
    & $x_1$ \newline $x_1 \lor x_2$ \newline $\overline{x_1} \lor x_3
      \lor x_2$ \newline $\overline{x_3}$ \newline $\overline{x_5}
      \lor x_3 \lor \overline{x_7} \lor x_9$
    & $x_1 \land x_2$ \newline $x_2 \lor \overline{x_3} \land x_4$
      \newline $x_1 \Rightarrow x_5$ \\ \hline
    \emph{CNF}
    & $x_1$ \newline $x_2 \lor x_5 \lor \overline{x_1}$ \newline $x_1 \land \overline{x_3}$ \newline $(x_1 \lor
      x_2) \land (\overline{x_3} \lor x_5 \lor x_9 \lor x_8 \lor x_2)$ \newline $x_4
      \land (x_1 \lor x_2) \land \overline{x_5} \land (x_1 \lor
      \overline{x_3} \lor x_5)$ \newline $x_1 \land (\overline{x_1}
      \lor x_3) \land (\overline{x_1} \lor \overline{x_3})$
    & $(x_1 \land x_3) \lor (x_2 \land x_5)$ \newline $x_1 \land (x_2
      \lor (x_3 \land (x_4 \lor (x_5 \land x_6))))$
  \end{tabular}
\end{model*}

\newcommand{\true}{\textsf{T}\xspace}
\newcommand{\false}{\textsf{F}\xspace}

Each variable $x_i$ represents a Boolean value (\true or \false).  An
\emph{assignment} consists in specifying a \true or \false value for
each variable.

\begin{questions}
\item Based on the examples and non-examples of \emph{terms} in the first row
  of the chart, write down a definition of a term.
\item Based on the examples and non-examples, write a definition of
  a \term{clause}.
\item Again based on the model, write a definition of a
  CNF.\marginnote{CNF stands for \term{conjunctive normal form}.}
\item Consider the assignment setting each $x_i$ to \true when $i$ is
  even, and \false otherwise.  For each CNF in the left-hand
  column,
  say whether it evaluates to \true or \false under this assignment
  (using the usual rules of Boolean logic).
\item Find an assignment that makes $x_1 \land (\overline{x_2} \lor
  \overline{x_3} \lor \overline{x_1}) \land (x_2 \lor \overline{x_1})$
  true. (You only need to specify values for $x_1$, $x_2$, and $x_3$.)
\item Does every \term{clause} have some assignment which makes it
  true?  If so, explain why; if not, give a counterexample.
\item Does every \term{CNF} have some assignment which makes it
  true?  If so, explain why; if not, give a counterexample.
\item Based on your previous answer, state an interesting decision
  problem about CNF formulas.
\end{questions}

This is a famous decision problem called \textsc{SAT}.  If we restrict
every \term{clause} to have \emph{exactly three} terms, the
corresponding decision problem is known as \textsc{3-SAT}.

\begin{questions}
\item Explain why $\textsc{3-SAT} \leq_P \textsc{SAT}$.
\end{questions}

XXX expand this part---get them to write down proof!

\begin{thm}
  $\textsc{3-SAT} \leq_P \textsc{Independent-Set}$.
\end{thm}

\end{document}
