% -*- compile-command: "./build.sh" -*-
\documentclass{tufte-handout}

\usepackage{algo-activity}
\usepackage{booktabs}

\title{\thecourse: SAT and 3-SAT}
\date{}

\begin{document}

\maketitle

% \begin{objective}
%   Students will XXX.
% \end{objective}

\newcommand{\True}{\ensuremath{\mathsf{T}}\xspace}
\newcommand{\False}{\ensuremath{\mathsf{F}}\xspace}

\begin{model*}{SAT}{SAT}
  Each variable $x_i$ represents a Boolean value (\True or \False).
  An \emph{assignment} consists in specifying a \True or \False value
  for each variable.  Recall that $\land$ denotes logical AND, and
  $\lor$ denotes logical OR. $\overline{x}$ denotes logical
  negation. For example, $x_1 \land \overline{x_3}$ means ``$x_1$ and
  not $x_3$''.\bigskip

  \begin{tabular}{l|p{2.5in}|p{2.5in}}
    & Examples & Non-examples \\ \hline
    \emph{Term}
    & $x_1$ \newline $x_3$ \newline $\overline{x_3}$
    & $x_1 \land x_2$ \newline $x_2 \lor x_4$ \\ \hline
    \emph{Clause}
    & $x_1$ \newline $x_1 \lor x_2$ \newline $\overline{x_1} \lor x_3
      \lor x_2$ \newline $\overline{x_3}$ \newline $\overline{x_5}
      \lor x_3 \lor \overline{x_7} \lor x_9$
    & $x_1 \land x_2$ \newline $x_2 \lor \overline{x_3} \land x_4$
      \newline $x_1 \Rightarrow x_5$ \\ \hline
    \emph{CNF}
    & $x_1$ \newline $x_2 \lor x_5 \lor \overline{x_1}$ \newline $(x_1 \lor
      x_2) \land (\overline{x_3} \lor x_5 \lor x_9 \lor x_8 \lor x_2)$ \newline $x_4
      \land (x_1 \lor x_2) \land \overline{x_5} \land (x_1 \lor
      \overline{x_3} \lor x_5)$ \newline $x_1 \land (\overline{x_1}
      \lor x_3) \land (\overline{x_1} \lor \overline{x_3})$ \newline $x_1 \land \overline{x_3}$
    & $(x_1 \land x_3) \lor (x_2 \land x_5)$ \newline $x_1 \land (x_2
      \lor (x_3 \land (x_4 \lor (x_5 \land x_6))))$
  \end{tabular}
\end{model*}

\begin{questions}
\item Based on the examples and non-examples of \term{terms} in the first row
  of the chart, write down a definition of a \term{term}.
\item Based on the examples and non-examples, write a definition of
  a \term{clause}. Be sure to use the word \term{term} in your definition.
\item Again based on the model, write a definition of a
  CNF. Be sure to use the word \term{clause}.\marginnote{CNF stands for \term{conjunctive normal form}.}
\item Consider the assignment setting each $x_i$ to \True when $i$ is
  even, and \False when $i$ is odd.  For each CNF in the left-hand
  column, say whether it evaluates to \True or \False under this
  assignment (using the usual rules of Boolean logic).
\item Find an assignment that makes $x_1 \land (\overline{x_2} \lor
  \overline{x_3} \lor \overline{x_1}) \land (x_2 \lor \overline{x_1})$
  true. (You only need to specify values for $x_1$, $x_2$, and $x_3$.)
\item Does every \term{clause} have some assignment which makes it
  true?  If so, explain why; if not, give a counterexample.
\item Does every \term{CNF} have some assignment which makes it
  true?  If so, explain why; if not, give a counterexample.
\item Based on your previous answer, state an interesting decision
  problem about CNF formulas.
\end{questions}

This is a famous decision problem called \textsc{SAT}.  If we restrict
every \term{clause} to have \emph{exactly three} terms, the
corresponding decision problem is known as \textsc{3-SAT}.

\begin{questions}
\item Explain why $\textsc{3-SAT} \leq_P \textsc{SAT}$.
\end{questions}

It turns out that $\textsc{SAT} \leq_P \textsc{3-SAT}$ as well, although
this is extremely nonobvious!  In fact, $3$ is the smallest $k$ for
which $\textsc{SAT} \leq_P k\textsc{-SAT}$ (\textsc{1-SAT} is trivial
(think about it) and \textsc{2-SAT} can be solved in linear time by a
very clever application of DFS).

\pause

It turns out that we can relate $\textsc{3-SAT}$ to another problem we
have studied before:

\begin{thm}
  $\textsc{3-SAT} \leq_P \textsc{Independent-Set}$.
\end{thm}

Let's prove it!

\begin{model*}{}{SATtoIS}
  \[ (x_1 \lor \overline{x_3} \lor x_4) \land (x_2 \lor x_1 \lor x_3) \land
    (\overline{x_1} \lor x_3 \lor \overline{x_4}) \]

  \begin{center}
  \begin{diagram}[width=200]
    import SAT
    dia = [[p 1, n 3, p 4], [p 2, p 1, p 3], [n 1, p 3, n 4]]
      # map vtri
      # map centerY
      # hsep 2
      # frame 2
      where
        p i = "x_" ++ show i
        n i = "\\overline{x_" ++ show i ++ "}"
  \end{diagram}

  \begin{diagram}[width=200]
    import SAT
    dia = [[p 1, n 3, p 4], [p 2, p 1, p 3], [n 1, p 3, n 4]]
      # zipWith (\c vs -> c .>> vtri vs) ["a", "b", "c"]
      # map centerY
      # hsep 2
      # connectPerim' styleA
          ("a" .> (1 :: Int)) ("c" .> (1 :: Int)) (1/8 @@ turn) (3/8 @@ turn)
      # connectPerim' (with & arrowHead .~ noHead)
          ("b" .> (2 :: Int)) ("c" .> (1 :: Int)) (1/8 @@ turn) (1/2 @@ turn)
      # connectPerim' styleB
          ("a" .> (2 :: Int)) ("b" .> (3 :: Int)) (7/8 @@ turn) (5/8 @@ turn)
      # connectPerim' styleB
          ("a" .> (2 :: Int)) ("c" .> (2 :: Int)) (7/8 @@ turn) (5/8 @@ turn)
      # connectPerim' styleB
          ("a" .> (3 :: Int)) ("c" .> (3 :: Int)) (7/8 @@ turn) (5/8 @@ turn)
      # frame 2
      where
        styleA = (with & arrowShaft .~ arc xDir (-1/8 @@ turn) & arrowHead .~ noHead)
        styleB = (with & arrowShaft .~ arc xDir (1/4 @@ turn) & arrowHead .~ noHead)
        p i = "x_" ++ show i
        n i = "\\overline{x_" ++ show i ++ "}"
  \end{diagram}
  \end{center}
\end{model*}

\begin{proof}
  To show $\textsc{3-SAT} \leq_P \textsc{Independent-Set}$, we assume
  that we have a black box to solve \blank, and show\\ how we can
  use it to construct a solution to \blank, \\ with only a polynomial
  amount of additional work.

  So suppose we have a logical formula in CNF, where every clause has
  exactly three terms.
\end{proof}
\end{document}
