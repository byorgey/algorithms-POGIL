% -*- compile-command: "rubber -d --unsafe dynamic-programming-intro.tex" -*-
\documentclass{tufte-handout}

% F'19:
% This one seems short but working through it completely, then having
% reporters compare with other teams took until around the 40-minute
% mark or so.  And it was really good, they all got it.  Took the last
% 10 minutes to sum up what we had done and put everything in a
% big-picture context.

\usepackage{algo-activity}

\title{\thecourse: Introduction to Dynamic Programming}
\date{}

\begin{document}

\maketitle

\begin{objective}
  Students will apply memoization techniques to speed up overlapping
  recursion.
\end{objective}

\begin{model}{Fibonaccis}{fibonaccis}
Here are three functions to compute Fibonacci numbers, implemented in
Python.  You may assume that they are all correct.

\begin{verbatim}
def fib1(n):
    if n <= 1:
        return n
    else:
        return fib1(n-1) + fib1(n-2)


def fib2(n):
    fibs = [0] * (n+1)   # Create initial array of all 0s
    fibs[1] = 1

    for i in range(2, n+1):
        fibs[i] = fibs[i-1] + fibs[i-2]

    return fibs[n]


fibtable = [0,1]

def fib3(n):

    while len(fibtable) < n+1:
        fibtable.append(-1)

    if fibtable[n] == -1:
        fibtable[n] = fib3(n-1) + fib3(n-2)

    return fibtable[n]
\end{verbatim}
\end{model}

\begin{questions}
\item Recall that the Fibonacci numbers are defined by the recurrence
  \begin{align*}
    F_0 &= 0 \\ F_1 &= 1 \\ F_n &= F_{n-1} + F_{n-2}
  \end{align*}
  Which of the three implementations corresponds most directly to
  this definition?
\item Draw the call tree for \verb|fib1(5)|. \vspace{1in}
\item How many times does \verb|fib1(2)| occur in the call tree?
  What about \verb|fib1(1)|?  \verb|fib1(0)|?
\item It turns out that \verb|fib1| is extremely slow.\footnote{In
    fact, it takes $\Theta(\varphi^n)$ time.}  Intuitively, why is it
  so slow?
\item Trace the execution of \verb|fib2(5)| and explain how it works. \vspace{1in}
\item Which does more work, \verb|fib2(5)| or \verb|fib1(5)|?  Why?
\newpage
\item In terms of $\Theta$, how long does \verb|fib2(n)|
  take?\footnote{For the purposes of this activity, you should assume
    that each addition takes constant time.  However, as you know from
    a previous activity, it is more accurate to say that addition
    takes linear time in the number of bits, which actually makes a
    difference here since Fibonacci numbers can get quite large.  You
    will analyze the situation more precisely on the HW.}
\item Suppose we switch the direction of the \verb|for| loop in
  \verb|fib2|, so \verb|i| loops from \verb|n| down to \verb|2|.
  Would it still work?  Why or why not?
\item Trace the execution of \verb|fib3(5)| and explain how it works. \vspace{1in}
\item In terms of $\Theta$, how long does \verb|fib3(n)| take?
\item Fill in this statement: \verb|fib3| is just like \verb|fib1|
  except that \\ \blank.
\item Fill in this statement: \verb|fib2| is just like \verb|fib3|
  except that \\ \blank.
\item Why don't we do something akin to \verb|fib2| or \verb|fib3| for
  merge sort?
\item Consider the following recursive definition of $Q(n)$ for $n
  \geq 0$:
  \begin{align*}
    Q(0) &= 0 \\
    Q(1) &= Q(2) = 1 \\
    Q(n) &= \max \begin{cases} Q(n - 3)^2 \\ Q(n-1) + Q(n-2) \end{cases}
  \end{align*}
  (Note that there are \emph{three} base cases.)  Using pseudocode,
  write an algorithm to calculate $Q(n)$ efficiently.
\end{questions}

\end{document}
