% -*- compile-command: "rubber -d --unsafe dynamic-programming-intro.tex" -*-
\documentclass{tufte-handout}

\usepackage{algo-activity}

\graphicspath{{images/}{../images/}}

\title{\thecourse: Introduction to Dynamic Programming}
\date{}

\begin{document}

\maketitle

\begin{objective}
  Students will apply memoization techniques to speed up recursion
  with overlapping subproblems.
\end{objective}

\begin{pobjective}
  Students will communicate high-level insights about recursive
  functions orally and in writing.
\end{pobjective}

Begin by \textbf{skimming} the following model.  You do not need to
understand all the code right away; the activity will guide you
through the process of understanding it.

\begin{model}{Fibonaccis}{fibonaccis}
Here are three functions to compute Fibonacci numbers, implemented in
Python.  You may assume that they are all correct.

\begin{verbatim}
def fib1(n):
    if n <= 1:
        return n
    else:
        return fib1(n-1) + fib1(n-2)


def fib2(n):
    fibs = [0] * (n+1)   # Create initial array of all 0s
    fibs[1] = 1

    for i in range(2, n+1):
        fibs[i] = fibs[i-1] + fibs[i-2]

    return fibs[n]


fibtable = [0,1]  # global table of Fibonacci numbers

def fib3(n):

    while len(fibtable) < n+1:
        fibtable.append(-1)

    if fibtable[n] == -1:
        fibtable[n] = fib3(n-1) + fib3(n-2)

    return fibtable[n]
\end{verbatim}
\end{model}

\section{Critical Thinking Questions: \texttt{fib1} (15 minutes)}

\begin{questions}
\item Recall that the Fibonacci numbers are defined by the recurrence
  \begin{align*}
    F_0 &= 0 \\ F_1 &= 1 \\ F_n &= F_{n-1} + F_{n-2}
  \end{align*}
  Which of the three implementations corresponds most directly to
  this definition?
\item Draw the call tree for \verb|fib1(5)|. \vspace{1in}
\item How many times does \verb|fib1(2)| occur in the call tree?
  What about \verb|fib1(1)|?  \verb|fib1(0)|?
\item It turns out that \verb|fib1| is extremely slow.\footnote{In
    fact, it takes $\Theta(\varphi^n)$ time.}  What do you think makes
  it so slow?
\end{questions}

  \begin{fullwidth}
  \begin{mdframed}
    \begin{itemize}
    \item Send your \textbf{reporter} to another group to check your
      understanding of what makes \verb|fib1| slow.
    \item While the reporter is gone, the \textbf{reflector} should
      take a minute to think of one \textbf{strength} of your group's
      work so far today and one \textbf{improvement} to suggest.
    \item When the reporter gets back, they should share what they
      learned.
    \item The reflector should then share their insights with the
      group.
    \end{itemize}
  \end{mdframed}
  \end{fullwidth}
\pause

\section{Critical Thinking Questions: \texttt{fib2} and \texttt{fib3}
  (25 minutes)}

\begin{questions}
\item Trace the execution of \verb|fib2(5)|. As a group, come up with
  one or two complete English sentences to explain how it
  works. \vspace{1in}
\item Which does more work, \verb|fib2(5)| or \verb|fib1(5)|?  Why?
\item In terms of $\Theta$, how long does \verb|fib2(n)|
  take?\footnote{For the purposes of this activity, you should assume
    that each addition takes constant time.  However, as you may know from
    a previous activity, it is more accurate to say that addition
    takes linear time in the number of bits, which actually makes a
    difference here since Fibonacci numbers can get quite large.  You
    will analyze the situation more precisely on a homework assignment.}
\item Suppose we switch the direction of the \verb|for| loop in
  \verb|fib2|, so \verb|i| loops from \verb|n| down to \verb|2|.
  Would it still work?  Why or why not?
\item Trace the execution of \verb|fib3(5)| and explain how it works. \vspace{1in}
\item In terms of $\Theta$, how long does \verb|fib3(n)| take?
\item Fill in this statement: \verb|fib3| is just like \verb|fib1|
  except that \\ \blank.
\item Fill in this statement: \verb|fib2| is just like \verb|fib3|
  except that \\ \blank.
\item Consider the following recursive definition of $Q(n)$ for $n
  \geq 0$:\marginnote{Note that there are \emph{three} base cases.}
  \begin{align*}
    Q(0) &= 0 \\
    Q(1) &= Q(2) = 1 \\
    Q(n) &= Q(n-1) + 2 \cdot \left[Q(n - 3)\right]^2
  \end{align*}
  \begin{enumerate}[label=(\alph*)]
  \item Calculate $Q(3)$, $Q(4)$, and $Q(5)$. \vspace{1in}
  \item Using pseudocode, or any language your group agrees to use, write an
    algorithm to calculate $Q(n)$ efficiently.
  \end{enumerate}
\end{questions}

\newpage

\section{Facilitation plan}
\label{sec:facilitation}

\section{CTQs 1}

15 minutes, including sending reporters and having reflectors share.
Use best judgment to see if a bigger class discussion is needed, but
hopefully just send them on to the next section.

\section{CTQs 2}

25 minutes.  This section is more involved.  Have reporters share out
at the end to check understanding.

\section{Wrap up}

10 minutes for questions and wrap-up.  Explain that they have learned
today the essence of ``dynamic programming'' and ``memoization''. DP
can sometimes be presented in confusing ways, but it's really just
about using memoization to speed up recursion with overlapping
subproblems.  Fibonacci numbers are too simple to really need DP, but
make a nice simple case study.  In the next activities we will move on
to more realistic applications.

\newpage

\section{Author notes}
\label{sec:author}

\begin{reflect}{F'19}
This one seems short but working through it completely, then having
reporters compare with other teams took until around the 40-minute
mark or so.  And it was really good, they all got it.  Took the last
10 minutes to sum up what we had done and put everything in a
big-picture context.
\end{reflect}

\end{document}
